\documentclass[12pt]{amsart}	
\usepackage{amssymb}
\usepackage{amsthm}

\usepackage{mycros}
\usepackage{mycrosROKs}

\title{On $q,\tau,$ and $s$ (or: on what`s in a name?)}
\author{Justin Katz}
\begin{document}
\maketitle

Cultural norms dictate that the complex variables $q$ and $\tau$ should be related by an identity like 
\begin{align*}
	 q = \exp(2\pi i \tau) \text{ or } \exp (\pi i \tau)
\end{align*}
and that $\tau$ should run about the upper halfplane $\half$.

Similarly, the variable $s$ is typically reserved for the exponent in dirichlet series, e.g. 
\begin{align*}
	\zeta(s)=\sum_{n=1}^\infty n^{-s}. 	
\end{align*}

The goal of this document is to explicate the `nature' of the variables $\tau,q,$ and $s$: to describe an intrinsic description of their domain and codomain.

\section{On $s$}


\begin{itemize}
	\item The variable $s$ seems to live as a parameter on some manner of dual space: e.g. $\Hom(\C^\times , \C^\times)= \{ \cdot \mapsto \cdot^s:  s\in \C\}$. 
\end{itemize}

 
\bibliographystyle{plain}
\bibliography{BigBib}

\end{document}
