\documentclass[11pt]{amsart}
\usepackage[margin=1in]{geometry}                % See geometry.pdf to learn the layout options. There are lots.
\geometry{letterpaper}                   % ... or a4paper or a5paper or ... 
%\geometry{landscape}                % Activate for for rotated page geometry
%\usepackage[parfill]{parskip}    % Activate to begin paragraphs with an empty line rather than an indent
\usepackage{graphicx}
\usepackage{amssymb}
\usepackage{epstopdf}
\usepackage{amsthm}
\usepackage{mathrsfs}
\usepackage{url}
\usepackage[all]{xy}
\renewcommand{\tilde}{\widetilde}
\newcommand{\hmod}{H^{1}(\modcurve)}
\newcommand{\cmod}{C^{\infty}_{b}(\modcurve)}
\newcommand{\tr}{\Tr}
\newcommand{\Ch}{\operatorname{Ch}}
\newcommand{\ind}{\operatorname{ind}}
\newcommand{\ip}[2]{\langle #1, #2 \rangle}
\newcommand{\C}{\mathbb{C}}
\newcommand{\Z}{\mathbb{Z}}
\newcommand{\R}{\mathbb{R}}
\newcommand{\Q}{\mathbb{Q}}
\newcommand{\til}[1]{\tilde{#1}}
\newcommand{\PSL}{\operatorname{PSL}}
\newcommand{\diag}{\text{diag}}
\newcommand{\half}{\mathfrak{H}}
\newcommand{\intring}{\mathcal{O}}
\newcommand{\sch}{\mathscr{S}}
\newcommand{\tor}{\mathbb{T}}
\newcommand{\bep}{\mathfrak{B}}
\newcommand{\ipd}{\ip{\cdot}{\cdot}}
\newcommand{\fredth}{\tilde{\theta}}
\newcommand{\resth}{\tilde{\theta}^{-1}}
\renewcommand{\sl}{\mathfrak{sl}}
\renewcommand{\phi}{\varphi}
\newcommand{\ltmod}{L^{2}(\modcurve)}
\renewcommand{\k}{\mathfrak{k}}
\newcommand{\halfred}{\tilde{\laphalf}}
\newcommand{\halfres}{(\lambda-\halfred)^{-1}}
\newcommand{\zpz}{\Z/p\Z}
\newcommand{\fredlapsn}{\tilde{\Delta}^{\nsphere}}
\newcommand{\ressn}{(1-\fredlapsn)^{-1}}
\newcommand{\vol}{\operatorname{vol}}
\newcommand{\ijsum}{\sum_{i < j}}
\newcommand{\poly}{\C[x_{1},\ldots,x_{n}]}
\newcommand{\rest}{\big|}
\newcommand{\dtx}{\frac{\partial^{2}}{\partial x^{2}}}
\newcommand{\dty}{\frac{\partial^{2}}{\partial y^{2}}}
\newcommand{\g}{\mathfrak{g}}
\newcommand{\dophalf}{\frac{\dop x \dop y}{y^{2}}}
\newcommand{\tbt}[4]{\left[ \begin{smallmatrix}
		#1 & #2 \\
		#3 & #4 
	\end{smallmatrix} \right] }
\newcommand{\Tbt}[4]{\left[ \begin{matrix}
		#1 & #2 \\
		#3 & #4 
	\end{matrix} \right] }
\newcommand{\Tr}{\operatorname{tr}}
\renewcommand{\r}{\mathfrak{r}}
\newcommand{\ciamod}{C^{\infty}_{a}(\modcurve)}
\newcommand{\cicmod}{C^{\infty}_{c}(\modcurve)}
\newcommand{\hamod}{H^{1}_{a}(\modcurve)}
\newcommand{\ltmoda}{L^{2}_{a}(\modcurve)}
\newcommand{\ltmodcts}{L^{2}_{\text{cts}}(\modcurve)}
\newcommand{\mel}{\mathcal{M}}
\newcommand{\ltmodcfm}{L^{2}_{\text{cfm}}(\modcurve)}
\newcommand{\res}{\operatorname{res}}
\newcommand{\re}{\operatorname{Re}}
\newcommand{\im}{\operatorname{Im}}
\newcommand{\Ad}{\operatorname{Ad}}
\newcommand{\Aut}{\operatorname{Aut}}
\renewcommand{\O}{\operatorname{O}}
\newcommand{\To}{\longrightarrow}
\newcommand{\Mapsto}{\longmapsto}
\newcommand{\inc}{\operatorname{inc}}
\newcommand{\gothic}[1]{\mathfrak{#1}}
\newcommand{\so}{\mathfrak{so}}
\newcommand{\Fund}{\mathcal{F}}
\newcommand{\partone}[1]{\frac{\partial}{\partial x_{#1}}}
\newcommand{\parttwo}[1]{\frac{\partial^{2}}{\partial x_{#1}^{2}}}
\newcommand{\PGL}{\operatorname{PGL}}
\newcommand{\F}{\mathbb{F}}
\newcommand{\ol}{\overline}
\newcommand{\inj}{\hookrightarrow}
\newcommand{\surj}{\twoheadrightarrow}
\newcommand{\trace}{\operatorname{Tr}}
\newcommand{\proj}{\operatorname{proj}}
\newcommand{\der}{\frac{d^2}{dx^2}}
\newcommand{\four}{\mathcal{F}}
\newcommand{\laphalf}{\Delta^{\half}}
\newcommand{\eps}{\varepsilon}
\newcommand{\dom}{\operatorname{dom}}
\newcommand{\id}{\operatorname{id}}
\newcommand{\Ind}{\operatorname{Ind}}
\newcommand{\Res}{\operatorname{Res}}
\newcommand{\End}{\operatorname{End}}
\newcommand{\SL}{\operatorname{SL}}
\newcommand{\GL}{\operatorname{GL}}
\newcommand{\SO}{\operatorname{SO}}
\newcommand{\Orth}{\operatorname{O}}
\newcommand{\dop}{\,{\rm d}}
\newcommand{\nsphere}{S}
\newcommand{\ltnsphere}{L^{2}(\nsphere)}
\newcommand{\honsphere}{H^{1}(\nsphere)}
\newcommand{\lthalf}{L^{2}(\half)}
\newcommand{\ltg}{L^{2}(G)}   
\newcommand{\modcurve}{\Gamma \backslash \half}
\newcommand{\Gal}{\operatorname{Gal}}
\newcommand{\ipn}[2]{\langle #1, #2 \rangle_1}
\newcommand{\Graph}{\operatorname{graph}}
\newcommand{\mhaar}[1]{\frac{\operatorname{d}#1}{#1}}
\newcommand{\rn}{\R^{n}}
\newcommand{\laprn}{\Delta^{\rn}}
\newcommand{\lapsn}{\Delta^{\nsphere}}
\newcommand{\ltrn}{L^{2}(\rn)}
\newcommand{\cirn}{C^{\infty}(\rn)}
\newcommand{\horn}{H^{1}(\rn)}
\newcommand{\inv}{{-1}}
\newcommand{\p}{\mathfrak{p}}
\renewcommand{\P}{\mathfrak{P}}
\newcommand{\frob}[1]{\operatorname{frob}(#1)}
\newcommand{\Ell}{\mathcal{L}}
\newcommand{\arccosh}{\operatorname{arccosh}}
\newcommand{\I}{\mathbb{I}}
\newcommand{\A}{\mathbb{A}}
\newcommand{\Of}{\mathcal{O}}
\newcommand{\Isom}{\operatorname{Isom}}
\newcommand{\lmod}{\backslash}
\newcommand{\rmod}{/}
\newcommand{\Com}{\operatorname{Com}}
\newcommand{\hecke}{\mathcal{H}}
\newcommand{\ord}{\operatorname{ord}}
\newcommand{\mf}{\mathfrak}
\newcommand{\q}{\textbf{q}}
\newcommand{\normset}{\mathcal{N}}
\renewcommand{\Ell}{\mathcal{L}}
\newcommand{\infl}{\operatorname{Infl}}
\newcommand{\vchar}{\operatorname{Vchar}}
\newcommand{\nspec}{\mathcal{N}}
\newcommand{\prim}{\operatorname{prim}}

\theoremstyle{definition}
\newtheorem{claim}{Claim}
\newtheorem*{question*}{Question}
\newtheorem{thm}{Theorem}
\newtheorem{prop}{Proposition}
\newtheorem{remark}{Remark}
\newtheorem*{remark*}{Remark}
\newtheorem{mydef}{Definition}
\newtheorem{fact}{Fact}
\newtheorem{lemma}{Lemma}
\newtheorem{cor}{Corollary}

\newcommand{\Of}{\mathcal{O}}
\newcommand{\lmod}{\backslash}
\usepackage{titlesec}

\titleformat{\section}
  {\normalfont\Large\bfseries}{\thesection}{1em}{}[{\titlerule[0.8pt]}]
\titleformat{\subsection}
  {\normalfont\Large\bfseries}{\thesubsection}{1em}{}[{}]

\usepackage[utf8]{inputenc}
\usepackage[english]{babel}
\newcommand{\ord}{\operatorname{ord}}
\setlength{\parindent}{0em}
\setlength{\parskip}{1em}
\newtheorem{exer}{Problem}
\newcommand{\PSL}{\operatorname{PSL}}
\newcommand{\amat}[1] {\Tbt{e^{#1}}{0}{0}{e^{-#1}}}
\newcommand{\nmat}[1]{\Tbt{1}{#1}{0}{1}}
\newcommand{\kmat}[1]{\Tbt{\cos #1}{\sin #1}{-\sin #1}{\cos #1}}
\newcommand{\dbdt}{\frac{d}{dt}}
\title{Trace formula talk}
\author{Justin Katz}
\begin{document}
\maketitle
%\section{The desired computation}
%Let $G$ be a unimodular locally compact group, $\Gamma$ be a discrete subgroup, and $R$ be the right regular representation of $G$ on $L^{2}(\Gamma \lmod G)$. Convert the group representation $R$ to an algebra representation of $C_{c}(G)$, still called $R$, by 
%	\begin{align*}
%		(R(f)\phi)(x)=\int_{G}f(y)\phi(xy)\dop y.
%	\end{align*}
%In situations where it is sensible, the following computation `winds up' this integral along $\Gamma$:
%	\begin{align*}
%		(R(f)\phi)(x)&= \int_{G}f(y)\phi(xy)\dop y \\
%				&= \int_{G}f(x^{-1}y)\phi(y)\dop y \\
%				&=\int_{\Gamma \lmod G}\sum_{\gamma\in\Gamma}f(x^{-1}\gamma y)\phi(\gamma y)\dop y \\
%				&=\int_{\Gamma \lmod G}\left (\sum_{\gamma \in \Gamma}f(x^{-1}\gamma y)\right ) \phi(y) \dop y,
%	\end{align*}
%where we've used $\Gamma$ invariance of $\phi$ (not $f$!) in the last step. The sum appearing in the last integrand manifests a kernel defining $R(f)$. Set
%	\begin{align*}
%		K_{f}(x,y)=\sum_{\gamma \in \Gamma}f(x^{-1}\gamma y).
%	\end{align*}
%Then, on the one hand, the trace of $R(f)$ is 
%	\begin{align*}
%		\Tr R(f)=\int_{\Gamma\lmod G} K_{f}(x,x)\dop x. 
%	\end{align*}
%Now, denote the set of conjugacy classes of elements in $\Gamma$ by $\{\Gamma\}$, and for a class $\gamma$ in $\{\Gamma\}$  denote the centralizers of $\gamma$ in $\Gamma$ and $G$ respectively by:
%	\begin{align*}
%		\Gamma_{\gamma}&=\{ g\in \Gamma: g^{-1}\gamma g =\gamma\} \\
%		G_{\gamma}&=\{g \in G:g^{-1}\gamma g= \gamma\}.
%	\end{align*}
%Now note that the set orbits of $\Gamma$ acting on itself by conjugation is precisely $\{\Gamma\},$ and the stabilizer of an element in an orbit $\gamma$ is $\Gamma_{\gamma}$.  Orbit stabilizer affords us the decomposition of $\Gamma$ (abusing notation, conflating classes and elements)
%	 \begin{align*}
%		\Gamma&=\bigcup_{\gamma \in \{\Gamma\}}\bigcup_{g \in \Gamma_{\gamma}\lmod \Gamma} \{ \delta \gamma \delta^{-1}\}
%	\end{align*}
%which manifests in the sum 
%	\begin{align*}
%		\sum_{\gamma \in \Gamma} f(x\gamma x^{-1}) = \sum_{\gamma \in \{\Gamma\}} \sum_{g\in \Gamma_{\gamma}\lmod \Gamma} f(xg \gamma g^{-1} x^{-1}).
%	\end{align*}
% pass the sum over $\{\Gamma\}$ through the integral, and then `wind' $\Gamma \lmod G$ along $\Gamma_{\gamma} \lmod \Gamma$ to obtain
%	\begin{align*}
%		\Tr(R(f))&= \sum_{\gamma \in \{\Gamma\}}\sum_{g\in \Gamma_{\gamma}\lmod\Gamma}f(x^{-1}g^{-1}\gamma gx) \\
%			&=\sum_{\gamma \in \{\Gamma\}}  \sum_{\Gamma_{g}\lmod G} f(x^{-1}\gamma x) \dop x.
%	\end{align*}
%As a last refinement, wind $\Gamma_{\gamma}\lmod G$ along $\Gamma_{\gamma}\lmod G_{\gamma}$ to obtain 
%	\begin{align*}
%				\Tr(R(f)) &=  \sum_{\gamma \in \{\Gamma\}}\vol(\Gamma_{\gamma}\lmod G_{\gamma}) \int_{G_{\gamma}\lmod G}f(x^{-1}\gamma x)\dop x.
%	\end{align*}
%
%
%Now, if as a $G$ rep (and thereby as a $C_{c}(G)$ rep,) we have a decomposition 
%	\begin{align*}
%		L^{2}(\Gamma \lmod G) = \bigoplus_{\pi \in \hat{G}} m_{\pi}^{\Gamma} \pi,  \qquad \text{where $m_{\pi}^{\Gamma}=\dim(\hom_{G}(\pi,L^{2}(\Gamma\lmod G)))$},
%	\end{align*}
%	and on the other hand, 
%		\begin{align*}
%			\Tr R(f)= \sum_{\pi \in \hat{G}} m_{\pi}\Tr \pi(f).
%		\end{align*}
%The trace formula is the equality of these two expressions. 
%
%Note, there are many things that are potentially nonsense here: 
%\begin{enumerate}
%	 \item Is the operator $R(f)$ of trace class? 
%	 \item Are the passages of sums through integrals sensible?
%	 \item Does $L^{2}(\Gamma  \lmod G)$ have such a decomposition? Need the multiplicities be finite? Are the irreducibles finite dimensional? If not, is $\pi(f)$ trace class on $\pi$?
%\end{enumerate}
%\section{Situations where it works, no problem: 1--Finite groups}
%Suppose $G$ is finite. Then any $\Gamma \leq G$ is discrete. Haar measure on $G$ and $\Gamma$ is counting, so integrals are merely sums over group elements. Note, in this case, that the space $L^{2}(\Gamma \lmod G)$ is finite dimensional and consists of (all) left $\Gamma$ invariant functions on $G$. This coincides with the representation induced from the trivial one on $\Gamma$.
%
%All sums being finite, and spaces finite dimensional, the discussion in section 1 applies with no stipulations. This gives us a formula
%	\begin{align*}
%		\sum_{\pi \in \hat{G}} m_{\pi}^{\Gamma}\Tr(\pi(f)) = \Tr R(f)= \sum_{\gamma \in \{\Gamma\}}\left |\Gamma_{\gamma}\lmod G_{\gamma}\right| \sum_{x \in G_{\gamma}\lmod G}f(x^{-1}\gamma x).
%	\end{align*}
%
%An application:
%\begin{thm}For an irreducible  $\pi$ of $G$, 
%	\begin{align*}
%		\dim_{\C}\hom_{G}(\pi,L^{2}(\Gamma \lmod G))=\dim_{\C}(\hom_{\Gamma}(1,\pi))
%	\end{align*}
%\end{thm}
%\begin{proof}
%Fix an irreducible $\sigma$ of $G$ and apply the trace formula to $f_{\sigma}g\mapsto \overline{\Tr(\sigma(g))}$, the (conjugate) character of $\sigma$. The operators $\pi(\overline{\Tr(\sigma)})$ appearing on the LHS act as
%	\begin{align*}
%		v\mapsto \sum_{g\in G}\overline{\Tr(\sigma)}(g)\pi(g)v.
%	\end{align*}
%By orthogonality (and linearity!) of characters, the trace of this operator is $0$ unless $\pi=\sigma$ and $|G|$ otherwise. 
%
%To attack the RHS, recall that characters are conjugacy invariant, so for each conjugacy class $\gamma$
%	\begin{align*}
%		\sum_{x\in G_{\gamma}\lmod G}\Tr(\sigma(x^{-1}\gamma x))&= \left| G_{\gamma} \lmod G \right| \overline{\Tr(\sigma(\gamma ))}.
%	\end{align*}
%Now the RHS is 
%	\begin{align*}
%		\Tr(f_{\sigma})&=\sum_{\gamma \in \{\Gamma\}} \left | \Gamma_{\gamma}\lmod G_{\gamma} \right | \left | G_{\gamma} \lmod G \right| \overline{\Tr \sigma (\gamma)} \\
%				&=|G| \overline{\sum_{\gamma\in \{\Gamma\}}1/|\Gamma_{\gamma}| \Tr(\sigma(\gamma))} \\
%			        &=|G| \dim \hom_{\Gamma}(1,\sigma)
%	\end{align*}
%\end{proof}
%\section{Situations where it works, with some effort: Compact Quotients}
%%Now using the discreteness of $\Gamma$ and the compact support of $f$ to conclude that the sum defining $K_{f}$ is locally finite, we
%Functional analytic preliminaries:
%
%\begin{enumerate}
%\item Compact operator means image of ball is precompact
%\item Equiv, compact means limit of finiite rNAK operators
%\item Collection of compact operators is closed in norm topology
%\item Take $A$ bounded on $H$. If for some orthonormal basis $B$, the sum $\sum_{b \in B}|Ab|^{2}$ is finite, then its value is independent of the orthonormal basis. Call such an operator \emph{Hilbert--Schmidt}. Proof: Take another basis $B'$ and write $Ab=\sum_{b' \in B'}\ip{Ab}{b'}b'$ then by orthonormality $|Ab|^{2}=\sum_{b' \in B'}|\ip{Ab}{b'}|^{2}$. Then
%	\begin{align*}
%		\sum_{b\in B} |Ab|^{2}&=\sum_{b\in B}\sum_{b'\in B'}|\ip{Ab}{b'}|^{2} \\
%						&=\sum_{b\in B}\sum_{b'\in B'}|\ip{b}{A^{*}b'}|^{2}\\ 
%	\end{align*}
%Switch the sums, note that $|A|_{2}=|A^{*}|_{2}$ and we're done.
%\item Hilbert-Schmidt operators are compact. Proof: manifest them as limit of finite rNAK operators-- Since the sum $\sum |Ae_{i}|^{2}$ converges, for any $n$ there is an $N_{n}$ with $\sum_{i>N_{n}}Ae_{i}<1/n^{2}$. So define $A_{n}$ as the projection of $A$ onto the space spanned by the first $N_{n}$ basis vectors.  
%\item Say that $A$ is \emph{of trace class} if for every orthonormal basis, the sum 
%	\begin{align*}
%		\sum_{b\in B}|\ip{Ab}{b} 
%	\end{align*}
%converges.  (Note, the requirement that this converge for every ONB is not vacuous. Delicate: if of trace class, then sum is absolutely convergent and independent of basis. If not of trace class, could be finite for some bases but not others). Define the trace of a trace class operator as that sum. 
%\item Trace class implies Hilbert Schmidt
%\item A product of Hilbert-Schmidt's is trace class. Proof: Use adjoint and cauchy schwarz.  
%\item $|\Tr(AB)|\leq |A||B|$
%\item If $A$ of trace class, then so is $A^{*}$ w/ $\Tr A^{*}=\overline{\Tr A}$. 
%\item An operator defined by an $L^{2}$ kernel is Hilbert--Schmidt
%\end{enumerate}
%RHS justifications:
%Since $R$ is a convolution homomorphism, the operator $R(f)$ defined by a function $f=f_{1}*f_{2}\in C_{c}(G)*C_{c}(G)\leq L^{2}(G)\times L^{2}(G)$,   is a product of Hilbert--Schmidt operators, and thus is trace class. A straightforward computation shows that, in fact, the integral kernel of the product is the product of integral kernels, in the sense that 
%	\begin{align*}
%		K_{f}(x,y)=\int_{G}K_{f_{1}}(x,z)K_{f_{2}}(z,y)\dop z. 
%	\end{align*}
%So, for such $f$, the RHS computations are justified.
%	\begin{remark}
%		For $G$ a Lie group, or adelic points of a reductive group, then any $C^{\infty}_{c}$ function is a finite sum of convolutions of $C_{c}$ functions. 
%	\end{remark}
%To justify the LHS, we need more functional analytic preliminaries:
%\begin{enumerate}
%	\item Compact self adjoint operators on a Hilbertspace have a nonzero eigenspace, and its dimension is finite. Proof: First, establish $|A|=\sup_{|v|=1}||\ip{Av}{v}|$. Then take a sequence $v_{n}$ of unit vectors such that $|\ip{ A v_{n}}{v_{n}}=|A|$. Pass to a subseq	uence to attain convergence to real $\alpha$ with $|\alpha|=|A|$.  Use compactness of $A$ to pass to subsequence so that $Av_{n}\to v$. Then compute 
%	\begin{align*}
%		|Av_{n}-\alpha v_{n}|^{2}&=|Av_{n}|^{2}-2\alpha \ip{Av_{n}}{v_{n}}+\alpha^{2}|v_{n}|^{2}\\
%							&\leq |A|^{2}-2\alpha \ip{Av_{n}}{v_{n}}+\alpha^{2}|v_{n}|^{2}\\
%							&=\alpha^{2}-2\alpha \ip{Av_{n}}{v_{n}}+\alpha^{2}|v_{n}|^{2}\to 0. 
%	\end{align*}
%Thus, $Av_{n}\to \alpha v_{n}$, but also $Av_{n}\to v$ so $v_{n}\to v/\alpha$. 
%	\item By induction, $A$ breaks its space into a direct some of eigenspaces, each with finite dimension. 
%	\item If $A$ is a product of Hilbert--Schmidt operators, hence of trace class, then 
%		\begin{align*}
%			\Tr A=\sum_{\lambda\in \spec A} (\dim H(\lambda))\lambda.
%		\end{align*}
%	\item IOU
%\end{enumerate}
%LHS holds
%
%Example: Poisson summation 
%
%Set $G=\R$ and $\Gamma=\Z$. For each integer $n$ define $\chi_{n}:x \mapsto e^{2\pi i n x}$. Then as $\R$ representations,
%	\begin{align*}
%		L^{2}(\Z\lmod \R)=\bigoplus_{n\in \Z}\C \chi_{n}. 
%	\end{align*}
%For $f$ smooth of compact support on $\R$, LHS gives
%	\begin{align*}
%		\Tr R(f)= \sum_{n} \Tr \chi_{n}(f)
%	\end{align*}
%where 
%	\begin{align*}
%		\chi_{n}(f) v=\int_{\R}f(y)\chi_{n}(y)v\dop y=(\int_{\R}e^{2\pi i n y}f(y)\dop y) v. 
%	\end{align*}
%Then 
%	\begin{align*}
%		\Tr \chi_{n}(f)=\hat{f}(-n). 
%	\end{align*}
%Thus, 	
%	\begin{align*}
%		\Tr R(f)=\sum_{n} \hat{f}(-n). 
%	\end{align*}
%To compute the RHS, note that conjugacy classes are singletons, centralizers are the whole group, so
%	\begin{align*}
%		\Tr R(f)&= \sum_{n\{\Z\}} \vol(\Gamma_{n}\lmod G_{n}) \int_{\{1\}} f(n) \dop x \\
%			&=\sum_{n\in \Z} \vol(\Z\lmod \R) f(n)=\sum_{n}f(n). 
%	\end{align*}
%	Thus, the trace formula is, for any smooth $f$ compactly supported on $\R$
%	\begin{align*}
%		\sum_{n}\hat{f}(-n) = \sum_{n}f(n). 
%	\end{align*}
\section{$\SL(2,\R)$}
Let $G=\SL(2,\R)$ and $\Gamma \leq G$ be a compact lattice. Taking for granted that there are no issues with convergence, the trace formula reads: (Recalling that $\cdot_{\gamma}$ is the centralizer of $\gamma$ in $\cdot$, and that $R^{\Gamma}$ is the right regular action of $G$ on $L^{2}(\Gamma \setminus G)$) for a test function $f\in C_{c}^{\infty}(G)$, 
	\begin{align*}
		\sum_ {\pi \in \hat{G}} m_{\pi}^{\Gamma}\Tr \pi(f)= \Tr R^{\Gamma}(f)= \sum_{\gamma\in \{\gamma\}} \vol (\Gamma_{\gamma} \lmod G_{\gamma}) \int_{G_{\gamma}\lmod G} f(g^{-1}\gamma g) \dop g.
	\end{align*} 
The aim of this talk is to demonstrate how this formula manifests in the following
\begin{thm}
	Let $\{\lambda_{j}\}$ be the eigenvalues of the hyperbolic Laplacian acting on $L^{2}(\Gamma \lmod \half)$, and $g\in C_{c}^{\infty}(\R)$. Then
		\begin{align*}
			\sum \hat{g}(\sqrt{\lambda_{j}-1/4}) = \frac{\vol (\Gamma \lmod \half)}{2\pi} \int_{0}^{\infty}\hat{g}(r) r\tanh(r)\dop r +1/2 \sum_{\gamma \in \{\Gamma\}} \frac{\ell(\gamma_{o})}{e^{\ell(\gamma)/2}-e^{-\ell(\gamma)/2}} g(\ell(\gamma))			
		\end{align*}
	where $\hat{g}$ is the Euclidean Fourier transform of $g$; where $\gamma_{o}$ is the generator of $\Gamma_{\gamma}$ (hence primitive geodesic in the class); and $\ell(\gamma)$ is the translation length of $\gamma$.   
\end{thm}




\subsection{Structure of $\SL(2,\R)$}
	Put coordinates on $G$ via the Iwasawa decomposition $G=NAK$:
	\begin{align*}
		A&=\{ a(u)=\amat{u}: u\in \R\}  \\ 
		N&=\{ n(x)=\nmat{x}: x \in \R\} \\
		K&= \{ k(\theta)=\kmat{\theta}: \theta \in [0,2\pi)\}. 
	\end{align*}
%With respect to these coordinates, the haar measure on $G$ is  
%	 \begin{align*}
%		\dop g = \frac{1}{2\pi} \dop u \dop x \dop \theta, 
%	\end{align*}
	Classification of conjugacy classes: 
	\begin{enumerate}
		\item $g \in G$ is hyperbolic if 
			\begin{enumerate}
				\item $\Tr g >2$ 
				\item $g$ is conjugate to an element in $A$
				\item The action of $g$ on $\overline{\half}$ has two distinct fixed points on the boundary
			\end{enumerate}
				\item $g \in G$ is elliptic if 
			\begin{enumerate}
				\item $\Tr g <2$ 
				\item $g$ is conjugate to an element in $K$
				\item The action of $g$ on $\overline{\half}$ has one fixed point in $\half$. 
			\end{enumerate}
					\item $g \in G$ is parabolic if 
			\begin{enumerate}
				\item $\Tr g =2$ 
				\item $g$ is conjugate to an element in $N$
				\item The action of $g$ on $\overline{\half}$ has exactly one point on the boundary
			\end{enumerate}
	\end{enumerate}
	Fact: a discrete subgroup $\Gamma$ is cocompact iff it doesnt contain any parabolic elements.
	
\subsection{Spectral side}
To understand the RHS, we need to compute $\Tr \pi (f)$ for those $\pi \in \hat{G}$ with $m_{\pi}^{\gamma}\neq 0$. To do this, we relate these $\pi$ with special functions on $\Gamma \lmod \half$.  Here's the theorem:
\begin{thm}
	Suppose $m_{\pi}^{\Gamma}\neq 0$. Then $\pi^{K}$  (that is, the right $K$ invariant vectors in $\pi$) is one dimensional, say $\pi^{K}=\C\phi$. Then, upon identifying $\phi$ with a function on $\Gamma \lmod \half$, we have
		\begin{align*}
			\Delta^{\half} \phi = \lambda \phi
		\end{align*}
	for $\lambda \in \R_{\geq 0}$ dependent only on the isomorphism class of $\pi$. 
\end{thm} 
	Essence of proof:  
	\begin{itemize}
		\item The convolution algebra of left-and-right $K$ invariant functions on $G$ is commutative: use Cartan decomposition $G=KAK$ to show first that functions on $K\lmod G / K$ are invariant under transpose, then note that transpose induces an anti-automorphism of $C_{c}^{\infty}(K\lmod G/ K)$, which is the identity. 
		\item Irreducible unitary reps of $G$ appearing in $L^{2}(\Gamma \lmod G)$ are admissible (meaning the restriction of the rep to $K$  is completely reducible, with irreducible constituents having finite multiplicity). 
		\item For an irreducible admissible rep $\pi$, the subspace $\pi^{K}$ affords a representation of $C_{c}^{\infty}(K\lmod G / K)$ which is irreducible. Irreducible reps of a commutative algebra are at most one dimensional. 
		\item Under the identification of functions on $\Gamma \lmod \half$ with right $K$ invariant, left $\Gamma$ invariant functions on $G$, the hyperbolic Laplacian agrees with the Casimir operator of G up to a constant. 
			\begin{itemize}
				\item Quick and dirty on Casimir:\\
				 The action of $G$ on $C^{\infty}(G)$ induces an action of its Lie algebra via the formula 
					\begin{align*}
						R(X)f(g)=\frac{d}{dt}|_{t=0} R(\exp(tX))f(g)=\frac{d}{dt}|_{t=0}f(g\exp(e^{tX})). 
					\end{align*}
					Thus, we can view $R$ as a Lie algebra homomorphism $\g$ into the associative algebra of differential operators on $G$ such that $R([X,Y])=R(X)R(Y)-R(Y)R(X)$. The universal object for such maps is $\mathcal{U} (\g)$, the universal enveloping algebra: this is an associative algebra and a map $\g \to \mathcal{U}(\g)$ such that any Lie algebra map of $\g$ into an associative algebra (turning Lie bracket into commutant) factors through a unique associative algebra map from  $\mathcal{U}(\g)$.  \emph{This is all to say that $\mathcal{U}(\g)$ is the smallest place in which it is meaningful to take products of Lie algebra elements}. 
					\item  The center of $\mathcal{U}(\g)$ is manifestly the algebra of differential operators on $G$ which commute with the action of $G$. The Casimir is a distinguished element of the center, constructed as follows:
					\begin{itemize}
						\item We'll contrive a nonzero algebra map $\End \g \to \mathcal{U}(\g)$ and take Casimir to be the image of the identity (which is certainly in the center): \\
						First, use the canonical map $\End \g \to \g \otimes \g^{*}$, then use a nondegenerate bilinear form on $\g$  to pin down an isomorphism $\g \to \g^{*}$ to obtain a map $\End \g \to \g \otimes \g$. Then use the canonical inclusion $\g \otimes \g$ into the tensor algebra $\bigotimes \g $, and quotient by brackets to get a map to $\mathcal{U}(\g)$. 
					\end{itemize}
				\item With the basis of $\g$ 
					\begin{align*}
						h=\Tbt{1}{0}{0}{-1} \quad e= \Tbt{0}{1}{0}{0}   \quad f= \Tbt{0}{0}{1}{0}
					\end{align*}
					and using the trace form to make the identification of $\g$ with its dual, we have  
					\begin{align*}
						h^{*}=h \quad e^{*}= f/2   \quad f^{*}= e/2.
					\end{align*}
					so Casimir is $h^{2}+2ef+2fe$.
					\item We want to compute the action of this differential operator on left $K$ invariant functions on $G$. To do this, first note that $e-f$ exponentiates into $K$, so will act by zero on left $K$ invariant functions. So we'll compute $h^{2}+4e^{2}$.   
					\item  Exponentials are
						\begin{align*}
							\exp(th)&= a(t)=\amat{t} \\
							\exp(te)&=n(t)=\nmat{t}
						\end{align*}
					so compute, for right $K$ invariant $f$ on $G$, using the fact that $A$ normalizes $N$ via $a(-u)n(x)a(a)=n(e^{u}x)$ so $n(x)a(u)=a(u)a(-u)n(x)a(u)=a(u)n(e^{u}x)$
						\begin{align*}
								R(e)f(n(x)a(u))&=\dbdt|_{t=0} f(n(x)a(u)n(t))=\dbdt|_{t=0}f(n(x+e^{ut})a(u))= e^{u}\frac{\partial}{\partial_{x}}f(n(x)a(u))\\
								R(h)f(n(x)a(u))&=\dbdt|_{t=0}f(n(x)a(u)a(t))=\dbdt|_{t=0}f(n(x)a(u+t))=\frac{\partial}{\partial u} f(n(x)a(u))
						\end{align*}
				%%%resolve this nonsense.
				\end{itemize}
			\item Because casimir commutes with $G$, it provides an $G$-invariant automorphism of any irreducible to itself. Thus, by schur, it acts as a scalar on that irreducible. I.e. the one dimensional subspace $\pi^{K}$ of $\pi$ appearing in $L^{2}(\Gamma\lmod G)$ is in fact an eigenspace of casimir.
			\item Since laplacian and casimir agree up to scale on left $K$ invariant functions on $G$,  this shows that the function in $\pi^{K}$ is in fact a laplacian eigenfunction. 
	\end{itemize}
		So now we look at the irreps $\pi$ of $G$ such that $\pi^{K}$ is one dimensional. These are called \emph{spherical} representations.
		\subsection{Spherical representations of $\SL(2,\R)$}.
			 Consider the subgroup 
				 \begin{align*}
					B=AN=\Tbt{*}{*}{0}{*}.
				\end{align*} 
		Parametrize the multiplicative characters on $B$ by
			\begin{align*}
				\chi_{s}(\Tbt{a}{b}{0}{a^{-1}})=|a|^{s},
			\end{align*}
			and induce from $\chi_{s}$ the representation
				\begin{align*}
					V_{s}=\{ \phi:G\to \C: \phi(bg)=\chi_{s+1}(b)\phi(g)\}.
				\end{align*}
		(the $+1$ is a normalization that we'll benefit from later). Since $G=NAK$, a function in $V_{s}$ is determined by its values on $K$. Denote by $\pi_{s}$ the $\phi \in V_{s}$ such that $\phi|_{K} \in L^{2}(K)$. Let $G$ act on $\pi_{s}$ by right translation. Then
	\begin{thm}
	When $s$ is not an odd integer, $\pi_{s}$ is irreducible with $\dim \pi_{s}^{K}=1$.  When  $s$ is an odd integer, $\pi_{s}$ has a unique irreducible subquotient which we'll renotate to $\pi_{s}$. Every irreducible spherical representation of $G$ arises as a $\pi_{s}$ and pairs $\pi_{s},\pi_{s'}$ are distinct unless $s=\pm s'$. 
	\end{thm}
	In order for  $\m_{\pi_{s}}^{K}\neq0$ we need $\pi_{s}$ to be unitary. The pairing on $\pi_{s_{1}}\times \pi_{s_{2}}$ 
		\begin{align*}
			\ip{\phi_{1}}{\phi_{2}}=\int_{K}\phi_{1}(k)\overline{\phi_{2}(k)}\dop k
		\end{align*}
	is $G$ invariant precisely when $s_{1}=-\overline s_{2}$ and thus gives a $G$ invariant innerproduct on $\pi_{s}$ when $s\in i\R$.  \\
	
	When $s \in(0,1)$, the map $M(s): \phi \mapsto  \int_{-\infty}^{\infty} f(\Tbt{0}{-1}{1}{0} \Tbt{1}{x}{0}{1} \cdot) \dop x$ is an intertwining map $\pi_{s}\to \pi_{-s}$, which induces an inner product on $\pi_{s}$ iff $s\in(-1,1)$. 
	\begin{thm}
		These irreducible unitary spherical reps of $G$ are the $\pi_{s}$ with $s\in i\R \cup (-1,1)$. 
	\end{thm}
Since each of these $\pi_{s}$ are irreducible, casimir acts on it by a scalar. To compute this scalar, it suffices to compute its value on a $\phi \in \pi_{s}^{K}$ which we normalize to take the value $1$ on $K$. Then identifying $\phi$ with a function on $\half$, the condition 
	\begin{align*}
		 \phi(bg)=\chi_{s+1}(b)\phi(g)
	\end{align*}
translates to
	\begin{align*}
		\phi(x+iy)=y^{(s+1)/2}
	\end{align*}
and since casimir is laplacian, the casimir eigenvalue is obtained by
	\begin{align*}
		\Delta^{\half}\phi(x+iy)=y^{2}(\frac{\partial^{2}l}{\partial x^{2}}+\frac{\partial^{2}l}{\partial y^{2}})y^{(s+1)/2} =\frac{(s+1)(s-1)}{4}y^{(s+1)/2}. 
	\end{align*}
	 Casmir acts on $\pi_{s}$  by $(s^{2}-1)/4$.  \\
From all of this, we conclude:
	\begin{align*}
		m_{\pi_{s}}^{\Gamma}=\dim \{ \phi : \half \to \C \vert \Delta^{\half}\phi = \frac{s^{2}-1}{4} \phi\}
	\end{align*}
To finish our analysis of the LHS, we'll compute the traces $\Tr \pi_{s}(f)$ for $f\in C^{\infty}_{c}(K\lmod G / K)$. Recalling that this algebra is commutative, it acts by a scalar on irreducibles, and its trace is that scalar. To compute that scalar, check it on the unique function $\phi \in \pi_{s}^{K}$ such that $\phi(1)=1$: compute
	\begin{align*}
		(\pi_{s}(f)\phi)(1)&=\int_{G}f(g) R(g)\phi(1))\dop g\\ 
					&=\int_{G}f(g)\phi(g)\dop g\\ 
					&=\frac{1}{2\pi} \int_{\R}\int_{\R}\int_{[0.2\pi)} f(n(x)a(u)k(\theta)) \phi(n(x)a(u)k(\theta))\dop \theta \dop u \dop x  \\
					&=\int_{\R}\int_{\R}f(n(x)a(u)) \phi(n(x)a(u)k(\theta))\dop u \dop x \\
					&=\int_{\R}\int_{\R}f(\nmat{x}\amat{u}) \chi_{s+1}(\amat{u}) \dop u \dop x\\
					&=\int_{\R}\int_{\R}f(\Tbt{e^{u/2}}{e^{u/2}x}{0}{e^{-u/2}}) e^{\frac{u(1+s)}{2}} \dop u \dop x\\
					&=\int_{\R}\int_{\R}f(\Tbt{e^{u/2}}{x}{0}{e^{-u/2}})e^{us/2}\dop u \dop x\\
	\end{align*}
The function $

\end{document}