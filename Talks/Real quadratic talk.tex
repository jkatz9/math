 \documentclass[11pt]{amsart}
\usepackage[margin=1in]{geometry}                % See geometry.pdf to learn the layout options. There are lots.
\geometry{letterpaper}                   % ... or a4paper or a5paper or ... 
%\geometry{landscape}                % Activate for for rotated page geometry
%\usepackage[parfill]{parskip}    % Activate to begin paragraphs with an empty line rather than an indent
\usepackage{graphicx}
\usepackage{amssymb}
\usepackage{epstopdf}
\usepackage{amsthm}
\usepackage{mathrsfs}
\usepackage{url}
\usepackage[all]{xy}
\renewcommand{\tilde}{\widetilde}
\newcommand{\hmod}{H^{1}(\modcurve)}
\newcommand{\cmod}{C^{\infty}_{b}(\modcurve)}
\newcommand{\tr}{\Tr}
\newcommand{\Ch}{\operatorname{Ch}}
\newcommand{\ind}{\operatorname{ind}}
\newcommand{\ip}[2]{\langle #1, #2 \rangle}
\newcommand{\C}{\mathbb{C}}
\newcommand{\Z}{\mathbb{Z}}
\newcommand{\R}{\mathbb{R}}
\newcommand{\Q}{\mathbb{Q}}
\newcommand{\til}[1]{\tilde{#1}}
\newcommand{\PSL}{\operatorname{PSL}}
\newcommand{\diag}{\text{diag}}
\newcommand{\half}{\mathfrak{H}}
\newcommand{\intring}{\mathcal{O}}
\newcommand{\sch}{\mathscr{S}}
\newcommand{\tor}{\mathbb{T}}
\newcommand{\bep}{\mathfrak{B}}
\newcommand{\ipd}{\ip{\cdot}{\cdot}}
\newcommand{\fredth}{\tilde{\theta}}
\newcommand{\resth}{\tilde{\theta}^{-1}}
\renewcommand{\sl}{\mathfrak{sl}}
\renewcommand{\phi}{\varphi}
\newcommand{\ltmod}{L^{2}(\modcurve)}
\renewcommand{\k}{\mathfrak{k}}
\newcommand{\halfred}{\tilde{\laphalf}}
\newcommand{\halfres}{(\lambda-\halfred)^{-1}}
\newcommand{\zpz}{\Z/p\Z}
\newcommand{\fredlapsn}{\tilde{\Delta}^{\nsphere}}
\newcommand{\ressn}{(1-\fredlapsn)^{-1}}
\newcommand{\vol}{\operatorname{vol}}
\newcommand{\ijsum}{\sum_{i < j}}
\newcommand{\poly}{\C[x_{1},\ldots,x_{n}]}
\newcommand{\rest}{\big|}
\newcommand{\dtx}{\frac{\partial^{2}}{\partial x^{2}}}
\newcommand{\dty}{\frac{\partial^{2}}{\partial y^{2}}}
\newcommand{\g}{\mathfrak{g}}
\newcommand{\dophalf}{\frac{\dop x \dop y}{y^{2}}}
\newcommand{\tbt}[4]{\left[ \begin{smallmatrix}
		#1 & #2 \\
		#3 & #4 
	\end{smallmatrix} \right] }
\newcommand{\Tbt}[4]{\left[ \begin{matrix}
		#1 & #2 \\
		#3 & #4 
	\end{matrix} \right] }
\newcommand{\Tr}{\operatorname{tr}}
\renewcommand{\r}{\mathfrak{r}}
\newcommand{\ciamod}{C^{\infty}_{a}(\modcurve)}
\newcommand{\cicmod}{C^{\infty}_{c}(\modcurve)}
\newcommand{\hamod}{H^{1}_{a}(\modcurve)}
\newcommand{\ltmoda}{L^{2}_{a}(\modcurve)}
\newcommand{\ltmodcts}{L^{2}_{\text{cts}}(\modcurve)}
\newcommand{\mel}{\mathcal{M}}
\newcommand{\ltmodcfm}{L^{2}_{\text{cfm}}(\modcurve)}
\newcommand{\res}{\operatorname{res}}
\newcommand{\re}{\operatorname{Re}}
\newcommand{\im}{\operatorname{Im}}
\newcommand{\Ad}{\operatorname{Ad}}
\newcommand{\Aut}{\operatorname{Aut}}
\renewcommand{\O}{\operatorname{O}}
\newcommand{\To}{\longrightarrow}
\newcommand{\Mapsto}{\longmapsto}
\newcommand{\inc}{\operatorname{inc}}
\newcommand{\gothic}[1]{\mathfrak{#1}}
\newcommand{\so}{\mathfrak{so}}
\newcommand{\Fund}{\mathcal{F}}
\newcommand{\partone}[1]{\frac{\partial}{\partial x_{#1}}}
\newcommand{\parttwo}[1]{\frac{\partial^{2}}{\partial x_{#1}^{2}}}
\newcommand{\PGL}{\operatorname{PGL}}
\newcommand{\F}{\mathbb{F}}
\newcommand{\ol}{\overline}
\newcommand{\inj}{\hookrightarrow}
\newcommand{\surj}{\twoheadrightarrow}
\newcommand{\trace}{\operatorname{Tr}}
\newcommand{\proj}{\operatorname{proj}}
\newcommand{\der}{\frac{d^2}{dx^2}}
\newcommand{\four}{\mathcal{F}}
\newcommand{\laphalf}{\Delta^{\half}}
\newcommand{\eps}{\varepsilon}
\newcommand{\dom}{\operatorname{dom}}
\newcommand{\id}{\operatorname{id}}
\newcommand{\Ind}{\operatorname{Ind}}
\newcommand{\Res}{\operatorname{Res}}
\newcommand{\End}{\operatorname{End}}
\newcommand{\SL}{\operatorname{SL}}
\newcommand{\GL}{\operatorname{GL}}
\newcommand{\SO}{\operatorname{SO}}
\newcommand{\Orth}{\operatorname{O}}
\newcommand{\dop}{\,{\rm d}}
\newcommand{\nsphere}{S}
\newcommand{\ltnsphere}{L^{2}(\nsphere)}
\newcommand{\honsphere}{H^{1}(\nsphere)}
\newcommand{\lthalf}{L^{2}(\half)}
\newcommand{\ltg}{L^{2}(G)}   
\newcommand{\modcurve}{\Gamma \backslash \half}
\newcommand{\Gal}{\operatorname{Gal}}
\newcommand{\ipn}[2]{\langle #1, #2 \rangle_1}
\newcommand{\Graph}{\operatorname{graph}}
\newcommand{\mhaar}[1]{\frac{\operatorname{d}#1}{#1}}
\newcommand{\rn}{\R^{n}}
\newcommand{\laprn}{\Delta^{\rn}}
\newcommand{\lapsn}{\Delta^{\nsphere}}
\newcommand{\ltrn}{L^{2}(\rn)}
\newcommand{\cirn}{C^{\infty}(\rn)}
\newcommand{\horn}{H^{1}(\rn)}
\newcommand{\inv}{{-1}}
\newcommand{\p}{\mathfrak{p}}
\renewcommand{\P}{\mathfrak{P}}
\newcommand{\frob}[1]{\operatorname{frob}(#1)}
\newcommand{\Ell}{\mathcal{L}}
\newcommand{\arccosh}{\operatorname{arccosh}}
\newcommand{\I}{\mathbb{I}}
\newcommand{\A}{\mathbb{A}}
\newcommand{\Of}{\mathcal{O}}
\newcommand{\Isom}{\operatorname{Isom}}
\newcommand{\lmod}{\backslash}
\newcommand{\rmod}{/}
\newcommand{\Com}{\operatorname{Com}}
\newcommand{\hecke}{\mathcal{H}}
\newcommand{\ord}{\operatorname{ord}}
\newcommand{\mf}{\mathfrak}
\newcommand{\q}{\textbf{q}}
\newcommand{\normset}{\mathcal{N}}
\renewcommand{\Ell}{\mathcal{L}}
\newcommand{\infl}{\operatorname{Infl}}
\newcommand{\vchar}{\operatorname{Vchar}}
\newcommand{\nspec}{\mathcal{N}}
\newcommand{\prim}{\operatorname{prim}}

\theoremstyle{definition}
\newtheorem{claim}{Claim}
\newtheorem*{question*}{Question}
\newtheorem{thm}{Theorem}
\newtheorem{prop}{Proposition}
\newtheorem{remark}{Remark}
\newtheorem*{remark*}{Remark}
\newtheorem{mydef}{Definition}
\newtheorem{fact}{Fact}
\newtheorem{lemma}{Lemma}
\newtheorem{cor}{Corollary}

\newcommand{\Of}{\mathcal{O}}
\renewcommand{\P}{\mathfrak{P}}
\renewcommand{\a}{\mathfrak{a}}
\usepackage{titlesec}

\titleformat{\section}
  {\normalfont\Large\bfseries}{\thesection}{1em}{}[{\titlerule[0.8pt]}]
\titleformat{\subsection}
  {\normalfont\Large\bfseries}{\thesubsection}{1em}{}[{}]

\usepackage[utf8]{inputenc}
\usepackage[english]{babel}

\newtheorem{exer}{Exercise}
\newcommand{\sm}{\setminus}
\newcommand{\f}{\mathfrak{f}}
\newcommand{\cue}{\mathfrak{Q}}
\newcommand{\q}{\mathfrak{q}}
\renewcommand{\b}{\mathfrak{b}}
\renewcommand{\i}{\iota}
\title{Secret seminar talk: Real quadratic zetas and periods of Eisenstein series}
\author{Justin Katz}
\begin{document}
\maketitle
\section{Arithmetic side}
\begin{itemize}
	\item Let $D$ be square free, for convenience $D=2,3 \mod 4$. Let $k=\Q(\sqrt{D})$ and $\Of_{k}=\Z[\sqrt{D}]$.
	\item The norm map $N:k\to \Q$ is given by $N(a+b\sqrt{D})=(a+b\sqrt{D})(a-b\sqrt{D})=a^{2}-Db^{2}$. 
	\item Let $I$ be the collection of integral ideals of $\Of_{k}$ and define
		\begin{align*}
			\zeta_{k}(s)=\sum_{\a \in I/\{0\}}|N(\a)|^{-s} \quad \text{ $\re s>1$}
		\end{align*}
	\item Let $C_{k}$ be the class group (which is finite), and let $\a_{1},..,\a_{h}$ be integral representatives of each ideal class.
	\item $\zeta_{k}$ breaks up 
		\begin{align*}
			\zeta_{k}(s)=\sum_{i=1}^{h}\sum_{\b \in [\a_{i}]}|N(\b)|^{-s}.
		\end{align*}
	\item $\b \in [\a_{i}]$ if and only if $\b=\theta \a_{i}$ for some $\theta \in \a_{i}^{-1}=\{\theta: \theta \a_{i} \subset \Of_{k}\}$, so 
		\begin{align*}
			\zeta_{k}(s)=\sum_{i=1}^{h}\sum_{\theta \in \a^{-1}_{i}}|N(\theta \a_{i})|^{-s}.
		\end{align*}
	\item Pick a $\Z$ basis $e_{1},e_{2}$ for $\a_{i}^{-1}$
		\begin{align*}
			\zeta_{k}(s)=\sum_{i=1}^{h}\sum_{(m,n) \in \Z^{2}/ \{0\}}|N((me_{1}+ne_{2})\a_{i})|^{-s}.
		\end{align*}
	\item By algebraic number theory, there are integers $A_{i},B_{i},C_{i}$ with $\disc Q_{i} = B_{i}^{2}-4A_{i}C_{i}=d$ and
		\begin{align*}
			N((me_{1}+ne_{2})\a_{i})=A_{i}m^{2}+B_{i}mn+C_{i}n^{2}=Q_{i}(m,n).
		\end{align*}
	\item Thus,
		\begin{align*}
			\zeta_{k}(s)=\sum_{i=1}^{h}\sum_{(m,n) \in \Z^{2}/ \{0\}}|Q_{i}(m,n)|^{-s}
		\end{align*} 
	\item Changing basis in the lattice of $\a_{i}^{-1}$  amounts to the action of $\SL_{2}(\Z)$ on $Q_{i}$, and by theory of quadratic forms, merely permutes the terms in the sum. 
\end{itemize}
\section{Geometro-analytic side}
\begin{itemize}
	\item To analyze the Dedekind zeta, we bring the geometry of the moduli space of lattices to bear on it. 
	\item Define a group and its subgroup, using convention that subscripts denote domain of entries
		\begin{align*}
			G=\SL_{2}, \quad P= \tbt{*}{*}{0}{*}.
		\end{align*}
	\item $G_{\R}$ and its subgroups act on the complex upper half plane $\half=\{z:\im z>0\}$ by linear fractional transformations
		\begin{align*}
			\tbt{a}{b}{c}{d} \tau = \frac{a \tau +b}{c\tau +d}
		\end{align*}
	\item (Aux. useful computation) 
		\begin{align*}
			 \im \tbt{a}{b}{c}{d} \tau= \frac{\im \tau}{ |c \tau +d|^{2}}
		\end{align*} 
	\item Aside: If $\half$ is endowed with the hyperbolic metric, $G_{\R}$ acts by isometries. 
	\item Action is transitive, isotropy of $i$ is $K=\SO(2)$: 
		\begin{align*}
			\tbt {a}{b}{c}{d} i= i \iff ai+b=-c + id  \iff a=d, \quad b=-c, \quad a^{2}+b^{2}=1
		\end{align*}
	\item By orbit stabilizer theorem
		\begin{align*}
			G_{\R}/K \approx \half  \quad \text{ via } \quad  gK \mapsto gi. 
		\end{align*}
	\item The moduli space of lattices is given by $G_{\Z} \setminus \half \approx G_{\Z} \setminus G_{\R} / K$
	\item The Real analytic Eisenstein series are functions either on $\half$ or $G_{\R}$ designed to descend to these quotients: 
		\begin{align*}
			E_{s}(g)&= \sum_{\gamma \in P_{\Z} \setminus G_{\Z} } (\im(\gamma g i))^{s}   \quad \text{ $\re(s)>1$}\\ 
			\text{ or }E_{s}(z)&= \sum_{\gamma \in P_{\Z}\setminus G_{\Z}}(\im(\gamma z))^{s} \\
						  &= \sum_{\tbt{*}{*}{c}{d} \in G_{\Z} }  \frac{(\im z)^{s}}{2|cz+d|^{2s}} 
		\end{align*}
	\item (Note, being in $G_{\Z}$ forces $(c,d)=1$)
	\item Convenient modification: a sum over all $(c,d)\neq 0$, can be written as $\sum_{e=1}^{\infty}\sum_{(c,d)=e}$. Upon multiplication by $\zeta(2s)$ we thus have
		\begin{align*}
			\zeta(2s)E_{s}(z)=\sum_{(c,d) \neq 0} \frac{(\im z)^{s}}{2|cz+d|^{2s}}
		\end{align*} 
\end{itemize}
\section{$D<0$ easy}
\begin{itemize}
	\item For imaginary quadratic, $D<0$ we've just shown that the Dedekind zeta breaks into a sum of a definite quadratic form at lattice points. 
	\item Erstwhile, 
		\begin{align*}
			|cz+d|^{2} & = (c\re(z)+d)^{2}+\im( z)^{2}c^{2}  \\
							 & =  |z|^{2}c^{2}+2\re(z) cd+d^{2}
		\end{align*}
		is a quadratic form of discriminant $4(\re z)^{2}-4|z|^{2}=-4 (\im z)^{2} $. 
	\item  By reduction theory, for each imaginary quadratic ideal class, with corresponding quadratic form $Q_{i}$  there is a unique $z_{i}$ in the standard fundamental domain $G_{\Z}\setminus \half$ with $\frac{\im z}{|mz+n|^{2}}=\frac{\sqrt{D}/2}{Q_{i}(c,d)}$.
	\item This gives 
		\begin{align*}
			\frac{(\sqrt{D}/2)^{s}}{\zeta(2s)} \zeta_{k}(s)= \sum_{i=1}^{h} E_{s}(z_{i}).
		\end{align*} 
\end{itemize}
\section{$D>0$ preparations}
\begin{itemize}
	\item For real quadratic, $D>0$ the quadratic forms $Q_{i}$ are \emph{indefinite}. There is no hope of finding $z_{i}$ to make the above work. 
	\item Instead, approach from the geometric side. There is a canonical embedding $\i:k^{*}\to \GL_{\Q}(k)$ via multiplication.
	\item With basis $1,\sqrt{D}$,  the embedding is 
		\begin{align*}
			a+b\sqrt{D} \mapsto\Tbt{a}{bD}{b}{a}.
		\end{align*}
	\item Dirichlet's theorem on units shows that the intersection $H_{\Z}=\i(k^{*})\cap G_{\Z}$ is nontrivial. This subgroup corresponds to $\i({\Of^{\times}_{k}}_{>0})$, positive normed units.
	\item $H_{\Z}$ is a subgroup of $H_{\R}=H_{\Z}\otimes \R$, a one parameter subgroup which is easy to describe
		\begin{align*}
			H_{\R}=\{ \Tbt{a}{bD}{b}{a}: a^{2}-b^{2}D=1\} = \{ h(t)=\Tbt{\cosh t}{ \sinh t \sqrt{D}}{\sinh t/\sqrt{D}}{\cosh t} : t\in \R\} 
		\end{align*}
	\item For any nonidentity $h=\tbt{a}{bD}{b}{a} \in H_{\Z}$,  $\Tr(h)=2a>2$ (since $a^{2}=1+b^{2}D$ if $b^{2}\neq 0$ then $a^{2}\geq 1+D$ and one can check for $D=2,3$ that always $2a>2$.) 
	\item That is, nontrivial units in ${\Of^{\times}_{k}}_{>0}$ correspond to hyperbolic transformations.
	\item Hyperbolic transformations have $2$ distinct real fixed points, one acting as a sink, the other as a source. 
	\item For $a+b\sqrt{D} \mapsto h H_{\Z}$, the fixed points of $h$ are its eigenvalues, which are the conjugates $a\pm b \sqrt{D}$.
	\item The group $H_{\R}$ acts freely and transitively on the geodesic from $a\pm b \sqrt{D}$, and by the existence of nontrivial units $H_{\Z} \setminus H_{\R}$ is closed (hence compact).
	\subsection{Computation for the quadratic form $a^{2}-Db^{2}$}
		\item The quadratic form corresponding to the trivial class is just the norm $a^{2}-Db^{2}$. Dehomogenizing, we have the quadratic polynomial $a^{2}-D=0$, with roots $a=\pm \sqrt{D}$. Let $C_{1}$ be the geodesic joining $\pm \sqrt{D}$.
		\item $i\sqrt{D}$ is on this geodesic, so
			\begin{align*}
				C_{1}= \{ h(t)\cdot i\sqrt{D} = \sqrt{D}\frac{i \cosh t+ \sinh t}{i \sinh t+\cosh t}: t\in \R\}. 
			\end{align*} 
		\item The integral to compute is then $\int_{\R} E_{s}(\sqrt{D}\frac{i \cosh t+ \sinh t}{i \sinh t+\cosh t}) \dop t $
			\begin{align*}
																	&= \int_{H_{\Z}\sm H_{\R}} E_{s}(h  i \sqrt{D}) \dop h \\
																	&= \int_{H_{\Z}\sm H_{\R}} \sum_{\gamma \in P_{\Z}\sm G_{\Z}}\im(\gamma h i \sqrt{D})^{s}\dop h \\
																	&= \int_{H_{\Z}\sm H_{\R}} \sum_{a \in P_{\Z}\sm G_{\Z}/H_{\Z}} \sum_{b \in a P_{\Z} a^{-1} \cap H_{z} \sm H_{\Z}} \im(ab h i\sqrt{D})^{s}\dop h  \\
																	&=\sum_{a \in P_{\Z}\sm G_{\Z}/H_{\Z}}  \int_{H_{\Z}\sm H_{\R}} \sum_{b \in \pm 1 \sm H_{\Z}} \im(ab h i\sqrt{D})^{s}\dop h  \\
																	&=\sum_{a \in P_{\Z}\sm G_{\Z}/H_{\Z}}  \int_{\pm 1 \sm H_{\R}}  \im(a h i\sqrt{D})^{s}\dop h  \\
																	&=\frac{1}{\zeta(2s)} \sum_{(c,d)\in \Z^{2}- \{0\}/H_{\Z}}  \int_{\pm 1 \sm H_{\R}}  \left(\frac{\im(h i \sqrt{D})}{|c h i \sqrt{D} +d|^{2}}\right)^{s}\dop h \\
																	&=\frac{1}{\zeta(2s)} \sum_{(c,d)\in \Z^{2}- \{0\}/H_{\Z}}  \int_{\R}  \left(\frac{\sqrt{D}}{|c  \sqrt{D} (i\cosh t +\sinh t) +d(i \sinh t + \cosh t)|^{2}}\right)^{s}\dop t\\ 
																	&=\frac{1}{\zeta(2s)} \sum_{(c,d)\in \Z^{2}- \{0\}/H_{\Z}}  \int_{\R}  \left(\frac{\sqrt{D}}{(c^{2}D+d^{2}+2cd \sqrt{D}) e^{2t}+(c^{2}D+d^{2}-2cd \sqrt{D})e^{-2t}}\right)^{s}\dop t 
			\end{align*}
\end{itemize}

\end{document}