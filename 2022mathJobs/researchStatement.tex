
\documentclass[12pt]{article}
\usepackage[margin=1 in]{geometry}
\usepackage[indent=0 em]{parskip}
\usepackage{mathptmx,amsmath,amscd,amssymb,amsthm,xspace}
\usepackage{titlesec, graphics,epsfig,wrapfig,verbatim,syntonly,cleveref}
\usepackage{hyperref,amssymb,color, url,fancyhdr,mdframed}
\usepackage{mycros}


\begin{document} 
\begin{comment}
	This is a concise summary of the research you’ve
	done so far and what you plan to do in the next few years. The total length is
	usually on the order of 5 pages. 
	
	The basic rule here is to keep it straightforward –
	it will probably be read by someone who is not an expert in your particular field,
	and who has hundreds of applications to get through. 
	
	It’s thus more important
	to indicate clearly how your research accomplishments and goals fit into the broad
	scheme of things, than to give all the technical details. 
	
	Try to format the research
	statement so that it can be easily skimmed: for example, you might put the most
	important problems and your most important results in bold. 	
\end{comment}

\end{document}		