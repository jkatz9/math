\documentclass[12pt]{article}
\usepackage[margin=1 in]{geometry}
\usepackage[indent=0 em]{parskip}
\usepackage{mathptmx,amsmath,amscd,amssymb,amsthm,xspace}
\usepackage{titlesec, graphics,epsfig,wrapfig,verbatim,syntonly,cleveref}
\usepackage{hyperref,amssymb,color, url,fancyhdr,mdframed}
\usepackage{mycros}

\begin{document} 
\begin{comment}
	In an application for a postdoctoral position, the
	teaching statement is not very important, so don’t waste much time on it. My
	opinion is that the teaching statement cannot help your application much or at all,
	but it might hurt your application if you write something really weird. (“When it
	comes to teaching, I believe it is better to be feared than to be loved.”) No one really
	cares what is your teaching philosophy. Just write a few conventional statements
	about what makes a good teacher (e.g. encouraging interaction with the students),
	and if you have received good teaching evaluations, then this is an opportunity to
	brag about them.
\end{comment}



\end{document}		
