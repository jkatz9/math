\documentclass{amsart}

\usepackage{amssymb}
\usepackage{amsthm}

\usepackage{hyperref}
\usepackage{cite}

\usepackage{mycros}
\usepackage{quiver}




\title{On a converse to Sunada}
\author{Justin Katz}

\begin{document}
\maketitle 


In \cite{sunada1985} Sunada introduced a group theoretic mechanism for producing pairs of isospectral manifolds. In this note, I formulate a converse to that construction and discuss some avenues by which one might prove them.

Suppose we are given two compact Riemannian manifolds, $M_{1}$ and $M_{2}$ which admit a finite covering map onto a common base $M_{o}$, and are both finitely covered by $M$.

\[\begin{tikzcd}
	&& M \\
	\\
	{M_{1}} &&&& {M_{2}} \\
	\\
	&& M_{o}
	\arrow[shorten <=6pt, shorten >=6pt, two heads, from=3-1, to=5-3]
	\arrow[shorten <=6pt, shorten >=6pt, two heads, from=3-5, to=5-3]
	\arrow[shorten <=6pt, shorten >=6pt, two heads, from=1-3, to=3-1]
	\arrow[shorten <=6pt, shorten >=6pt, two heads, from=1-3, to=3-5]
\end{tikzcd}\]


Taking  a finite cover, if necessary, we assume that the cover $M \to M_{o}$ is regular, and set $G=\Gal(M/M_{o})$. Let $H_{i} = \Gal(M/M_i)$ denote the subgroups corresponding to the intermediate covers $M \epi M_i$ for $i=1,2$. Letting $G$ act on $M$ by deck transformations, the diagram above is

% https://q.uiver.app/?q=WzAsNCxbMiwwLCJNIl0sWzAsMiwiSF8xIFxcYmFja3NsYXNoIE0iXSxbNCwyLCJIXzIgXFxiYWNrc2xhc2ggTSJdLFsyLDQsIkdcXGJhY2tzbGFzaCBNIl0sWzAsMSwiIiwwLHsic3R5bGUiOnsiaGVhZCI6eyJuYW1lIjoiZXBpIn19fV0sWzAsMiwiIiwyLHsic3R5bGUiOnsiaGVhZCI6eyJuYW1lIjoiZXBpIn19fV0sWzEsMywiIiwwLHsic3R5bGUiOnsiaGVhZCI6eyJuYW1lIjoiZXBpIn19fV0sWzIsMywiIiwyLHsic3R5bGUiOnsiaGVhZCI6eyJuYW1lIjoiZXBpIn19fV1d
\[\begin{tikzcd}
	&& M \\
	\\
	{H_1 \backslash M} &&&& {H_2 \backslash M} \\
	\\
	&& {G\backslash M}
	\arrow[two heads, from=1-3, to=3-1]
	\arrow[two heads, from=1-3, to=3-5]
	\arrow[two heads, from=3-1, to=5-3]
	\arrow[two heads, from=3-5, to=5-3]
\end{tikzcd}\]



\bibliographystyle{plainurl}
\bibliography{bigBib}


\end{document}
