%!TEX root = main.tex

For $\tbt{a}{b}{c}{d} \in G = \SL(2,k)$, if $c \in k^\times$ then one has 
\al{ 
    \Tbt{a}{b}{c}{d} &= \Tbt{-1}{}{}{-1}  \Tbt{c^\inv}{}{}{c} \Tbt{1}{ac}{}{1} \Tbt{}{1}{-1}{} \Tbt {1}{dc^\inv}{}{1} \\
    				 &= \abf (-1)\cdot \abf (c^\inv)\cdot \ubf (ac) \cdot \wbf \cdot \ubf(dc^\inv)
}
while if $c=0$, then $d=a^\inv$
\al{
	\Tbt{a}{b}{0}{a^\inv} &= \Tbt{a}{}{}{a^\inv} \Tbt{1}{ba^\inv}{}{1} \\
						  &= \abf(a)\ubf(ba^\inv) 
}
Here, $\abf: \Gbb_m \to G$ is the cocharacter corresponding to the split maximal torus $A \leq G$, $\ubf: \Gbb_a \to G$ parameterizes the associated root subgroup, and $\wbf = \tbt{}{1}{-1}{}$ is the weyl element.

For $b \in k^\times$, one has 
\al{
	\abf(b) &= w \cdot \ubf(b^\inv)\cdot w\cdot \ubf (b) \cdot w \cdot \ubf(b^\inv)
			}


Some computations for posterity: 
\al{
	w*g*w &= \left(\begin{array}{rr}
-d & c \\
b & -a
\end{array}\right) \\
	w*g	  &= \left(\begin{array}{rr}
c & d \\
-a & -b
\end{array}\right) \\
	g*w   &= \left(\begin{array}{rr}
-b & a \\
-d & c
\end{array}\right)
}
