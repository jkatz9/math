%!TEX root = main.tex

% Stealing some notation from Sally and Shalika: $k$ is the local field, with integers $\Ocal$, and $G=\SL_2(k)$. We let $K_m \leq \SL_2(k)$ be the kernel of the reduction mod $\Pcal^m$ map. $A$ is the diagonal split torus. 

%From Labesse and Langlands:


$F$ is a local field of characteristic zero, $G=\SL(2)$. Take a cartan subgroup $T$ of $G$, defined over $F$.  A special role will be played by the group $H^1(F,T)$. 

Set $\tilde{G}=\GL_2$, then $\Tilde{T} = C_{\tilde{G}} (T)$ is a cartan subgroup of $\tilde{G}$. That $\tilde{G}$ has a single $\tilde{G}(F)$ conjugacy class of cartans, one has $H^1(F,\tilde{T}) = 1.$  

Any $h\in \tilde{G}(F)=\GL(2,F)$ is a product $s^\inv g$ with $g\in G(\Fbar)=\SL(2,\Fbar)$ and $s \in \tilde{T}(\Fbar)$.

Let $L$ be the centralizer of $T(F)$ in the algebra of $2\times 2$ matrices over $F$. Then the \emph{set} of determinants of elements in $\tilde{T}(F)$ coincides with the algebra-theoretic norms of elements in $L^\times$.

The map $g \to \det h \mod \Nm{L/F} L^\times$ yields an identification of $F^\times / \Nm{L/F}$ with $H^1(F,T)$. 

If $F$ is an extension of a field $E$, consider groups $G'$ defined over $E$ sandwhiched between the restrictions of scalars: $\Res_{L/E}G \leq G' \leq \Res_{L/E}\tilde{G}$. Then $G'$ is defined by a subgfroup $A$ of $\Res_{L/E} \Gbb_m$ and $G'(F) = \{ g \in \tilde{G}(F) \vert \det g \in A (E)\}$

Take $T' = C_{G'}(\Res_{F/E}T)$ and set $\Dfrak(T';E) = F^\times / A(E)\Nm{L/F}L^\times $. A slight extension is to consider $G' = \{ g \in \tilde{G} (F) \det g \in A\}$ for any closed subgroup $A$ of $F^\times$, which may or may not be the set of points of a group rational over some field. 

Let $\kappa : X_* (T) \to \C^\times $ which is $\Gal(\Fbar)$ invariant. 

The LLC associates to the pair $(T,\kappa)$ a group $H$, which must be either $G$ or $T$. 

Fix haar measures on $G'$ and $T'$ and let $\gamma \in T'$ be regular. For $ h \in \tilde{G}(F)=\GL (2,F) $, we can transfer the measure on $T'$ to $h^\inv T' h $. 

Def: $G'$ over $F$ is an inner twist if there exists an isomorphism $\psi: G' \to G$ defined over an extension $K/F$ such that $\sigma(\psi) \sigma ^\inv$ is inner for all $\sigma \in \Gal(K/F).$ Langlands' prediction is that there should be an injection of the automorphic representation of $G'(\Abb_F)$ into those of $G(\Abb_F)$


Consider $\gamma' \in G'(F)$, semi-simple, then the conjugacy class of $\psi(\gamma')$ is defined\footnote{what does this mean here? I think this means that the conjugacy class of $\psi(\gamma')$ is stable under $\Gal(K/F)$, perhaps pointwise so}  over $F$. Steinberg assures us that the conjugacy class of $\psi(\gamma')$ in $G(\Fbar)$ contains an element $\gamma \in G(F)$. This means that there is an injection of the elliptic (i.e. nonsplit) conjugacy classes of $G'(F)$ into those of $G(F)$. These classes form the indexing set for the respective trace formula for $G'$ and $G$. One wonders if this injection respects the orbital integrals in some sense. 

For $\gamma \in T$ let $\gamma_1,\gamma_2 \in \bar{F}$ be its eigenvalues. When $T$ is split, the function $d(\gamma)= \frac{|(\gamma_1 -\gamma_2)^2|^{1/2}}{|\gamma_1\gamma_2|^{1/2}} $ will play a special role. 

Fix a regular element $\gamma^0$ in $\Tti(F))$ and let $\psi$ be a fixed nontrivial additive character of $F$. Fix an ordering $\gamma_1^0, \gamma_2^0$ on the eigenvalues of $\gamma^0$, which in turn determines an order on those for $\gamma$. 

For a given regular $\gamma$, the quotient $\Tti(F) \lmod \Gti(F)$ may be identified with the orbit $\Ocal(\gamma)$ of $\gamma$ under $\Gti(F)$ conjugation.

We arrange for the measure on $\Tti(F)\lmod \Gti))$ is of the ofrm $|\omega_\gamma |/|\gamma_1 - \gamma_2|$ for a certain form $\omega_\gamma$.

For $a\in F^\times$, set $\gamma(a) = a\tbt{1}{1}{0}{1}$. If $\Tti$ is split and $a\in Z(T'),$ set $\Phi^{T'}(a,f)= |a|^\inv \int_{\Ocal(\gamma(a))} f(h) \dop h$. 

For a quadratic extension $L/F$, regard $\Gti(F)$ as the group of invertible $F$ linear transformations of  $L$. Then $\Tti(F)= L^\times$ acting on $L$ by multiplication. 

Pick an $F$ basis $\{1,\tau\}$ for $L$. Then there are $u,v \in F$ so that $\tau^2 = u \tau +v$. 

If $\gamma = a+b\tau \in \Tti(F)$ or $L^\times$, its eigenvalues are of the form $\gamma_1=a+b\tau$ and $\gamma_2 = a+b \bar{\tau}$ so that $\gamma_1-\gamma_2  = b (\tau -\bar{\tau})$. 

In these coordinates, $\gamma$ corresponds to $\tbt{a}{bv}{b}{a+bu} $

For $g = \tbt{a_1}{b_1}{c_1}{d_1}$, then 
$$\gamma^g = \tbt{*}{-b \Nm{L/F}(b_1+d_1v)/\det g}{b \Nm{L/F}(a_1+c_1v)/\det g}{*}$$

Let $\Gti(\Ocal_F)$ be the stabilizer of $\Ocal_L$ in $\Gti(F)$. After averaging a function $f$ on $\Gti(F)$ over $\Gti(\Ocal_F),$ we can assume that $f(g^k)=\kappa'(\det(k))f(g)$. 

If $\pi$ is a uniformizer for $\Ocal_F$, then every double coset in $\Tti(F)\lmod \Gti(F) \rmod \Gti(\Ocal_F)$ contains a $g$ so that $g\Ocal_L=\Ocal_F + \pi ^m \Ocal_F \tau$, for some $m \geq 0$.

Equivalently, in coordinates, it contains a representative of the form $\tbt{1}{0}{0}{\pi^m}$ for $m\geq 0$.



Every $g \in G'$ can be written as $g = n a k$ with $k \in K' = G' \cap \Gti(O_F).$ Set $\beta(g) = \|\alpha/\beta\|$ if $g= n a k$ with $a =\tbt{\alpha}{}{}{\beta}$ and $\lambda(g) = \beta(g) + \beta(wg),$ so that $\lambda(g) = \beta(wn)$ when $g=nak$. For $\gamma \in A'$ regular, set $\Delta(\gamma) = \| \alpha - \beta \|/\| \alpha \beta \|^{1/2}$ 

Define distributions
\[ F(\gamma, f) = \Delta(\gamma) \int_{A \lmod G} f(g^\inv \gamma g ) \dop g \]
and 
\[ A_1(\gamma, f) = \Delta(\gamma) \int_{A\lmod G} f(g^\inv \gamma g ) \ln \lambda(g) \dop g\]


Given a character $\eta$ of $A$, consider the representation $g \mapsto \rho(g,\eta)$ of $G$ acting by (right) translation on the space of smooth left $N(F)$ invariant functions on $G$ satisfying 
\[ \phi(ag) = \eta (a)\beta(a)^{1/2}\phi(g) \]
for all $a \in A$. We can regard the space of $\rho(\eta)$ as a space of functions on $G(O_F).$ The space of functions is the same for $\eta$ as it is for $\eta_s: a \mapsto \eta(a) \beta(a)^s$

The kernel for $\rho(f,\eta)$ is given by 
\[ K_\eta(k_1,k_2) = \int_A \int_{N(F)} f(k_1^\inv a n k_2 ) \lambda(a)^{1/2} \dop a \dop n\]
with the measure on $K'$ chosen so that 
\[ \int_G f(g) \dop g = \int_A \int_{N(F)} \int_K f(ank) \dop a \dop n \dop k\]






% Langlands' setup:
% \begin{itemize}
% 	\item $T=T^G$  is a cartan subgroup of a reductive group  $G$ over a field $F$ of characteristic zero. 
% 	\item $\Afrak(T)=\Afrak(T,F)$ is the set of all $g\in G(\Fbar)$ for which $\Ad(g)T'=T$, and such that the morphism $t \mapsto t' = \Ad(g^\inv)t$ as well as $T'$ are defined over $F$. 
% 	\item $\Dfrak(T)=\Dfrak(T,F)=T(\Fbar)\lmod \Afrak(T)\rmod G(F)$. An element $g\in G(\Fbar)$ lies in $\Afrak(T)$ if and only if $a_\sigma = \sigma(g)g^\inv$ lies in $T(\Fbar)$ for all $\sigma \in \Gal(\Fbar/F)$.
% 	\item The collection  $\{ a_\sigma \vert \sigma \in \Gal(\Fbar/F) \}$ defines a cohomology class in $H^1(F,T)$ and the map $g \mapsto \{a_\sigma\} =\{\sigma(g)g^\inv\}$ yields an injection $\Dfrak(T) \to H^1(F,T)$. 
% 	\item The image of this injection is the kernel of $H^1(F,T)\to H^1(F,G)$ (which is somehow not always a group.) 
% 	\item The set $\Dfrak(T)$ parameterizes the conjugacy classes within a stable conjugacy class of $T$ (in $G(F)$). 
% 	\item $X_*(T)$ and $X_*(T_sc)$ are the lattices of coweights in $T$ and $T_{sc}$.  The latter can be identified with the sublattice of the former, generated by the co-roots. \end{itemize}


% 	