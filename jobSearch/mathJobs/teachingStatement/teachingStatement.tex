\documentclass[12pt]{article}
\usepackage[margin=1 in]{geometry}
\usepackage[indent=0 em]{parskip}
\usepackage{mathptmx,amsmath,amscd,amssymb,amsthm,xspace}
\usepackage{titlesec, graphics,epsfig,wrapfig,verbatim,syntonly}
\usepackage{hyperref,amssymb,color, url,fancyhdr,mdframed}
\usepackage{mycros}

\title{Teaching Statement}
\author{Justin Katz}
\date{}
\begin{document} 
	\maketitle
\begin{comment}
	In an application for a postdoctoral position, the
	teaching statement is not very important, so don’t waste much time on it. My
	opinion is that the teaching statement cannot help your application much or at all,
	but it might hurt your application if you write something really weird. (“When it
	comes to teaching, I believe it is better to be feared than to be loved.”) No one really
	cares what is your teaching philosophy. Just write a few conventional statements
	about what makes a good teacher (e.g. encouraging interaction with the students),
	and if you have received good eaching evaluations, then this is an opportunity to
	brag about them.
\end{comment}
\section*{Core Principles}
I consider teaching to be an essential role of an active mathematician.  My teaching methodology is adaptive and prioritizes the learning objective of my audience.  I strive to make my classroom a safe, inclusive, and accessible environment, wherein students can freely, fully, and productively engage with the material. I make it a priority to be available to provide feedback to my students and to be sensitive to the needs of groups that are underrepresented in higher education. 
\section*{Teaching experience}
I have experience teaching in a variety of contexts: as a teaching assistant, as a lecturer, and as a mentor.
\subsection*{As a recitation leader:} I have served as a recitation leader for 21 sections of multivariable calculus, at Purdue University, over the course of 7 semesters. In this role, I supplemented the primary lecturer by addressing students' questions, and by developing students' problem-solving abilities through active participation and weekly quizzes.  I also actively encouraged students to probe beyond the course objectives and to think critically about challenging problems.  To this end I wrote several expository articles in response to specific students' questions, and made them freely available on my personal webpage. I supplemented several of these articles with animations, written in Mathematica, and made the source code freely accessible to students. 

\subsection*{As an instructor:} I have served as the instructor for a semester of two sections of  MA16010: Applied Calculus.  This course served as many of my students' first (and only) interaction with calculus. The topics that we covered include trigonometric and exponential functions; limits and differentiation, rules of differentiation, maxima, minima and optimization; curve sketching, integration, anti-derivatives, the fundamental theorem of calculus. Properties of definite integrals and numerical methods. The course concluded with applications to life, managerial, and social sciences.  In this role, I wrote, administered, and graded exams, and held weekly office hours.   

\newpage

\subsection*{As a mentor:}
I have served as the primary mentor for several reading courses for groups of undergraduate students. 

\paragraph{Differential geometry:} I lead two groups of students' independent studies into differential geometry; the first consisting of one undergraduate student in fall 2019, and the second with two undergraduates in spring semester of 2020. In both cases, we met weekly and discussed students' reading. We focused primarily on the textbook by DeCarmo, but also made use of several contemporary expository articles. Each week students solved a collection of problems, and presented their solutions. 

\paragraph{Abstract algebra:} In the fall of 2021, I lead two reading courses in abstract algebra for two groups of undergraduates. The first section consisted of 5 students, and was their first interaction with modern algebra. We worked through the first several chapters of Herstein's text. In our weekly meeting, students presented their solutions to problems of their choosing. The second section, consisting of 3 students, covered more advanced subjects; primarily focusing on Galois theory, with an aim towards algebraic number theory. The students in this section collaborated on a computational project, based in Sage, in which they formulated and verified certain statistical claims about quadratic and cubic extensions of the rational numbers.  

\paragraph{Representation theory:} I am currently leading an independent study with a single undergraduate into the basics of representation theory of finite groups. We are reading the standard texts by Fulton and Harris, and by Serre. We meet weekly, and the student is preparing as a term project an expository paper on the representation theory of $\operatorname{PSL}_2$ over finite fields.

\subsection*{As an organizer:}
\paragraph{Open mic night} From the summer of 2019 until March of 2020 I organized and hosted a weekly seminar, playfully called `open mic night.' This was a casual forum wherein graduate students could speak on topics about which they had been thinking, problems on which they were working, and concepts with which they were struggling. 

\end{document}		
