
\documentclass[12pt]{article}
\usepackage{mathptmx,amsmath,amscd,amssymb,amsthm,xspace}
\usepackage{titlesec, graphics,epsfig,wrapfig,verbatim,syntonly}

\usepackage{hyperref,amssymb,color, url,fancyhdr}
\usepackage{mycros}
\usepackage[margin=1 in]{geometry}
\usepackage{verbatim}
\usepackage[]{cleveref}
\usepackage[indent=0 em]{parskip}

\newcommand{\kres}{\overline{\kbf}} 
\setlength{\parskip}{6pt}


%\titlespacing\section{0pt}{12pt plus 4pt minus 2pt}{0pt plus 2pt minus 2pt}
%\titlespacing\subsection{0pt}{12pt plus 4pt minus 2pt}{0pt plus 2pt minus 2pt}
%\titlespacing\subsubsection{0pt}{12pt plus 4pt minus 2pt}{0pt plus 2pt minus 2pt}





\newtheorem{q}{Question}



\usepackage{mdframed}


\newtheorem{question}{Question}
\title{Project Description \vspace{-1 em} }
\author{Justin Katz \vspace{-1 em}}
\date{	}
\begin{document} 
\maketitle


\section{Introduction}\label{intro}

\begin{comment}
The primary focus of my research is the spectral geometry of Riemannian mani folds, with an emphasis on the locally symmetric spaces associated to arithmetic lattices in Lie groups. My work makes essential use of geometric topology, harmonic analysis, algebraic groups, number theory, arithmetic/algebraic/differential geometry, and automorphic  forms. This topic interacts with class field theory, Galois cohomology and $K–$forms of linear algebraic g roups, Bruhat–Tits theory, and the geometry/topology of symmetric and locally symmetric spaces. It has direct connections to Riemannian geometry, geometric analysis, algebraic/arithmetic geometry, and the Langlands’ program.
\end{comment}
A principal technique in geometry  is to reduce  problems of the form:  
 
\begin{problem}\label{prob:geo}
Given two objects $X_1$ and  $X_2$ living in some class $C$, how does one determine if $X_1$ and $X_2$ are identical?
\end{problem}

to  one of the form:

\begin{problem}\label{prob:grp} 
	Given subgroups $\Gamma_1$ and $\Gamma_2$ of a group $G$, how does one determine if $\Gamma_1$ is conjugate to $\Gamma_2$ in $G$?
\end{problem}

In many cases, this latter problem is more tractable. For example, as will be the case in this proposal, when the ambient group $G$ is a semisimple Lie group, one may approach the problem through its harmonic analysis and dynamics. When $\Gamma_1$ and $\Gamma_2$ are known to be lattices in $G$, then one gains powerful tools through trace formulae of various forms.

In the context of this paper, the class $C$ is the moduli space $\Mcal_g$ of
compact, smooth, hyperbolic surfaces of a fixed genus $g$,  and we investigate
the former question under certain hypotheses on the eigenvalue spectrum of the
Laplace-Beltrami operator\ acting on  square integrable sections of certain flat
vector bundles over such surfaces. The space of such bundles can be identified
with a certain character variety $\Ucal_{g}$. Using an analogy between
geometry and number theory, popularized by Sunada, we use the Artin-Takagi
formalism, to encode this problem in certain entire fuctions called
Selberg zeta functions. A principal aim of the proposed project is to study this spectrum as a function
of both the \textit{metric} and the \textit{bundle}. 

Investigation of spectra as a function of the metric was popularized by the
famous question posed by Kac:  \emph{Can you hear the shape of a drum?}   The answer to depends on the drum. On the one hand, there exist
Riemannian manifolds of all manners of shapes which sound identical; these
manifolds are referred to as \textbf{isospectral}. Constructing  new examples of
such spaces is the subject of my first publication, and will be discussed
further in the background section.

On the other hand, there are some very special drums which are determined by
their sounds. We call a closed Riemannian manifold (absolutely)
\textbf{spectrally rigid} if isospectrality implies isometry. It is known that
spectral rigidity is often a generic property
, though there are very few
explicit examples. The content of my graduate this is the identification of some
specific arithmetic hyperbolic $2$ and $3$ manifolds which are spectrally rigid. 

As a function of the bundle, much less is known. In the case that the bundles
are $1$ dimensional and the underlying space is a compact Riemann surface, the
spectrum can be viewed as a function on the universal Hodge bundle \cite{fay1973}, and has been studied in \cite{takhtajan1991} as a function of moduli.	

Using an analogy between geometry and number theory, popularized by Sunada, we
use the Artin-Takagi formalism \cite{venkov1981} to encode this problem with Selberg zeta
functions. These functions can be thought of as an infinite dimensional
analogues of the characteristic polynomial of the twisted Laplacian. 

\section{Background}
\paragraph{Past work: flexibility} The failure of spectral rigidity was first observed in Milnor in 1964 \cite{milnor1964}, where he constructed a pair of isospectral, nonisometric $16$ dimensional flat tori. In a sweeping generalization of Milnor's construction,  Vignéras in \cite{}  described a procedure for producing infinite families of compact hyperbolic $2$ and $3$  manifolds which fail to be spectrally rigid. Her methodology  exploits  a  particular mode of failure of the local-to-global principle for rational conjugacy classes of maximal orders in a quaternion algebra over a number field with sufficiently complicated arithmetic.  A critical step in her argument is an application of the Selberg trace formula to relate the eigenvalue spectrum on the associated locally symmetric spaces with their respective geodesic length spectrum; which in turn is closely related to the solution of an explicit family of diophantine equations.  As her examples are arithmetic, it follows from work of Borel \cite{borel1989} that only finitely such pairs from her construction can have genus at most $g$.

In $1985$, Sunada \cite{sunada1985}  constructed many more examples of isospectral, non-isometric manifolds; his construction produces positive dimensional subspaces of the moduli space of Riemann surfaces of genus $g$ that fail to be spectrally rigid provided $g$ is sufficiently large (e.g. if $g > 168$). His construction was motivated by an old construction by Gassman \cite{gassmann1926}. The
construction is elementary, using finite covers and the existence of groups with pairs of subgroups satisfying a condition called {\bf almost conjugate}. 

In my first paper \cite{arapura2019} , in collaboration with Donu Arapura, Partha Solapurkar, and Ben McReynolds, we applied a refined notion of almost conjugacy to  constructed locally symmetric manifolds and complex projective surfaces that share many algebraic and analytic invariants. For example, we produce non-isometric closed hyperbolic n-manifolds, as covers of a fixed manifold, that have isomorphic integral cohomology in such a way that the isomorphism commute with the natural maps induced by the cover. We also produced arbitrarily large collections of pairwise non-isomorphic smooth projective surfaces where the isomorphisms are natural with respect to the Hodge structure, or as Galois modules. In particular, the projective surface have isomorphic Picard and Albanese varieties, and have isomorphic effective Chow motives. All of these examples also have the same eigenvalue and geodesic length spectrum for their associated Riemannian structures. The construction based on a refinement of Sunada's method, based on examples first discovered by L. Scott \cite{scott1993integral}   and recently used by D. Prasad \cite{prasad2017}, in a construction that partly motivated ours.


\paragraph{Spectral rigidity} \label{par:specrigidity}
Despite the failure of spectral rigidity, there have been several positive
results. In 1982, Wolpert  \cite{wolpert1978}  proved that a generic (in the
Baire sense) Riemann surface is determined by its spectrum. In 1992,
Reid proved that if $X$ is an arithmetic Riemann surface, then any Riemann
surface $Y$ that is isospectral to $X$ must be commensurable with $X$ \cite{reid1992a}. In
particular, $Y$ must be arithmetic. The content of my thesis is the
determination of a particular collection of arithmetic  hyperbolic surfaces
which are, in fact, spectrally rigid. Before I can state my theorem, we need
some terminology.


First,  we define arithmetic hyperbolic surfaces. Let  $\kbf$ be a totally real number field with ring of integers $\obf$,  and $\Bbf$ an indefinite quaternion algebra over $\kbf$ which is split at a unique real place of $\kbf$. That is to say, among all of the inclusions $\rho:  \kbf \to \Rbb$, there is a unique  one $\rho_o$ such that $\Bbf \otimes_{\kbf}  \kbf_{\rho_o} $ is isomorphic to the algebra of two by two matrices over $\kbf_{\rho_o} =\Rbb$, the completion of $\kbf$ at $\rho_o$.  We will also use the symbol $\rho_o$ to denote the resulting inclusion $\Bbf \to \Bbf \otimes_{\kbf}  \kbf_{\rho_o} $. An {\bf order} $\Obf$ in $\Bbf$ is an $\obf$ subalgebra of $\Bbf$ of maximal rank.  Let $\Obf^1$ denote the multiplicative subgroup  of $\Obf$ consisting of elements of reduced norm  $1$.  Then $\rho_o$ yields an identification $\Obf^1$ with a lattice $\Gamma_\Obf$ in $\SL(2,\Rbb)$.  We say that a lattice $\Lambda$ in $\SL(2,\Rbb)$ is {\bf arithmetic}  if there exists some $\Obf$ in some $\Bbf$ over some $\kbf$ such that some $\GL(2,\Rbb)$ conjugate of $\Lambda$ is commensurable with $\Gamma_{\Obf}$.  For any lattice $\Lambda$ in $\SL(2,\Rbb)$, we let $X(\Lambda)$ denote the associated hyperbolic orbifold  $\Lambda \lmod \SL(2,\Rbb) \rmod  \SO(2)$.  We say that a quaternion algebra $\Bbf$ over $\kbf$ has {\bf type number one} if it has a unique $\Bbf^\times$ conjugacy class of maximal orders.  Let $\mathbf{I}$ be an ideal in $\obf$ not divisible by any prime over which $\Bbf$ is ramified.   Write $\Gamma_\Obf(\mathbf{I}) \leq \Gamma_\Obf$ for the  {\bf principal congruence kernel mod $\mathbf{I}$} . Note that normal inclusion of $\Gamma_\Obf(\mathbf{I})$ in $\Gamma_\Obf$ induces a regular cover $X(\Gamma_\Obf(\mathbf{I}))$ over $X(\Gamma_\Obf)$, with deck group isomorphic to  $\Gamma_\Obf(\mathbf{I})  \lmod \Gamma_\Obf = \SL(2,\obf /\mathbf{I}).$  We are now prepared to state the theorem:
 \begin{thm}
 	Let $\kbf, \obf, \Bbf, \mathbf{I}$ be as above. Suppose further that $\Bbf$ has {\bf type number $1$}. Let $\Obf$ be a representative of the single conjugacy class of maximal orders in $\Bbf$. The surface $X(\Gamma_\Obf(\pbf))$  is spectrally rigid. 
 \end{thm}
This theorem is the first, to my knowledge, which  produces infinitely many infinite families of Riemannian manifolds which are demonstrably spectrally rigid. By carefully  choosing $\kbf,$ and $\Bbf,$ one can  apply this theorem to the principal congruence Hurwitz surfaces, partially confirming a conjecture of Alan Reid \cite{reid2014}. 

\section{Proposal}
To study the relationship among Dedekind zeta functions of number fields, number theorists study their factorizations into Artin $L$-functions associated to Galois representations representations. Following their lead, we introduce the analogous $L$ functions and study factorizations of Selberg's zeta function. Fix a compact negatively curved Riemannian manifold $M$, with fundamental group $\Gamma$. Let $\rho: \Gamma \to U(V)$ be a unitary representation on a finite dimensional Hilbert space $V$. Let
\[	L_{M,\rho}(s) = \prod_{\gamma} \prod_{k=0}^\infty   \det ( \id_V -  \rho(\gamma)\exp^{-(s+k) \ell_M(\gamma) }  ) .\label{slf}\]
These functions satisfy the so-called Artin-Takgai formalism: for a cover $M' \to M$ :
\begin{enumerate}
	
	\item  For $\rho$ a unitary representation of $\Gamma' = \pi_1( M' ) \leq \Gamma$, 
	\[ L_{M',\rho} (s)  = L_{M, \ind_{\Gamma'}^{\Gamma_o} \rho} (s),
	 \]  
	 and for a unitary representation $\sigma$ of $\Gamma$,
	\[ L_{M,\sigma} (s)  = L_{M',\res_{\Gamma'}^{\Gamma}\sigma} (s). \]
	\item For a pair $\sigma, \sigma'$ of unitary representations of $\Gamma$, 
	\[ L_{M,\sigma \oplus \sigma' }(s) = L_{M,\sigma}(s) L_{M,\sigma'}(s) \]
\end{enumerate}
Applying the second property above, we may extend the function $L_{M,\cdot} (s)$ from the set ${\rm URep}(M)$ of unitary reps of $\pi_1(M)$ to the additive group ${\rm VURep}(\pi_1(M))$ of virtual unitary representations. Let $T(M)$ be some reasonable space of negatively curved metrics on $M$. Then we may view the assignment $(\mu,\rho) \mapsto L_{\mu,\rho}$ as a meromorphic-function valued function on the space $T(M) \times {\rm VURep(M)}$ which is a homomorphism in the second variable. 

\begin{question}\label{q:multind}
	For which metrics $\mu$ is the homomorphism $L_{\mu, \cdot}$ injective? When it isn't injective, how large is its kernel? 
\end{question} 

This question has a meaningful analogue in number theory, whereat the answer is positive \cite{funakura1978}. It has as an immediate consequence that all arithmetically equivalent number fields arise from Gassman equivalence\cite{perlis1977}. A positive answer to \cref{q:multind} will show that all instances of isospectrality within a commensurability class must arise from Sunada's construction. 
 
\begin{question}
 Fix a twist $\rho$. For which spaces of metrics $T(M)$ on $M$ is the function $L_{\cdot, \rho}$ injective on $T(M)$?
\end{question}
When $M$ is a compact surface of genus $g>1$, a natural candidate is the space $\Mcal_g$ of hyperbolic metrics.

Inside of ${\rm VURep(M)}$ is the subgroup ${\rm FVUrep(M)}$ consisting of those virtual representations which admit a continuous extension to the profinite completion of $\pi_1(M)$). 
\begin{question}\label{q:multind2}
	Supposing the answer to \cref{q:multind} is negative, must the kernel intersect ${\rm FVURep(M)}$ nontrivially? 
\end{question}
Now suppose that $M$ is an arithmetic hyperbolic $2$ or $3$ manifold of the form $ \mathbf{G}(k) \lmod \mathbf{G}(\Abb_k) / \prod_{\pfrak<\infty} \mathbf{G}(\Ocal_\pfrak) $  for some inner form $\mathbf{G}$ of $\SL_2$ and write $\Gamma $ for the corresponding  Fuchsian or Kleinian lattice in $G:= \mathbf{G}(k^\infty)$. The unitary representations of $\Gamma$ which factor through a finite \emph{congruence quotient} are precisely those which extend continuously to $K:= \prod_{\pfrak<\infty} \mathbf{G}(\Ocal_\pfrak)$. Write $\mathrm{CFVURep}(M)$ the subgroup of $\mathrm{FVURep}(M)$ generated by these representations.  
\begin{question}\label{q:multind3}
	Supposing the answer to \cref{q:multind2} is negative, must the kernel intersect ${\rm CFVURep(M)}$ nontrivially? 
\end{question}

As mentioned in \cref{par:specrigidity}, Alan Reid proved that isospectral arithmetic hyperbolic $2$ and $3$ manifolds must be commensurable.  
\begin{question}
  Let $M$ and $M'$ be isospectral nonarithmetic hyperbolic $2$ or $3$ manifolds
  $M$ and $M'$ be commensurable?  
  \end{question}
If this is so, then if paired with a positive answer to \cref{q:multind}, one would have a complete characterization of isospectrality for such manifolds. 
 
\section{Career development}
In pursuing the research objectives of this project, further investigating related literature, interacting with
other researchers and collaborators, and producing articles related to this work, I will develop my skills as
a mathematical researcher. Working within the 
Geometry and Topology group at Temple university,  will give me the opportunity to extend my knowledge of these fields dramatically. In particular, communicating with this large and active group of faculty and graduate students will provide me with an excellent opportunity to broaden my mathematical and collaborative abilities. 
 	
\section{Sponsoring scientist}
Matthew Stover is a world expert in the geometry of locally symmetric spaces. Many of his published results make prove incisive theorems regarding the geometry of closed geodesics on such spaces. I believe that many of these results can be passed through relevant trace formulae to produce interesting theorems about their spectral geometry. I have met with Stover at several conferences over the last three years; most recently at the Geometry, Topology, and Groups conference in honor of his advisor Alan Reid's 60th birthday. His record of mentorship is exemplary. While at the 2019 summer school at MSRI, I had the pleasure of interacting with two of Stover's Ph.D. students, with both of whom I have had since fruitful mathematical correspondences.   


\section{Broader impact}
I believe that mathematics is a fundamentally collaborative enterprise; that \emph{communicating} is an essential aspect of \emph{research}; and that teaching is a critical component of learning. I will strive to ensure a positive and inclusive community both within Temple university and throughout the Philadelphia area. I believe that access to mathematics is a fundamental human right, and I will work to eliminate barriers from its pursuit.    

\newpage

\bibliographystyle{plain} 
\bibliography{zotBib}

\end{document}		