
\documentclass[12pt]{article}
\usepackage{mathptmx,amsmath,amscd,amssymb,amsthm,xspace}
\usepackage{titlesec, graphics,epsfig,wrapfig,verbatim,syntonly}
\usepackage{hyperref,amssymb,color, url,fancyhdr}
\usepackage{mycros}
\usepackage[margin=1 in]{geometry}
\usepackage{verbatim}
\usepackage[]{cleveref}
\usepackage[indent=0 em]{parskip}

\newcommand{\kres}{\overline{\kbf}} 
\setlength{\parskip}{6pt}


%\titlespacing\section{0pt}{12pt plus 4pt minus 2pt}{0pt plus 2pt minus 2pt}
%\titlespacing\subsection{0pt}{12pt plus 4pt minus 2pt}{0pt plus 2pt minus 2pt}
%\titlespacing\subsubsection{0pt}{12pt plus 4pt minus 2pt}{0pt plus 2pt minus 2pt}





\newtheorem{q}{Question}



\usepackage{mdframed}


\newtheorem{question}{Question}
\title{Project Summary}
\author{Justin Katz \vspace{-1 em}}
\date{	}
\begin{document} 
	\maketitle

\paragraph{Overview} The overall goal of the proposed project is to investigate the Laplace eigenvalue spectrum of certain closed Riemannian manifolds, and their interaction with twists by unitary representations. The primary focus of this program is to analyze said spectrum for compact hyperbolic surfaces and to give a complete classification of isospectrality for them. In his graduate thesis, the PI has established an absolute spectral rigidity theorem for a certain class of hyperbolic arithmetic $2$ and $3$ manifolds. The techniques used to prove this result have promising further applications.

The PI will examine the  extent to which certain properties of hyperbolic surfaces are determined by their spectrum, or twists thereof. A primary goal is to determine whether the commensurability class of a non-arithmetic hyperbolic surface is such a property. Another property that will be investigated is whether the spectrum detects the location of a surface within certain strata in the moduli space of such surfaces. 

The PI will investigate the analogy between the geodesic geometry  negatively curved Riemannian manifolds and arithmetic of ideals in number fields. In particular, he will pursue analogues of standard number theoretic results about Dedekind zeta functions and their factorizations into Artin L-functions, in terms of Selberg's zeta functions. 
\paragraph{Intellectual Merit}
The novelty of the techniques employed by the PI will serve to advance the state of knowledge in geometry and topology. His application of the aforementioned analogy will serve to further the development of both number theory and geometry. More specifically, the concrete interpretation of zeroes of Selberg's zeta functions as eigenvalues of twisted Laplace operators will lead to interesting heuristics for the much less understood zeroes of Dedekind's zeta and Artin's $L$ functions. 
\paragraph{Broader Impact}
Through the course of the proposed project, the PI will be an active member in the mathematical community at Temple University. He will actively pursue opportunities to mentor undergraduate and graduate students in the department. Furthermore, the PI will strive to expose a broader audience, both within the university and in the surrounding Philadelphia area. The PI has a robust history of teaching mathematics and mentoring undergraduates and junior graduate students. The PI is committed to fostering a safe environment for learning. He will actively seek to ensure that the environment is welcoming to groups which are marginalized or are otherwise underrepresented in mathematics. 

\end{document} 