\documentclass[11pt]{amsart}

\usepackage[margin=1in]{geometry} 
\geometry{letterpaper}

\usepackage{titlesec}                   % ... or a4paper or a5paper or ... 
\usepackage{graphicx}
\usepackage{amssymb}
\usepackage{amsthm}
\usepackage{titlesec}
\usepackage[utf8]{inputenc}
\usepackage[english]{babel}
\usepackage{cite}
\usepackage{tikz-cd}


\renewcommand{\tilde}{\widetilde}
\newcommand{\hmod}{H^{1}(\modcurve)}
\newcommand{\cmod}{C^{\infty}_{b}(\modcurve)}
\newcommand{\tr}{\Tr}
\newcommand{\Ch}{\operatorname{Ch}}
\newcommand{\ind}{\operatorname{ind}}
\newcommand{\ip}[2]{\langle #1, #2 \rangle}
\newcommand{\C}{\mathbb{C}}
\newcommand{\Z}{\mathbb{Z}}
\newcommand{\R}{\mathbb{R}}
\newcommand{\Q}{\mathbb{Q}}
\newcommand{\til}[1]{\tilde{#1}}
\newcommand{\PSL}{\operatorname{PSL}}
\newcommand{\diag}{\text{diag}}
\newcommand{\half}{\mathfrak{H}}
\newcommand{\intring}{\mathcal{O}}
\newcommand{\sch}{\mathscr{S}}
\newcommand{\tor}{\mathbb{T}}
\newcommand{\bep}{\mathfrak{B}}
\newcommand{\ipd}{\ip{\cdot}{\cdot}}
\newcommand{\fredth}{\tilde{\theta}}
\newcommand{\resth}{\tilde{\theta}^{-1}}
\renewcommand{\sl}{\mathfrak{sl}}
\renewcommand{\phi}{\varphi}
\newcommand{\ltmod}{L^{2}(\modcurve)}
\renewcommand{\k}{\mathfrak{k}}
\newcommand{\halfred}{\tilde{\laphalf}}
\newcommand{\halfres}{(\lambda-\halfred)^{-1}}
\newcommand{\zpz}{\Z/p\Z}
\newcommand{\fredlapsn}{\tilde{\Delta}^{\nsphere}}
\newcommand{\ressn}{(1-\fredlapsn)^{-1}}
\newcommand{\vol}{\operatorname{vol}}
\newcommand{\ijsum}{\sum_{i < j}}
\newcommand{\poly}{\C[x_{1},\ldots,x_{n}]}
\newcommand{\rest}{\big|}
\newcommand{\dtx}{\frac{\partial^{2}}{\partial x^{2}}}
\newcommand{\dty}{\frac{\partial^{2}}{\partial y^{2}}}
\newcommand{\g}{\mathfrak{g}}
\newcommand{\dophalf}{\frac{\dop x \dop y}{y^{2}}}
\newcommand{\tbt}[4]{\left[ \begin{smallmatrix}
		#1 & #2 \\
		#3 & #4 
	\end{smallmatrix} \right] }
\newcommand{\Tbt}[4]{\left[ \begin{matrix}
		#1 & #2 \\
		#3 & #4 
	\end{matrix} \right] }
\newcommand{\Tr}{\operatorname{tr}}
\renewcommand{\r}{\mathfrak{r}}
\newcommand{\ciamod}{C^{\infty}_{a}(\modcurve)}
\newcommand{\cicmod}{C^{\infty}_{c}(\modcurve)}
\newcommand{\hamod}{H^{1}_{a}(\modcurve)}
\newcommand{\ltmoda}{L^{2}_{a}(\modcurve)}
\newcommand{\ltmodcts}{L^{2}_{\text{cts}}(\modcurve)}
\newcommand{\mel}{\mathcal{M}}
\newcommand{\ltmodcfm}{L^{2}_{\text{cfm}}(\modcurve)}
\newcommand{\res}{\operatorname{res}}
\newcommand{\re}{\operatorname{Re}}
\newcommand{\im}{\operatorname{Im}}
\newcommand{\Ad}{\operatorname{Ad}}
\newcommand{\Aut}{\operatorname{Aut}}
\renewcommand{\O}{\operatorname{O}}
\newcommand{\To}{\longrightarrow}
\newcommand{\Mapsto}{\longmapsto}
\newcommand{\inc}{\operatorname{inc}}
\newcommand{\gothic}[1]{\mathfrak{#1}}
\newcommand{\so}{\mathfrak{so}}
\newcommand{\Fund}{\mathcal{F}}
\newcommand{\partone}[1]{\frac{\partial}{\partial x_{#1}}}
\newcommand{\parttwo}[1]{\frac{\partial^{2}}{\partial x_{#1}^{2}}}
\newcommand{\PGL}{\operatorname{PGL}}
\newcommand{\F}{\mathbb{F}}
\newcommand{\ol}{\overline}
\newcommand{\inj}{\hookrightarrow}
\newcommand{\surj}{\twoheadrightarrow}
\newcommand{\trace}{\operatorname{Tr}}
\newcommand{\proj}{\operatorname{proj}}
\newcommand{\der}{\frac{d^2}{dx^2}}
\newcommand{\four}{\mathcal{F}}
\newcommand{\laphalf}{\Delta^{\half}}
\newcommand{\eps}{\varepsilon}
\newcommand{\dom}{\operatorname{dom}}
\newcommand{\id}{\operatorname{id}}
\newcommand{\Ind}{\operatorname{Ind}}
\newcommand{\Res}{\operatorname{Res}}
\newcommand{\End}{\operatorname{End}}
\newcommand{\SL}{\operatorname{SL}}
\newcommand{\GL}{\operatorname{GL}}
\newcommand{\SO}{\operatorname{SO}}
\newcommand{\Orth}{\operatorname{O}}
\newcommand{\dop}{\,{\rm d}}
\newcommand{\nsphere}{S}
\newcommand{\ltnsphere}{L^{2}(\nsphere)}
\newcommand{\honsphere}{H^{1}(\nsphere)}
\newcommand{\lthalf}{L^{2}(\half)}
\newcommand{\ltg}{L^{2}(G)}   
\newcommand{\modcurve}{\Gamma \backslash \half}
\newcommand{\Gal}{\operatorname{Gal}}
\newcommand{\ipn}[2]{\langle #1, #2 \rangle_1}
\newcommand{\Graph}{\operatorname{graph}}
\newcommand{\mhaar}[1]{\frac{\operatorname{d}#1}{#1}}
\newcommand{\rn}{\R^{n}}
\newcommand{\laprn}{\Delta^{\rn}}
\newcommand{\lapsn}{\Delta^{\nsphere}}
\newcommand{\ltrn}{L^{2}(\rn)}
\newcommand{\cirn}{C^{\infty}(\rn)}
\newcommand{\horn}{H^{1}(\rn)}
\newcommand{\inv}{{-1}}
\newcommand{\p}{\mathfrak{p}}
\renewcommand{\P}{\mathfrak{P}}
\newcommand{\frob}[1]{\operatorname{frob}(#1)}
\newcommand{\Ell}{\mathcal{L}}
\newcommand{\arccosh}{\operatorname{arccosh}}
\newcommand{\I}{\mathbb{I}}
\newcommand{\A}{\mathbb{A}}
\newcommand{\Of}{\mathcal{O}}
\newcommand{\Isom}{\operatorname{Isom}}
\newcommand{\lmod}{\backslash}
\newcommand{\rmod}{/}
\newcommand{\Com}{\operatorname{Com}}
\newcommand{\hecke}{\mathcal{H}}
\newcommand{\ord}{\operatorname{ord}}
\newcommand{\mf}{\mathfrak}
\newcommand{\q}{\textbf{q}}
\newcommand{\normset}{\mathcal{N}}
\renewcommand{\Ell}{\mathcal{L}}
\newcommand{\infl}{\operatorname{Infl}}
\newcommand{\vchar}{\operatorname{Vchar}}
\newcommand{\nspec}{\mathcal{N}}
\newcommand{\prim}{\operatorname{prim}}

\theoremstyle{definition}
\newtheorem{claim}{Claim}
\newtheorem*{question*}{Question}
\newtheorem{thm}{Theorem}
\newtheorem{prop}{Proposition}
\newtheorem{remark}{Remark}
\newtheorem*{remark*}{Remark}
\newtheorem{mydef}{Definition}
\newtheorem{fact}{Fact}
\newtheorem{lemma}{Lemma}
\newtheorem{cor}{Corollary}



\setcounter{secnumdepth}{4}
\setcounter{secnumdepth}{4}



\title{Research Statement}
\author{Justin Katz}
\begin{document}
\maketitle 

The subject of my research is the spectral geometry of Riemannian manifolds, with a particular emphasis on the locally symmetric spaces associated to arithmetic lattices in Lie groups. My work makes essential use of geometric topology, harmonic analysis, algebraic groups, number theory, arithmetic/algebraic/differential geometry, and automorphic forms.     


In my first paper \cite{arapura2019}, in collaboration with Donu Arapura, Partha Solapurkar, and Ben McReynolds, we constructed locally symmetric manifolds and complex projective surfaces that share many algebraic and analytic invariants. For example, we produce non-isometric closed hyperbolic n-manifolds, as covers of a fixed manifold, that have isomorphic integral cohomology in such a way that the isomorphism commute with the natural maps induced by the cover. We also produced arbitrarily large collections of pairwise non-isomorphic smooth projective surfaces where the isomorphisms are natural with respect to the Hodge structure, or as Galois modules. In particular, the projective surfaces have isomorphic Picard and Albanese varieties, and have isomorphic effective Chow motives. All of these examples also have the same eigenvalue and geodesic length spectrum for their associated Riemannian structures. The construction based on a refinement of Sunada's method, based on examples first discovered by L. Scott and recently used by D. Prasad, in a construction that partly motivated ours. In this paper, we answer a question \cite{prasad2017} of D. Prasad.

While my first paper provided infinite families of pairwise isospectral, but nonisometric Riemannian manifolds, my current project proves that certain infinite families of Riemannian manifolds are spectrally rigid. I have proven the following 
 
\begin{thm}
	Let $B$ be an indefinite quaternion algebra over $\Q$, and let $\mathcal{O}\leq B$ be a maximal order and $\Gamma=\mathcal{O}^1$ be the norm $1$ units in $\mathcal{O}$. For any prime $p$ not dividing the discriminant of $B$, let $\Gamma(p)$ denote the principal congruence subgroup of $\Gamma$, and $X(p)(\C)$ be the complex points of the associated Shimura curve. When endowed with its canonical Riemannian metric of constant curvature,  $X(p)(\C)$ is spectrally rigid. That is, any Riemannian manifold that has the same Laplace eigenvalue spectrum as $X(p)(\C)$ is isometric to $X(p)(\C)$.  
\end{thm}

This is the strongest formulation that I am currently able to prove, but I am in the process of the following improvements: first, the hypothesis that $B$ be defined over $\Q$ can be weakened to $B$ being defined over a totally real number field $K$ such that $B$ has type number $1$. Second, the order $\mathcal{O}$ need not be a maximal, but rather could be an Eichler order of squarefree level. Third, the prime $p$ can be replaced by any ideal in $K$, coprime to the discriminant of $\O$. This theorem is, to the best of my knowledge, the first to establish spectral rigidity for any specific family of compact Riemann surfaces, and is sufficient to conclude that infinitely many Hurwitz surfaces are spectrally rigid. This partially confirms a conjecture \cite{reid2014} of Alan Reid. 


\bibliographystyle{plain}
\bibliography{mybib.bib}


\end{document} 