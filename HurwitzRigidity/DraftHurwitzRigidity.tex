\documentclass[11pt]{amsart}

\usepackage[margin=1in]{geometry} 
\geometry{letterpaper}

\usepackage{titlesec}                   % ... or a4paper or a5paper or ... 
\usepackage{graphicx}
\usepackage{amssymb}
\usepackage{amsthm}
\usepackage{titlesec}
\usepackage[utf8]{inputenc}
\usepackage[english]{babel}
\usepackage{cite}
\usepackage{tikz-cd}

\usepackage{mycros}

\setlength{\parindent}{0em}
\setlength{\parskip}{1em}

\setcounter{secnumdepth}{4}
\setcounter{secnumdepth}{4}

\titleformat{\paragraph}
{\normalfont\normalsize\bfseries}{\theparagraph}{1em}{}
\titlespacing*{\paragraph}
{0pt}{3.25ex plus 1ex minus .2ex}{1.5ex plus .2ex}

\title{Spectral Rigidity of Hurwitz Surfaces}
\author{Justin Katz}
\begin{document}
\maketitle

For an element $\gamma \in \PSL(2,\R)$, there is a unique lift $\tilde{\gamma} \in \SL(2,\R)$ such that $\trace(\tilde{\gamma})\geq 0$.
Define $\trace(\gamma):= \trace(\tilde{\gamma})$.
We say that an element $\gamma \in \PSL(2,\R)$ is hyperbolic if $\trace(\gamma)>2$.
In this case, there is a unique real number $N(\gamma)>1$ such that $\tilde{\gamma}$ is conjugate in $\SL(2,\R)$ to $\tbt{N(\gamma)}{0}{0}{N(\gamma)^{-1}}$.
We call $N(\gamma)$ the \emph{norm} of $\gamma$.
The norm and trace are related by the equation $\trace(\gamma)=N(\gamma)+N(\gamma)^{-1}$.
In particular, they uniquely determine one another.

Let $\Gamma \leq \PSL(2,\R)$ be a cofinite Fuchsian group.
We say that a hyperbolic element $\gamma$ is primitive if it is not a proper power.
Let $[\Gamma]_{{\rm prim}}$ denote the collection of conjugacy classes of primitive hyperbolic elements of $\Gamma$.
Since trace and norm are both conjugation invariant, we may extend their definition to $[\Gamma]_{{\rm prim}}$.
\[ Z_{\gamma}(s,\rho):= \prod_{[\gamma]\in [\Gamma]_{p}}\prod_{k=0}^{\infty}\det(1-\rho(\gamma)N(\gamma)^{-(s+k)}) \]

\paragraph{Multiplicative independence}


Set $\Lambda_\Gamma(s,\rho)=\log Z_\Gamma(s,\rho)$.
Compute for $\re(s)\gg 0$,
\begin{align*}
	\Lambda_\Gamma(s,\rho) & =\sum_{[\gamma]\in [\Gamma]_p} \sum_{k=0}^\infty \log \det (1-\rho(\gamma)N(\gamma)^{-(s+k)})                        \\
	                       & =\sum_{[\gamma]\in [\Gamma]_p} \sum_{k=0}^\infty \trace \log (1-\rho(\gamma)N(\gamma)^{-(s+k)})                      \\
	                       & =-\sum_{[\gamma]\in [\Gamma]_p} \sum_{k=0}^\infty\sum_{m=1}^\infty \frac{\chi_\rho(\gamma^m)}{m}N(\gamma)^{-m(s+k)},
\end{align*}
since $\log(1-A)=-\sum_{m=1}^\infty \frac{A^m}{m}$ for any matrix $A$ with sufficiently small entries.
In the last line, $\chi_\rho(\gamma)=\trace(\rho(\gamma))$ is the character of $\rho$.

The expression in the last line of the previous display is sensible for $\chi_\rho$ for any class function on $\Gamma$.
For $\chi$ a class function on $\Gamma$, we take this as a definition of $\Lambda_\Gamma(s,\chi)$ and thereby $Z_\Gamma(s,\chi)$.
For a class function $\chi$, suppose there are finitely many nonzero rational numbers $a_\rho$, for $\rho \in \hat{\Gamma}$ such that  $\chi=\sum_{\rho \in \hat{\Gamma}}a_\rho \chi_\rho$.
We call such a $\chi$ a \emph{virtual character}, and write $\chi \in \vchar(\Gamma)$.
For such a $\chi$,
\[ \Lambda_\Gamma(s,\chi)=\sum_{\rho\in \hat{\Gamma}} a_\rho \Lambda_\Gamma(s,\rho) \]
and
\[ Z_\Gamma(s,\chi)=\prod_{\rho\in \hat{\Gamma}} Z_\Gamma(s,\rho)^{a_\rho} \]

Suppose there is some multiplicative dependency among the functions $Z_\Gamma(s,\rho_1),...,Z_\Gamma(s,\rho_n)$, i.e.
there are rational numbers $a_{\rho_i}$ so that $\prod_{i=1}^n Z_\Gamma(s,\rho_i)^{a_{\rho_i}}=1$.
Then this may be expressed concisely as $Z_\Gamma(s,\chi)=1$, where $\chi=\sum_{i=1}^n a_{\rho_i} \chi_{\rho_i}$.

Let $\Gamma'\leq \Gamma$ be a finite index normal subgroup, and set $G=\Gamma'\lmod \Gamma$.
For each class function $\chi$ of $G$, let $\tilde{\chi}$ be its inflation to $\Gamma$.
Now suppose $\chi$ is a virtual character of $G$ such that $Z_\Gamma(s,\tilde{\chi})=1$.
This is equivalent to
\[ \Lambda_\Gamma(s,\tilde{\chi})= -\sum_{[\gamma]\in [\Gamma]_p} \sum_{k=0}^\infty\sum_{m=1}^\infty \frac{\tilde{\chi}(\gamma^m)}{m}N(\gamma)^{-m(s+k)}=0 \]

Recall (find ref) the uniqueness principle for generalized Dirichlet series: let $\nu_1, \nu_2,...$ be a sequence of distinct positive real numbers and $a(\nu_1),a(\nu_2),...$ a sequence of complex numbers such that the series
\[ \sum_{i=1}^\infty a(\nu_i)\nu_i^{-s}\]
converges absolutely to an analytic function $f(s)$ for $\re s \gg 0$.
Then $f(s)=0$ identically if and only if $a(\nu_i)=0$ for all $\nu_i$.


In order to apply the uniqueness principle for (generalized) Dirichlet series, we must collect summands according to their base.
To this end, let $\mathcal{N}_\Gamma$ denote the \emph{primitive norm spectrum of $\Gamma$}.
That is, $\mathcal{N}_\Gamma$ is the (multi-)set of values taken by the norm map $N$ at primitive hyperbolic conjugacy classes in $\Gamma$.
For every real number $\nu$, let $\mathcal{N}_\Gamma(\nu)$ denote the collection of classes $[\gamma] \in [\Gamma]_p$ with $N(\gamma)=\nu$.
By definition, $\mathcal{N}_\Gamma(\nu)$ is empty unless $\nu \in \mathcal{N}_\Gamma$.
Now we have
\[ \Lambda_\Gamma(s,\tilde{\chi})=-\sum_{\nu \in \mathcal{N}_\Gamma} \sum_{[\gamma] \in \mathcal{N}_\Gamma(\nu)} \sum_{k=0}^\infty\sum_{m=1}^\infty \frac{\tilde{\chi}(\gamma^m)}{m}\nu ^{-m(s+k)}=0.
\]
The sum over $k$ is a geometric series with base $\nu^{-m}$, so
\[ \Lambda_\Gamma(s,\tilde{\chi})=-\sum_{\nu \in \mathcal{N}_\Gamma} \sum_{[\gamma] \in \mathcal{N}_\Gamma(\nu)} \sum_{m=1}^\infty \frac{\tilde{\chi}(\gamma^m)}{m(1-\nu^{-m})}\nu ^{-ms}=0  \]

The set consisting of all powers of a fixed primitive norm and the set consisting of a fixed power of all primitive norms may intersect nontrivially.
Indeed, by McReynolds-Lafont, arithmetic non-compact hyperbolic surfaces admit arbitrarily long arithmetic progressions in their length spectrum.

For each primitive norm $\nu$, and each integer $m>0$, let $A_m(\nu)$ denote the set of primitive hyperbolic conjugacy classes $[\gamma]$ in $\Gamma$, such that $N(\gamma^m)=\nu$.
For a fixed $\nu \in \mathcal{N}_\Gamma$, only finitely many $A_m(\nu)$ are nonempty.
Let $m(\nu)$ be the largest $m$ such that $A_m(\nu)$ is nonempty.
Then we have a partition $\mathcal{N}_\Gamma(\nu)=A_1(\nu)\sqcup ...\sqcup A_{m(\nu)}(\nu)$.
Thus, for each $\nu \in \mathcal{N}_\Gamma$, the  coefficient $b(\nu,\chi)$ of $\nu^{-s}$ in $\Lambda_\Gamma(s,\chi)$  is
\[ b(\nu,\chi):= -\sum_{m=1}^{m(\nu)} \frac{1}{m(1-\nu^{-m})} \sum_{[\gamma]\in A_m(\nu)} \tilde{\chi}(\gamma^m).
\]
which we conclude is zero for every primitive length $\nu$.
This does not immediately imply that the class function $\tilde{\chi}$ is zero.

Indeed, for each fixed $\nu$, as we vary $\chi$ among the virtual characters, the function  $\lambda_\nu: \chi \mapsto b(\nu,\chi)$ is actually a linear functional on the vectorspace $\vchar(\Gamma' \lmod \Gamma)$.
In order to conclude that $\chi=0$, we must demonstrate that the (infinite!) set of functionals $\{ \lambda_\nu: \nu \in \mathcal{N}_\Gamma\}$ spans the (finite dimensional!) dual space to $\vchar(\Gamma'\lmod \Gamma)$.
%%%%%%%%%%%%%%%%%%%%
\paragraph{Conjugacy classes in $\GL(2,q)$ and $\PGL(2,q)$}
Let $q$ be an odd prime power and set $G=\GL(2,q)$ and $\bar{G}=\PGL(2,q)=G/ Z$ where $Z$ is the center of $G$.
For an element $g\in G$, define the \emph{characteristic polynomial} $p_{g}:=x^{2}-\tr(g)x+\det(g) \in F_{q}[x]$.
Among non central conjugacy classes, the characteristic polynomial of an element completely determines its class:
\begin{lemma}
	Suppose $g$ and $h$ are non central.
	Then $g$ and $h$ are conjugate in $G$ if and only if $p_{g}=p_{h}$
\end{lemma}
\begin{proof}
	This follows from the theory of Jordan normal forms.
\end{proof}
Conjugation in $\bar{G}$ can be described in terms of conjugation in $G$:
\begin{lemma}
	Elements $gZ$ and $hZ$ in $\bar{G}$ are conjugate if and only if $g$ is conjugate to $\lambda h$ in $G$ for some $\lambda \in Z$.
\end{lemma}
For an element $gZ$ of $\bar{G}$, let $\bar{p}_{gZ}$ denote the collection of characteristic polynomials of lifts of $gZ$ to $G$.
That is, $\bar{p}_{gZ}=\{p_{\lambda g}: \lambda \in Z\}$.
Combining the preceding lemmas, we obtain a characterization of nonidentity conjugacy classes in $\bar{G}$:
\begin{lemma}
	Nonidentity elements $gZ$ and $hZ$ are conjugate in $\bar{G}$ if and only if $\bar{p}_{gZ}=\bar{p}_{hZ}$.
\end{lemma}
%%%%%%%%%%%%%%%%%%
\paragraph{Multiplicative independence for principal congruence covers of semi-arithmetic surfaces}

Now suppose $\Gamma$ is a subgroup of $\PGL(2,\O_K)$  where $K$ is a totally real number field, and $\O_K$ is its ring of integers.
We further suppose that under some embedding $K\to \R$, the image of $\Gamma$ is a lattice in $\PGL(2,\R)$.
For any prime ideal $\pfrak$ of $\O_K$, the reduction mod $\pfrak$ map $\O_K \to \Fbb_\pfrak:=\O_K \rmod \pfrak \O_K$ induces a map $\PGL(2,O_K)\to G(\pfrak):=\PGL(2,\Fbb_\pfrak)$.
Let $\pi_\pfrak$ be its restriction to $\Gamma$.
For each $\pfrak$, define the \emph{principal congruence subgroup}  $\Gamma(\pfrak) := \ker \pi_\pfrak$ \emph{of level $\pfrak$}.
Let $S$ be the set of primes $\pfrak$ such that the map $\pi_\pfrak$ is surjective.
Thus, when $\pfrak \in S$, $\pi_\pfrak$ induces an isomorphism of $\Gamma(\pfrak) \lmod \Gamma$ with $G(\pfrak)=\PGL(2,O_K\rmod \pfrak O_K)$.
If $H$ is one of the groups $\Gamma,\Gamma(\pfrak)$ or $G(\pfrak)$, let $\tilde{H}$ denote its lift to $\GL(2,\Fbb_{\pfrak})$.


\begin{thm}
	Suppose $\gamma\in \Gamma - \Gamma(\pfrak)$ is hyperbolic and that the characteristic polynomial $p_{\gamma}$ is irreducible over $K$.
	If $\lambda \in \Gamma$ is such that $N(\gamma)=N(\lambda)$, then $\pi_{\pfrak}(\gamma)$ and $\pi_{\pfrak}(\lambda)$ are conjugate in $G(\pfrak)$.
\end{thm}

\begin{proof}
	It suffices to show that $N(\gamma)$ completely determines the characteristic polynomial $p_{\gamma}$.
	To this end, observe that the characteristic polynomial $p_{\gamma}$ of $\gamma$ is also the minimal polynomial for the unit $N_{\gamma}$ in the quadratic extension $K_{\gamma}:= K(N(\gamma))$ of $K$.
	Let $\sigma$ denote the nontrivial Galois automorphism of $K_{\gamma}/K$.
	Then $p_{\gamma}=(t-N(\gamma))(t-N(\gamma)^{\sigma}),$ as claimed.
	Specifically, the constant term $\det(\gamma)$ of $p_{\gamma}$ is the Galois norm $N(\gamma)N(\gamma)^{\sigma}$ of the quadratic unit $N(\gamma)$.
\end{proof}

\begin{corollary}
	Suppose $\chi$ is the inflation of a class function from $G(\pfrak)$ to $\Gamma$.
	Then $\chi$ is constant along each set $\mathcal{N}_{\Gamma}(\nu)$ such that $\nu \neq 1 \mod \pfrak$.
	If $\nu=1 \mod \pfrak$,  and $\mathcal{N}_{\Gamma}(\nu)\cap \Gamma(\pfrak)= \emptyset$, then $\chi$ is constant on $\mathcal{N}_{\Gamma}(\nu)$.
\end{corollary}

\begin{thm}
	Suppose $\Gamma$  is as above, and $\pfrak \in S$.
	Let $\chi$ be a class function on $\Gamma$ inflated from one on $G(\pfrak)$.
	Then $\Lambda_{\Gamma}(s,\chi)=0$ if and only if $\chi=0$.
\end{thm}
\begin{proof}
	If $\Lambda_{\Gamma}(s,\chi)=0$, then for all $\nu \in \mathcal{N}_{\Gamma}$, we have $b(\nu,\chi)=0$.

	First, suppose $\nu\neq 1 \mod \pfrak$.
	Then along the set $\mathcal{N}_{\Gamma}(\nu)$ of primitive hyperbolic $\gamma \in \Gamma$ with norm $\nu$, the function $\chi$ takes a common value which we denote $\chi(\nu)$.
	Let $a_{m}(\nu)$ denote the cardinality of $A_{m}(\nu)$, the set of primitive hyperbolic $\gamma \in \Gamma$ such that $N(\gamma^{m})=\nu$.
	Then
	\[
		b(\nu,\chi)=-\sum_{m=1}^{m(\nu)}\frac{a_{m}(\nu)}{m(1-\nu^{-m})}\chi(\nu)=0
	\]
	To conclude that $\chi(\nu)=0$, it suffices to demonstrate the existence of a primitive hyperbolic $\gamma \in \Gamma$ such that $N(\gamma)= \nu \mod \pfrak$.
	This follows from the Cheboratev density theorem of (cite Sarnak):

	\begin{thm}
		Let $C$ be a conjugacy class in $G(\pfrak)$ and $F(T,C)$ denote the collection of primitive hyperbolic $\gamma$ in $\Gamma$ with $N(\gamma)<T$ such that $\pi_{\pfrak}(\gamma) \in C$.
		Then as $T \to \infty$,
		\[F(T,C) = \frac{|C|}{|G(\pfrak)|} \frac{T}{\log T} +o(1).\]
	\end{thm}

	Thus, for any primitive hyperbolic $\gamma \in \Gamma$ with $N(\gamma)\neq 1 \mod \pfrak$, we have $\chi(\gamma)=0$.

	Now suppose $\nu =1 \mod \pfrak$.
	Then if $\gamma \in \mathcal{N}_{\Gamma}(\nu)$, either $\pi_{\pfrak}(\gamma) \in I$ the identity conjugacy class, or $\pi_{\pfrak}(\gamma) \in P$ the unique parabolic conjugacy class in $G(\pfrak)$.
	Note that $|I|=1$, and $|P|=q^{2}-1$, where $q= |O_{K}/\pfrak O_{K}|$.
	By Sarnak's version of Chebotarev's density theorem, there are infinitely $\nu \in \mathcal{N}_{\Gamma}$  with $\nu=1 \mod \pfrak$ such that $\mathcal{N}_{\Gamma}(\nu)\cap \Gamma(\pfrak) =\emptyset$.
	For any such $\nu$, $\chi$ is constant along $\mathcal{N}_{\Gamma}(\nu)$.
	As above, for such $\nu$, $b(\nu,\chi)=0$ implies $\chi(\nu)=0$.

\end{proof}





\bibliography{SRofHS}{}
\bibliographystyle{plain}

\end{document}