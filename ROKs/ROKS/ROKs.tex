\documentclass[11pt]{amsart}

\usepackage[margin=1in]{geometry} 
\geometry{letterpaper}

\usepackage{titlesec}                   % ... or a4paper or a5paper or ... 
\usepackage{graphicx}
\usepackage{amssymb}
\usepackage{amsthm}
\usepackage{titlesec}%
\usepackage[utf8]{inputenc}
\usepackage[english]{babel}
\usepackage{cite}
\usepackage{tikz-cd}
\usepackage{braket}
\usepackage{mydefs}

 %Character abbreviations
\newcommand{\D}{\mathcal{D}}
\newcommand{\T}{\mathcal{T}}
\newcommand{\N}{\mathcal{N}}
\renewcommand{\L}{\mathcal{L}}
\newcommand{\Tb}{\overline{\mathcal{T}}}
\newcommand{\Adeles}{\mathbb{A}}
\newcommand{\Ideles}{\mathbb{I}}
\newcommand{\Order}{\mathcal{O}}
\newcommand{\kron}[2]{\left( \frac{#1}{#2}\right)}
\newcommand{\Sig}{\Sigma}
\newcommand{\sig}{\sigma}


%Typography Shortcuts
\newcommand{\lp}{\left(}
\newcommand{\rp}{\right)}
\newcommand{\abs}[1]{\lvert #1 \rvert}
\newcommand{\nrm}[1]{\lVert #1 \rVert}

%Operatornames
\newcommand{\Ord}{\operatorname{Ord}}
\newcommand{\Ric}{\operatorname{Ric}}
\newcommand{\stab}{\operatorname{stab}}
\newcommand{\fp}{\operatorname{f.p.}}
\newcommand{\Ann}{\operatorname{Ann}}
\newcommand{\Exp}[1]{\operatorname{exp} \left( #1 \right)}
\newcommand{\Rep}{\operatorname{Rep}}
\newcommand{\Spec}{\operatorname{Spec}}
\newcommand{\Hom}{\operatorname{Hom}}
\newcommand{\Sym}{\operatorname{Sym}}
\newcommand{\image}{\operatorname{Im}}
\newcommand{\Diff}{\operatorname{Diff}}
\newcommand{\MCG}{\operatorname{MCG}}
\newcommand{\grad}{\operatorname{grad}}
\newcommand{\Cont}{\operatorname{Cont}}
\newcommand{\del}{\partial}
\newcommand{\Hex}{\operatorname{Hex}}
\newcommand{\Invo}{\operatorname{Inv}}
\newcommand{\Fix}{\operatorname{Fix}}
\newcommand{\Lie}{\operatorname{Lie}}
\newcommand{\SU}{\operatorname{SU}}

\setlength{\parindent}{0em}
\setlength{\parskip}{1em}
\newtheorem{prob}{Problem}

\setcounter{secnumdepth}{4}
\setcounter{secnumdepth}{4}

\titleformat{\section}{\normalfont\normalsize\bfseries}{\thesection}{1em}{}%%
\titlespacing*{\section}{0pt}{3.25ex plus 1ex minus .2ex}{1.5ex plus .2ex}

\titleformat{\subsection}{\normalfont\normalsize\bfseries}{\thesubsection}{1em}{}
\titlespacing*{\subsection}{0pt}{3.25ex plus 1ex minus .2ex}{1.5ex plus .2ex}




\title{Remarks, observations, keepsakes}
\author{Justin Katz}
\begin{document}
\maketitle 

\begin{itemize}
	\item $\GL(n,R)$ is the set of ordered bases, as a free $R$ module, of $R^n$.
	\item Elements of a quotient $G/K$ are subsets of $G$. So $G/K\subset 2^{2^G}$. 
	\item $M(2,\Z) \cap \GL(2,\R)=\{A \in M(2,\Z): \det(A) \in \Z\}$, which is strictly larger than $\cup_{n\in \Z} \tbt{n}{0}{0}{n} \GL(2,\Z)$. 
	\item $\SL(2,\Z)$ and $\Gamma(2)$ are, respectively, (finite index) subgroups of the mapping class group of a genus $1$ surface $S_{1,0}$ and a genus $1$ surface with $1$ puncture, $S_{1,1}$. 
	\item $\GL(2,\Q)$ acts transitively on $\Q^2-\{(0,0)\}$.
	\item The submonoid $\GL(2,\Q)\cap M(2,\Z)$ acts transitively on $\Z^2$.
	\item There is a one to one correspondence between primitive geodesics on $\PSL(2,\Z)\lmod H$ and ideal classes in real quadratic orders. To specify a class function supported on primitive hyperbolic elements of $\PSL(2,\Z)$ is to define a function on all ideal class groups of all real quadratic orders.
	\item For a real quadratic field $K$, the idele class group is $ K^\times \lmod \Ideles_K$. The norm $1$ units $K^\times  \lmod \Ideles^1_K$ is compact, and surjects onto every class group of every quadratic order contained in $K$. Consequently, any $\SL(2,\Z)$ class function induces a $K^\times$ invariant function on $\Ideles_K^1$ for every real quadratic field $K$ of $\Q$, though it need not be continuous. Will it at least be integrable? When the class function is the character of a representation of $\SL(2,\Z)$ which factors through a finite index subgroup, one knows at the very least that it takes only finitely many values. 
	\item For any field $K$, any order $\Order$ of $K$,and any fractional ideal $J$ of $\Order$, and any integral ideal $a$ of $\Order$, the quotient $K/aJ^{-1}$ is an $\Order$ module, and is (as an abelian group) isomorphic to $\Q^2/\Lambda$ for some lattice $\Lambda$.
	\item For a real quadratic field $K$, with real embeddings $i_1$ and $i_2$, the most natural way to embed $K$ in $\R$ is via $x\mapsto (i_1(x),i_2(x))$. It is precisely in this embedding that the norm one units $\Order^\times$ act by translation along hyperbolas with standard basis vectors as axes.  There is an element of $\GL(2,K)\subset \GL(2,\R)$ which conjugates the image of this embedding to $\Q^2$.
	\item The fundamental domain for the cyclic group generated by a hyperbolic element $\eta$ of $\SL(2,\R)$ acting linearly on $\R^2$ is the union of half open sectors spanned by $v$ and $\eta v$, and $\eta v$ and $v$ for any nonzero $v$ away from the eigenspaces of $\eta$.
	\item A subset of $\Z_p$ is open/closed if and only if it is the preimage under reduction mod $p^n$ of some subset $A$ of $\Z_p/p^n \Z_p$. Given any subset $A$ of $\Z_p$, one can certainly look at its image under reduction mod $p^n$, its preimage under each such reduction as an open/closed approximation. 
	\item Kronecker Limit formula for real quadratic $K$ of (fundamental) discriminant $d$ (from "On a zeta function associated to a Quadratic order"):
	\begin{align*}
		\zeta_K(s)&=\frac{2 h_K \log \eps_d}{\sqrt{d}(s-1)} -\frac{1}{6\sqrt{d}} \sum_{Q\in PBQF(d)/\Gamma} \int_{-\log \eps_d}^{\log \eps_d} \log(y_Q(v)^6 |\Delta(z_Q(v))|)\dop v \\
				  &+\frac{h_K \log \eps_d}{\sqrt{d}}(4\gamma -\log d) +O(s-1) 
	\end{align*}
	where $\Delta$ is the modular discriminant and $z_Q(v)=\frac{\alpha_Q-i \alpha_Q ' e^{-v}}{1-ie^{-v}}$ with $\alpha_Q,\alpha_Q'$ the roots of dehomogenized $Q$. 
	\item Tanaka duality manifesting itself for groups: Let $\Gamma$ be a residually finite group, and $A$ a commutative ring. Let $\Rep_A(\Gamma)$ be the category of $A$-representations of $\Gamma$. Then a morphism $u:\Gamma_1\to \Gamma_2$ between groups should induce an isomorphism between profinite completions if and only if the dual (restriction) map $u*: \Rep_A(\Gamma_2) \to \Rep_A(\Gamma_1)$ is an equivalence of categories, for every $A$.
	\item There's a hierarchy of information vis-a-vis understanding the rep theory of a group. For now let $G$ be finite. The first tier is knowing the number $n$ of irreducible representations $V_i$ of $G$ and their degrees $d_i$, this is enough to give the polynomial ring generated by reps of $G$ a natural grading. The second tier is knowing the splitting data of tensor products: $V_i \otimes V_j= \bigoplus_{k} m_{ijk} V_k$. This, apparently, is enough to give the whole character table of $G$. Note, this only tells us about the character of $V_i \times V_j$; it is not enough to tell us what the character of $V_i$ times $V_j$ is. 
\end{itemize}
\section{Reps and stuff}
From Mostow, on "Representative functions": Let $G$ be a group, $V$ a vector space, $\rho: G\to \Aut(V)$ a representation over some field $k$. 

\begin{itemize}
	\item A representation, above all else, is an action of a group by automorphisms of an abelian group.
	\item Associate to each pair of vectors $v\in V$, $w\in V^*$, the $k$-valued function $\rho_{v,w}(g)=\ip{\rho(g)v}{w}$, and call it a coefficient function, and write $[\rho]$ for the $k$-linear span of such functions. 
	\item When $V$ is finite dimensional, so is $[\rho]$. As a space of functions on $G$, $[\rho]$ has commuting left and right $G$ actions, by translation. Set $\Rep_k[G]_X= \cup_{\rho \in X} [\rho]$ for some collection of representations $X$ of $G$. Take as default, $X$ the collection of finite dimensional reps over $k$. Then a $k$ valued function $f$ on $G$ is in $\Rep[G]$ if and only if the $k$ linear span of its $G$ translates is finite dimensional. 
	\item Note $\Rep[G]$ is a unital $k$-algebra.
	\item If $V_\rho$ is the underlying space of some finite dimensional rep $\rho$ of $G$, then $V_\rho$ embeds as a $G$ module into $[\rho] \oplus ... \oplus [\rho]$ ($\dim V_\rho$ summands), via the $G$ map $v \mapsto \rho_{v,w_1}+...+\rho_{v,w_{\dim(V_\rho)}}$, where $w_1,...,w_{\dim V_\rho}$ is a basis for $V_\rho ^*$.
	\item Let $\tau$ denote the right translation rep of $G$ on $\Rep_k(G)$. Suppose $P$ is a bi-invariant $k$-subalgebra of $\Rep_k(G)$ containing the constant function $1$, and $\alpha$ an algebra endomorphism of $P$, a) preserving fixing the constant function $1$, b)commuting with $\tau(G)$. Then every subspace of $P$ which is $\tau(G)$ stable is also $\alpha$ stable. 
	\item If, further, $P$ is invariant under the involution $f\mapsto f'$ (inversion), then $\alpha$ preserves the  constant functions, and $\alpha$ is actually an automorphism with inverse given by $(\alpha^{-1} f)(s)=\alpha (f\cdot s)'(1)$.
	\item It follows that there is a bijection between proper automorphisms of $\Rep_k(G)$ and $\Hom_{k-alg}(\Rep_k(G),k)$.  Call this common space $A(G)$ or $\Spec(\Rep_k(G),k)$. One should regard $A(G)$ as a sort of underlying space, and $\Rep_k(G)$ as the algebra of coordinate functions on it. Note, $A(G)$ is actually a pro-algebraic group: it's the projective limit of algebraic groups. 
	\item A homomorphism $\theta:G'\to G$ induces a $k$ algebra homomorphism $\theta^t:R[G]\to R[G']$ (a transpose), and thus a homomorphism $\hat{\theta}:A(G') \to A(G)$. 
	\item Let $L$ be a LAG/$\Q$, and $\Gamma$ a subgroup such that each everywhere defined rational $\Q$ function on $L$ has bounded denominators on $\Gamma$. Let $\rho$ be a $\Q$ rep of $L$ on the $\Q$ space $V$. Then $\Gamma$ stabilizes a lattice in $V_\Q$. 
	\item The proalgebraic topology is akin to looking at the congruence topology relative to all possible arithmetic realizations. 
\end{itemize}
\section{Shintani stuff}
From "The p-adic Shintani modular symbol and evil Eisenstein series." 
\begin{itemize}
	\item Let $V$ denote $\Q^2$. Write $S(V)=S(\Adeles_V^(\infty))$ as the group of test functions on the finite adeles of $V$. Concretely, $S(V)$ consists of the functions $f:\Q^2\to \Z$ that are supported on a lattice, and also periodic with respect to some lattice, hence factor through a finite quotient of their support.
	\item Key point: continuity of such functions comes from factoring through a finite quotient of support, while compact supportedness comes from being supported on a single lattice. For example, the constant $1$ function on $\Q^2$ is continuous but not compactly supported, hence not Schwartz. By contrast, the indicator function for $\Z[1/p]^2$ is neither continuous nor compactly supported. 
	\item Shintani's setup: $G=\GL(2,\Q)^+$ and $X$ the space of rational symmetric two by two matrices. Let $\hat{G}$ and $\hat{X}$ be their congruence completions. Then $\hat{G}$ consists of those $g\in \GL(2,\Adeles^f)$ with $\det g \in \Q_{>0}$. Here, $\Gamma=\SL(2,\Z)$ and $\hat{\Gamma}$ its closure in $\hat{G}$. 
	\item There is a natural bijection $\hat{\Gamma}\lmod \hat{X}$ with $\Gamma \lmod X$ and $\hat{\Gamma}\lmod \hat{G} / \hat{\Gamma}$ with $\Gamma \lmod G / \Gamma$. Note that, under this identification, a function is compactly supported on the latter if and only it is finitely supported on the former. 
	\item Picking a haar measure on $\hat{G}$ normalized so $\int_{\hat{\Gamma}} \dop g=1$, the action of the hecke algebra $H(\Gamma,G)$ on $C^\infty(\Gamma \lmod X)$  is by convolution. For $x\in X$, let $\Gamma_x$ be stabilizer. Fix base point $x_o \in X$ which is primitve of conductor $1$. Let $\dop \mu$ be compatible measure on $\hat{\Gamma} \lmod \hat{X}$, normalized so $\hat{\Gamma \cdot x} \subset \hat{X}$ has measure $1$. 
	\item Then for any other $x\in X$, and picking $g_x \in G$ so that $g_x \cdot x= x_o$, one has 
	\begin{align*}
		\mu(x):= \int_{\hat{\Gamma \cdot x}} \dop \mu =|\Gamma_{x_o} : g_x \Gamma_x g_x^{-1}].  
	\end{align*}
	\item Suppose $x\in X$ has conductor $f$, and discriminant $d$. Then $\mu(x)=[\Order^1_{d,1}:\Order_{d,f}^1]$. 
	\item There is a bijection: 
		\begin{align*}
			\{ \alpha \in a | N(\alpha)>0 (\alpha, f) =1 \} / \Order_f^1 \approx \{b \in C^{-1} |b+f f\Order_f=\Order_f \}
		\end{align*}
		where $a$ is an $\Order_f$ ideal in the  class  $C \in Pic(\Order_f)$. The identification is $\alpha \mapsto \alpha \overline{a}/ N(a)$ where $\overline{\cdot}$ is galois conj. 
	\item The fourier--eisenstein transform: let $\phi$ be a $\Gamma$ invariant complex valued function on $\Gamma \lmod X$ which is supported on finitely many orbits. Set $F(\phi)(s_1,s_2) = \int_{\hat{X}} \phi(x) E(x;s_1,s_2) \dop \mu(x)$, where $E(x,s_1,s_2)=\mu(x)^{-1} \sum_{v\in \Z^2 /\SO(Q_x)} |Q_x(v)|^{-s_1 -1/2}|\det x| ^{-s_2+1/4}$ where $Q_x$ is the BQF defined by $X$. 
	\item When $\phi$ is supported on finitely many orbits, the fourier--eisenstein transform of $\phi$ is just a finite linear combination of eisenstein series: $F(\phi)(s_1,s_2)=\sum_{x \in \Gamma \lmod X} \phi(x) [\Order^1:\Order^1_{f(x)}] E(x,s_1,s_2)$
	\item $K$ is real quadratic field with discriminant $D$, let $X_{D,1}=X$ be two by two rational symmetric matrices with determinant $D/4$. Set $K'=K-\Q$. The map $\alpha \mapsto S_\alpha:=\frac{\sqrt{D}}{\alpha - \overline{\alpha}} \tbt{1}{-\tr \alpha/2}{-\tr \alpha /2}{N \alpha}$ is a bijection $K'$ with $X$.
	\item Letting $g=\tbt{a}{b}{c}{d}$ act on $\alpha$ by $g\cdot \alpha = \frac{d \alpha -c}{-b \alpha +a}$, one has $g\cdot S_\alpha = S_{g \cdot \alpha}$ . This gives an identification of $C^\infty (\Gamma \lmod X)$ with $C^\infty(\Gamma \lmod K')$. 
	\item Following Arakawa, define for $\alpha \in K'$: $\xi(s,\alpha)= \sum_{n=1}^\infty \frac{\cot \pi n \alpha}{n^s}$. The $\xi(s,\alpha)$ cvges absolutely for $\re(s)>1$, has meromorphic continuation to $\C$ with simple pole at $s=1$. Let $c_{-1}(\alpha)$ denote the residue at $s=1$. 
	\item Apparently: $c_{-1}(\alpha)$ is left $\Gamma$ invariant, and is a common $H(\Gamma,G)$ eigenfunction: for $f\in H(\Gamma,G)$ one has $f \cdot c_{-1} = \hat{f}(-1/2) c_{-1}$. 
\end{itemize}

\section{An effective grunwald wang}
\begin{itemize}
	\item $K$ number field, $C_K=\Ideles_K/K^\times$ idele class group. Grunwald: given a finite set $S$ of places of $K$, and a family of characters $\chi^\nu$ of $K^\times_\nu$ for $\nu \in S$ of orders $m_\nu$ (local characters), there exists a continuous character $\chi$ of $C_K$ of finite order (global character) whose local component $\chi_\nu$ at each $\nu \in S$ restrict to $\chi^\nu$ on the copy of $K^\times_\nu $ in $C_K$.  T
	\item Given the local data, there are infinitely many global characters satisfying Grunwald's conclusion. Most are highly ramified. The game: control the ramification of possible such. 
	\item For $\chi_\nu$ a continuous character of $K_\nu^\times$, define arithmetic conductor: $N(\chi_\nu)$ to be $1$ if $\chi_\nu$ is unramified or $\nu$ is archimedean, and $q_\nu^n$ when $n$ is the smallest integer such that $(1+p_\nu^n)^\times \subset \ker(\chi_\nu)$, for $p_\nu$ the unique maximal ideal in $O_nu$. For global characters, take a product. 
	\item A theorem (S version of multiplicity one for $\GL(1)$): Let $\chi$ be a nontrivial global character of $C_K$ of finite order. Set $A(\chi,S)=d_K N(\chi)N_S$. Then there exists a place $\nu$ of $K$ with:
		\begin{itemize}
			\item $p_\nu$ is not in $S$
			\item $\chi_\nu \neq 1$ and is not ramified
			\item $\log N(p_\nu)  \	ll \log A(\chi,S)$ 
			\item $N(p_\nu) \ll_{\eps,K} N(\chi)^{1/2+\eps} N_S^\eps$ for every $\eps>0$
			\item With GRH, $N(p_\nu) \ll (\log A(\chi,S))^2$. 
		\end{itemize}		 
\end{itemize}

\section{Useful identities}
\begin{itemize}
	\item For $A\in M(2,R)$: $\tr (A^2)=(\tr A)^2-2\det(A)$.
	\item For $u,v\in \SL(2,R)$: $\tr(uv)=\tr(u)\tr(v)-\tr(u^{-1}v)$
\end{itemize}

\section{notes on Goldfeld's textbook}
Let $(\pi,V)$ be an automorphic cuspidal representation of $\GL(2,\Adeles_\Q)$. Fix $f_c1,f_2 \in V$  and set, for $g\in \GL(2,\Adeles_\Q)$: 
	\begin{align*}
		\beta(g):= \int_{\GL(1,\Adeles_\Q)\GL(2,\Q) \lmod \GL(2,\Adeles_\Q)} f_1(hg)\overline{f_2(h)} \dop ^\times h,
	\end{align*}
Let $\phi$ be a Bruhat-Schwarz function. Define for $s\in \C$ with $\re(s) \gg 0$ define 
	\begin{align*}
		Z(s,\Phi,\beta):= \int_{\GL(2,\Adeles_\Q)} \Phi(g) \beta(g) |\det(g)|^{s+1/2} \dop^\times g.
	\end{align*}
\begin{itemize}
	\item An idea: take $\pi$ to be the right regular representation on $L^2_0(\GL(2,\Q) \mod \GL(2,\Adeles_\Q))$ as the automorphic cuspidal representation. Then the game becomes: picking vectors $f_1$ and $f_2$ which decay fast enough (on the spectral side)to make these integrals coverge.  
\end{itemize}

\section{notes on (local) densities}
Every element of $S\in \Sym (2,\Z_p) $ gives rise to a binary quadratic form, and yields a function  $\Z_p \to \C$ (really to $\Q$),  $x\mapsto \alpha(S,x)$ the \textbf{local representation density of $x$ by $S$}. Explictly, letting 
$$A_t(S,x)= |\{ X \in M(2,\Z_p /p^t \Z_p): S[X]=x \mod p^t \Z_p\}|,$$ 
one has
$$ \alpha(S,x) = \lim_{t\to \infty} p^{-t} A_t(S,x) $$ 

It turns out that this local density has an integral expression. Letting
$$W(S,x):= \int_{\Q_p} \int_{\Z_p^2}\psi(b S[v]) \psi(-xb) \dop v \dop b,$$
one has $W(S,x)=\alpha(S,x)$ for all $S \in \Sym(2,\Z_p)$ and all $x\in \Z_p$. 

Note: as a function of $S$, the value of $W(S,x)$ depends only on the $\GL_2(\Z_p)$ orbit of class of $S$. 
\section{10/31/2021}

\begin{itemize}
	\item Any open subgroup of of $\PGL(2,k)$, or $\SL(2,k)$, or $\GL(2,k)$, contains unipotent elements. Consequently, if $\Gamma$ is arithmetic fuchsian (cocompact or not), the $p$-adic closure of $\Gamma$ will contain unipotent elements at all primes $p$ unramified in the ambient quat alg.  
	\item 	
\end{itemize}

\section{Notes on: Clifford theory for representations, by Karpilovsky}
\begin{itemize}
  \item In this section, all rings are unital. 
  \item For a (finite) family of rings $R_{1},...,R_{n}$, and a given ring $R$, one has $R \approx R_1\times \cdots \times R_n$  (as rings) if and only if there exist pairwise orthogonal central idempotents $e_1, \cdots , e_n \in R$ such that 
  		\subitem a)   $1=e_1+ \cdots + e_n$, and 
  		\subitem b) $R_i \approx Re_i$  (as rings), for all $i$.
  \item For a ring $R$, denote the left $R$ module with underlying abelian group $R$ and $R$ action given by multiplication on the left by   $_R R$ (this is the left regular $R$ module). 
  \item If $M$ is an $(S,R)$ bimodule, and $N$ a left  $R$ module, then $M \otimes_R N$ (which a priori is a priori only an abelian group) is naturally a left $S$ module.
  \item To summarize: tensor product is a functor from the category of pairs (left $R$ modules,right $R$) modules  to the category of abelian groups. 
  \item In particular, if $R$ is a commutative ring, and $M,N$ are $R$ modules, then $M\otimes_R N$ is canonically an $R$ module.
  \item Let $M,M',M''$ be right $R$ modules, and $N,N',N''$ be left $R$ modules.  Suppose we have right $R$ modules maps $u: M' \to M$ and $v: M \to M''$ such that $v$ surjects $M$ onto $M''$, and $v$ is $0$ on the image of $M'$ in $M$ under $u$ (i.e. the sequence $M' \to  M \ to M'' \to 0$ is exact), and similarly for maps $s: N' \to N$ and $t:N \to N''$. Then the sequence (of abelian groups) \begin{equation*}  M' \otimes_R N \to M\otimes_R N \to M'' \otimes_R N \to 0 \end{equation*} is exact.  Similarly for $M \otimes_R  -$ 
  \item In particular, the homomorphism (of abelian groups) $v \otimes t :M\otimes_R N \to M'' \otimes_R N''$ is surjective, with kernel $\image(u\otimes 1) + \image(1\otimes s)$
  \item Let $M$ be an $(S,R)$ bi-module. Then $M\otimes _R R \approx M$ as (left) $S$ modules.
  \item If $R$ is a subring of a ring $S$, and $M$ is a free left $R$ module with $R$ basis $e_1, \cdots, e_n$, then $S \otimes_R M$ is a free left $S$ module with $S$ basis $\{1 \otimes e_1, \cdots , 1 \otimes e_n\}$. 
  \item For $M$ a right $R$ module, and $A$ a left ideal of $R$. Then $M\otimes_R A \to M$ via $m\otimes a \mapsto ma$ is the `canonical' map. Its image is the (additive) subgroup $MA$ of $M$ consisting of sums $\sum_i m_i a_i$ for $m_i \in M$ and $a_i \in A$.
  \item The property that a right $R$ module $M$ is flat is equivalent to the canonical map $M\otimes _R A \to MA$ being an isomorphism (of abelian groups) for every left $R$ ideal $A$. 
  \item Definition: for a commutative ring $R$, an $R$-algebra is a ring $A$ which is also an $R$ module, in such a way that $ r(xy)=(rx)y=x(ry)$  for all $x,y \in A$ and $r\in R$.
  \item Tensor products of $R$ algebras are again $R$ algebras. If $A_1$ and $A_2$ are $R$ algebras with units $e_1$ and $e_2$ respectively, then the unit of $A_1 \otimes_R A_2$ is $e_1 \otimes e_2$.  The maps $f_1: A_1 \to A_1 \otimes_R A_2$ and $f_2:A_2 \to A_1 \otimes_R A_2$ such that $f_1(x)=x \otimes e_2$ and $f_2(x)= e_1 \otimes x$ are homomorphisms of $R$ algebras such that $f_1(a_1)f_2(a_2)=f_2(a_2)f_1(a_1)$ for all $a_i \in A_i$. The tensor product is the universal such object: so tensor products are universal in the \emph{domain}.
  \item For any ring $R$, write $M_n(R)$ for the ring of $n\times n$ matrices with coefficients in $R$, and for $i,j \leq n$, let $e_{ij}$ be the elt in $M_n(R)$ which is $1$ in the $i,j$th spot and $0$ elsewhere. The elements $e_{ij}$ are referred to as the \emph{matrix units}. 
  \item Key point: the matrix units $\{ e_{ij} : 1\leq i,j\leq n\}$ satisfy the following properties:
 	\begin{enumerate}
  		\item $e_{ij}e_{ks} = 0$ if $j\neq k$ and $e_{ij}e_{ks} = e_{is}$ if $j=k$. 
  		\item $ \id = 1 = e_{11} + \cdots + e_{nn}$
  		\item  The centralizer of $\{ e_{ij}: 1\leq i,j \leq n\}$ in $M_n(R)$ is $R\id =R$. 
  		\item $R\approx e_{11} M_n(R) e_{11}$ (as rings).  
  	\end{enumerate} 
  \item In fact, for any ring $S$ containing elements $v_{ij}$ for $1 \leq i,j \leq n$ satisfying the first three properties above, let $R$ be the centralizer of $\{ v_{ij} :  1\leq i,j\leq n\}$ in $S$. Then the map $M_n(R) \to S$ defined by $(a_{ij}) \mapsto \sum a_{ij} v_{ij} \in S$  is an isomorphism of $R$ algebras such that $R \approx v_{11} S v_{11}$  (as rings).
  \item For any $R$ module $V$ and $n\geq 1$, one has $\End_R(V^n) \approx M_n(\End_R(V))$.
  \item Key point: the map $W \mapsto W^n$ is an isomorphism of the lattice of $R$ submodules $W$ of $V$ onto the lattice of $M_n(R)$ submodules of $V^n$.
  \item $\End_{M_n(R)} (V^n ) \approx \End_R(V)$ 
  \item The map $V \mapsto V^n$ induces a bijective correspondence between the isomorphism classes of $R$ modules and $M_n(R)$ modules. The inverse map is given by $W \mapsto e_{11} W$. 
  \item For an ideal $I$ of $R$, set $M_n(I)=\{ (a_{ij}) \in M_n(R) : a_{ij} \in I \: {\rm for}\: 1 \leq i,j \leq n\}$. Note, this is not a subring of $M_n(R)$ in the author's definition: unless $I=R$, $M_n(I)$ will not have  a unit. It is, however, an $M_n(R)$ submodule of $M_n(R)$. That is, it is a (left and right) $M_n(R)$ ideal. 
  \item In fact, the correspondence $I\mapsto M_n(I)$ is a bijection between ideals of $R$ and $M_n(R)$. In particular, $R$ is simple if and only if $M_n(R)$ is. 
  \item The correspondence is natural: $M_n(R)/M_n(I) \approx M_n(R/I)$ (as rings) 
  \item If $R=I_1 \oplus \cdots \oplus I_s$ is a two sided decomposition of $R$, then $M_n(R) = M_n(I_1) \oplus \cdots \oplus M_n(I_s)$. 
  \item An $R$ module is artinian if all descending chains of submodules terminates, and is noetherian if all ascending chains of submodules terminate. Example: $\Z$ is noetherian but not artinan, since $\cdots \leq p^n\Z \leq p^{n-1} \Z \leq \cdots \leq p\Z \leq \Z$ does not terminate. 
  \item Say a ring $R$ is artinian/noetherian if the left regular $R$ module $_R R$ is artinian/noetherian.
  \item Say $V$ is finitely generated if there exist $v_1,\cdots, v_n$ so that $V=Rv_1+\cdots Rv_n$, and finitely co-generated if for every family $\{V_i: i\in I\}$ of submodules of $V$ with $\bigcap_{i \in I} V_i =0$, there exists a finite subset $J$ of $I$ such that $\bigcap_{j \in J} V_i =0$.
  \item Artinian modules are characterized by the property: every nonempty set of submodules has a minimal element. Alternatively, every quotient is finitely cogenerated. Consequently, every nonzero artinian module has an irreducible submodule.  
  \item noetherian modules are characterized by the property: every nonempty set of submodules has a maximal element. ALternatively, every submodule is finitely generated. 
  \item Key point: a nonzero $R$ module has a composition series if and only if it is both artinian and noetherian. 
  \item Let $W$ be an $R$ submodule of an $R$ module $V$. Then $V$ is artinian/noetherian if and only if both $W$ and $V/W$ are artinian/noetherian. 
  \item For finitely generated modules $V$ over $R$: if $R$ is artinian/noetherian then $V$ is artinian/noetherian. 
  \item If $R$ is a ring and $V$ a nonzero $R$ module which is either artinian or noetherian. Then $V$ is a direct sum of indecomposable submodules. 
  \item If $V$ is completely reducible, then every submodule is a homomorphic image of $V$, and every homomorphic image of $V$ is isomorphic to a submodule of $V$. If $V\neq 0$ then $V$ contains an irreducible submodule. 
  \item For a nonzero $R$ module $V$, the following are equivalent: 1) $V$ is completely reducible, 2) $V$ is the \emph{direct} sum of irreducible submodules, 3) $V$ is the sum (not necessarily direct) of irreducible submodules. 
  \item Definition: a completely reducible $R$ module $V$ is \emph{homogeneous} if it can be written as a sum of mutually isomorphic irreducible submodules. The sum of irreducible submodules isomorphic to a given irreducible one is called a homogeneous component.
  \item A submodule $W$ of an $R$ module $V$ is called \emph{fully invariant} if it is mapped into itself by all elts of $\End_R(V)$. In a completely reducible $R$ module, a submodule is fully invariant if and only if it is a sum of a set of homogeneous components.
  \item If $V$ is completely reducible and can be written as $V=\bigoplus_{j \in J} W_j$ for homogeneous components $W_j$,  then $\End_R(V)\approx \prod \End_R(W_j)$ (as rings).
  \subsection{1.4}
  \item The radical of an $R$ module $V$, written as $J(V)$ is the intersection of all maximal submodules of $V$ (if there are no such submodules, the `empty intersection' is $V$).  If $V$ is finitely generated and nonzero, then $J(V)\neq V$.
  \item The jacobson radical of a ring $R$ is the radical of the left regular module $_R R$. Equivalently, this is the intersection of all maximal left ideals of $R$.  Say $R$ is semisimple if $J(R)=0$. 
  \item The annihilator of an $R$ module $V$ is the kernel of the module-defining-map $R \to \End(V)$. That is, $\Ann(R)$ is the collection of $r\in R$ which act as the zero operator on $V$. Then $\Ann(V)$ is an ideal in $R$, and $V$ is naturally an $R/\Ann(V)$ module.  
  \item Call an $R$ module faithful if $\Ann(V)=0$. That is, if for all $r \in R$, there is some $v \in V$ such that $rv\neq 0$. 
  \item An $R$ module is irreducible if and only if $V \approx R/X$ (as $R$ modules) for some maximal left ideal $X$ of $R$. 
  \item Call an ideal $I$ of $R$ primitive if $R/I$ (a ring) has a faithful irreducible module. Then $I$ is primitive if and only if $I$ is the annihilator of an irreducible $R$ module.
  \item Homomorphisms of modules map jacobson radicals to jacobson radicals (surjectivity if and only if the homomorphism is surjective with kernel contained in jacobson radical of the image.) 
  \item If $W \leq V$ as $R$ modules, then $J(W) \leq J(V)$ and $(J(V)+W)/W \leq J(V/W). $
  \item If $W \leq J(V)$ as $R$ modules, then $J(V/W) = J(V)/W$.
  \item For any ring $R$ and $x\in R$, then $x\in J(R)$ if and only if for all $r\in R$, $1-rx$ is a left unit of $R$. This implies $J(R)$ contains no nonzero idempotents
  \item Key computation: suppose $I \leq R$ is a left nil ideal of $R$. Then for $x\in I$ and any $r\in R$, we have $rx\in I$, so $(rx)^n=0$ for some $n$. In particular, the sequence of sums $1+\cdots+ (rx)^k$ stabilizes for $k\geq n$. The stable value is the left inverse of $1-rx$. 
  \item For an artinian ring $R$, $J(R)$ is nilpotent. 
  \item For any ring $R$ and any $n>0$, $J(M_n(R))=M_n(J(R))$.
  \item Schur: if $V$ is an irreducible $R$ module, then $\End_R(V)$ is a divison ring. 
  \item For $V$ an $R$ module and $e\in R$ an idempotent. Then $\Hom_R(eR,V) \approx Ve$ as additive groups. Similarly for a right $R$ module. Proof: the isomorphism is $f \mapsto f(e)$. 
  \subsection{1.6: group algebras}
  \item In this section $R$ is a commutative ring. Key fact: if $H \leq G$ and $T$ is a right/left transversal for $H$ in $G$, then $RG$ is a free left/right $RH$ module, freely generated by $T$.
  \item For any groups $G,H$, $R(G\times H) \approx RG \otimes_R RH$. 
  \item Key lemma: Let $H \leq G$ finite index subgroup, and $T \subset G$ a left transversal for $H$ in $G$ containing $1$. Let $V$ be an $RG$ module, define $f: (V_H)^G \to V$ (subscript is restriction, superscript is induction) by $f(\sum_{t \in T} t \otimes v_t ) = \sum_{t\in T} tv_t$ (for $v_t\in V$). Then $f$ is a surjective homomorphism of $RG$ modules, and $\ker(f) \leq (V_H)^G$ is a direct summand of $((V_H)^G)_H$. 
  \item Key lemma: for $H,G,T$ as above, and $V,W$ both $RG$ modules, then for any $f\in \Hom_{RH}(V_H,W_H)$ the map $f^*:V\to W$ defined by $f^*(v) = \sum_{t\in T} (tft^{-1} )(v)$ is an $RG$ homomorphism. Furthermore, a different choice of $T$ does not change $f^*$. 
\end{itemize}
\section{Notes on Traces of Hecke Operator}
\begin{itemize}
  \item Let $R$ denote the right regular representation of $G(\A)$ on $L^2(\omega)$, the space of square integrable functions on $G(\Q)\lmod G(\A)$ which transform by $\omega$ on $Z(\A)$. So $R(g)f(x)=f(gx)$ for $f\in L^2(\omega)$. The cuspidal subspace $L_o^2(\omega)$ consists of those $f\in L^2(\omega)$ such that $\int_{N(\Q) \lmod N(\A)} f(ng)=0$ for almost all $g \in G(\A)$.  Set $R_o = R(\cdot) |_{L_o^2(\omega)}$. 
  \item Let $C_c(G(\A),\omega^\inv)$ be the subspace of continuous $f:G(\A) \to \C$ with compact-mod-center support such that $f(zg)=\omega^\inv(z)f(g)$ for all central $z$. Extend the definition of $R$ to $C_c(G(\A),\omega^\inv)$ by $R(f)\phi(g) = \int_{\overline{G}(\A)} f(x) \phi(gx) \dop x$.
  \item Then $R_o(f)$ is trace class for all such $f$ (though $R(f)$ may not be).
  \item The following is the Arthur/Selberg trace formula for $\GL_2$: 
%\begin{align*}
%	\Tr R_o(f) = &  \vol(\Gbar(\Q) \lmod \Gbar(\A)) f(1)  \\
%				 &+ \sum_{[\gamma] \subset \Gbar(\Q)\text{ elliptic} } \int_{\Gbar_\gamma (\Q) \lmod \Gbar(\A)} f(g^\inv \gamma g ) \dop g \\
%				 &+ -\frac{1}{2} \vol (\Q^\times \lmod \A^1 ) \sum _{\gamma \in \Mbar(\Q)-\{1\} } \int_{\Mbar(\Q) \lmod \Gbar(\A)} f(g^\inv \gamma g) v(g) \dop g \\
%				 &+ \fp_{s=1} Z_F(s) \\
%				 &+\frac{1}{4\pi} \sum_{\chi_1 \chi_2=\omega} \int_{-\infty}^{\infty} \Tr(M(-it) M'(it) \rho(\chi ,it) (f) ) \dop t \\
%				 &- \sum_{\chi^2=\omega} \frac{1}{4} \Tr(M(0) \rho(\chi,0)(f)) \\
%				 &-\sum_{\chi^2=\omega } \int_{\Gbar(\A)} f(g) \chi(\det(g)) \dop g. 
%\end{align*}
 		where $v(g)=H(g)+H(wg)$ is the weight function, with $H$ the height function and $w$ the weyl element.  The unipotent term is given by $F(a)=\int_K f(k^\inv (\tbt{1}{a}{0}{1} k ) \dop k$, and $Z_F(s)= \int_{\A^\times } F(a) |a|^s \dop^\times a$ is the tate zeta 
   \item The remaining terms are `noncuspidal.' 
   \item  Some observations regarding the congruence subgroups $\Gamma(N)$, $\Gamma_0(N)$, and $\Gamma_1(N)$: One has $\Gamma(1)/\Gamma(N) \approx \SL_2(\Z/N\Z)$,  that $\Gamma_0(N)/ \Gamma_1(N) \approx (\Z/N\Z)^\times$ and that $\Gamma_1(N)/\Gamma(N) \approx \Z/N\Z$.
   \item Set $\psi(N)=|\Gamma(1): \Gamma_0(N)|$. Then $\psi(N) = N \prod_{ p \vert N} (1+1/p)$. 
   \item For $g\in G(\R)^+$ (elts of $\GL_2(\R)$ with positive determinant), write $j(g,z)= (cz+d) \det(g)^{-1/2}$. The action of $G(\R)^+$ on $\half$ by linear fractional transformations can be written in terms of the natural action of $G(\R)^+$ on $\C^2$  via $g\colvec{z}{1} = \det(g)^{1/2} j(g,z) \colvec{gz}{1}$
   \item For $z\in  \half$, one has $\im(g z)= |j(g,z)|^{-2} \im(z)$ and $j(g'g,z)=j(g',gz)j(g,z)$. 
   \item For a dirichlet character $\omega' : (\Z/N\Z)^* \to \C^\times$, there is a unique positive integer $N_{\omega'}$  minimal among those such that $\omega'$ factors through $\Z/N_{\omega'}\Z$. Call this the conductor of $\omega'$.
   \item For a dirichlet character $\omega'$ mod $N$, extend first to $\Z/N\Z$ by $0$ off of $(\Z/N\Z)^\times$, and then to $\Z$ by $\N\Z$ invariance.
   \item Then $\omega'$ determines a character of $\Gamma_0(N)$ by $\tbt{a}{b}{c}{d} \mapsto \omega'(d)$.  
   \item $G(\R)^+$ acts on functions $h:\half \to \C$ from the right (with action dependent on a choice of $k \geq 0$: $h_\delta(z) = j(\delta ,z)^{-k}h(\delta z)$. A weight $k$ modular form is a fixed vector under the aforementioned action. Write $W_k(\Gamma)$ and $M_k(\Gamma)$ the weak modular forms and modular forms respectively. 
   \item For a congruence subgroup $\Gamma$ and a finite order character $\chi: \Gamma \to \C^\times$, set $\Gamma_\chi = \ker(\chi)$. Let $W_k(\Gamma,\chi) \subset W_k(\Gamma_\chi) $ be those $h$ satisfying $h_\gamma = \chi(\gamma)^\inv h$. 
   \item For $\delta \in G(\R)^+$ let $\chi_\delta : \delta^\inv \Gamma \delta \to \C^\times$ be defined by $\chi_\delta (\alpha)=\chi(\delta \alpha \delta^\inv)$ for $\alpha \in \delta^\inv \Gamma \delta$. Then $h\mapsto h_\delta$ is an isomorphism $W_k(\Gamma, \chi) \to W_k(\delta^\inv \Gamma \delta , \chi_\delta)$. When $\Gamma=\Gamma_0(N)$ and $\chi=\omega'$ write $W_k(N,\omega')$.
   \item In this case, $\Gamma_1(N) \subset \Gamma_{\omega'}\subset \Gamma_0(N)$. Set $M_k(N,\omega') = W_k(N,\omega')\cap M_k(\Gamma_1(N))$. Generally speaking: $W_k(\Gamma_1(N))= \bigoplus_{\omega' } W_k(N,\omega')$ with the sum over dirichlet characters $\omega'$ of conductor dividing $N$. 
   \item Let $N$ be the unipotent radical of $B$, the borel of upper triangular matrices. Then for $\Gamma\leq \Gamma(1)$ there exists a positive integer $M$ such that $\Gamma \cap N(\Z) = N(M\Z)$.
   \item If $A=\{ \delta\}$ is transversal of $\Gamma \lmod \Gamma(1)$ in $\Gamma(1)$, then $\{ \delta \infty: \delta \in A\}$ contains a set of representatives of the cusps of $\Gamma$. 
   \item Suppose $\Gamma(N) \leq \Gamma$. Then since $\Gamma(N)$ is normal in $\Gamma(1)$, one has $\tau \tbt{1}{N}{0}{1} \tau^\inv  \in \Gamma(N)\leq \Gamma$ for all $\tau \in \Gamma(1)$.
   \item Given $\delta\in G(\Q)^+$ let $s=\delta \infty$, and choose $\tau \in \Gamma(1)$ so that $\tau \infty =s$. Then $\tau^\inv \delta \infty =\infty$, so $(\delta^\inv \tau \tbt{1}{N}{0}{1} \tau^\inv \delta) \in (\delta^\inv \Gamma \delta)_\infty$ is upper triangular. In fact,  $\delta^\inv \tau \tbt{1}{N}{0}{1} \tau^\inv \delta= \tbt{1}{b}{0}{1}$ for some $b\in \Q^\times$.
   \item Ultimately, for $\Gamma$ a congruence subgroup and $\delta \in G(\Q)^+$, there exists an $M_\delta=M_\delta(\Gamma) \in \Q_{>0}$ so that $N(\Q) \cap  \delta^\inv \Gamma \delta = N(M_\delta \Z)$.
   \item If $\alpha \in M_2(\Z)$ with $\det(\alpha)=m \neq 0$, then $m\alpha^\inv \in M_2(\Z)$. 
   \item If $\{\gamma_i\}$ is a set of coset representatives for $(\alpha^\inv \Gamma \alpha \cap \Gamma ) \lmod \Gamma$, then $\Gamma \alpha \Gamma = \bigsqcup_i \Gamma \alpha \gamma_i$. 
   \item If $|\Gamma \lmod \Gamma \alpha \Gamma|= |\Gamma \alpha \Gamma \rmod \Gamma|$ then there is a common set of representatives in $\Gamma \alpha \Gamma$ for $\Gamma \lmod \Gamma \alpha \Gamma$ and $\Gamma \alpha \Gamma \rmod \Gamma$.
   \item For $g\in G(\Q)$, $g\colvec{\Z}{\Z}$ is the lattice spanned by the columns of $g$. 
   \item For $g\in G(\Q)$, then $g\in \Gamma_0^\pm (N)$ if and only if both $g\colvec{\Z}{\Z} = \colvec{\Z}{\Z}$ and $g\colvec{\Z}{N\Z} = \colvec{\Z}{N\Z}$
   \item Let $L=\colvec{\Z}{\Z}$. If $\alpha \in M_2(\Z)$ with $\alpha L \subset L$ and $\det(\alpha) \neq 0$, then $|L:\alpha L| = |\det \alpha|$.
   \item There is an involution $\cdot ^\ell$ on $\Delta_0(N)$ given by $\alpha^\ell = \tbt{1}{0}{0}{N} ^t \alpha \tbt{1}{0}{0}{N}^\inv$. This amounts to $\tbt{w}{x}{Ny}{z}^\ell = \tbt{w}{y}{Nx}{z}$. 
   \item Set $T(n) = \{ \alpha \in \Delta_0(N) : \det(\alpha) =n\}$. Then $T(n) = \bigsqcup_{ad=n, a>0 , a\vert d, \gcd(a,N)=1} \Gamma_0(N)\tbt{a}{0}{0}{d} \Gamma_0(N)$.
   \item  Suppose that $\pi_o$ and $\pi$ are representations of $G$ on $V_o$ and $V$ respectively. Let $T:V_o \to V$ be a bounded linear map. Suppose $W_o \subset V_o$ is a $G$ stable subspace, and that for each $w_o \in W_o$ the function $g\mapsto \pi(g) T \pi_o(g^\inv) w_o$ is an integral $V$ valued function on $G$ w/r/t the left Haar measure $\dop g$. Then $L:W_o \to V$ defined by $Lw_o = \int_G \pi(g) T \pi_o(g^\inv) w_o \dop g$ intertwines $\pi_o |_{W_o}$ and $\pi$. 
\end{itemize}
\section{Notes on GGP-S}
\begin{itemize}
  \item Let $G_o= \{ g_{a,b}:=\tbt{a}{b}{0}{1}: a \in K^\times, b\in K\} \leq G=\GL_2(K)$. Then every unitary irreducible representation of $G$ restricts to an irreducible representation of $G_o$.
  \item All but one unitary irreducible representation of $G_o$ is one dimensional: given by multiplicative characters on $K^\times$. The remaining unitary irreducible representation is infinite dimensional, given by $L^2(K^\times, \dop^\times x)$, and is of the form $U(g_{a,b})\varphi(x)= \chi(bx)\varphi(ax)$ where $\chi$ is a fixed nontrivial additive character of $K$. 
\end{itemize}

\section{Notes on Complex analytic stuff (Werner Basler)}

\begin{itemize}
  \item Wronski's identity: suppose $X:G \to M_n(\C)$ is a holomorphic function on $G\subset \C$ such that $X'(z)=A(z) X(z)$ for all $z\in G$. Set $w(z)=\det X(z)$ and $a(z)=\tr A(z)$. Then \[ w(z)= w(z_o) \Exp{\int_{z_o}^z a(u) \dop u} \] for any $z_o \in G$. In particular, either $w(z)=\det X(z)$ is identically zero on $G$ or is nonvanishing on $G$. 
  \item Given $A:G\to M_n(\C)$, an $X:G \to M_n(\C)$ as above is called a fundmanetal solution for $A$ if $\det X(z) \neq 0$ for some, hence all, $z \in G$.
  \item Inhomogeneous systems: $G$ simply connected region in $\C$, and $A:G \to M_n(\C)$ and $b: G \to \C^n$ holomorphic. Consider the equation $x' = A(z) x +b(z)$ for $z\in G$. Call $x'=A(z) x$ the associated homogeneous system. 
  \item Variation of constants formula: the general solution to the inhomogeneous equation in the preceding bullet can be obtained from the fundamental solution $X$ to the homogeneous problem via  \[ x(z)=X(z) \left( c+ \int_{z_o}^z X^\inv (u) b(u) \dop u \right) \]  where $z_o\in \C$ and $c\in C^n$ arbitrary. 
  \item Homogeneous systems with a first order pole at $z_o$: given a fundamental solution $X: G \to M_n(\C)$ find an $M \in M_n(\C)$ such that $X(z)=S(z) (z-z_o)^M$ with $S(z)$ holomorphic and single valued for $z$ in a punctured disk about $z_o$.
  \item The equation we consider is \[zx' = A(z)x, \quad A(z) = \sum_{n=0}^\infty A_n z^n, \quad |z|<\rho.  \] Say this equation has good spectrum if none of the eigenvalues of $A_o$ differ by a natural number. Equivalently, as $n$ runs through $\Z_{>0}$, the spectrum of $A_o + n I$ are pairwise disjoint. 
  \item When the equation has good spectrum, it has a fundamental solution of the form  \[ X(z)= S(z)z^{A_o}, \quad S(z)=\sum_{n=0}^\infty S_n z^n, \quad S_o=I, \quad |z|<\rho \] with the coefficients $S_n$ determined by the relations  \[S_n \lp A_o +n I \rp -A_o S_n = \sum_{m=0}^{n-1} A_{n-m} S_m\]. 
  \item Confluent hypergeometric systems: for $A,B \in M_n(\C)$, consider the equation  \[ zx' = \lp zA+B \rp x\]
  \item Hypergeometric systems: for $A,B \in M_n(\C)$, consider \[\lp A-zI \rp x' = Bx\]. For now, suppose $A= \diag(\lambda_1 , \cdots , \lambda_n)$ is diagonal.  By making a change of variable $z=au+b$ for some $a\neq 0$, one can achieve $\lambda_1 =0$ and $\lambda_n= 1$.
  \item Suppose $z\mapsto T(z) \in \GL_n(\C)$ is holomorphic and nonvanishing on a neighborhood of the origin in $\C$. Then $x$ is a solution to $zx'=A(z)x$ if and only if for $x=T(z)\tilde{x}$ one has $\tilde{x}$ is a solution to $z \tilde{x}' = B(z) \tilde{x}$ where $B(z)$ given by $zT'(z)=A(z)T(z)-T(z)B(z)$. 
\end{itemize}

\section{Notes on Kobayashi's Complex Hyperbolic Spaces}

\begin{itemize}
  \item For a bounded domain $X\subset \C^n$ define the Caratheodory distance $c_X(p,q)$ to be the supremum over all holomorphic $f:X \to D$ (where $D$ is a disk in $\C$)  of $\rho(f(p),f(q))$ where $\rho$ is the poincare metric on $D$. Then any holomorphic map $(X,c_X) \to (Y,c_Y)$ is distance decreasing. 
  \item Arzela-Ascoli: suppose $X$ is locally compact seperable and $Y$ is a locally compact metric space with distance $d_Y$. Then a family $\mathcal{F} \subset C(X,Y)$ is relatively compact in $C(X,Y)$ if and only if both $\mathcal{F}$ is equicontinuous at every point $x\in X$ and for ever $x\in X$, the set $\{ f(x): f\in \mathcal{F} \} \subset Y$ is relatively compact. 
  \item $D(X,Y)$ is the set of distance decreasing maps $X\to Y$. 
  \item $V$ an $n$ dimensional complex vector space, and $V^*$ its dual space. Let $F$ be nonegative real defined on a subset of $V$ such that if $F$ is defined at $v$ then it is defined at $tv$ for all $t\in \C$ and $F(tv)=|t| F(v)$.  
  \item Let $X$ be a Riemann surface and $\dop \sigma^2 = 2\lambda \dop z \dop \zbar$ be a hermitian pseudo metric on $X$ and let $\omega = i\lambda \dop z \wedge \dop \zbar$ be its associated Kahler form. 
  \item Set $\dop ^c= i(\dop '' - \dop ' )$ so that $\dop \dop ^c =2 i \dop' \dop''$. 
  \item The Ricci form associated to $\omega$ is $\Ric(\omega)= -\dop \dop^c \log \lambda = 2K\omega.$ where \[K= - \frac{1}{\lambda} \frac{\partial^2 \log \lambda }{\dop z \dop \zbar} \] is the Gaussian curvature of $\dop \sigma$. 
  \item Let $D_a$ be the open disk of radius $a$ in $\C$. The poincare metric is \[ \dop s_a^2 \frac{4a^2 \dop z \dop \zbar}{ A (a^2-|z|^2 )^2} \]. This metric is complete and has curvature $-A$. Take $a=1$ so that $A=1$ and let $\dop s^2 =\dop s_1^2$. Let $\phi$ be the Kahler form for $\dop s^2$. Then $\Ric(\phi) =-2\phi$ since $K=-1$.
  \item Generalization of Ahlfors--Pick: let $\dop \sigma^2$ be any hermitian pseudometric on $D$ with curvature bounded above by $-1$. Then $\dop \sigma^2 \leq \dop s^2$. 
  \item Let $X$ be a Riemann surface with Hermitian pseudometric $\dop s_X^2$ with curvature bounded above by $-1$. Then every holomorphic $f:D\to X$ is distance decreasing: $f^* \dop s_X^2  \leq \dop s^2$. 
  \item Let $H$ be the upper half plane. Then $w\mapsto \frac{i-w}{i+w}$ is biholomorphism $H \to D$. The pulled back poincare metric is $\dop s_H^2 = \frac{\dop w \dop \bar{w}}{v^2}$ for $w=u+iv$. 
  \item $D^*$ is the punctured disk, and $p:H \to f^*$ the covering defined by $z=p(w)=e^{2\pi i w}$. Then \[ \dop s_{D^*}^2 = \frac{4 \dop z \dop \zbar} {|z|^2 ( \log 1/|z|^2)^2} \] is a complete metric on $D^*$  of curvature $-1$. Its area element (i.e. its Kahler form) is  \[ \mu_{D^*} = \frac{i \dop z \wedge \dop \zbar} { |z|^2 (\log |z|^2)^2}.  \]
\end{itemize}

\section{Complex Analytic stuff}
\begin{itemize}
  \item A concrete Riemann surfaces is a `branched Riemann domain' $\pi:X \to \C$ or $\pi: X \to \Pone^1(\C)$.
  \item Serre duality: If $X$ is a compact Riemann surface and $V$ is a holomorphic vector bundle over $X$, then $H^1(X,V)$ and $H^0(X,K_X \otimes V^*)$ are finite dimensional vectorspaces of equal dimension. 
  \item In particular, try $V=\Of_X = X\times \C $, then $H^1(X,\Of_X) \approx H^0(X,K_X)$. 
\end{itemize}

\section{Notes on Wolpert's book}
\begin{itemize}
  \item Margulis' lemma for the hyperbolic plane: elements of a discrete torsion free group acting on the hyperbolic plane by isometries which move a base point a distance $<2$ are contained in a cyclic subgroup. 
  \item For a hyperbolic subgroup, an area $2 \ell \cot \ell/2$ collar about a geodesic of length $\ell$ will embed into the quotient.
  \item For a parabolic subgroup, an area $2$ neighborhood of a cusp will embed. 
  \item Consider the fibration $\pi:P= \{ (z,w,t)| zw=t, \; |z|,|w|,|t| <1\} \to D = \set{ t | |t|<1}$. The differential of $zw-t$ is nowhere vanishing on $\C^3$ so $P$ is a smooth complex submanifold. Since $\dop z$, $\dop w,$ and $\dop(zw-t)$ are $\C$ linearly independent, $(z,w)$ form global coordinates for $P$. Then $\pi(z,w)=zw=t$ and $\dop \pi = z \dop w+ w \dop z$, which vanishes only at the origin. The fibers of $\pi$ away from the origin in $\C^3$ make up the family of hyperbolas in $\C^2$ with coordinates $(z,w)$ which limit to the union of coordinate axes. 
  \item For $t \neq 0$, a fiber projected to $z$ axis is $\set{ |t|<|z|<1}$, or projected to the $w$ axis is $\set{ |t|<|w|<1}$. For $t=0$, the fiber $\set{ (z,0) | |z|<1} \cup \set{(0,w) | |w|<1}\subset \C^2$. 
  \item A vector $v$ in $\C^3$ is tangent to $P$ provided $v(zw-t)=0$ and is tangent to a fiber of $\pi: P\to D$ provided $ \dop \pi (v)=v(zw)=0$.
  \item The \emph{relative cotangent bundle} of a fiber is \[ \frac{\text{cotangents of }P}{\text{pullback of contangents of } D} = \frac{ \Of(\dop z) + \Of(\dop w) }{ \Of ( \pi^* \dop t)} \]
  \item Look for nonsingular change of basis $f \dop z + g \dop w = a \omega +b \dop t$ (for a suitable differential $\omega$) of $\Of(\dop z) + \Of(\dop w) $. 
  \item When $\omega = \frac{\dop z}{z} - \frac{\dop w}{w}$ the change of basis is \[ \Tbt{z/2}{-w/2}{1/w}{1/z} \colvec{f}{g} = \colvec{a}{b}. \] withe determinant $1$. 
  \item This gives a direct sum decomposition $\Of(w \dop z) +\Of( z \dop w) = \Of(\omega) \bigoplus \Of(\pi^* \dop t)$. So the relative cotangent bundle is $\Of(\omega)$. 
  \item $F$ is a fixed reference Riemann surface of genus $g$ with $n$ distinguished points. $S(F)$ is the set of conformal structures on $F$. $\Diff^+(F)$ is the group of orientation preserving diffeomorphisms of $F$ which fix the marked points. The subgroup $\Diff_o(F)$ is the normal subgroup consisting of those diffeomorphisms homotopic to the identity.  $M(F)=S(F)/\Diff^+(F)$ is the moduli space of conformal structures on $F$ with labeled distinguished points. $T(F)=S(F)/\Diff_o(F)$ is the Teichmuller space of marked such structures. There is a branched covering map $T(F) \to M(F)$ with deck group $\MCG=\Diff^+(F)$ the pure mapping class group.
  \item For a map $f$ of Riemann surfaces, given in local coordinates as $w(z)$ the complex differential of $f$ is $\partial f= w_z \dop z + w_{\zbar} \dop \zbar$. Setting $\mu = w_{\zbar} / w_z$ this is then $\partial f = w_z ( \dop z + \mu \dop \zbar )$. Then $\arg(\dop z + \mu \dop \zbar)$ is a well defined angle measure. For the deformation given by $\mu$, the direction $(\arg \mu )/2$ resp. $(\arg\mu + \pi)/2$ is the direction of maximal resp. minimal stretch. 
  \item A beltrami differential on a Riemann surface $R$ is a section of $K^\inv \otimes \overline{K}$ with $ |\mu |_\infty$ finite. Write $B(R)$ for the $C$ banach space of $L^\infty$ such sections. Write $B_1(R)$ for the open unit ball in that Banach space. 
  \item The schwarzian derivative (an infinitesimal form of the cross ratio) for a holomorphic map $h$ is \[ \{ h \}:= \lp\frac{h''}{h'}\rp'-\frac{1}{2}\lp\frac{h''}{h'}\rp^2 \] which satisfies $\{ h \circ A\} = \{ h\} \circ \lp A\cdot \lp A'\rp ^2\rp$ for all linear fractional $A$. 
  \item Here's how to define a map $\Phi:B_1(R) \to Q(\overline{R})$. Given $\mu$ lift to a tensor $\tilde{\mu}$ on $\Hbb$, the upper half plane, and extend by $0$ to $\Lbb$ the lower halfplane. The equation $w_{\zbar} = \tilde{\mu} w_z$ has a unique solution $w^{\tilde{\mu}}$ fixing $0,$ $1,$ and $\infty$. THe map $\Phi$ sends $\mu$ to $\{ w^{\tilde{\mu}} \} \vert_\Lbb \in Q(\Lbb/ \Gamma)$ where $R=\Hbb/\Gamma$.
  \item There is a canonical pairing b/w $B(R)$ and $Q(R)$ given by integration: $(\mu, \phi) \mapsto \int_R \mu \phi$. Let $Q(R)^\top \subset B(R)$ be the annihilator.
\end{itemize}
\begin{thm}
The main theorem of Ahlfors--Bers deformation theory: The quotient $T(R)= B_1(R)/\Diff_o(R)$ is a complex manifold with $\Phi: T(R) \to Q(\overline{R})$ a holomorphic embedding, with image containing the $|\cdot |_\infty $ ball of radius $1/2$ and contained in the ball of radius $3/2$. At the origin, the $\C$ tangent space of $T(R)$ is $B(R)/Q(R)^\top$ and the $\C$ cotangent space is $Q(R)$, with $( \cdot , \cdot )$ the tangent-cotangent pairing.
\end{thm}

\begin{itemize}
  \item The coset representatives for $B(R)/Q(R)^\top$ are given by harmonic beltrami differentials $:= H(R)$. These are of the form $\mu = \overline{\phi} (\dop s^2 ) ^\inv$ with $\phi \in Q(R)$ and $\dop s^2$ the $R$ hyperbolic metric.
  \item Beltrami differentials are dense in the space of square integrable sections $L^2(K^\inv \otimes \overline{K})$. 
  \item Let $\Acal$ be the concentric annulus with inner radius $\Exp{-2 \pi^2 / \log \lambda }$  (for $\lambda >1$) and outer radius $1$. Then $\Hbb$ covers $\A$ via the map $z \mapsto w = \Exp{2\pi i \log z /\log \lambda }$. The deck group of this cover is infinite cyclic generated by $z \mapsto \lambda z$
  \item Let $\Hcal$ be the horizontal strip, consisting of $\zeta$ with $0< \im \zeta < \pi$. Then $\Hcal$ is also the universal cover of $\Acal$ via the map $z \mapsto \zeta =\log z$. Its deck group is infinite cyclic, generated by $\zeta \mapsto \zeta +\ell$ where $\lambda = \Exp{\ell}$. We seek to determine the deformation of $\Acal$ under the variation $ \zeta \mapsto \zeta + \ell +\eps$. 
  \item On the left side of the fundamental domain $\{ 0 < \re(\zeta) < \ell \}$ use the coordinate $\zeta$, and on the right side use the coordinate $\zeta_1=\zeta - \ell $. Introduce the overlap $\zeta_1= p(\zeta)= \zeta+\eps$ identification. The first order variation of $p$ is the vectorfield $\partial/ \partial \zeta_1$ on the overlap of charts.
  \item Take a unit step function $\phi$ which is $0$ close to $0$ and $1$ close to $\ell$. On the fundamental domain, consider $f(\zeta)=\zeta + \eps \phi(\re(\zeta))$ and extend by periodicity. Then $f$ satisfies $f(\zeta +\ell )= f(\zeta)+ \ell +\eps$, so conjugates $\zeta \mapsto \zeta+\ell$ to $\zeta \mapsto \zeta+\ell + \eps$. The first order variation is $\phi(\re\zeta) \partial / \partial\zeta $. Note that $f$ is a smooth function, though not holomorphic. 
  \item $f$ has $\C$ derivatives $f_\zeta = 1 + \frac{\eps}{2} \phi ' ( \re \zeta) $ and $f_{\bar{\zeta}} = \frac{\eps}{2} \phi' (\re \zeta)$. So the beltrami differentials are $\mu_f = \frac{\frac{\eps}{2} \phi ' }{1+ \frac{\eps}{2} \phi'}$. Its first variation is $\dot{\mu} = \frac{\dop }{\dop \eps} \mu_f \mid_{\eps =0} = \frac{1}{2} \phi'$. 
  \item A quasi conformal map satisfies $f_{\bar{\zeta}} = \mu f_\zeta$. For a family of beltrami differentials $\mu(\eps)$, and a variation from the identity map: $f(\zeta; \eps) = \zeta +\eps f_1(\zeta)+ \dots$, substituting into the preceeding equation gives $\dot{f}_{\bar{\zeta}} = \dot{\mu}$ ( this is the Kodaira spencer story apparently). 
  \item Variation of translation length: for a horizontal strip $\Hcal$, one has \[ \frac{\dop ^n }{\dop \eps^n}\ell(\eps) = \frac{1}{\pi} \int_{F_{\ell} }  \frac{\dop ^n}{\dop \eps^n} (u_x^\eps -v_y^\eps ) \dop E= \frac{2}{\pi}\re  \int_{F_{\ell} }  \frac{\dop ^n}{\dop \eps^n} \frac{\partial}{\partial \bar{\zeta}} h^\eps \dop E \] where $\dop E$ is the Euclidean volume element, $F_\ell$ is the fundamental domain consisting of $0\leq \re \zeta \leq \ell$ and $0\leq \im \zeta \leq \pi$,  and $h^\eps = u^\eps +i v^\eps : \Hcal \to \Hcal$ is such that $h^\eps(\zeta+\ell) = h^\eps (\zeta) + \ell (\eps)$. 
  \item  For $n=1$ (so that we are looking at first variation), and substituting $z=e^{\zeta}$ we get Gardiner's formula:    \[ \dot{\ell} = \frac{2}{\pi} \re \int_{F_\ell} \dot{\mu} \dop E = \frac{2}{\pi} \re \int_{|z| =1}^{e^\ell} \int_{\arg z= 0}^\pi  \dot{\mu}(\frac{\dop z}{z})^2 \dop E \] where, formally, $\dop \ell \frac{2}{\pi }(\dop \zeta)^2 = \frac{2}{\pi} (\frac{\dop z}{z})^2 \in Q$. 
  \item The Teichmuller space of the annulus is one $\R$ dimensional, parameterized by core length. Thus, this is not quite an exemplar for the computations on the teichmuller spaces of more complicated surfaces. Note, the geodesic length of the core geodesic determines only the conformal type of the annulus. 
  \item We can extend the Teichmuller space of the annulus to one $\C$ dimensional by allowing for the consideration of maps of $\Hcal$ which are translations on-or-near the boundaries.  These are twist deformations. 
\end{itemize}

\subsection{Geodesic lengths, twists, and symplectic geometry}
\begin{itemize}
  \item Let $F$ be a smooth surface of genus $g$, and $R$ a Riemann surface diffeomorphic to $F$.  A point in $\Tcal(F)$ represents an isomorphism of $\pi_1(F)$ with the deck group $\pi_1(R)$ of the universal covering map $\Hbb \to R$. 
  \item For a free homotopy class $[\alpha]$, viewed as a conjugacy class in $\pi_1(F)$. The length of the geodesic on $R$ in the free homotopy class of $[\alpha]$ is denoted $\ell_\alpha (R)$. 
  \item Let $\Gamma  \leq \PSL(2,\R)$ be the uniformizing group for $R$. Pick a representative $\alpha \in [\alpha]$ and let $\Gamma_\alpha$ be its centralizer, arranged so that the axis of $\alpha$ is the imaginary line.
  \item Define the $\alpha$-petersson series  \[\Theta_\alpha = \frac{2}{\pi} \sum _{ \gamma \in \Gamma_\alpha \lmod \Gamma} \gamma^* \lp \frac{\dop z}{z}\rp^2\]
  \item Key point: $\Theta_\alpha  \in \Qcal(R)$ and is $\dop \ell_\alpha$ (i.e. represents a cotangent vector to the point $R$ in $T(F)$).
  \item The length-length formula: for geodesics $\alpha,\beta$ on a surface $R$ of finite type, \[ \ip{\grad \ell_\alpha}{\grad \ell_\beta} = \frac{2}{\pi} \lp \ell_\alpha \delta_{\alpha \beta} +\sum_{\Gamma_\alpha \lmod \Gamma \rmod \Gamma_\beta }' u \log{\vert\frac{u+1}{u-1}\vert } -2 \rp \] where for $C\in \Gamma$,  $u(\tilde{\alpha}, C(\tilde{\beta}))$ is either $\cos \theta$ provided $\tilde{\alpha}$ and $C(\tilde{\beta})$ intersect, and $\cosh d(\tilde{\alpha}, C(\tilde{\beta}))$ otherwise. 
  \item When $\alpha$ and $\beta$ are disjoint, the terms in the sum are all positive. It follows that the gradient for a simple geodesic is nowhere vanishing.
  \item A key computation: for $h \in \Diff/\Diff_o$, one has $\ell_\alpha \circ [h] = \ell_{h^\inv (\alpha)}$. Consequently, for any finite subgroup $G\subset \MCG$ and maximal collection of disjoint simple closed curves $\{ \alpha_j\}$, the function $\Lcal= \sum_{g\in G} \sum_{j} \ell_{h(\alpha_j)}$. Then $\Lcal$ is proper, strictly convex, and $G$ invariant (as a function on teichmuller space). Thus $\Lcal$ has a unique minimum, which must be $G$ fixed point. 
\end{itemize}

\section{Notes on Surfaces, Circles, and Solenoids (Robert Penner)}
\begin{itemize}
  \item The \textbf{lambda length} of a pair of disjoint horocycles in the upper half plane centered at $u,v\in \R$, with euclidean diameters $c,d$ respectively is $\sqrt{\frac{2}{cd}}\abs{u-v}$, which is roughly the exponential of the hyperbolic distance between those horocycles.
  \item $G=\PSL(2,\Z)$, and $\hat{G}$ is its profinite completion, $\Dbb$ is the open unit disk and $\Hcal$ is $(\Dbb \times \hat{G})/G$ where $\gamma \in G$ acts on $(z,t) \in \Dbb \times \hat{G}$ by $\gamma(z,t)=(\gamma z, t \gamma^\inv)$.
  \item Let $\ip{\cdot}{\cdot}$ denote the Minkowski form (with signature $2,1$) on $\R^3$, and \[\Hbb=\set{u \in \R^3 : \ip{u}{u}=-1 \text{ and } z>0}\] is the upper sheet. Then projection to the open unit disk at height $0$ about the origin in $\R^3$ establishes an isometry b/w $\Hbb$ and $\Dbb$.
  \item The open positive light cone $L^+=\{ u \in \R^3: \ip{u}{u}=0 \text{ and, } z>0\}$ identifies with the collection of horocycles in $\Hbb$ via `duality' $u \mapsto h(u)=\{ w \in \Hbb: \ip{w}{u}=-1\} $. Then $L^+$ identifies with the boundary $S^1$ of $\Dbb$, via the map $\Pi$ which sends a horocycle in $L^+$ to its center in $S^1$. 
  \item A \textbf{decorated geodesic} is an unordered pair $\{h_0,h_1\}$ of horocycles with distinct centers: so there is a well defined geodesic connecting the centers of $h_0$ and $h_1$.
  \item There is a well defined signed distance $\delta$ associated to the decorated geodesic $\{ h_0,h_1\}$ with sign positive if and only if the horocycles are disjoint. Define the \textbf{lambda length} of the decorated geodesic $\{h_0,h_1\}$ to be $\sqrt{2 \exp{\delta}}$
  \item Then $\lambda ( h(u_0),h(u_1))= \sqrt{-\ip{u_0}{u_1}}$ with $h$ the duality defined above (where $u_0,u_1$ are in $L^+$).
  \item Let $F=F_g^s$ be a smooth surface of genus $g$ with $s\geq 1$ punctures (with $2-2g-s<0$.) Let $G=\pi_1(F)$, and let $\Tcal(F)=\hom'(G,\PSL(2,\R))\rmod \PSL(2,\R)$ be its teichmuller space, where the prime indicates that we're looking only at those discrete faithful representations such that no element is elliptic, and loops about punctures are parabolic. 
  \item Define the \textbf{decorated teichmuller space} $\tilde{\Tcal}(F)\to \Tcal(F)$ to be the trivial $\R_{>0}^s$ bundle such that the fiber over a point is the set of all $s-$tuples of horocycles, one about each puncture, parameterized by hyperbolic length. 
  \item An  \textbf{arc family} in $F$ is an isotopy class of a family of essential arcs disjointly embedded in $F$ and connecting the punctures, w/ not two elements in the family are homotopic (relative to the punctures). If $\alpha$ is a maximal arc family, so that each component of $F-\alpha$ is a triangle, then we say that $\alpha$ is an ideal triangulation of $F$. 
  \item A theorem: Fix an ideal triangulation $\tau$ of $F$. Then the assignment of $\lambda$ lengths is a surjective homeomorphism $\Tcal(F) \to \R^\tau_{>0}$. 
\end{itemize}
	\subsection{Coordinates for the solenoid}
	\begin{itemize}
  		\item $G$ is now a finite index subgroup of $\PSL_2(\Z)$, and $M=\Dbb \rmod G$. Let $\Ccal_M$ be the category of finite sheeted orbifold covers $\pi:F \to M$, where $F$ is a punctured Riemann surface. Then $\Ccal_M$ is a directed set.
  		\item The \textbf{punctured solenoid} is $\Hcal_M = \underset{\leftarrow}\lim \; \Ccal_M$. So a point of $\Hcal_M$ is a sequence of points $y_i \in F_i$, for each cover $\pi_i:F_i \to M$, chosen compatibly with the projections.
  		\item Apparently, finite index subgroups yield homeomorphic solenoids.
  		\item $G$ has characteristic subgroups $G_N$, which for $N\in \Z_\geq 0$ consist of the intersection of all subgroups of index $\leq N$, which form a nested sequence $G_{N+1} <G_N<G_{N-1}<\cdots G_2<G_1=G$.
  		\item Define a metric $G\times G \to \R$ via $\gamma \times \delta \mapsto \min{ N^\inv: \gamma \delta^\inv \in G_N }$. Then the profinite completion of $G$ is the metric completion with respect to this metric. 
  		\item As above, form $\Hcal_G= \Dbb\times_G \hat{G}$. Then $\Hcal_G$ is homeomorphic to $\Hcal_{\PSL_2(\Z)}$ for any finite index subgroup $G$ of $\PSL_2(\Z)$. Henceforth, $G=\PSL_2(\Z)$. 
  		\item Let $\hom'(G\times \hat{G}, \PSL_2(\R))$ denote the collection of continuous functions such that:
  			\subitem{1)}  For all $\gamma_1,\gamma_2\in G$, and $t\in \hat{G}$, we have \[\rho(\gamma_1 \circ \gamma_2, t) = \rho(\gamma_1, t\gamma_2^\inv) \circ \rho(\gamma_2,t), \] 
  			\subitem{2)} For every $t\in \hat{G}$, there is a quasiconformal $\phi_t: \Dbb \to \Dbb$ which conjugates the action of $G$ on $\Dbb\times \hat{G}$ from 
	  			\begin{equation*}
  					\gamma_1:(z,t) \mapsto (\gamma z, t \gamma^\inv) 
				\end{equation*}
				to
				\begin{equation*}
  					\gamma_\rho: (z, t) \mapsto (\rho(\gamma,t) z,t\gamma^\inv).
				\end{equation*}
		\item Then we let $G_\rho= \{ \gamma_\rho: \gamma \in G\}$ and set $\Hcal=\Dbb \times_\rho \hat{G}$.
		\item Write $\Cont(\hat{G}, \PSL_2(\R))$ for group (under pointwise composition) of continuous functions $\hat{G} \to \PSL_2(\R)$. Then $\Cont(\hat{G}, \PSL_2(\R))$ acts continuously on $\hom'(G\times \hat{G}, \PSL_2(\R))$ via $(\alpha \rho ) (\gamma,t)= \alpha^\inv (t\gamma^\inv ) \circ \rho(\gamma,t) \circ \alpha(t)$. 
		\item Key structural theorem: there is a natural homeomorphism of $\Tcal(\Hcal)$ with 
			\begin{equation*}
			  \hom'(G\times \hat{G}, \PSL_2(\R))/\Cont(\hat{G}, \PSL_2(\R)).
			\end{equation*}
		\end{itemize}

\section{Notes on Will Harvey's essay on Teichmuller spaces, triangle groups, and dessins.}
	\begin{itemize}
		\item The real affine linear map $\tilde{f_\tau}:\C \to \C$ which sends $x+yi$ to $x+y\tau$ and induces the homeomorphism $f_\tau: C/\Lambda_i \to \C/\Lambda_\tau$ is \textbf{extremal} in its homotopy class (in teich's sense) in the sense that it has the least overall distortion measured by taking the supremum of the local stretching function on $X$. In this case, the local distortion is $\frac{\bar{\partial} f_\tau}{\partial f_\tau} = \frac{1+i\tau}{1-i\tau} \frac{\dop \bar{z}}{\dop z}$
		\item Teichmuller geodesic (disks): start with a Riemann surface $X_o$, and a quadratic differential form $\phi$ on $X_o$. This gives a (complex) one parameter family of deformations of $X_o$. 
		\item First description: away from the zeroes of $\phi$, write $\phi=\dop w^2$ to get a local parameter $w$ up to transition functions of the form $w \mapsto \pm w +c$. Equivalently, set $w= \int_{z_o}^z \sqrt{\phi(t)}$. Then, for each $\eps$ with $\abs{\eps}<1$, define a new structure on the underlying topological surface $S_o$ by rotating each chart through $\arg\eps$ and expanding the real foliation of $\R^2 =\C$ while contracting the imaginary foliation via the mapping $z=x+iy \mapsto w= K_\eps^{1/2} x+iK_\eps ^{-1/2} y$ where $K_\eps = (1+\abs{\eps})/(1-\abs{\eps})$. Note: if $\arg\eps=0$ so that $\eps \in (0,1)$ then the family of such structures is called the \textbf{teichmuller ray} at $X_o$ in the direction $\phi$. 
		\item Second description: write $\nu_\eps(z)=\eps \bar{\phi}(z)/\abs{\phi(z)}$. Then solve the \textbf{beltrami equation} 
			\begin{equation*}
				\frac{\partial w}{\partial \zbar} = \nu_\eps \frac{\partial w}{\partial z}.
  			\end{equation*}
  		Then conjugating $X_o$ by the one parameter family of solutions gives a holomorphic curve of deformations.
  		\item Given a $\phi \in \Omega^2(X)$, nonzero, obtain (via Teichmuller deformation) a holomorphic mapping $e_\phi$ of $\Ucal$ into $\Tcal (X)$ via the map $e_\phi(\eps) = \eps \frac{\bar{\phi}}{\abs{\phi}} \in B_1(X) \subset L_{-1,1}^\infty(X)$ for $\abs{\eps}<1$. The global metric on teichmuller space is in fact realized by the poincare measure for points in a teichmuller geodesic disk. 
  		\item Veech's examples: let $X_n$ with $n\geq 5$ and odd, be the genus $g=(n-1)/2$ hyperelliptic surface with affine equation $y^2=1-x^n$, and consider the holomorphic $1$ form $\omega= \dop x/y$ on it. 
  		\item $\omega$ has a zero of order $2g-2$ at the single point such that $x=\infty$. Then $q= \omega^2$ defines a teichmuller disk in $\Tcal_g$. 
  		\item Let $\zeta = \exp( 2\pi i /n)$ and let $T_\zeta$ be the triangle with vertices at $0,1$ and $\zeta$. Let $P=\cup_{\ell=0}^{n-1} \zeta ^\ell T_\zeta$. Then multiplication by $\zeta$ induces a cyclic group of euclidean symmetries of $P$ such that the 	quotient mapping is an $n$ fold covering of the plane near $0$. 
  		\item Repeat this construction but with $-T_\zeta$ to produce a (distinct, since $n$ is odd) polygon $Q$. Then identify pointwise in $P\cup Q$ the pairs of outside edges in corresponding triangles using a translation. 
  		\item This gives a closed surface, with local complex structure away from the corners. Filling this in conformally gives a compact riemann surface $X_\zeta$ (with nontrivial symmetry group) having a local structure over $\Pbb^1$ given by the map $p_2(z)=f\circ p_1:X_\zeta \to X_\zeta /\langle z \mapsto \pm \zeta z\rangle $, where $p_1$ is the projection $X_\zeta \to \Cbb \Pbb^1 =X_\zeta / \langle z \mapsto \pm z\rangle $   and $f : \Cbb \Pbb^1  \to \Cbb \Pbb^1$ is the mapping $z\mapsto z^n$. 
  		\item The resulting order $n$ symmetry of $X$ induced by lifting $z\mapsto \zeta z$ fixes \emph{three} points: the two center points, and the single orbit of corner points. 
  		\item Big theorem of Veech: the stabilizer in the mapping class group (of genus $g$) of the Teichmuller disk determined by the differential $q=\omega^2$ is a fuchsian triangle group $H_n = \langle \Tbt{1}{2\cot \pi/n}{0}{1}, \Tbt{\cos 2\pi/n}{-\sin 2\pi /n}{\sin 2\pi /n}{\cos 2\pi/n} \rangle = \langle \sigma , \beta\rangle $, which is isomorphic to $\langle x_1^n= x_2^2=1\rangle$ via $x_1^{g+1} x_2 =\sigma$ and $x_1=\beta$. 
  		\item Let $C_n=\left[ H_n, H_n \right] \leq H_n$ be the commutator subgroup. It is finite index and normal in $H_n$ with quotient $\Z/2 oplus \Z/n$. Harvey claims that the quotient of $\Ucal$ by $H_n$ or by $C_n$ yield ``the same'' surface $X_n$. How is this so? 
	\end{itemize}

\section{Notes on Goldman's bit in the teichmuller handbook}
A lovely lemma:
\begin{lemma}
	For $x,y\in \SL_2(\Cbb)$, the following are equivalent: 
	\begin{itemize}
		\item $x,y$ generate an irreducible representation on $\C^2$. 
		\item $\tr(xyx^\inv y^\inv ) \neq 2$ (multiplicative commutator)
		\item $\det( xy -yx ) \neq 0$ (Lie algebra commutant)
		\item The pair $(x,y)$ is not $\SL_2(\C)$ conjugate to a representation by upper triangular matrices. 
		\item Either the group $\langle x,y\rangle$ is not solvable, or there exists a splitting $\C^2 = L_1 \oplus L_2$ into an invariant pair of lines $L_i$ such that one of $x$ or $y$ interchanges $L_1$ and $L_2$.
		\item $\{ \id , x , y , xy\}$ is a $\C$ basis for $M_2(\C)$. 
	\end{itemize}
\end{lemma}

The fundamental group of the \emph{three holed sphere} $\Sigma_{0,3}$ is free on two generators, but has a \emph{redundant geometric presentation:} 
\begin{equation*}
	\pi=\pi_1(\Sig_{0,3})= \langle X,Y,Z \mid XYZ=1\rangle 
\end{equation*}
where $X,Y,Z$ correspond to the three components of $\del \Sig_{0,3}$. Denote the corresponding trace functions (on the representation variety $\hom (\pi,G)$) by the lower case letters. 
Here's a theorem: 
\begin{thm}
	The equivalence class of a flat $\SL_2(\C)$ bundle over $\Sig_{0,3}$ with irreducible holonomy is determined by the equivalence class of its restrictions to the three components of $\del \Sig_{0,3}$. Any triple of isomorphism class of flat $\SL_2(\C)$ bundles over the (disconnected, $1$ dimensional space) $\del \Sig_{0,3}$ whose holonomy traces satisfy $x^2+y^2+z^2 -xyz \neq 4$ extends to a flat $\SL_2(\C)$ bundle over $\Sigma_{0,3}$. 
\end{thm}

Every irreducible representation $\rho$ of $\pi$ in $\SL_2(\C)$ apparently corresponds to an object in $\Hbb^3$: a triple of geodesics. Any two of these geodesics admits a unique common perpendicular geodesic.  These perpendiculars cut off a hexagon bounded by geodesic segments, with all (six) right angles. 

The surface $\Sigma_{0,3}$ admits an orientation \emph{reversing} involution $\iota_{\Hex}$  whose restriction to each boundary component is a reflection. The quotient $\Sigma_{0,3}/\iota_{\Hex}$ is topologically a disk, combinatorially a hexagon. The three boundary components map to three intervals $\del_i (\Hex)$ in the boundary of $\Hex$. The other three edges in $\del \Hex$ correspond to the three arcs comprising the fix point set of $\iota_{\Hex}$. The orbifold fundamental group of $\Hex$ is $\Z/2 * \Z/2 *\Z/2$ (free product). Denote the generators by $\iota_{YZ}, \iota_{ZX},$ and $\iota_{XY}$. The covering map $\Sigma_{0,3} to \Hex$ induces the embedding of fundamentals $\pi_1(\Sigma_{0,3}) \to \pi_1{\Hex}$ defined on generators by 
\begin{align*}
	X &\mapsto	\iota_{ZX}\iota_{XY} \\
	Y &\mapsto \iota_{XY}\iota_{YZ} \\
	Z &\mapsto \iota_{YZ}\iota_{ZY}.
\end{align*}

Some observations about involutions: we're looking at projective transformations of $\C\Pbb^1$ which have order exactly two. Any such is given by a matrix $g\in \GL_2(\C)$ such that $\det(g)=1$ and $\tr(g)=0$. Let $\tilde{\Invo}=\SL_2(\C) \cap \sl_2(\C)$ denote the set of all such matrices. Since $\tilde{\Invo}$ is invariant under multiplication by $\pm \id$, we can take its quotient, call it $\Invo$. So we view $\Invo \subset \PGL_2(\C)$, and it consists precisely of the collection of involutions of $\C\Pbb^1$. It naturally identifies with the set of unordered pairs of distinct points in $\C\Pbb^1$. We identify $\tilde{\Invo}$ as the space of \emph{oriented} geodesics in $\Hbb^3$. 

Every nonidentity element $g$ of $\PGL_2(\C)$ stabilizes a unique element $\iota_g$ of $\bar{\Invo}$ (the closure of $\Invo$ in $\Pbb(\sl_2(\C))$.) If  $g$ is semisimple, then $\iota_g$ is the unique involution with the same fixed points as $g$. Otherwise, $g$ is parabolic, and $\iota_g$ corresponds to the line $\Fix(\Ad(g))=\ker(\id -\Ad(g))\subset \sl_2(\C)$, which is the Lie algebra centralizer of $g$ in $\sl_2(\C)$. 

If $g\in \SL_2(\C)$ is nonidentity semisimple, then the two lifts of $\iota_g \in \Invo$ to $\tilde{\Invo} \subset \SL_2(\C)$ differ by $\pm \id$. Let $g'= g-\frac{1}{2} \tr(g) \id$  be the \emph{traceless projection} of $g$. Then $\tr(g')=0$, $g'$ commmutes with $g$, and $\det(g')\neq 0$ (since assumed semisimple).  Choose $\delta \in \C^\times$ so that $\delta^2 =\det(g')=\frac{4 -\tr(g)^2}{4}$. Then $\delta^\inv g' \in \tilde{\Invo}$ and represents the involution $\iota_g$ centralizing $g$: \[\tilde{\iota_g}=\pm \frac{2}{\sqrt{4-\tr(g)}}\lp g-\frac{\tr(g)}{2} \id \rp\]

Elements of even order $2k$ in $\PGL(2,\C)$ correspond to elements in $\GL(2,\C)$ of order $4k$, while elements of odd order $2k+1$ in $\PGL(2,\C)$ have two lifts to $\GL(2,\C)$, one of order $2k+1$ and one of order $2(2k+1)$. 

Upper halfspace model of hyperbolic three space.  Let $\Hbb$ be hamilton's quaternions, generated by $i$ and $j$ with $i^2=j^2=-1$ and $ij+ji=0$. Inside $\Hbb$ is the subalgebra $\C$ with $\R$ basis $\{1,i\}$. Write $\Hcal^3=\set{z+uj\mid  z\in \C, u\in \R_{>0}}$ and let $g=\tbt{a}{b}{c}{d}$ act by 
		\[z+uj \mapsto (a(z+uj) +b)(c(z+uj)+d)^\inv.\]
Then $\PGL(2,\C)$ is the group of orientation preserving isometries of $\Hcal^3$ under the Poincare metric $u^{-2} (\abs{\dop z}^2 + \dop u^2)$.

For $g\in \SL(2,\C)$ noncentral, the following are equivalent:
	\begin{itemize}
		\item $g$ has two distinct eigenvalues
		\item $\tr g \neq \pm 2$,
		\item the corresponding collineation of $\C \Pbb^1$ has two fixed points 
		\item the corresponding orientation preserving isometry of $\Hcal^3$ leaves invariant a unique geodesic $\ell_g$ each of whose endpoints are fixed
		\item a unique involution $\iota_g$ centralizes $g$
	\end{itemize}
Set $\Lie(g,h)=gh-hg$ for $g,h \in \SL(2,\C)$. Then $\Lie: \SL(2,\C) \times \SL(2,\C) \to \sl(2,\C)$ and vanishes at $(g,h)$ if and only if $g$ and $h$ commute. Further, $\Lie(g,h) \in \GL(2,\C)$ if and only if $\langle g, h \rangle $ acts irreducibly on $\C^2$. 

For $g,h$ such that $\langle g,h\rangle$ acts irreducibly, let $\lambda$ be the image of $\Lie(g,h)$ in $\PSL(2,\C)$. Then, since $\tr(\Lie(g,h))=0$, the isometry $\lambda$ has order two. Further, $\tr(g \Lie(g,h))=0$ so that $g\lambda$ also has order two, so that $\lambda g \lambda^\inv =\lambda g \lambda =g^\inv$. Similarly of $h$. We find that $\lambda$ acts by reflection on the invariant axes $\ell_g,\ell_h$ of $g$ and $h$. So its fixed axis $\ell_\lambda$ is orthogonal to $\ell_g$ and $\ell_h$. To summarize:

\begin{thm}
If $g,h\in \GL(2,\C)$, then the Lie product $\Lie(g,h)=gh-hg$ represents the common orthogonal geodesic $\perp (\ell_{\Pbb(g)},\ell_{\Pbb(g)})$ to the invariant axes of $\Pbb(g)$ and $\Pbb(h)$. 	
\end{thm}

A real character $(x,y,z)\in\R^3$ corresponds to a representation of a rank two free group into one of the two \emph{real forms} $\SU(2)$ or $\SL(2,\R)$ of $\SL(2,\C)$. Geometrically, the $\SU(2)$ representations correspond to those which fix a point in $\Hcal^3$ and $\SL(2,\R)$ representations correspond to those which preserve a plane $\Hcal^2 \subset \Hcal^3$ (as well as an orientation on that plane). 

\begin{thm}
Let $(x,y,z)\in \R^3$, and $\kappa(x,y,z):= x^2+y^2+z^2 -xyz-2$. Let $\rho:\pi \to \SL(2,\C)$ be a representation with character $(x,y,z)$ such that $\kappa(x,y,z)\neq 2$. 
\begin{itemize}
	\item If $x,y,z \in [-2,2]$ and $\kappa <2$, then $\rho(\pi)$ fixes a unique point in $\Hcal^3$ and is conjugate to an $\SU(2)$ representation. 
	\item Otherwise, $\rho(\pi)$ preserves a unique plane in $\Hcal^3$ and the restriction to that plane preserves orientation. 	
\end{itemize}
	If $\kappa(x,y,z)=2$, then $\rho$ is reducible and one of the following must occur:
\begin{itemize}
	\item $\rho(\pi)$ acts identically on $\Hcal^3$, so $\rho(\pi) \subset \pm \id$ has central image. 
	\item $\rho(\pi)$ fixes a line in $\Hcal^3$ in which case $x,y,z\in [-2,2]$ and $\rho$ is conjugate to a representation taking values in $\SO(2)=\SU(2)\cap \SL(2,\R)$. 
	\item $\rho(\pi)$ acts by transvections along a unique line in $\Hcal^3$ in which case $x,y,z\in \R -(-2,2)$, and $\rho$ is conjugate to a representation taking values in $\SO(1,1)\subset \SL(2,\R)$. Note, $\SO(1,1)\approx \R^\times $ 
	\item $\rho(\pi)$ fixes a unique point on $\del_\infty \Hcal^3$. 
\end{itemize}
	
\end{thm}

	\section{Notes on Mukai's introduction to moduli and invariants}
	\begin{itemize}
		\item 
	\end{itemize}
   
\end{document}
