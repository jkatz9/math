\documentclass[11pt]{amsart}


\usepackage[margin=1in]{geometry}
	\geometry{letterpaper}
\usepackage{titlesec}
\usepackage{amssymb}
\usepackage{amsthm}
\usepackage{braket}
\usepackage[utf8]{inputenc}
\usepackage[english]{babel}




\renewcommand{\tilde}{\widetilde}
\newcommand{\hmod}{H^{1}(\modcurve)}
\newcommand{\cmod}{C^{\infty}_{b}(\modcurve)}
\newcommand{\tr}{\Tr}
\newcommand{\Ch}{\operatorname{Ch}}
\newcommand{\ind}{\operatorname{ind}}
\newcommand{\ip}[2]{\langle #1, #2 \rangle}
\newcommand{\C}{\mathbb{C}}
\newcommand{\Z}{\mathbb{Z}}
\newcommand{\R}{\mathbb{R}}
\newcommand{\Q}{\mathbb{Q}}
\newcommand{\til}[1]{\tilde{#1}}
\newcommand{\PSL}{\operatorname{PSL}}
\newcommand{\diag}{\text{diag}}
\newcommand{\half}{\mathfrak{H}}
\newcommand{\intring}{\mathcal{O}}
\newcommand{\sch}{\mathscr{S}}
\newcommand{\tor}{\mathbb{T}}
\newcommand{\bep}{\mathfrak{B}}
\newcommand{\ipd}{\ip{\cdot}{\cdot}}
\newcommand{\fredth}{\tilde{\theta}}
\newcommand{\resth}{\tilde{\theta}^{-1}}
\renewcommand{\sl}{\mathfrak{sl}}
\renewcommand{\phi}{\varphi}
\newcommand{\ltmod}{L^{2}(\modcurve)}
\renewcommand{\k}{\mathfrak{k}}
\newcommand{\halfred}{\tilde{\laphalf}}
\newcommand{\halfres}{(\lambda-\halfred)^{-1}}
\newcommand{\zpz}{\Z/p\Z}
\newcommand{\fredlapsn}{\tilde{\Delta}^{\nsphere}}
\newcommand{\ressn}{(1-\fredlapsn)^{-1}}
\newcommand{\vol}{\operatorname{vol}}
\newcommand{\ijsum}{\sum_{i < j}}
\newcommand{\poly}{\C[x_{1},\ldots,x_{n}]}
\newcommand{\rest}{\big|}
\newcommand{\dtx}{\frac{\partial^{2}}{\partial x^{2}}}
\newcommand{\dty}{\frac{\partial^{2}}{\partial y^{2}}}
\newcommand{\g}{\mathfrak{g}}
\newcommand{\dophalf}{\frac{\dop x \dop y}{y^{2}}}
\newcommand{\tbt}[4]{\left[ \begin{smallmatrix}
		#1 & #2 \\
		#3 & #4 
	\end{smallmatrix} \right] }
\newcommand{\Tbt}[4]{\left[ \begin{matrix}
		#1 & #2 \\
		#3 & #4 
	\end{matrix} \right] }
\newcommand{\Tr}{\operatorname{tr}}
\renewcommand{\r}{\mathfrak{r}}
\newcommand{\ciamod}{C^{\infty}_{a}(\modcurve)}
\newcommand{\cicmod}{C^{\infty}_{c}(\modcurve)}
\newcommand{\hamod}{H^{1}_{a}(\modcurve)}
\newcommand{\ltmoda}{L^{2}_{a}(\modcurve)}
\newcommand{\ltmodcts}{L^{2}_{\text{cts}}(\modcurve)}
\newcommand{\mel}{\mathcal{M}}
\newcommand{\ltmodcfm}{L^{2}_{\text{cfm}}(\modcurve)}
\newcommand{\res}{\operatorname{res}}
\newcommand{\re}{\operatorname{Re}}
\newcommand{\im}{\operatorname{Im}}
\newcommand{\Ad}{\operatorname{Ad}}
\newcommand{\Aut}{\operatorname{Aut}}
\renewcommand{\O}{\operatorname{O}}
\newcommand{\To}{\longrightarrow}
\newcommand{\Mapsto}{\longmapsto}
\newcommand{\inc}{\operatorname{inc}}
\newcommand{\gothic}[1]{\mathfrak{#1}}
\newcommand{\so}{\mathfrak{so}}
\newcommand{\Fund}{\mathcal{F}}
\newcommand{\partone}[1]{\frac{\partial}{\partial x_{#1}}}
\newcommand{\parttwo}[1]{\frac{\partial^{2}}{\partial x_{#1}^{2}}}
\newcommand{\PGL}{\operatorname{PGL}}
\newcommand{\F}{\mathbb{F}}
\newcommand{\ol}{\overline}
\newcommand{\inj}{\hookrightarrow}
\newcommand{\surj}{\twoheadrightarrow}
\newcommand{\trace}{\operatorname{Tr}}
\newcommand{\proj}{\operatorname{proj}}
\newcommand{\der}{\frac{d^2}{dx^2}}
\newcommand{\four}{\mathcal{F}}
\newcommand{\laphalf}{\Delta^{\half}}
\newcommand{\eps}{\varepsilon}
\newcommand{\dom}{\operatorname{dom}}
\newcommand{\id}{\operatorname{id}}
\newcommand{\Ind}{\operatorname{Ind}}
\newcommand{\Res}{\operatorname{Res}}
\newcommand{\End}{\operatorname{End}}
\newcommand{\SL}{\operatorname{SL}}
\newcommand{\GL}{\operatorname{GL}}
\newcommand{\SO}{\operatorname{SO}}
\newcommand{\Orth}{\operatorname{O}}
\newcommand{\dop}{\,{\rm d}}
\newcommand{\nsphere}{S}
\newcommand{\ltnsphere}{L^{2}(\nsphere)}
\newcommand{\honsphere}{H^{1}(\nsphere)}
\newcommand{\lthalf}{L^{2}(\half)}
\newcommand{\ltg}{L^{2}(G)}   
\newcommand{\modcurve}{\Gamma \backslash \half}
\newcommand{\Gal}{\operatorname{Gal}}
\newcommand{\ipn}[2]{\langle #1, #2 \rangle_1}
\newcommand{\Graph}{\operatorname{graph}}
\newcommand{\mhaar}[1]{\frac{\operatorname{d}#1}{#1}}
\newcommand{\rn}{\R^{n}}
\newcommand{\laprn}{\Delta^{\rn}}
\newcommand{\lapsn}{\Delta^{\nsphere}}
\newcommand{\ltrn}{L^{2}(\rn)}
\newcommand{\cirn}{C^{\infty}(\rn)}
\newcommand{\horn}{H^{1}(\rn)}
\newcommand{\inv}{{-1}}
\newcommand{\p}{\mathfrak{p}}
\renewcommand{\P}{\mathfrak{P}}
\newcommand{\frob}[1]{\operatorname{frob}(#1)}
\newcommand{\Ell}{\mathcal{L}}
\newcommand{\arccosh}{\operatorname{arccosh}}
\newcommand{\I}{\mathbb{I}}
\newcommand{\A}{\mathbb{A}}
\newcommand{\Of}{\mathcal{O}}
\newcommand{\Isom}{\operatorname{Isom}}
\newcommand{\lmod}{\backslash}
\newcommand{\rmod}{/}
\newcommand{\Com}{\operatorname{Com}}
\newcommand{\hecke}{\mathcal{H}}
\newcommand{\ord}{\operatorname{ord}}
\newcommand{\mf}{\mathfrak}
\newcommand{\q}{\textbf{q}}
\newcommand{\normset}{\mathcal{N}}
\renewcommand{\Ell}{\mathcal{L}}
\newcommand{\infl}{\operatorname{Infl}}
\newcommand{\vchar}{\operatorname{Vchar}}
\newcommand{\nspec}{\mathcal{N}}
\newcommand{\prim}{\operatorname{prim}}

\theoremstyle{definition}
\newtheorem{claim}{Claim}
\newtheorem*{question*}{Question}
\newtheorem{thm}{Theorem}
\newtheorem{prop}{Proposition}
\newtheorem{remark}{Remark}
\newtheorem*{remark*}{Remark}
\newtheorem{mydef}{Definition}
\newtheorem{fact}{Fact}
\newtheorem{lemma}{Lemma}
\newtheorem{cor}{Corollary}


 %Character abbreviations
\newcommand{\D}{\mathcal{D}}
\newcommand{\T}{\mathcal{T}}
\newcommand{\N}{\mathcal{N}}
\renewcommand{\L}{\mathcal{L}}
\newcommand{\Tb}{\overline{\mathcal{T}}}
\newcommand{\Adeles}{\mathbb{A}}
\newcommand{\Ideles}{\mathbb{I}}
\newcommand{\Order}{\mathcal{O}}
\newcommand{\kron}[2]{\left( \frac{#1}{#2}\right)}
\newcommand{\Sig}{\Sigma}
\newcommand{\sig}{\sigma}


%Typography Shortcuts
\newcommand{\lp}{\left(}
\newcommand{\rp}{\right)}
\newcommand{\abs}[1]{\lvert #1 \rvert}
\newcommand{\nrm}[1]{\lVert #1 \rVert}

%Operatornames
\newcommand{\Ord}{\operatorname{Ord}}
\newcommand{\Ric}{\operatorname{Ric}}
\newcommand{\stab}{\operatorname{stab}}
\newcommand{\fp}{\operatorname{f.p.}}
\newcommand{\Ann}{\operatorname{Ann}}
\newcommand{\Exp}[1]{\operatorname{exp} \left( #1 \right)}
\newcommand{\Rep}{\operatorname{Rep}}
\newcommand{\Spec}{\operatorname{Spec}}
\newcommand{\Hom}{\operatorname{Hom}}
\newcommand{\Sym}{\operatorname{Sym}}
\newcommand{\image}{\operatorname{Im}}
\newcommand{\Diff}{\operatorname{Diff}}
\newcommand{\MCG}{\operatorname{MCG}}
\newcommand{\grad}{\operatorname{grad}}
\newcommand{\Cont}{\operatorname{Cont}}
\newcommand{\del}{\partial}
\newcommand{\Hex}{\operatorname{Hex}}
\newcommand{\Invo}{\operatorname{Inv}}
\newcommand{\Fix}{\operatorname{Fix}}
\newcommand{\Lie}{\operatorname{Lie}}
\newcommand{\SU}{\operatorname{SU}}

%\setlength{\parindent}{0em}
%\setlength{\parskip}{1em}


%\setcounter{secnumdepth}{4}
%\setcounter{secnumdepth}{4}

\titleformat{\section}{\normalfont\normalsize\bfseries}{\thesection}{1em}{}%%
\titlespacing*{\section}{0pt}{3.25ex plus 1ex minus .2ex}{1.5ex plus .2ex}
\titleformat{\subsection}{\normalfont\normalsize\bfseries}{\thesubsection}{1em}{}
\titlespacing*{\subsection}{0pt}{3.25ex plus 1ex minus .2ex}{1.5ex plus .2ex}


\newcommand{\CP}{\Cbb \Pbb}
\newcommand{\CPone}{\Cbb \Pbb^1}

\title{Slightly more refined ROKs}
\author{Justin Katz}
\begin{document}
\maketitle
These are notes on the article \cite{girondoFieldsDefinitionUniform2014} from the journal issue
\cite{izquierdoRiemannKleinSurfaces2015}.

\begin{itemize}
  \item A \emph{dessin} is a hypermap on a compact oriented two-manifold. Equivalently, they are bipartite graphs embedded on a surfaces, with simply connected complementary regions.
  \item Dessins arise on all smooth complex projective curves defined over a numberfield. Belyi demonstrated that there exist nonconstant meromorphic functions $\beta: C\to \CP^1$ on a smooth curve $C$ ramified at most over $0,1,\infty \in \CP^1$ if and only if $C$ can be defined over a numberfield. The dessin arises as the preimage of the interval $[0,1]$ under $\beta$.
  \item Conversely any topological hypermap on a compact surface admits a unique conformal structure,   corresponding to an algebraic curve defined over a number field, together with a belyi function producing that hypermap as a dessin.
  \item Belyi functions on a curve $C$ correspond to inclusions of fuchsian groups $\Gamma \leq \Delta(r,s,t)$ (with finite index) and $\Delta=\Delta(r,s,t)$ is a fuchsian triangle group, such that $\Gamma \lmod \Hbb$ identifies with $C(\C)$. Identifying\footnote{In what sense are we making this identification? Surely, we are doing so as topological spaces, but what more?} the quotient $\Delta \lmod \Hbb$ with the Riemann sphere $\CPone$ (arranging the ramification points to occur at $0,1,\infty$), the belyi function $C \to \CPone$ identifies with  the canonical quotient map $\Gamma \lmod \Hbb \to \Delta \lmod \Hbb$ induced by the inclusion $\Gamma \to \Delta$.
  \item A \emph{regular} dessin is one for which the group of color--and--orientation preserving automorphisms acts transitively on the edges. The surfaces of genus $g>1$ admitting regular dessins are called \emph{quasi-platonic}. Such dessins on such surfaces are distinguished as arising from those $\Gamma$ which are \emph{normal} subgroups of the relevant triangle group.
  \item A \emph{uniform} dessin is characterized by $\Gamma < \Delta$ being a \emph{torsion free} finite index subgroup, though need not necessarily be normal.
  \item A (tantalizing) theorem in \cite{girondoShimuraCurvesMany2012} seems to characterize those surfaces admitting several different uniform dessins \emph{of the same signature} as those arising from those \emph{contained in}\footnote{Note: contained \emph{in} rather than \emph{containing}} a congruence subgroup of an arithmetic triangle group.
\end{itemize}

\section*{Background on fields of moduli/fields of definition}
Here, $S$ is a smooth algebraic curve\footnote{over $\C$?} and $k \subset \C$ is a field. Say that $k$ is \emph{a} \textbf{field of definition} for $S$ if there exist $J$ homogeneous polynomials $F_j\in k[x_0, \cdots, x_n]$ such that $S(\C)$ and
		\[S_F= \set{x\in \CP^n \mid F_j(x) \; \text{ for all j}}\]
are isomorphic\footnote{complex analytically?}.

Given a set $F$ of homogeneous generators for the ideal defining  $S$, and an element $\sigma \in  \Gal(\Qbar / \Qbb)$,  one can define $S_{ F^\sigma}$ as above, with coefficients of the polynomials conjugated by $\sigma$. The \textbf{inertia subgroup}  $I_S \leq \Gal ( \Qbar / \Qbb)$  consists of those $\sigma \in \Gal(\Qbar/\Qbb)$ such that $S_{F^\sigma} \approx S_F$  as\footnote{presumably} complex analytic varieties. Apparently this subgroup is independent (perhaps up to conjugation?) of the choice of model.  The fixed field $\Qbbar^{I_S}= \Fix(I_S)$ is the \textbf{field of moduli} for $S$, which we denote by $M(S)$.  The field of moduli is contained in any field of definition, though not every curve is defined over its field of moduli. That is, $M(S)$ may not be a field of definition for $S$.

If $G<\Aut(S)$ say that $k$ is a field of definition for the pair $(S,G)$ if there exists a model $S_F$ of $S$ over $k$ and an isomorphism $\phi:S_F \to S$ such that $\phi^\inv G \phi < \Aut(S_F)$ is also defined over $k$. Then the inertia group of the pair is
\[ I_{(S,G)} =  \set{\sigma \in \Gal(\Qbbar/\Q) \mid \parbox[t]{.5\linewidth}{ there exists an isomorphism $f_\sigma: S\to S^\sigma$ such that $\alpha^\sigma \circ f_\sigma = f_\sigma \circ \alpha$ for all $\alpha \in G$}}. \]
Write $M(S,G)$ for the field of moduli for the pair.

Now, given a belyi function $\beta:S\to \CPone$ defining a dessin $\Dcal$ on $S$, define the group of automorphisms of $(S,\beta)$ to be the subgroup of $\Aut(S)$ such that $\beta\circ f= \beta$ (equality of functions $S\to \CPone$). We also write $\Aut(\Dcal)$ for $\Aut(S,\beta)$.

For any model $S_F$ as above, $\beta \circ \phi$ is a rational function on $S_F$. Say that $k$ is a field of definition of $(S,\beta)$ (or $\Dcal$) if there exists a model $S_F$ of $S$ such that both $F$ and the covering $\beta \circ \phi$ are defined over $k$. Define the inertia subgroup similarly to above and define $M(\Dcal)=M(S,\beta)$ to be its fixed field: the field of moduli for $\Dcal$. Note $I_(S,\beta) < I_S$ so $M(S)< M(S,\beta)$

Say a belyi function is regular if it defines a normal covering $S\to S/G\approx \CPone$, for some subgroup $G$ of $\Aut(S)$.

\begin{lemma}
	If $S$ is quasiplatonic with surface group $\Gamma \triangleleft \Delta$ for a maximal triangle group $\Delta$, then $M(S)=M(S,\beta)$ for the belyi function $\beta:\Gamma \lmod \Hbb \to \Delta \lmod \Hbb$.
\end{lemma}

If $\Delta$ is not maximal, (say, for example, $r=s \neq t$) then there exist a $\sigma \in I_S$ fixing $S$ but exchanging the zero set of its belyi function $B$ and the zero set of $1-B$ such that $M(S,B)$ (can be) a quadratic extension of $M(S)$.

\begin{thm}
	Let $S$ be a quasiplatonic curve of genus $g>1$ with full automorphism group $G=\Aut(S)$. If $Z(G)\leq G$ is a direct factor off $G$: i.e. $G=G'\times Z(G)$, then $(S,G)$ can be defined over its field of moduli $M(S,G)$.
\end{thm}

Note: this theorem applies to $\PGL_2$ and $\PSL_2$ and direct products of them with $C_2$'s. A consequence: if $U\leq G$ and $C=U\lmod S$ (a cover of $G\lmod S$, covered by $S$), then $C$ can be defined over $M(S,G)$. Further, all such $S$ can be simultaneously defined over $M(S,G)$ in the sense that all the function fields $M(S,G)(C)$ are subfields of $M(S,G)(S)$.

\section*{Hurwitz groups and surfaces}
Say $G$ is a hurwitz group if it is a quotient of $\Delta(2,3,7)$ by a finite index torsion free normal subgroup. A theorem of Macbeath characterizes those $q$ for which $\PSL_2(\Fbb_q)$ is a hurwitz group. The cases are as follows:
\begin{itemize}
	\item $q=7$,
	\item $q=p$ for $p$ prime $\pm 1$ mod $7$.
	\item $q=p^3$ for $p$ prime and $p= \pm 2$ or $\pm 3$ mod $7$.
\end{itemize}

See \cite{dzambicMacbeathsInfiniteSeries2007} for details on the following arithmetic construction of Macbeath's curves. Given a number field $k$ let $\Ocal_k$ be its ring of integers. Let $A$ be the indefinite quaternion algebra defined over $k=\Q(\cos\pi/7)$ unramified at all finite places. Then $\Delta(2,3,7)$ is the image of the norm $1$ subgroup of $A$ under canonical inclusion $A \to \GL(2,\R)$ induced by the unique split real place, followed by the surjection $\GL(2,\R) \to \PGL(2,\R)$. Further, $\Delta$ can be realized as a subgroup of $\PSL(2,\Ocal_L)$ for an (at worst) quadratic extension of $k$.

For a rational prime $p \in \Z$, the ideal $p\Ocal_k$ is:
	\begin{itemize}
		\item ramified if (and only if) $p=7$; in this case $p\Ocal_k=\pfrak^3$ for some prime $\pfrak \leq \Ocal_k$ with $N(\pfrak)=7$,
		\item split if (and only if) $p=\pm 1$ mod $7$; in this case $p\Ocal_k=\pfrak_1 \pfrak_2 \pfrak_2$ with primes $\pfrak_i \leq \Ocal_k$ of norm $p$
		\item inert if(and only if) $p=\pm 2$ or $\pm 3$ mod $7$; in this case $p\Ocal_k=\pfrak$ remains prime, and has norm $p^3$.
	\end{itemize}

Now let $\Delta=\Delta(r,s,t)$ be any \emph{arithmetic} triangle group, realized as the norm $1$ units $\Mcal^1$ of a maximal order $\Mcal$ of a quaternion algebra $A$ over a totally real field\footnote{apparently, as first observed by Takeuchi, all of the arithmetic triangle groups have trace fields with class number one. Consequently, for all such $k$, every ideal $\pfrak=\pi \Ocal_k$ for some prime element $\pi$ of $\Ocal_k$.} $k$. Let $A_\pfrak$ be the local quaternion algebra defined over the $\pfrak$-adic field $k_\pfrak$.

So long as $\pfrak$ doesnt divide the discriminant of $A$, the local quaternion alg $A_\pfrak$ is isomorphic to $M_2(k_\pfrak)$. Let $\Delta(\pfrak)$ denote the principal congruence subgroup of level $\pfrak$. Then all of the surfaces $\Delta(\pfrak) \lmod \Hbb$ have a regular belyi function $\beta: S=\Delta(\pfrak) \lmod \Hbb \to \Delta \lmod \Hbb$ and has automorphism group of order $\abs{Aut(S)}=\abs{N(\Delta(\pfrak))/\Delta(\pfrak)}$ where $N$ denotes the $\PSL(2,\R)$ normalizer of $\Delta(\pfrak)$.

The principal congruence subgroups of level $\pfrak^n$ of the local quaternion algebra correspond to the intersections of certain collections of maximal orders. More precisely, the principal congruence subgroup of level $\pfrak^n$ corresponds to (norm one units of) the intersection of all of the vertices in the Bruhat-Tits tree of distance $\leq n$ from the root.

Let $\Ecal_\pfrak$ denote the Eichler order of level $\pfrak$ in $A_\pfrak$: that is, the intersection of the two maximal orders $\Mcal_\pfrak$ and $\gamma^\inv \Mcal_\Pfrak \gamma $, where $\gamma=\tbt{\pi}{0}{0}{1}$. Let $\Phi_0(\pfrak)=\Ecal_\pfrak^1$ denote the norm $1$ subgroup of $\Ecal_\pfrak$. The \textbf{fricke element} $\tbt{0}{\pi ^\inv}{-1}{0} \in A_\pfrak$ conjugates one maximal order into the other, and so preserves their intersection $\Ecal_\pfrak$, and induces an involution on $\Phi_0(\pfrak)$. Globally, the fricke involution may give rise to a matrix in $\GL(2,\R)$ which conjugates two triangle groups. By the rigidity of triangle groups, such a conjugation can be realized inside $\PSL(2,\R)$, hence gives rise to an involution $\alpha_\pfrak$ which generates a $C_2$ extension $\Delta_{f}(\pfrak)= \langle\alpha_\pfrak, \Delta_0(\pfrak) \rangle$ of $\Delta_0(\pfrak)$, called the Fricke extension.

Now $\alpha_\pfrak$ normalizes $\Delta_0(\pfrak)$ but \emph{not} $\Delta$, though $\alpha_\pfrak^2 \in \Delta_0(\pfrak)$. Conjugation by $\alpha_\pfrak$ induces an involution on the curve $\Delta_0(\pfrak)\lmod \Hbb$. Note: $\alpha_\pfrak \in \PSL(2,\R)$ may not have order two, despite it inducing an involution on the curve $\Delta_0(\pfrak) \lmod \Hbb$.

Consequently, any group $K\leq \Delta_0(\pfrak)$ is a subgroup of both $\Delta$ and $\alpha_\pfrak \Delta \alpha_\pfrak^\inv$, and consequently possess two distinct uniform dessins on $K\lmod \Hbb$.

The discussion also applies verbatim for prime powers: we obtain subgroups $\Delta_0(\pfrak^j) \leq \Delta_0(\pfrak)$, involutions $\alpha_{\pfrak^j}$, and extensions $\Delta_{fr}(\pfrak^j)=\langle \Delta_0(\pfrak^j),\alpha_{\pfrak^j}\rangle $. Now, $\Delta_{0}(\pfrak^j)$ is normal in $\Delta_{fr}(\pfrak^j)$, but none of $\Delta_0(\pfrak^\ell)$ is normal in $\Delta_{fr}(\pfrak^j)$ for $j<\ell$. Nor are any of its $\Delta$ conjugates $\Delta_0^i(\pfrak^\ell)$.

\begin{lemma}
	\begin{itemize}
		\item For each $j=1,\cdots,n$ there are $q^{j-1}(q+1)$ congruence subgroups $\Delta_0^i(\pfrak^j)$, each conjugate to $\Delta_0(\pfrak^j)$ (in $\Delta$) for $i=0,\cdots,q^{j-1}(q+1)-1$.
		\item  Each of them is contained in $\Delta$, and in $j$ different triangle groups conjugate to $\Delta$, \textbf{in which $\Delta(\pfrak^n)$ is included non-normally}.
		\item Every $\Delta_0^i(\pfrak^j)$ is the intersection of $\Delta$ with a conjugate triangle group $\Delta^{j,i}$, and for fixed $j$, the different $\Delta^{i,j}$ form an orbit under $\Delta$ conjugation.
	\end{itemize}
\end{lemma}
	\begin{proof}
		Proceed by induction on $j$. The group $\Delta_0(\pfrak)$ has index $q+1$ in $\Delta$, and for each class $\rho_i \in \Delta$ modulo $\Delta_0(\pfrak)$ for $i=0,\cdots, q$ set $\Delta_0^i(\pfrak):= \rho_i  \Delta_0(\pfrak) \rho_i^\inv$ for $\rho_0=\id, \cdots, \rho_q$ such that $\Delta(\pfrak^n) \triangleleft \Delta_0^i(\pfrak) < \Delta$.

		For each of them, let $\alpha_i:= \rho_i \alpha_\pfrak \rho_i^\inv$ be its fricke involution (where $\alpha_\pfrak$ is the one for $\Delta_0(\pfrak)$). Then form the corresponding Fricke extension $\Delta_{fr}^i(\pfrak)$ which is properly contained in $\Delta$.

		The conjugacy of the different $\Delta_0^i(\pfrak^j)$ for fixed $j$ comes from the following interpretation: consider the \textbf{fake projective line} $\Pbb^1(\Ocal_k/\pfrak^j)$ consisting of pairs $(x,y) \in (\Ocal_k / \pfrak^j)^2$ not both coordinates divisible by $\pfrak$, modulo the diagonal action of $(\Ocal_k / \pfrak^j)^\times $. The action of $\Delta$ on this fake projective line is transitive, and the subgroups $\Delta_0^i(\pfrak^j)$ are the point stabilizers.
	\end{proof}



\bibliographystyle{plain}
\bibliography{BigBib}

\end{document}
