\begin{lemma}\label{lemma:MangosteenPicosdeEuropa}
    Let $A$ be a quaternion algebra over a number field $k$ with ring of integers $R$, and let $O$ be a maximal order in $A$.
    Let $\pfrak$ be a prime ideal of $R$ over which $A$ is unramified.
    Suppose $\Lambda$ be a subgroup of $O^1$ satisfying
    \begin{align}\label{eq:trace}
        \tr g \equiv 2 \mod \pfrak^{2n}, \quad \text{for all $g \in \Lambda$}.
    \end{align}
    then there exists an $\alpha \in A^\times$ such that $\alpha \Lambda \alpha^\inv \subset O^1(\pfrak^n)$.
\end{lemma}
\begin{proof}
    % For a positive integer $n$, recall that $\pi_{\pfrak^n}: O^1 \to O^1[\pfrak^n]=O^1/O^1(\pfrak^n)$ is the reduction map.

    First, we suppose that $A$ is unramified at $\pfrak$.
    Thus
    \begin{align}\label{eq:split}
        O^1 \otimes R_\pfrak \approx SL(2,R_\pfrak)
    \end{align}, and $O^1[\pfrak^n] \approx \SL(2,R/\pfrak^n)$. Fixing an isomorphism in as \ref{eq:split}, we identify $O^1$ with a group of matrices $g = \tbt{a}{b}{c}{d}$ with entries $a,b,c,d \in \R_k$.
    % Picking a uniformizer $\varpi \in R_\pfrak$ for $\pfrak$, each $g\in O^1$ admits a unique decomposition
    % \begin{align}\label{eq:decomposition}
    %     % g                & = g_0 + g_1 \varpi + g_2 \varpi^2 + \dots                                                                \\
    %     \Tbt{a}{b}{c}{d} & =  \Tbt{a_0}{b_0}{c_0}{d_0} +\Tbt{a_1}{b_1}{c_1}{d_1} \varpi + \Tbt{a_2}{b_2}{c_2}{d_2} \varpi^2 + \dots
    % \end{align}
    % $O^1[\pfrak] \approx \SL(2,\kfrak_\pfrak)$
    % for matrices $g_n \in M(2,R_\pfrak)$ at least one of each is not in $\pfrak$.

    We will argue by induction. For the base case, suppose $n=1$ so that $\tr(g) \equiv 2 \mod \pfrak^2$ for all $g \in \Lambda$.


    Write $\Lambda[\pfrak] = \pi_{\pfrak}(\Lambda)\leq \SL(2,\kfrak_\pfrak)$. Now, as $\tr g \equiv 2 \mod \pfrak$ for all $g \in \Lambda$, it follows that each $h \in \Lambda[\pfrak]$ is unipotent.
    Indeed, by Cayley-Hamilton (in $M(2,\kfrak_\pfrak)$) each element $h$ satisfies its characteristic polynomial $p(x;h)=x^2 - \tr h x + 1= (x-1)^2$.
    It follows that $\Lambda[\pfrak]$ is conjugate in $\SL(2,\kfrak_\pfrak)$ to a subgroup of the group $U(\kfrak_\pfrak)$ of upper triangular unipotent matrices. As $O^1 \to O^1[\pfrak]\approx \SL(2,\kfrak_\pfrak)$ is surjective, we may pull back any such conjugation to one in $O^1$.

    Thus, without any loss of generality, we may assume that each $g$ in $\Lambda$ may be written as
    \begin{align*}
        g = \Tbt{1}{x}{0}{1} + \varpi \gamma
    \end{align*}
    for some $\gamma = \tbt{a}{b}{c}{d} \in M(2, R_\pfrak)$.

    Since $\Lambda \leq O^1$, one has $\det g = 1 $ for all $g\in \Lambda$, thus
    \begin{align*}
        1 & = \det g                                     \\
          & = 1 + (a+d - cx ) \varpi + (ad -bc) \varpi^2
    \end{align*}
    so that
    \begin{align}\label{eq:MandarinLeyden}
        (a+d - cx) \varpi + (ad-bc) \varpi^2 =0.
    \end{align}
    By hypothesis (\ref{eq:trace}), we have $(a+d)\varpi = \tr g -2 \equiv 0 \mod \pfrak^2$ so that $a+d \in \pfrak$. Consequently, $xc \in \pfrak$. As $\pfrak$ is prime, it follows that at least one of $x,c$ lies in $\pfrak$. If $x \in \pfrak$, then $g \in O^1(\pfrak)$. If this is so for all $g \in \Lambda$ then the claim is proven. Supposing otherwise, there exists a $ g = \Tbt{1}{x}{0}{1} + \varpi \tbt{a}{b}{c}{d}$ for which $x \in R_\pfrak^\times$. In this case one has $c \in \pfrak$. Let $\alpha_o = \Tbt{\varpi}{0}{0}{1}$. Then
    \begin{align*}
        \alpha_o g\alpha_o^\inv  = \Tbt{1}{\varpi x}{0}{1} + \varpi \tbt{a}{\varpi b}{\varpi^\inv c}{d}
    \end{align*}
    lies in $O^1(\pfrak)$.

    To complete the proof of the base case, we must prove that there exists an element $\alpha \in A^\times $ such that $\alpha g \alpha^\inv =  \alpha_o g \alpha_o^\inv$. Note that multiplying $\alpha_o$ on the left by an element of $\GL(2,R_\pfrak)$ preserves the condition that $\alpha_o g \alpha_o^\inv \in O^1(\pfrak)$, so it suffices to prove that the intersection $\alpha_o \GL(2,R_\pfrak) \cap A^\times $ is nonempty. This, in turn, follows from the fact that $A^\times$ is dense in $\GL(2,k_\pfrak)$ and that the coset  $\alpha_o \GL(2,R_\pfrak)$ is open in $\GL(2,k_\pfrak)$.

    Before we proceed, we need the following lemmata:
    \begin{lemma}\label{lemma:iso}
        Suppose $n\geq 2$. Then there is an isomorphism of $O^1(\pfrak^{n-1}) / O^1(\pfrak^{n})$ with the underlying additive group of $\sl(2,\kfrak_\pfrak)$ which intertwines the conjugation action of $O^1$ on the former with the adjoint action of $O^1$ on the latter.
    \end{lemma}
    \begin{proof}
        The isomorphism is given by composing the map $g = 1 + \varpi^{n-1}\gamma \mapsto \gamma$  of $O^1(\pfrak^{n-1}) \to M(2,R_\pfrak)$ with reduction modulo $\pfrak$.
    \end{proof}

    \begin{lemma}\label{lemma:killingform}
        Let $\Ncal$ be the quadric in $\Pbb(\sl(2,\kfrak_\pfrak))\approx \Pbb^2(\kfrak_\pfrak)$, cut out by the equation $det X =0$. Then $\Ncal$ is a rational normal curve of degree $2$ in $\Pbb^2(\kfrak_\pfrak)$. The action of $\Ad(\SL(2,\kfrak_\pfrak))$ on $\Ncal$ is transitive.

        Consequently, any additive subgroup of $\sl(2,\kfrak_\pfrak)$ consisting of elements satisfying $\det X = 0$ is conjugate via $\SL(2,\kfrak_\pfrak)$ to a subgroup of the form $\Tbt{0}{\kfrak_\pfrak}{0}{0}$.
    \end{lemma}
    Now suppose that $\tr g \equiv 2 \mod \pfrak^{2n}$ for every $g\in \Lambda$. By inductive hypothesis, after conjugating by an element of $A^\times$ if necessary, we may assume $\Lambda \leq O^1(\pfrak^{n-1})$.

    Thus, each element $g\in \Lambda$ can be written as
    \begin{align}\label{eq:AppleWensleydale}
        g = \id +\varpi^{n-1} \gamma
    \end{align}
    for some $\gamma =\tbt{a}{b}{c}{d} \in M(2, R_\pfrak)$, and the assignment $g \mapsto \gamma = \tbt{a}{b}{c}{d} \mod \pfrak$ identifies $\Lambda[n] = \Lambda /\Lambda \cap O^1(\pfrak^n)$ with an additive subgroup of $\sl(2,\kfrak_\pfrak)$.

    Note that since $\tr g \equiv 2 \mod \pfrak^{2n}$, one has $\tr(\gamma) \equiv 0 \mod \pfrak^{n+1}$.
    Computing determinants as in \ref{eq:AppleWensleydale},
    \begin{align*}
        1 & = \det g                                                  \\
          & = 1 + \varpi^{n-1} \tr \gamma + \varpi^{2n-2}\det \gamma,
    \end{align*}
    and since $\det g =1$, we find $\tr \gamma +\varpi^{n-1}\det \gamma =0$. From $\tr \gamma  \equiv 0 \mod \pfrak^{n+1}$ we find $ \varpi^{n-1}\det \gamma \equiv 0 \mod \pfrak^{n+1}$, so that $\det \gamma \equiv 0 \mod \pfrak^2$.





    % To proceed, we need the following lemmata
    % \begin{lemma}[Lemma 4.1 in \cite{leiningerLengthEigenvalueEquivalence2007}]
    % 	Suppose $n \geq 1$. There is an exact sequence
    % 	\begin{align*}
    % 		1 \to \sl(2,R_\pfrak/\pfrak^n) \to \SL(2,R_\pfrak/ \pfrak^{2n}) \to \SL(2,R_\pfrak/\pfrak^n) \to 1
    % 	\end{align*}
    % 	The conjugation action of $\SL(2,R_\pfrak/\pfrak^n)$ on $\sl(2,R_\pfrak/\pfrak^n)$ induced by this sequence is the adjoint action.
    % \end{lemma}
    % \begin{proof}
    % 	The inclusion $R_\pfrak/\pfrak^n \to R_\pfrak / \pfrak^{2n}$ by $a \mapsto \varpi^n a$ induces an inclusion of $M(2, R_\pfrak/\pfrak^n) \to M(2,R_\pfrak/\pfrak^{2n})$ by $X \mapsto \varpi^n X$.

    % 	Reduction modulo $\pfrak^n$ induces a surjection $\pi : \SL(2,R_\pfrak/\pfrak^{2n}) \to \SL(2,R_\pfrak/\pfrak^n)$ which has kernel
    % 	\begin{equation}\label{eq:kerpi}
    % 		\ker \pi = \{ \id + \varpi^n X : \det(\id + \varpi^n X)=1\}.
    % 	\end{equation}
    % 	Compute, for $\id + \varpi^n X \in \ker \pi$:
    % 	\begin{equation*}\det(\id + \varpi^n X) = 1 + \varpi^n \tr X +\varpi^{2n} \det X =1 + \varpi^n \tr X \in R_\pfrak / \pfrak^{2n}\end{equation*}
    % 	so that $\tr X = 0 \in R_\pfrak/ \pfrak^n$, which is to say $X \in \sl(2,R_\pfrak/\pfrak^n)$.

    % 	Last, observe that the map $X \mapsto 1 + \varpi^n X$ is indeed a homomorphism $\sl(2,R_\pfrak/\pfrak^n) \to \SL(2,R_\pfrak / \pfrak^{2n})$: for  $X,Y \in \sl(2,R/\pfrak^n)$ compute
    % 	\begin{align*}
    % 		(1+\varpi^n X)(1+\varpi^n Y) = 1 + \varpi^n (X+Y) + \varpi^{2n}(XY) = 1 + \varpi^n (X+Y)
    % 	\end{align*}
    % 	in $\SL(2,R_\pfrak/\pfrak^{2n})$, as claimed.
    % \end{proof}
    \newpage
    % \begin{lemma}
    % 	For each $n>1$, and each prime $\pfrak$ over which $A$ is unramified, the map
    % 	\begin{align}\label{eq:liealgebra}
    % 		g \mapsto (g - 1)/\varpi^n
    % 	\end{align}
    % 	induces an isomorphism $O^1(\pfrak^n)/O^1(\pfrak^{n+1}) \approx \sl(2,\kfrak_\pfrak)$. The action of $O^1$ on $O^1(\pfrak^n)/O^(\pfrak^{n+1})$ by conjugation factors through the quotient $O^1[\pfrak] \approx \SL(2,\kfrak_\pfrak)$, and the isomorphism induced by \ref{eq:liealgebra} intertwines the conjugation action with the adjoint action.
    % \end{lemma}

    % \begin{lemma}
    % 	The quadratic form $X \mapsto \det x$ on $\sl(2,\kfrak_\pfrak)$ is invariant under the adjoint action of $\SL(2,\kfrak_\pfrak)$. Let $\Ncal$ denote the
    % \end{lemma}


    % If each $g\in \Lambda$ is such that $\gamma \in \pfrak M(2,\R_\pfrak)$, then the claim is proven. Supposing otherwise. 

    % To this end, we apply the following form of strong approximation:

    % \begin{proposition}[Strong Approximation Theorem \cite{maclachlanArithmeticHyperbolic3Manifolds2003}]
    % 	Let $A$ be a quaternion algebra over a number field $k$, and let $S$ be a finite set of places of $k$ such that $S\cap V_k^\infty \neq \emptyset$ and, for at least one $v_o \in S$, $v_o \notin \Ram(A)$. Then $A^1_k A^1_S $ is dense in $A^1_{\Abb_k}$.
    % \end{proposition}












\end{proof}

\paragraph{finite computations}
In this section $\kfrak$ is a finite field. Set
\begin{align*}
    \SL(2,\kfrak) = \{ x \in M(2,\kfrak) : \det x =1 \}
\end{align*}
and
\begin{align*}
    \sl(2,\kfrak) = \{ x \in M(2,\kfrak) : \tr x = 0 \}.
\end{align*}
The adjoint representation is
\begin{align*}
    \Ad : \SL(2,\kfrak) \to \Aut(\sl(2,\kfrak)) \\
    \Ad(g):  X \mapsto  g X g^\inv.
\end{align*}


% \begin{align*}
% 	\tilde{G} = \GL(2,k), \quad G = \SL(2,k), \quad \bar{G}  = \PGL(2,k)
% \end{align*}
\cite{leiningerLengthEigenvalueEquivalence2007}
\cite{voightQuaternionAlgebras2021}
