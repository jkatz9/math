% !TeX root = main.tex
Throughout $k$ will denote a number field, and $R$ its ring of integers. We write $V_k$ for its set of places, and write $\nu \in V_k^\infty$ (resp. $\nu \in V_k^f$) if $\nu$ is archimedian (resp. nonarchimedian).
We identify nonarchimedian places $\nu \in V_k^f$ with discrete valuations $\nu : k \to \Zbb$ on $k$ and in turn, prime ideals $\pfrak = \{ x \in k: \nu(x) > 0\}$ in $R$.
For a place $\nu \in V_k$ we write $k_\nu$ for the corresponding local field, and if $\nu \in V_k^f$ corresponding to a prime ideal $\pfrak$ of $R$ we write $R_\nu$ or $R_\pfrak$ for the closure of $R$ in $k_\nu$.
We say $\nu \in V_k^\infty$ is real (resp. complex) and write $\nu \in V_k^{\infty,\Rbb}$ (resp. $V_k^{\infty,\Cbb}$) if $k_\nu \approx \Rbb$ (resp. $\Cbb$).

The adele ring $\underline{k}$ over $k$ is the restricted direct product
\begin{align}
    \underline{k} = \prod_{\nu \in V_k}' k_\nu.
\end{align}
Thus, an adele $x \in \underline{k}$ has components $x_\nu \in k_\nu$ subject to the constraint that for all but finitely many places $\nu$, one has  $x_\nu \in R_\nu$. Equivalently,
\begin{align}
    \underline{k} = \{x \in \prod_{\nu\in V_k} k_\nu : |x_\nu| \leq 1, \text{ for all but finitely many $\nu$}\}.
\end{align}
For a finite set $S$ cotaining $V_k^\infty$, we write
\begin{align}
    U_S = \prod_{\nu \in S}k_\nu \times \prod_{\nu \notin S}R_\nu,
\end{align}
and topologize $\underline{k}$ such that the sets $U_S$ form a basis of open neighborhoods of $0$. With this topology, the $\underline{k}$ forms a locally compact topological ring. For each $|nu$ there is a continuous field embedding $k \to k_\nu$. Taking the product of these embeddings we get the diagonal inclusion $k \to \underline{k}$, which is a continuous injective homomorphism of rings. The image is discrete and the quotient $k \lmod \underline{k}$ is compact.

The idele group of $k$ is the locally compact group
\begin{align}
    \underline{k}^\times = \{ x \in \prod_\nu k_\nu^\times : |x_\nu| =1, \forall' \nu \in V_k
    \}
\end{align}
topologized as a subspace of $\ul{k} \times \ul{k}$ via the embeddding $

% We denote by $\underline{k}$ the ring of adeles of $k$. 

Let $A$ be a quaternion algebra over $k$. We say that $A$ \textbf{ramifies} over a place $\nu \in V_k$, and write $\nu \in \Ram A$, if $A_\nu = A \otimes_k k_\nu$ is a division algebra, and otherwise \textbf{unramified} in which case $A_\nu \approx M(2,k_\nu)$.

By class field theory, the set $\Ram A$ of places over which $A$ ramifies is a finite set with even cardinality. Conversely, given any finite set $S \subset V_k$ of even cardinality and such that $S\cap V_k^{\infty,\Cbb}=\emptyset$ there is a unique quaternion algebra $A$ with $\Ram A = S$, up to isomorphism. $A$ is a division algebra if and only if $\Ram A$ is nonempty, wheras $\Ram A = \emptyset$ if and only if $A$ is isomorphic to $M(2,k)$.

The adele ring of $A$ is
\begin{align}
    \underline{A} = \prod_{\nu}' A_\nu = \{\alpha \in \prod_{\nu \in V_k} A_\nu : \alpha_\nu \in O_\nu \text{ for all but finitely many $\nu$} \}
\end{align}


Associate to the quaternion algebra $A$ the algebraic groups over $k$ defined by
\begin{align*}
    \tilde{\Gbf}(E) = (A\otimes_k E)^\times,\quad \Gbf (E) = (A\otimes_k E)^1,\quad  \bar{\Gbf}(E) = E^\times \lmod (A\otimes_k E)^\times
\end{align*}
for any commutative $k$-algebra $E$. If $\Hbf$ is any of $\Gbf$, $\tilde{\Gbf}$, or $\bar{\Gbf}$, we write $H$ for $\Hbf(k)$, $H_\nu$ for $\Hbf(k_\nu)$, and $H_{\underline{k}}$ for $\Hbf(\underline{k})$.

An order in a quaternion algebra $A$ over $k$ is a subring $O$ containing $R$ such that $kO = A$. For any order $O$, we define affine group schemes over $R$ by
\begin{align*}
    \Gambfti_O(S) = (O\otimes_R S)^\times, \quad \Gambf_0(S) = (O\otimes_R S)^1, \quad \Gambfbar_O = S^\times \lmod (O\otimes_R S)^\times
\end{align*}
for any $R$-algebra $S$. Using the inclusion $R \to k$, we identify $\Gambf_O(R)$ with a subgroup $\Gamma_O$ of $G=\Gambf_O(k)$ and, in turn, a subgroup of $G_\nu$ for any place $\nu$.






\paragraph{Principal congruence subgroups}
For an ideal $\afrak$ of $R$, the reduction map $R \to R/\afrak$ induces an epimorpshim of groups $\Gamma \to \Gambf (R/\afrak)$. We write $\Gamma_O(\afrak)$ for the kernel, and refer to it as the \textbf{principal congruence subgroup of level } $\afrak$ in $\Gamma_O$. To summarize, for each ideal $\afrak$ of $R$, there is a short exact sequence of groups
\begin{align*}
    1 \to \Gamma_O (\afrak) \to \Gamma_O \to \Gambf (R/\afrak) \to 1.
\end{align*}
where
\begin{align*}
    \Gamma_O(\afrak) = \{ g \in O^1: g \in 1 + \afrak O\}
\end{align*}
and $\Gambf_O (R/\afrak)$ is a finite group. We define principal congruence subgroups of  $\Gamti$ and $\Gambar$ similarly.

\paragraph{Type number}
We say that $O$ is a maximal order in $A$ if it is not properly contained in any other order. Maximal orders exist in any quaternion algebra over $k$. The group $\Gti = A^\times$ (through its quotient $\Gbar = k^\times \lmod A^\times$) on the set of maximal orders in $A$. We say that $A$ has  \textbf{type-number 1} if this action is transitive, i.e. all maximal orders in $A$ are conjugate by an element of $A^\times$.
\begin{remark}\label{remark:classnum}
    The number conjugacy classes of maximal orders in a quaternion algebra $A$ over a number field $k$ is always finite, and divides the order of the $2$-torsion subgroup of the class group of $k$. If $k$ class number $1$, or more generally if it has odd class number, then any quaternion algebra over $k$ will have type number $1$.  In particular, any quaternion algebra $\Qbb$ has type number $1$.
\end{remark}


\section{Arithmetic $2-$ and $3-$ manifolds}
\begin{definition}
    We say $A$ is of
    \begin{itemize}
        \item    \textbf{Fuchsian-type} if: $k$ is totally real, and $A$ is unramified over exactly one real place $\nu_o$. In thise case, we use $\nu_o$ to identify $k$ with a subfield of $\Rbb$, and replace $\nu_o$-subscripts with $\Rbb$. Thus if $A$ is of fuchsian type, one has
              \begin{align}\label{eq:Fuchsian}
                  G_\Rbb \approx \GL(2,\Rbb),\quad G_\Rbb \approx \SL(2,\Rbb),\quad  \bar{G}_\Rbb \approx \PGL(2,\Rbb)
              \end{align}
        \item \textbf{Kleinian-type} if: $k$ has exactly one complex place $\nu_o$, and $A$ is ramified over all real places. In this case, use $\nu_o$ to identify $k$ with a subfield of $\Cbb$, and replace $\nu_o$-subscripts with $\Cbb$. Thus if $A$ is of Kleinian type, one has
              \begin{align}\label{eq:Kleinian}
                  G_\Cbb \approx \GL(2,\Cbb),\quad G_\Cbb \approx\SL(2,\Cbb),\quad  \bar{G}_\Cbb \approx \PGL(2,\Cbb)
              \end{align}
    \end{itemize}
\end{definition}
If $A$ is a quaternion algebra of Fuchsian or Kleinian type, we pick once and for all the isomorphisms in \ref{eq:Fuchsian} and \ref{eq:Kleinian}  respectively.









If $A$ is of Fuchsian (resp. Kleinian) type then  for any order $O$,   $\Gamma_O$ is a lattice in $G_\Rbb  \approx \SL(2,\Rbb)$ (reps. in $G_\Cbb\approx \SL(2,\Rbb)$). We call $\Gamma_O$ the \textbf{arithmetic lattice associated to} $O$. Let $\Hbb$ be hyperbolic $2$- or $3$-space, according to whether $A$ is Fuchsian or Kleinian. Then $\Gamma_O$ acts on $\Hbb$ properly discontinuously by isometries, and we write $X(\Gamma_O)$ for the quotient orbifold $\Gamma_O \lmod \Hbb$.





\begin{definition}.\linebreak
    \begin{enumerate}
        \item     A subgroup $\Lambda$ of $\SL(2,\Rbb)$ (resp. $\SL(2,\Cbb)$) is \textbf{arithmetic} if there exists a quaternion algebra $A$ over a number field $k$ of Fuchsian (resp. Kleinian) type and an order $O$ in $A$ such that $\Lambda$ is commensurable (in the wide sense) with the arithmetic lattice $\Gamma_{O}$ associated to $O$.
        \item  We say that $\Lambda$ is \textbf{derived from} $O$ if it is conjugate in $\SL(2,\Rbb)$ (resp. $\SL(2,\Cbb)$) to a subgroup of $\Gamma_O$.
        \item Say that $\Lambda$ is \textbf{derived from} $A$ if it is derived from some order in $A$. If $M$ is a hyperbolic $- or $- manifold, we say that $M$ is arithmetic (resp. derived from an order $O$ or a quaternion algebra $A$) if there exists an arithmetic lattice $\Lambda$  such that $M$ is isometric to $\Lambda \lmod \Hbb$ for some arithmetic lattfice $\Lambda$ (resp. arithmetic lattice derived from $O$ or $A$).
    \end{enumerate}
\end{definition}

\paragraph{Invariants of arithmetic Fuchsian and Kleinian groups}
Let $\Lambda$ be a subgroup of $\SL(2,\Cbb)$.  Write $\tr \Lambda$ for the set $\{\tr g : g\in \Lambda\} \subset \Cbb$ and  $\Qbb(\tr \Lambda)$ for the \textbf{trace field of $\Lambda$}, the subfield of $\Cbb$ generated by $\tr \Lambda$. We write
\begin{align*}
    A_0\Lambda = \{ \sum_i a_i g_i : a_i \in \Qbb \tr \Lambda, g_i \in \Lambda\},
\end{align*}
for the subring of $M(2,\Cbb)$ generated by $\Lambda$ as an algebra over $\Qbb \tr \Lambda$. Then we have
\begin{proposition}
    Suppose $\Lambda$ is a finitely generated, nonelementary subgroup of $\SL(2,\Cbb)$. Then $A_0\Lambda$ is a quaternion algebra over $\Qbb \tr \Gamma$. Furthermore
    \begin{enumerate}
        \item If $\Lambda$ is a lattice in $\SL(2,\Cbb)$ or an arithmetic lattice in $\SL(2,\Rbb)$ then $\Qbb \tr \Lambda$ is a number field.
        \item $\Lambda$ is conjugate to a subgroup of $\SL(2,\Rbb)$ if and only if $\Qbb \tr \Lambda$ is contained $\Rbb \subset \Cbb$.
        \item If $\Lambda$ is an arithmetic lattice in $\SL(2,\Cbb)$ or $\SL(2,\Rbb)$, then $\Lambda$ is derived from $A_0 \Lambda$ if and only if $\tr \Lambda$ consists of algebraic integers in $\Qbb \tr \Lambda$.
    \end{enumerate}
\end{proposition}



The following basic facts will be used in what follows.
\begin{proposition}
    % Let $A$ be a quaternion algebra of Fuchsian or Kleinian type.
    \begin{enumerate}
        Let $\Kbb = \Rbb$ or $\Cbb$.
        \item\label{item:orders} If $O,O'$ are two orders in a quaternion algebra $A$, then $\Gamma_O$ and $\Gamma_{O'}$ are commensurable in $G=A^1$. Consequently, the property of being commensurable in $G_\Kbb$ to an order in $A$ depends only on $A$.
        \item If $O,O'$ are orders in quaternion algebras $A,A'$ over number fields $k,k'$ with $A_\Kbb = A'_\Kbb$ such that $\Gamma_O$ is commensurable to $\Gamma_O'$ in $A_\Kbb^1 = {A^1}{\Kbb}^1$ then $k=k'$ and $A = A'$. Consequently, the quaternion algebra is an invariant of the commensurability class of an arithmetic lattice in $\SL(2,\Kbb)$.
        \item The quaternion algebra of an arithmetic lattice in $\SL(2,\Kbb)$ is a complete invariant of its commensurability class.
        \item An arithmetic lattice is cocompact if and only if its quaternion algebra is a division algebra.
    \end{enumerate}
\end{proposition}



\newpage