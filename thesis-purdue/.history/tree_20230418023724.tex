\documentclass[draft]{article}
\usepackage{amsthm,amssymb,amsmath,amsfonts,braket}
\usepackage[inline]{enumitem}
\usepackage{mycros}
\usepackage{graphicx} % Required for inserting images
\usepackage{url}
\usepackage{verbatim}
\usepackage{todonotes}
\usepackage{mathtools,thmtools}
\usepackage{tikz-cd}
\usepackage{quiver}
\usepackage{comment}

\newcommand{\LaSpec}{\operatorname{Spec}_\Delta}
\newcommand{\Vol}{\operatorname{Vol}}
\newcommand{\Comm}{\operatorname{Comm}}
\newcommand{\Clo}{\operatorname{Clo}}
\newcommand{\Autbf}{\operatorname{\mathbf{Aut}}}
\newcommand{\MOcal}{\mathcal{MO}}
\newcommand{\HLcal}{\mathcal{HL}}
\newcommand{\Span}{\operatorname{Span}}
\newcommand{\CDef}{\operatorname{Cdef}}
\newcommand{\Int}{\operatorname{Int}}
\newcommand{\disc}{\operatorname{disc}}
\newcommand{\hilbert}[3]{\left(\frac{#1, #2}{#3}\right)}
\newcommand{\sang}[1]{\langle\langle #1\rangle\rangle}
\newcommand{\sdang}[1]{\langle\langle\langle\langle #1 \rangle\rangle\rangle\rangle}
\newcommand{\Ram}{\operatorname{Ram}}
\newcommand{\Sing}{\operatorname{Sing}}
%\newcommand{\nrd}{\operatorname{nrd}}
%\newcommand{\trd}{\operatorname{trd}}
\title{tree}
\author{justin katz}
\date{March 2023}

\setlength{\parindent}{0 em}
\setlength{\parskip}{6 pt}
\begin{document}

Let $A$ be a quaternion algebra over a number field $k$, and let $\Ocal$ be a maximal order in $A$. For each finite place $\pfrak$ of $k$ over which $A$ is split, pick an isomorphism of $A_\pfrak =A\otimes_k k_\pfrak$ with $M(2,k_\pfrak)$ such that $\Ocal_\pfrak = \Ocal \otimes_R R_\pfrak$ maps to $M(2,R_\pfrak)$ and write $\rho_\pfrak:A^\times \to \GL(2,k_\pfrak)$ for the composition of the natural embedding $A^\times \to (A\otimes_k k_\pfrak)^\times$ followed by our chosen isomorphism $A_\pfrak^\times \approx \GL(2,k_\pfrak)$. 

\paragraph*{p-adic computation}
\begin{definition}[Bruhat-Tits tree]
The Bruhat-Tits tree $\Tcal$ for $\GL(2,k)$ is the following graph: the vertices of $\Tcal$ are homothety classes of $R$-lattices in $k^2$.
If $L$ is a lattice in $k^2$, we write $[L]$ for its homothety class. 

Two vertices $\Lambda_1,\Lambda_2 $ of $\Tcal$ are adjacent if there are representative lattices $L_i \in \Lambda_i$ such that $\pi L_1 < L_2 < L_1$ with each inclusion proper.

We write $V(\Tcal)$ for the vertices of $\Tcal$, and $E(\Tcal)$ for the edges of $\Tcal$ (which consist of ordered pairs of adjacent vertices.)
\end{definition}

We regard $k^2$ as a space of column vectors, with distinguished basis vectors $e_1,e_2$. We write $L_0 = Re_1+Re_2$ and refer to this as our distinguished lattice, and refer to its homothety class $\Lambda_0 = [L_0]$ as our distingushed vertex.  
 
$\GL(2,k)$ acts on the set of lattices in $k^2$: for $g\in \GL(2,k)$ and a lattice $L$, the set $gL = \{ g v : v \in L\}$ is also a lattice in $k^2$. This action is transitive: given a lattice $L$, pick an $R$ basis $(v,w)$ for $L$ and define $g_L$ by the assignment $ge_1 = v$ and $ge_2 = w$. Then $g_L L_0 =L$, where $L_0$ is our distinguished lattice $R^2$ in $k^2$ as above. Under this action, the stabilizer of $L_0$ is $\GL(2,R)$. Thus, the orbit map $g \mapsto gL_0$ induces a bijection $\GL(2,k)/ \GL(2,R)$ with the set of lattices in $k^2$. Under this bijection the map $L\mapsto [L]$, sending a lattice to its homothety class, corresponds to the quotient $\GL(2,k) / \GL(2,R) \to k^\times \lmod \GL(2,k) \rmod \GL(2,R)$ with $k^\times$ acting by scaling. As $k^\times \cap \GL(2,R) = R^\times$, we thus have an identification of $V(\Tcal)$ with $\PGL(2,k)/\PGL(2,R)$ as left $\PGL(2,k)$ sets. 

The resulting action of $\PGL(2,k)$ on $V(\Tcal)$ maps ordered edges to ordered edges: if $(\Lambda_1,\Lambda_2) \in E(\Tcal)$, and $L_1,L_2$ are representative lattices with $\pi L_1 < L_2 < L_1$  

\begin{remark}
    \begin{enumerate*}
        \item The action of the center $k^\times$ of $\GL(2,k)$ on lattices $L$ factors through $R^\times \lmod k^\times $, as $xL = L$  if (and only if) $x \in R^\times$. 
    \end{enumerate*}
\end{remark}
 


\paragraph*{Local Computations} 
For now, let $k$ nonarchimedian local field of characteristic $0$, with ring of integers $R$ and maximal ideal $\pfrak$, with uniformizer $\pi$. Let $V$ be a two dimensional $k$-vectorspace, and $A = \End_k(V)$. Then every maximal order in $A$ takes the form 
\begin{align*}
    \End_R(L) = \{ x \in A : xL\leq L\}
\end{align*}
where $L$ is a free $R$-module of rank $2$ in $V$. If $M = xL$ for some $x\in A^\times = \Aut_k(V)$, then $\End_R(M) = x\End_R(L)x^\inv$. As $A^\times$ acts transitively on the set of lattices in $V$ (by left-translation), it acts transitively on the set of maximal orders in $A$ (by conjugation). Further, if $\End_R(L)=\End_R(M)$ for two lattices $L,M$ in $V$, then there exists an $x \in k^\times$ such that $L = x M$. In this case, we say $L$ and $M$ are homothetic, and conclude that there is an $k^\times \lmod A^\times = \PGL(V)$ equivariant bijection between homothety classes of lattices in $V$ and maximal orders in $A$. 

Given two lattices $L$ and $M$ in $V$, there exists a basis $u,v$ of $L$ such that $\pi^a u , \pi^b v$ is a basis for $M$ (elementary divisor theorem) and we set $d(L,M) = |a-b|$. If $x,y\in k^\times$ then $d(xL,yM)=d(L,M)$ so that $d$ descends to a function on homothety classes, or equivalently a function on maximal orders. 


\cite{sallyFourierTransformOrbital1983}
\bibliographystyle{plain}
\bibliography{references}
\end{document}
