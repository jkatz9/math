\documentclass{book}
\usepackage{amsthm,amssymb,amsmath,amsfonts}
\usepackage{mycros}
\usepackage{graphicx} % Required for inserting images
\usepackage{url}
\usepackage{verbatim}
\usepackage{todonotes}
\usepackage{mathtools,thmtools}
\usepackage{tikz-cd}
\usepackage{quiver}
\usepackage{comment}

\usepackage{hyperref}
\hypersetup{
    colorlinks,
    citecolor=black,
    filecolor=black,
    linkcolor=black,
    urlcolor=black
}
\newcommand{\LaSpec}{\operatorname{Spec}_\Delta}
\newcommand{\Vol}{\operatorname{Vol}}
\newcommand{\Comm}{\operatorname{Comm}}
\newcommand{\Clo}{\operatorname{Clo}}
\newcommand{\Autbf}{\operatorname{\mathbf{Aut}}}
\newcommand{\MOcal}{\mathcal{MO}}
\newcommand{\HLcal}{\mathcal{HL}}
\newcommand{\Span}{\operatorname{Span}}
\newcommand{\CDef}{\operatorname{Cdef}}
\newcommand{\Int}{\operatorname{Int}}
\newcommand{\disc}{\operatorname{disc}}
\newcommand{\hilbert}[3]{\left(\frac{#1, #2}{#3}\right)}
\newcommand{\sang}[1]{\langle\langle #1\rangle\rangle}
\newcommand{\sdang}[1]{\langle\langle\langle\langle #1 \rangle\rangle\rangle\rangle}
\newcommand{\Ram}{\operatorname{Ram}}
\newcommand{\Sing}{\operatorname{Sing}}
%\newcommand{\nrd}{\operatorname{nrd}}
%\newcommand{\trd}{\operatorname{trd}}
\hfuzz=5.1pt 
\title{thesis}
\author{justin katz}
\date{March 2023}

\setlength{\parindent}{0 em}
\setlength{\parskip}{6 pt}
\begin{document}

\maketitle
%\tableofcontents
\newpage
\begin{comment}
	\chapter{Introduction}
\section{background}
\subsubsection{notation}
\begin{itemize}
	\item For a bundle $\pi : E \to B$, we write $E_x$ for the fiber $\pi^\inv (x) \subset E$ over a point $x\in B$. We denote the space of (global) sections by $\Gamma(E)$.
	\item For a smooth manifold $M$, we denote by $TM,T^*M$ the tangent and cotangent bundles, respectively. Linear algebraic constructions applied applied to vector bundles over $M$ are understood to be defined. We refer to elements of $\Gamma(\bigwedge^k T^*M)$ as $k$-forms on $M$.
	\item
\end{itemize}
\subsubsection{Riemannian geometry}
Given a smooth manifold $M$ a Riemannian metric on $M$ is a choice of positive definite symmetric bilinear forms
\[g_x{\cdot}{\cdot}: T_x M \otimes T_x M, \to \Rbb\]
on the tangent space $T_x M$ to $M$, which varies smoothly relative to $x\in M$. We call to the pair $(M,g)$ a Riemannian manifold, or a Riemannian structure on $M$.
%     Fixing $(M,g)$ introduces several canonically associated fobjects, as follows:
% \subfsection{length} 
%     For a curve $\gamma$ on $M$, define its length:
%         \[ \ell_g(\gamma) = \int_{a}^{b} g_{r(t)}(r'(t),r'(t)) \dop t, \]
%     where $r:[a,b] \to \gamma$ is any smooth parameterization of $\gamma$.  
Canonically ssociated to a metric $g$ on $M$ are several objects:
\begin{description}
	\item[\textbf{gradient}] The gradient is an operator $\nabla: C^\infty(M) \to  \Gamma(TM)$ sending a smooth function $f$ to the vector field $\nabla f$ characterized by the condition: \[ g_x(\nabla f(x),X(x))=X_x(f) \]
		for all vectorfields $X \in \Gamma(TM)$.
	\item[\textbf{codifferential}] $\delta$
	\item[\textbf{hodge star}] $*:\bigwedge^k T^* M \to \bigwedge^{\dim M - k} T^*M$
	\item[textbf{volume form}] $\dop \vol : \bigwedge^{\dim M} T^* M \to \Rbb$
	\item[\textbf{density}] $|\dop \vol|$
	\item[Laplacian] $\Delta$
\end{description}


\section{Spectral geometry}
\subsection{Laplace spectrum}
Using the Riemannian density, we equip $C^\infty_c(M)$ with an inner product:
\[ \ip{\phi}{\psi} = \int_M f(x) \overline{g(x)} |\dop \Vol|(x). \]
We define $L^2(M)$ to be the completion of $C^\infty(M)$ with respect to this inner product.

On $C^\infty(M)$ the Laplace operator $\Delta$ is a densely defined, negative semi-definite, and symmetric:
\[ \ip{\Delta \phi}{\psi} = - \ip{\phi}{\Delta \psi} \]
unbounded operator on $L^2(M)$. Consequently, $\Delta$ admits a canonical (though, not necessarily unique, if $M$ is not compact) to a an unbounded operator $\tilde{\Delta}$ defined on a dense subspace of $L^2(M)$ on which it is \textit{self-adjoint}. If we assume further that $M$ is closed, then the $\tilde{\Delta}$ has \textit{compact} resolvent. By the spectral theorem for compact self adjoint operators, there exists an orthonormal basis $1=\psi_1,\psi_2,\cdots$ of eigenfunctions for $\tilde{\Delta}$ with real, nonpositive eigenvalues $0=\lambda_1>\lambda_2>\cdots$. By elliptic regularity, the eigenfunctions $\psi_i$ of the extension $\tilde{\Delta}$ are actually smooth, thus lie in the original domain $C^\infty(M)$ of $\Delta$.

We refer the sequence $\lambda_i$ of eigenvalues for $\Delta$ as the \textbf{Laplace-spectrum} of $M$ and write it as $\LaSpec(M)$.

It will be useful to encode $\LaSpec(M)$ in various other forms.
\begin{description}
	\item[multiplicity function] Denote by $m_M: \Cbb \to \Zbb_{\geq 0}$ the function which assigns to a complex number $\lambda$ the number of indices $i\in \Zbb_{\geq 0}$ such that $\lambda_i = \lambda$; an alternative description is:
		\[m_M(\lambda) = \dim \ker_{C^\infty(M)}( \Delta - \lambda) \]
	\item[spectral measure] Letting $\delta_x$ denote the dirac delta distribution on $C^\infty(\Cbb)$ based at $x$, we set
		\begin{align*}
			\dop m_M = \sum_{i=1}^\infty \delta_{\lambda_i}.
		\end{align*}
	\item[theta] Define $\Theta_M: \Rbb_{>0} \to \Rbb$ by
		\begin{align*}
			\Theta_M(t) & = \sum_{i=1}^\infty e^{t\lambda_i}           \\
			            & = \int_\Cbb e^{t \lambda} \dop m_M (\lambda)
		\end{align*}
	\item[Minakshisundaram--Pliejel zeta] Define, for $s$ in a suitable right half-plane,
		\begin{align*}
			\zeta_M(s) & = \sum_{i=1}^\infty \lambda_{i}^{-s}        \\
			           & = \int_\Cbb \lambda^{-s} \dop m_M(\lambda).
		\end{align*}
\end{description}
\subsection{Length spectrum}
In this section, we suppose that the Riemannian metric $g$ on $M$ has strictly negative curvature.

In this case, within each free homotopy class of curve on $M$, there is a unique element with minimal length with respect to $g$. This representative is a geodesic on $M$, and we refer to it as the \textbf{geodesic representative} of a free homotopy class of curve. For any point $x_o \in M$, there is an identification of the set of free homotopy classes of curves with conjugacy classes in the fundamental group $\pi_1(M,x_o)$ based at $x_o$. Consequently, the Riemannian metric induces a class function $\ell_g: \pi_1(M,x_o) \to \Rbb$.


% \subsubsection{topological preliminaries}
%     For a connected, locally simply connected topological spaces $X$, we let $\tilde{X}$ denote the universal cover. For each point $x_o \in X$ the fundamental group $\pi_1(X,x_o)$ acts freely on $\tilde{X}$ by homeomorphisms preserving the fibers of $\tilde{X} \to X$. It follows that the projection $\tilde{X} \to X$ induces a homeomorphism $\pi(X,x_o)\lmod \tilde{X} \to X$. Consequently, for any topological space $Y$, we may identify $C(X,Y)$ with the subpsace $C(\tilde{X} , Y)^{\pi_1(X)}$ of $C(X,Y)$ of functions on $\tilde{X}$ invariant under $\pi_1(X)$. If $f \in C(X,Y)$ is a function, we write $\tilde{f}$ for the corresponding function on $\tilde{X}$. 

%     When $(M,g)$ is a Riemannian manifold, the lift $\tilde{g}$ of the metric to $\tilde{M}$ is the unique one such that the universal covering map $\tilde{M} \to M$ is a local isometry. 
% Taking $\Rbb/\Zbb$ as a model for the circle $S^1$, we identify the space $\Lcal(M)$ of continuous maps $S^1 \to M$ with the space $C(\Rbb,M)^\Zbb$ of continuous functions $f: \Rbb \to M$ which are invariant under the action of $\Zbb$ (by precomposition with translation). The Riemannian metric $g$ on $M$ induces a  \textbf{length function} $\ell_g : \Lcal(M) \to \Rbb_{\geq 0}$. 

% Fixing a point $x_o \in M$, let $\Lcal_{x_o} (M)$ denote the subspace of $\Lcal(M)$ consisting of those $f$ satisfying $f(0)=x_o$. We refer to elements of $\Lcal_{x_o} (M)$ as \textit{loops based at $x_o$}. For loops $\gamma,\gamma':\Lcal_{x_o}(M)$, define their concatenation $\gamma \cdot \gamma' \in \Lcal_{x_o} (M)$ by the rule 
% \begin{align}
%     \left(\gamma \cdot \gamma' \right)(t)=   \begin{cases}\gamma_0(2 t) & 0 \leq t \leq \frac{1}{2} \\ \gamma_1(2 t-1) & \frac{1}{2} \leq t < 1\end{cases}
% \end{align}
% defined initially on $[0,1)$, extended by $\Zbb$ invariance to $\Rbb$. 







\subsection{relationships between spectra}
\subsection{Behavior under covers}
\subsection{twisting(by flat vector bundles)}
\subsection{Artin Takagai formalism}
\subsection{Isospectrality(of various forms)}
\section{Arithmetic locally symmetric spaces}
This section will follow the presentation in \cite{helgasonDifferentialGeometryLie2001}.
\subsection{Locally symmetric spaces}
Let $(M,g)$ be a Riemannian manifold, and $x\in M$. Let $N_0 \subset (T M)_x$ denote a neighborhood of $0$ which is stable under the map $v \mapsto -v$ and set $N_x = \exp_x N_0$ and small enough so that the map $\exp_x: N_0 \to N_p$ is a diffeomorphism.  For each point $y \in N_x$, let $\gamma_{x,y} : [0,1] \to N_p$ denote the unique geodesic segment satisfying $\gamma_{x,y}(0)=x$ and $\gamma_{x,y}(1)=y$. Set $s_x(y) = \gamma_{x,y}(-1)$. The map $s_x$ is a diffeomorphic involution on $N_x$, which we call the \textbf{geodesic symmetry} with respect to $x$. By design, it satisfies $(ds_x)_x = -\id \in \Aut((TM)_x)$.
\begin{defn}
	A Riemannian manifold $M$ is a \textbf{locally symmetric space} if each point $x\in M$ admits a neighborhood on which the geodesic symmetry $s_x$ is an isometry. A locally symmetric space is \textbf{globally symmetric} if for each point $x\in M$, the geodesic symmetry $s_x$ extends to a global isometry of $M$.
\end{defn}
When $(M,g)$ is Riemannian locally symmetric space for which the metric $g$ is complete, its universal cover $\tilde{M}$, when equipped with the pullback metric $\tilde{g}$, is globally symmetric. For any $x\in M$, the action of the fundamental group $\pi_1(M,x)$ on $\tilde{M}$ as group $\Gamma$ of deck transformations of $\tilde{M} \to M$ is by isometries (with respect to $\tilde{g}$, and induces an isomtery $\Gamma \lmod \tilde{M}\to M$. To summarize, all locally symmetric spaces arise as quotients $\Gamma \lmod X$ of globally symmetric spaces, where $\Gamma < \Isom(X)$ is a discrete subgroup acting freely on $X$.

As discussed in the next section,  we may realize any globally symmetric space $X$ as a quotient $G/K$ for a semisimple Lie group $G$ and a compact subgroup $K$.
Combining this claim with the preceeding paragraph, we conclude:
\begin{thm}
	Let $(M,g)$ be a locally symmetric space. Then there exists a semisimple Lie group $G$, a compact subgroup $K$, and a discrete subgroup $\Gamma$ of $G$ such that $(M,g)$ is isometric to $(\Gamma \lmod G \rmod K , B)$ where $B$ is a  $G$ invariant Riemannian metric on $G\rmod K$.
\end{thm}



\subsection{Symmetric spaces}

A Riemannian symmetric space is a Riemannian manifold $M$ such that, for each point $x\in M$, there exists a global involutive isometry $s_x \in \Isom(M)$ which fixes $x$ and induces the map $v \mapsto -v$ on the tangent space $(TM)_x$ at $x$. In this case, the group $G=\Isom(M)$
\begin{itemize}
	\item acts transitively on $M$,
	\item admits a unique structure as a smooth manifold, making it into a Lie group,
	\item if, for a point $x \in M$, we write $K_x$ for its stabilizer in $G$, then $K$ is a compact subgroup and the orbit map $G \to M$ sending $g$ to $gx$ induces a diffeomorphism $G/K_x \to M$,
	\item the map $\sigma_x: G \to G$ defined by $g \mapsto s_x g s_x$ is an involutory automorphism of $G$ such that $(G^{\sigma_x})^\circ \leq K_x \leq G^{\sigma_x}$ where $G^{\sigma_x}$ is the group of fixed points of $\sigma_x$, and $(G^{\sigma_x})^\circ$ its connected component containing the identity.
\end{itemize}
Fix $x \in M$ and supress it from the notation for $K$, $\sigma$, and $s$. Let $\gfrak$ be the Lie algebra of $G$. As $\sigma \in \Aut(G)$ is an involution, the eigenvalues $\lambda$ of $(d\sigma)_e \in \Aut(\gfrak)$ and $\gfrak = \ker ((d\sigma)_e-1)\oplus \ker ((d\sigma)_e+1).$  From the third property above, one finds that the Lie algebra $\kfrak$ of $K$ is precisely the $+1$ eigenspace of $(d \sigma)_e$. Let $\pfrak$ denote the $-1$ eigenspace.

Letting $\pi : G \to M$ denote the orbit map $g \mapsto gx$ one finds that $\ker (d\pi)_e = \kfrak$, and it maps $\pfrak$ isomorphically to $T_x M$.  In particular, if $v \in \pfrak$, then the geodesic based at $x$ with tangent vector $(\dop \pi)_e v$ is given by $t \mapsto \exp (t v) x$. %Furhtermore, for $w \in (TM)_x$ the parallel translate of $w$ along this geodesic is given by $(d \exp (tv) )_x (w) $

Consequently the study of Riemannian globally symmetric spaces is tantamount to the study of \textbf{symmetric pairs} $(G,H)$, where $G$ is a connected Lie group, $H$ a compact subgroup such that there exists an involutory automorphism $\sigma \in \Aut(G)$ such that $(G^\sigma)^\circ \leq H \leq G^\sigma$.



\subsection{Arithmetic locally symmetric spaces}
% Arithmetic locally symmetric spaces are those which arise from a certain construction, which I now outline. First, we need some algebro-geometric background.
% \subsubsection{Affine schemes}
%    Let $R$ be a commutative ring with $1$. We associate to $R$ the topological space $X=\spec(R)$ defined as follows: 
%    \begin{itemize}
%        \item The elements of $X$ are prime ideals $\pfrak \leq R$. 
%        \item For each ideal $\afrak\leq R$, let $V(\afrak)$ denote the set of prime ideals $\pfrak$ containing $\afrak$. Then we take the sets $V(\afrak)$ as the closed sets of $X$. Alternatively, if one defines, for $f\in R$, the set $D_f$ of prime ideals not containing $f$, one may take the set $\{D_f: f\in R\}$ as  basis for the topology on $X$. 
%    \end{itemize}
%    We equip the topological space $X=\spec (R)$ with a sheaf $O_{X}$ of rings, defined by specifying sections on the open sets $D_f$ as $\Gamma(D_f,O_X) = R_f$, the localization of $R$ with respect to the powers of $f$.
%    \begin{defn}
%        A pair $(X,O_X)$ consisting of a topological space $X$ and a sheaf of rings $O_X$ on $X$ is an \textbf{affine scheme} if $X=\spec(R)$ for some ring $R$.
%    \end{defn}

%    \begin{example}
%        The closure of a singleton $
%        \{\pfrak\} \subset X = \spec R$ is the set of prime ideals $\pfrak'$ containing $\pfrak$. It follows that the set $\{\pfrak\}$ is closed in $X$ if and only if $\pfrak$ is a maximal ideal of $R$.  
%    \end{example}
%    \begin{example}
%         For any ring $R$, the affine scheme $\Abb^n_R:=\spec(R[x_1,...,x_n])$ is called the $n$-dimensional affine space over $R$. There is a subset of $\Abb_R^n$ which may be identified with the familiar $R^n$, by associating to an $n$-tuple $(a_1,...,a_n)\in R^n$ the maximal ideal $\langle x_1 - a_1 , ...,x_n-a_n\rangle \in \Abb_n^R$. There are, however, always more points than these in $\Abb^n_R$.
%         % When $R=F$ is an algebraically closed field, the maximal ideals of $F[x_1,...,x_n]$ take the form $\langle x_1-a_1, ....,x_n-a_n\rangle$ for some $n$-tuple of elements $(a_1,...,a_n) \in F^n$.  Thus, only the \textit{closed points} of $\Abb^n_F$ may be identified with $F^n$.
%    \end{example}
% \subsubsection{General Schemes}

%    More generally, a \textbf{scheme} is a pair $(X,O_X)$ where $X$ is a topological space and $O_X$ is a sheaf of rings such that $X$ admits a covering $X = \bigcup_i U_i$ by open sets $U_i$ which are affine schemes. 


\section{Algebraic groups}
In this section $F$ is an arbitrary field.
\begin{defn}
	An algebraic group $\Gbf$ over $F$ is a group object in the category of schemes over $F$.
\end{defn}
We will freely identify such a $\Gbf$ with its \textbf{functor of points}:
\begin{align}
	X \mapsto \Gbf(X) = \Hom_F(X,\Gbf),
\end{align}
for any scheme $X$ over $F$. When $X=\spec A$ for $A$ a commutative $F$-algebra, then we abuse notation in writing $\Gbf(A)$ for $\Gbf(\spec(A))$.
Thus, we think of $\Gbf$ as a family of groups $\Gbf(A)$ parameterized by commutative $F$-algebras $A$, such that each $A \to B$ of $F$-algebras induces a group homomorphism $\Gbf(A)\to \Gbf(B)$.
In particular, the canonical inclusion $x\mapsto x \cdot 1_A \in A$ defining $A$ as an $F$-algebra, induces a canonical map $\Gbf(F) \to \Gbf(A)$.

\paragraph{$\Gbb_a,\Gbb_m$} The fundamental building blocks for linear algebraic groups over $F$ are the \textbf{additive group} $\Gbb_a$ and and \textbf{multiplicative group} $\Gbb_m$, which are defined by via their functor of points:
\begin{align}
	\Gbb_a(A)= (A,+),\quad \Gbb_m(A)=(A^\times,\cdot), \quad \text{for any $F$-algebra $A$.}
\end{align}

\paragraph{$\Autbf$:} if $\Gbf$ is any algebraic group over $F$, then the assignment
\begin{align}
	A\mapsto \Aut_{gp}(\Gbf(A))
\end{align}
defines an algebraic group, which we'll call $\Autbf_{gpsch}(\Gbf)$ [TODO: think about whether these automorphisms should fix/stabilize $\Gbf(F)$.]

The multiplicative structure of an algebra $A$ allows for a \textbf{canonical} inclusion $A^\times \to \Aut((A,+))$, which in turn yeilds a map of algebraic groups $\Gbf_m \to \Autbf(\Gbb_a)$.
\begin{rem}
	If one forgets about the group structure on $\Gbb_a$ and just considers it as a scheme, then one obtains the \textbf{affine line} $\Abb^1 = \spec(F[x])$  over $F$. The algebraic group $\Autbf_{sch}(\Abb^1)$ defined by $A \mapsto \Aut_{sch}(\Abb^1(F))$ is larger than $\Autbf_{gpsch}(\Gbb_a)$.
\end{rem}

The natural action of $A^\times$ on $A$ induces a homomorphism $\Gbb_m \to \Aut(\Gbb_a)$, of algebraic groups.
\subsection{foo}
\paragraph{Over a field}
Let $F$ be any field.

\paragraph{Over a nonarchimedian local field}

\paragraph{Over an archimedian local field}

\paragraph{Over a number field}

\subsubsection{Some comments on integral structures}

\chapter{Results}
\section{Spectral flexibility}

\section{Spectral Rigidity}
\subsection{Commensurability invariants of hyperbolic surfaces}
\subsubsection{Trace fields and quaternion algebras}
\begin{defn}
	Let $\Gamma$ be a nonelemnentary subgroup of $\PSL(2,\Cbb)$ and let $\hat{\Gamma} = P^\inv (\Gamma)$, where $P:\SL(2,\Cbb) \to \PSL(2,\Cbb)$. The \textbf{trace field} $\Qbb(\tr \Gamma)$ of $\Gamma$ is the field
	\begin{align*}
		\Qbb(\tr \hat{\gamma}: \hat{\gamma} \in \hat{\Gamma}).
	\end{align*}
\end{defn}
\begin{rem}
	If $M$ is a finite volume hyperbolic $2$- or $3$-manifold, then there exists a lattice $\Gamma$ in $\PSL(2,\Rbb)$ or $\PSL(2,\Cbb)$ such that $M$ is isometric to $\Gamma \lmod \Hbb^2$ or $\Gamma \lmod \Hbb^3$, and the set of such $\Gamma$ consitutes a full $\PSL(2,\Rbb)$ or $\PSL(2,\Cbb)$ conjugacy class. Since $\Qbb(\tr \Gamma)$ is a conjugcay invariant of $\Gamma$, we may understand it as an invariant of the Riemannian manifold $M$.
\end{rem}
\begin{def/prop}
For a nonelementary subgroup $\Gamma \leq \PSL(2,\Cbb)$, let $A_0\Gamma\subset M(2,\Cbb)$ denote the algebra generated over $\Qbb(\tr \Gamma)$ by the elements $\hat{\gamma}$ for $\hat{\gamma} \in \hat{\Gamma}$. Then, in fact, $A_0\Gamma$ is a quaternion algebra over $\Qbb(\tr \Gamma)$.
\end{def/prop}
\begin{rem}
	As before, the isomorphism type of $A_0 \Gamma$ as a quaternion algebra over $\Qbb(\tr \Gamma)$ is an invariant of the hyperbolic $2$- or $3$-manifold that $\Gamma$ uniformizes.
\end{rem}
\subsubsection{Invariant trace fields and quaternion algebras}
While the trace field $\Qbb(\tr \Gamma)$, and the quaternion algebra $A_0 \Gamma$ over it, are invariants of the isometry class of the hyperbolic manifold associated to $\Gamma$, they are not constant along the commensurability class of $\Gamma$. The following construction refines the preceeding one to get a genuine commensurability invariant.
\begin{def/prop}
Let $\Gamma$ be a finitely generated non-elementaryy subgroup of $\PSL(2,\Cbb)$ and let $\Gamma^{(2)}$ denote the subgroup of $\Gamma$ generated by squares of elements of $\Gamma$. The trace field $\Qbb(\tr\Gamma^{(2)})$ of $\Gamma^{(2)}$ is an invariant of the commensurability class of $\Gamma$. We refer to it as the \textbf{invariant trace field} of $\Gamma$, and write it as $k\Gamma$.

Furthermore, the quaternion algebra $A_0\Gamma^{(2)}$ over $k\Gamma$ is an invariant of the commensurability class of $\Gamma$. We refer to it as the \textbf{invariant quaternion algebra} of $\Gamma$, and denote it by $A\Gamma$.
\end{def/prop}



\subsection{Arithmeticity}
Historically, arithmeticity of a lattice $\Gamma$ in a semisimple Lie group $G$ was defined by stipulating that $\Gamma$ be commensurable with groups arising from a particular construction, which we will detail in a moment.

This definition of arithmeticity for a lattice does not clearly reflect the sense in which it is an intrinsic property of the Riemannian geometry of the locally symmetric space that it uniformizes.

The following celebrated theorem of Margulis characterizes arithmeticity in terms of the existence of an abundant group of \textbf{virtual symmetries}.

To state this theorem, we first recall that the commensurator $\Comm_G(\Gamma)$ of a subgroup $\Gamma$ of a group $G$ is the group of elements $g\in G$ such that $g\Gamma g^\inv \cap \Gamma$ has finite index in both $\Gamma$ and $g\Gamma g^\inv $.

\begin{rem}\label{remark:hiddensymm}
	If $M=\Gamma \lmod G \rmod K$ is a locally symmetric space, then for an element $g \in \Comm_G(\Gamma)$, let $M' = (g\Gamma g^\inv) \cap \Gamma \lmod G / K$.  Then $M' \to M$ is a Riemannian cover, and $g$ induces an isometry of $M'$. When $g \notin N_G(\Gamma) \leq \Comm_G(\Gamma)$, then this isometry is not the lift one on $M$. In this case, following \cite{farbHiddenSymmetryArithmetic2004}, we say $g$ is a hidden symmetry of $M$.
\end{rem}

\begin{thm}%[Theorem (1) of \cite{margulisDiscreteSubgroupsSemisimple1991b}]
	Let $G =\PSL(2,\Rbb)$ or $G= \PSL(2,\Cbb)$. A lattice $\Gamma\leq G$ is arithmetic if and only if $\Comm_G(\Gamma)$ is dense in $G$. Equivalently, $\Gamma$ is arithmetic if it has infinite index in $\Comm_G(\Gamma)$.
\end{thm}
\subsection{Arithmetic lattices in SL(2,R)}
In this section, we will identify a particular representative of each commensurability class of arithmetic lattice in $\SL(2,\Rbb)$ or $\SL(2,\Cbb)$. Each of these represenatives will be maximal lattices, and will be uniquely detemrined by an explicit tuple of arithmetic datum. The main theorem will identify specific subgroups of these represenatives which are spectrally rigid, provided certain conditions on the arithemtic data are met.
\end{comment}

\subsection{Main Theorem}
\subsubsection*{Definitions}
\begin{comment}
	Let $R$ be a commutative ring (with 	$). An algebra $(A,\iota)$ over $R$ is an associative ring $A$ with 	$ equipped with an embedding $\iota : R \to A$ of rings, mapping 	 \in R$ to 	 \in A$ such that $\iota(R)$ is contained in the center of $A$. Using $\iota$, we identify $R$ with a central subring of $A$ and suppress reference to $\iota$.

A \textbf{standard involution} on $B$ is an anti-involution $\overline{\cdot}: A \to A$ such that $x\overline{x} \in A$ for all $x\in A$. Given a standard involution on $A$, define the \textbf{reduced trace},\textbf{reduced norm}, and \textbf{characteristic polynomial} of an element of $A$ by
\begin{align*}
	\trd:& A \to R, \quad \trd(x)&= x + \overline{x} \\
	\nrd:& A \to R, \quad \nrd(x)&= x\overline{x} \\
	\mu(\cdot;T):& A \to R[T] \quad \mu(x;T) &= T^2 - \tr(x) T +\nrd(x).
\end{align*}

\end{comment}
\begin{comment}
	In this section $R$ is an integral domain, and $k$ is its field of fractions, and $A$ a finite dimensional algebra over $k$ with unit element 	_A$. A finitely generated $R$-submodule $M$ of $A$ is a \textbf{lattice}, and is \textbf{full} if $kM=A$. A full lattice in $A$ is an \textbf{order} if it is a subring containing 	_A$. 

Given a full lattice $M$ in $A$, the sets
\begin{align*}
	\Ocal_\ell(M)= \{ x \in A: xM \subset M\}\\
 	\Ocal_r(M)= \{ x \in A: Mx \subset M\}
\end{align*}
are orders in $A$, and are called the \textbf{left}/\textbf{right} orders of $M$ respectively.
\begin{defn}

\end{defn}

\end{comment}
\subsubsection{Theorem}
Let $k \leq \Rbb$ be a totally real number field, and $A$ a quaternion algebra over $k$ which is unramified over the identity embedding $k\to \Rbb$, and ramified over all other infinite places, and write $A^1$ for the subgroup of $A^\times$ consisting of elements with reduced norm $1$. An isomorphism $\phi : A\otimes_k \Rbb \to M(2,\Rbb)$ of algebras, restricts to an isomorphism $(A\otimes_k \Rbb)^1 \to \SL(2,\Rbb)$ of groups. Composing $\phi$ with the canonical projection $\SL(2,\Rbb)  \to \PSL(2,\Rbb)$ we obtain a homomorphism $\psi : A^1 \to \PSL(2,\Rbb)$ which factors through $\pm 1 \lmod A^1$.  
 
For an order  $\Ocal \subset A$ in $A$, write $\Ocal^1 = A^1 \cap \Ocal$. We call the subgroup $\psi(\Ocal^1)\leq \PSL(2,\Rbb)$ the arithmetic Fuchsian group associated to the datum $k$,$A$,$\psi$,$\Ocal$. 

We write this group as $\Gamma_{k,A,\psi,\Ocal}$, or just by $\Gamma_\Ocal$ if $k,A,\psi$ are fixed. A fuchsian group is \textbf{arithmetic} provided it is commensurable to one of the form $\Gamma_{k,A,\psi,\Ocal}$, and say that it is \textbf{derived} from the quaternion algebra $A$ if is a subgroup of some $\Gamma_{k,A,\psi,\Ocal}$.

 \begin{rem}
	Some comments about this definition:
\begin{enumerate}
	\item Given any Fuchsian group $\Lambda$, there is at most one number field $k$, and quaternon algebra $A$ over $k$, such that $\Lambda$ is commensurable to $\psi(\Ocal^1)$ for some order $\Ocal$ in $A$. Thus, for arithmetic Fuchsian groups, the pair $(k,A)$ is a complete invariant of commensurability classes. 
	\item If $\Lambda$ is an arithmetic lattice derived from the quaternion algebra $k,A$, then there are only finitely many orders $\Ocal$ in $A$ such that $\Lambda$ is contained in $\psi(\Ocal^1)$.
	\item For any order $\Ocal$ in a quaterion algebra $A$, the set of maximal orders containing $\Ocal$ is finite and nonempty.
\end{enumerate}
 \end{rem}

For an integral ideal $\afrak$ in $R$, the map $\pi_\alpha: \Ocal \to \Ocal \otimes_R R/\afrak$ is a surjective ring homomorphism.   
the \textbf{principal congruence subgroup } of level $\afrak$ in an order $\Ocal$ is the subgroup of $\Ocal^1$ given by
\begin{align*}
	\Ocal^1(\afrak) = \{ x \in \Ocal^1 : x - 1 \in \afrak \Ocal^1\}.
\end{align*}

We let $\pi_\afrak$ denote the projection $\Ocal^1 \to (\Ocal \otimes_R R/\afrak)^1$.   % there is a short exact sequence  \begin{align*} 1 \to \Ocal^1(\afrak) \to \Ocal^1 \to (\Ocal \otimes_R R/\afrak)^1 \to 1 \end{align*}

If $\Lambda$ is any subgroup of $\PSL(2,\Rbb)$ which is commensurable to $\psi(\Ocal^1)$, we write $\Lambda(\afrak,\Ocal)$, or just $\Lambda(\afrak)$ if $\Ocal$ is understood, for $\ker \pi_\alpha \vert_\Lambda = \Lambda \cap \psi(\Ocal^1(\afrak))$. 
\begin{rem}
	The reduction map $R \to R/\afrak$ induces a projection $\Ocal \to \Ocal \otimes_R R/\afrak \approx \Ocal / \afrak \Ocal$ of algebras, and in turn a projection $\pi_\afrak:\Ocal^1 \to (\Ocal \otimes_R R/\afrak)^1$ of groups. Then $\Ocal^1(\afrak) = \ker \pi_\alpha$.
\end{rem}

\subsection{Length spectra of arithmetic Fuchsian groups}
Fix an arithmetic datum $(k,A,\psi,\Ocal)$ as above such that $\Ocal$ is a \emph{maximal order} in $A$, and set $\Gamma = \Gamma_{k,A,\psi,\Ocal}$. In this section, we will compute the length spectrum of the surface $X=\Gamma \lmod \Hbb$ in terms of the arithmetic of $k$. 

\subsubsection*{Geodesics as conjugacy classes}
The quotient map $\Hbb \to \Gamma \lmod \Hbb = X$ is the universal cover, and identifies $\Gamma$ with $\pi_1(X)$. Consequently, we may identify free homotopy classes of closed curves in $X$ with conjugacy classes in $X$. As $X$ has negative curvature, there is a unique geodesic representative in each free homotopy class on $X$. Thus, we may identify the set of geodesics on $X$ with conjugacy classes in $\Gamma$. Given a conjugacy class $\gamma \subset \Gamma$, we write $\ell(\gamma)$ for the length of the corresponding geodesic.  Each $g\in \gamma$ has a unique lift $\tilde{g}$ to $\SL(2,\Rbb)$ with positive trace. We write $\tr(g)$ for $\tr(\tilde{g})$, and set 
\begin{align*}
	\lambda_g = \frac{t_g + \sqrt{t_g^2 - 4}}{2}.
\end{align*}
Then the length of the geodesic corresponding to $\gamma$ is related to $\lambda_g$ by the formula:
\begin{align*}
	\lambda_g^2 = \exp(\ell(g)).
\end{align*}
 
\subsubsection*{Conjugacy classes in quaternion algebras and orders}
In this section $F$ is a number field with number field $R$, $A$ a quaternion algebra over $F$, and $\Ocal$ a maximal order in $A$.  

The algebra $A$ is equipped with an (anti-)involution $\overline{\cdot}$ and reduced norm/trace maps: 
\begin{align*}
	\trd(x) &= x+\overline{x} \\
	\nrd(x) &= x\overline{x},
\end{align*} 
and every element $x\in A$ satisfies its characterstic polynomial 
\begin{align*}
	\chi(x;T)= T^2 - \trd(x) T + \nrd (x).
\end{align*}

Given an element $x\in A$, let $F[x] \subset A$ be the smallest $F$-subalgebra of $A$ containing $x$. 
\begin{claim}	
	If $x\in A^\times$ is central, then $F[x] = F$. Otherwise, for oncentral $x$, with $\delta = \delta(x)$,
	\begin{align*}
		F[x] \approx \begin{cases}	
						F \oplus F 			& \text{if $\delta \in  (F^\times)^2$ } \\
						F[\sqrt{\delta}] & \text{if $\delta \in  F^\times \setminus (F^\times)^2$} \\
						F[\eps]/(\eps^2 ) & \text{if $\delta(x)=0$ },
					\end{cases}
	\end{align*}
	as $F$-algebras. 
\end{claim}
We say a noncentral element $x$ is split semisimple ( or $F$-hyperbolic), nonsplit semisimple (or $F$-elliptic), and parabolic correspondingly.

\begin{claim}
	If $x \in A^\times$ is noncentral, then the centralizer of $x$ in $A^\times$ is $F[x]^\times$. 
\end{claim}

We view $A$ as four dimensional affine space over $F$, and write $F[A]$ for its coordinate ring. Then $A^\times$ is the principal open subvariety of $A$ defined by the nonvanishing of $\nrd$.

For $g \in A^\times$ and $f\in F[A]$, and write  $f^g$ for the function $x \mapsto  f( g x g^\inv)$. For $x\in A$, write $O(x) = \{ x^g : g \in A^\times\}\subset A$ for its orbit under conjugation by $A^\times$. 

\begin{claim}\label{orbits}
	For noncentral $x\in A^\times$, the orbit $O(x)$ is Zariski-closed in $A$ if and only if $x$ is not parabolic. All parabolic elements in $A^\times$ constitute a single conjugacy class, and the Zariski closure of this class is zero locus of $\delta = \trd^2-4\nrd$.
% say $\Ucal$. The complement of $\Ucal$ in its Zariski closure $\overline{\Ucal}$ is the center $F^\times$ of $A^\times$. 
\end{claim}
\begin{claim}
	The algebra $F[A]^{A^\times}$ of invariants under $A^\times$ is the subalgebra $F[\trd,\nrd]$.  The map $x \mapsto (\trd(x),\nrd(x))$.
\end{claim}

Recall that the characterstic polynomial $\chi(x;T)$ has repeated roots if and only if its discriminant $\delta(x) = \trd(x)^2 - 4\nrd(x)$ is zero. 

\begin{thm}
	Let $\Gamma=\Gamma_{k,A,O}$ denote a maximal arithmetic lattice $G$, where $G=\PSL(2,\Rbb)$ or $G=\PSL(2,\Cbb)$, arising from the arithmetic datum $(k,A,O)$ (as above) such that $A$ has type number $1$. Then for all ideals $I$ in $R_k$ which are not divisible by any prime over which $A$ ramifies, nor any dividing $2$ or $3$, the hyperbolic $2$- or $3$-manifold $X(I)=\Gamma(I) \lmod G \rmod K$ is absolutely spectrally rigid.
\end{thm}
\subsection{the proof}
First, we prove the theorem for hyperbolic $2$-manifolds.

We suppose $M$ is a Riemannian manifold such that the Laplace spectra  $\LaSpec(M)$ and $\LaSpec X(I)$ are identical.

\begin{claim}\label{claim:heat}
	The dimension, volume, and mean scalar curvature of $M$ and $X(I)$ coincide. 
\end{claim}

 

\begin{lemma}
	Let $X$ be a Riemannian manifold, with Laplace spectrum $0\geq \lambda_0 \geq \lambda_1 \geq \cdots$, and let $K_X(t)=\sum_{i=1}^\infty e^{\lambda_i t}$. Then $K_X(t)$ admits an asymptotic expansion as $t\to 0^+$ of the form
	\[ K_X(t) \sim (4\pi t)^{-\dim(X)/2} \sum_{n=0}^\infty a_n(X) t^n \]
	for constants $a_i(X)$ which are integrals over $X$ of polynomials in the entries of the curvature tensor of $X$ and its covariant derivatives. The first two are:
	\[a_0(X) = \Vol(X),\quad a_1(X) = \frac{1}{6} \int_X \tr {\rm Ricc} (x) \dop \Vol(x).\]
\end{lemma}

Indeed, since $K_M(t) = K_{X(I)}(t)$ identically, we may read off the invariants in the claim from their common asymptotic expansions as $t\to 0^+$. 
\begin{rem}
	The total scalar curvature of a smooth, closed surface is its Euler characteristic by Gauss-Bonnet. As the euler characteristic is a complete invariant of the diffeomorphism type for closed surfaces, we may thus conclude that $M$ and $X(I)$ are in fact diffeomorphic.
\end{rem}

[TODO]: argue that $M$ must have constant negative curvature $=-1$. We'll do so either via one of the following routes:
\begin{itemize}
	\item look at the the full set of heat invariants, and argue that they detect variation in curvature eventually
	\item apply the theorem of patodi which says that if a closed $d$ manifold $X$ has constant negative curvature, and $M$ is isospectral to $X$ on $p$-forms for $0\leq p \leq d$, then $M$ also has constant negative curvature. Need to argue tho that isospectrality on $0$ forms implies the same for $1$ forms on surfaces (is this true in general?)
	\item apply the theorem of BCG which says that, at least for closed $2$ manifolds, within a conformal class of metrics on a closed surface $X$ with fixed volume, the entropy of the geodesic flow is minimized at the unique metric of constant curvature. Here we need to argue that that entropy is, in fact, a spectral invariant.
\end{itemize}


As $M$ is a compact hyperbolic $2$-manifold, it admits a Fuchsian uniformization $M \approx \Lambda \lmod \Hbb^2$, where $\Lambda$ is a uniform lattice in $\PSL(2,\Rbb)$. 
\begin{claim}
	After conjugating by an element of $\PSL(2,\Rbb)$ we may take $\Lambda$ to be a finite index subgroup of $\Gamma = \rho(O^1) \leq  \PSL(2,\Rbb)$. 
\end{claim}
\begin{proof}
	This claim follows from two theorems. 

	First, we apply the following theorem to conclude that $\Lambda$ is arithmetic.
	\begin{thm}[Takeuchi \cite{takeuchiCharacterizationArithmeticFuchsian1975}]\label{thm:takeuchi}
		Let $\Gamma$ be a Fuchsian group of the first kind. Then $\Gamma$ is an arithmetic Fuchsian group derived from a quaternion algebra if and only if $\Gamma$ satisfies the following conditions
	\begin{enumerate}
			\item The subfield $k$ of $\Cbb$, generated over $\Qbb$ by the traces of elements of $\Gamma$, has finite degree over $\Qbb$
			\item $\tr(\Gamma)$ is contained in the ring of integers $R_{k}$ of $k$
			\item For any isomorphism $\phi: k \to \Cbb$ such that $\phi \neq \id$, the set $\phi(\tr(\Gamma))$ is bounded in $\Cbb$.
	\end{enumerate}
	\end{thm}
	As these conditions are satisfied for $\Gamma(I)$, and $\tr(\Gamma(I)) = \tr(\Lambda)$, there are also satisfied for $\Lambda$. Thus, $\Lambda$ is a finite index subgroup of the norm one units of \emph{some} maximal order in \emph{some} quaternion algebra over $k$. 
	
	The second theorem demonstrates that this quaternion algebra is in fact $A$.
	\begin{thm}[Reid \cite{reidIsospectralityCommensurabilityArithmetic1992}]
		Let $M_1$ and $M_2$ be isospectral arithmetic hyperbolic $2$ or $3$ manifolds. Then $M_1$ and $M_2$ are commensurable.
	\end{thm}
	For arithmetic hyperbolic $2$- and $3$-manifolds, commsurability is tantamount to an isomorphism of invariant quaternion algebras. 
	
	From theorem \ref{thm:takeuchi}, $\Lambda$ is contained in $\rho(O'^1)$ for some maximal order $O'$ in $A$. Since $A$ has type number $1$, all maximal orders in $A$ are conjugate (by an element of $A^\times$). Consequently we may take $O'=O$, so that $\Lambda \leq \Gamma = \rho(O^1)$ as claimed. 
\end{proof}

% From claim \ref{claim:heat},  $\vol(\Lambda \lmod \Hbb^2) = \vol(\Gamma(I) \lmod \Hbb^2)$ and each of these is equal to the respective group's index in $\Gamma$, we find $|\Gamma: \Gamma(I)| = |\Gamma: \Lambda|$.

% \begin{rem}
% 	As $\Gamma(I)$ and $\Lambda$ are both contained in $\Lambda = PO^1 = \pm 1 \lmod O^1$, there is no harm in lifting everything to $O^1$. We do so, and reuse the same notation for lifts.
% \end{rem}

For each prime $\pfrak$ of $R$, and subset $E$ of a vectorspace $V$ over $k$, let $\Clo_\pfrak(E)$ denote its closure in $V\otimes k_\pfrak$. In particular, we set $\Lambda_\pfrak = \Clo_\pfrak(\Lambda)\leq A_\pfrak = A\otimes k_\pfrak$.

\begin{comment}
	At this point, we do not know if $\Lambda$ is a \emph{congruence} subgroup of $\Gamma$. That is, we do not know if $\Lambda = A^1 \cap \prod_{\pfrak} \Clo_\pfrak(\Lambda) = \Lambda$ (product taken inside $(A\otimes \Abb_k)^1$)

(Equivalently, we do not know if $\Gamma(J) \leq \Lambda$ for some ideal $J$ of $R$)
\end{comment}

	Let $\tilde{\Lambda}$ be the \textbf{congruence closure} of $\Lambda$ in $\Gamma$. That is, $\tilde{\Lambda}$ is the smallest congruence subgroup of $\Gamma$ containing $\Lambda$. Alternatively, one has the description of $\tilde{\Lambda}$ as $\tilde{\Lambda} = \bigcap_{J \triangleleft R}\Gamma(J)\Lambda$.

	\begin{claim}\label{claim:modJ}
		Let $J$ be an ideal in $R$  and $H$ a subgroup of $\Gamma$. Then the following are equivalent: 
		\begin{enumerate}
			\item $H\Gamma(J) = \Gamma$
			\item $H\cap \Gamma(J) = \Gamma(J)$
			\item The restriction of the natural projection $\Gamma \to \Gamma/\Gamma(J)$ to $H$ is surjective
		\end{enumerate}
		Furthermore, if $J=\prod_\pfrak \pfrak^{j_\pfrak}$ then each of these conditions holds if and only if it holds at $\pfrak^{j_\pfrak}$. 
	\end{claim}

\begin{claim}
	Let $J$ be an ideal in $R$ which is coprime to $I$. Then $\Lambda \Gamma(J) = \Gamma$. 
\end{claim}
\begin{proof}
	By the previous claim, it suffices to assume that $J= \pfrak^n$ is a prime power. 
\end{proof}


\begin{comment}
	\begin{rem}
	As $\Lambda$ and $\Gamma(I)$ are contained in $\Gamma$ and $\Gamma(I)$ is normal in $\Gamma$, the subgroup $\Gamma(I)\Lambda$ of $\Gamma$ is a congruence subgroup containing $\Lambda$.
	Consequently, $\tilde{\Lambda}\leq \Gamma(I)\Lambda$. 
	Further, if we write $\Lambda(I) = \Gamma(I) \cap \Lambda$ for the kernel of the restriction to $\Lambda$ of $\pi_I$ (reduction mod $I$), then $\Gamma(I)\Lambda/\Gamma(I) \approx \Lambda / \Lambda(I)$ may be identified with a subgroup of $\SL(2,R/I)$ in such a way that the mod-$I$ reduction of trace on $\Lambda$ (as an element of $A$) coincides with the trace on $\SL(2,\R/I)$ %(as autormophisms of the free $R/I$ module $(R/I)\bigoplus (R/I)$.)
\end{rem}
\end{comment}


\begin{lemma}
	Let $\pfrak$ be a prime ideal of $R$, not dividing $2$ or $3$. Suppose $H$ is a closed subgroup of $\SL(2,R_\pfrak)$ such that $\tr(H) = R_\pfrak$. Then $H = \SL(2,R_\pfrak)$.
\end{lemma}
\begin{proof}
	The claim will follow from the analogous statement over the residue field $\kfrak = R/\pfrak$. Indeed, if $H$ is a closed, proper subgroup of $\SL(2,R_\pfrak)$ for which $\tr(H) = R_\pfrak$, then its reduction $H(\pfrak)$ mod $\pfrak$ would be a proper subgroup of $\SL(2,\kfrak_\pfrak)$ for which $\tr H(\pfrak) = \kfrak_\pfrak$. 

	Suppose then that $G \leq \SL(2,\Fbb_q)$ is a subgroup for which $\tr(G) = \Fbb_q$. Let $\zeta$ be a primitive $q-1$st root of unity in $\Fbb_q$, and let $\delta$ be a primitive $q+1$st root of unity in the unique quadratic extension $\Fbb_{q^2}$ over $\Fbb_{q}$. Let $t_\zeta = \zeta+ \zeta^\inv$ and $t_\delta = \delta +\delta^\inv$
	
	As $\tr(G) = \Fbb_q$, there exist elements $g_\zeta$ and $g_\delta$ in $K$ such that $\tr(g_\zeta)=t_\zeta$ and $\tr(g_\delta)= t_\delta$ respectively. Note that $g_\zeta$ (resp. $g_\delta$) has order $q-1$ (resp. $q+1$), as can be seen by diagonalizing. Set $A = \langle g_\zeta\rangle$ and $K = \langle g_\delta \rangle$, and note that $A\cap K = \{1\}$. 
	
	Let $C$ be the subgroup of $G$ generated by $A$ and $K$. Then $|C|\geq (q-1)(q+1)=|A||K|$, so that $|\SL(2,\Fbb_q) : C| \leq q$. As the minimal index of a proper subgroup of $\SL(2,\Fbb_q)$ is $q+1$, we conclude that $C$, and therefore $G$, is equal to $\SL(2,\Fbb_q)$. 
\end{proof}


\section{local computations}
In this section, $k$ is a nonarchimedian local field of characteristic zero, $\nu$ a normalized additive valuation, $R$ its valuation ring, $\mfrak$ the unique maximal ideal of $R$, uniformizer $\varpi \in \mfrak$, and residue field $\kfrak$ of characterstic $p$ and cardinality $q$.


\subsection*{Split case}
 In this section, let $G=\SL(2,k)$. Define closed subgroups
\begin{align*}
 A & = \{ \Tbt{x}{}{}{x^\inv}: x \in k^\times \} \\
 U & = \{ \Tbt{1}{y}{}{1}: y \in k \} \\
 B & = \{\Tbt{x}{y}{}{x^\inv} : x \in k^\times, y \in k\}=AU,
\end{align*}
and elements 
\begin{align*}
	w &= \Tbt{0}{-1}{1}{0} \\ 
	\alpha &= \Tbt{\pi}{}{}{1}  
\end{align*} 
of $\GL(2,k)$.

For a positive integer $n$, let $\pi_n : \SL(2,R) \to \SL(2,R/\mfrak^n)$ be the natural projection. For any subgroup $H$ of $G$, we set $H(R) = H \cap \SL(2,R)$, and $H(\mfrak^n)= \ker (\pi_n\vert_H)$. 

\begin{claim}
	(a) $K=\SL(2,R)$ is a maximal compact open subgroup of $G$. (b) The subgroups $K(\mfrak^n)$ constitute neighborhood basis of the identity in $K$. (c) There are exactly two conjugacy classes of maximal compact open subgroups of $G$. Representatives for these classes are given by $K$ and $\alpha K \alpha^\inv$.
\end{claim}






\subsection*{poo}
Let $V$ be a $2$-dimensional vectorspace over $k$. Then $\End_k(V)$ is a quaternion algebra over $k$. Let $\Mcal$ denote the set of maximal orders in $\End_k(V)$.

For any lattice $L$ in $V$, the set
\begin{align*}
	\End(L) = \{x \in \End_k(V): x L \leq L\}
\end{align*}
is a maximal order in $\End_k(V)$, and every maximal order in $\End_k(V)$ takes the form $\End(L)$ for some lattice $L$ in $V$. Thus, the map $L \mapsto \End(L)$ is a surjection from the set  $\Lcal$ of lattices in $V$ to the set $\Mcal$ of maximal orders in $\End_k(V)$. 

Two lattices $L,M$ satisfy $\End(L)=\End(M)$ if and only if $L=s M$ for some $s\in k^\times$. In this case, $L$ and $M$ are said to be \textbf{homothetic}, and we write $[L]$ for the homothety class of a lattice $L$.  Evidently, the map $L \mapsto \End(L)$ factors through the projection $L\mapsto [L]$ and establishes a bijection between the set $\Hcal$ of homothety classes of lattices in $V$ and the set $\Mcal$ of maximal orders in $\End_k(V)$. 

There is a natural action of $\PGL(V)$ on $\Hcal$ by left muliplication, and on $\Mcal$ by conjugation. The bijection $\Hcal \to \Mcal$ described above is $\PGL(V)$ equivariant. The stabilizer of a homothety class $\Lambda$ is $\PGL(L) = R^\times \lmod \GL(L)$ where $L$ is a lattice representing $\Lambda$, and $\GL(L) = \{x \in \End_k(V): xL = L\}$, and the stabilizer of a maximal order $\Ocal$ is the image in $\PGL(V)$ of the group $\{x \in \GL(V) : x \Ocal x^\inv = \Ocal\}$. 



% For a pair of homothety classes $\Lambda,\Lambda' \in \Hcal$, choose representatives $L \in \Lambda,L'\in\Lambda'$ so that $L'\leq L$. Then by the invariant factor theorem over $R$, we may pick a basis $u,v$ for $L$ such that $\pi^a u, \pi^b v $ is a basis for $L'$. The quantity $|a-b|$ is independent of the choice of representatives, and defines a distance function $d$ on $\Hcal$. 

We say that two classes $\Lambda,\Lambda' \in \Hcal$ are adjacent, and write $\Lambda\sim\Lambda'$ if there exist representative lattices $L,L'$ such that $\pi L< L'< L$ with strict containment, and we say that two maximal orders $\Ocal,\Ocal' \in \Mcal$ are adjacent, and write $\Ocal\sim \Ocal'$ if  $|\Ocal:\Ocal\cap \Ocal'| = |R:\mfrak|=q$.

\begin{proposition}
	The bijection $\Hcal \to \Mcal$ described above is a $\PGL(V)$ equivariant isomorphism of graphs.  
\end{proposition}



For a pair of maximal orders $\End(L)$ and $\End(L')$, the set 
\begin{align*}
	\Hom(L,L')= \{ x \in \End_k(V): xL\leq L'\}
\end{align*}
is a right $\End(L)$ module, and a left $\End(L')$ module. The quotients  





\section{Some diagrams}
% https://q.uiver.app/?q=WzAsMTIsWzAsMCwiMSJdLFsxLDAsIlxcR2FtbWEoSikiXSxbMiwwLCJcXEdhbW1hIl0sWzMsMCwiXFxHYW1tYS9cXEdhbW1hKEopIl0sWzQsMCwiMSJdLFswLDEsIjEiXSxbMSwxLCJcXExhbWJkYShKKSJdLFsyLDEsIlxcTGFtYmRhIl0sWzMsMSwiXFxMYW1iZGEvXFxMYW1iZGEoSikiXSxbNCwxLCIxIl0sWzEsMiwiXFxHYW1tYShKKVxcY2FwIFxcTGFtYmRhIl0sWzMsMiwiXFxMYW1iZGFcXEdhbW1hKEopL1xcR2FtbWEoSikiXSxbMCwxXSxbMSwyXSxbMiwzXSxbMyw0XSxbNSw2XSxbNiw3XSxbNyw4XSxbOCw5XSxbNiwxLCIiLDEseyJzdHlsZSI6eyJ0YWlsIjp7Im5hbWUiOiJob29rIiwic2lkZSI6InRvcCJ9fX1dLFs3LDIsIiIsMSx7InN0eWxlIjp7InRhaWwiOnsibmFtZSI6Imhvb2siLCJzaWRlIjoidG9wIn19fV0sWzgsMywiZV9cXExhbWJkYShKKSIsMix7InN0eWxlIjp7InRhaWwiOnsibmFtZSI6Imhvb2siLCJzaWRlIjoidG9wIn19fV0sWzYsMTAsIiIsMSx7ImxldmVsIjoyLCJzdHlsZSI6eyJoZWFkIjp7Im5hbWUiOiJub25lIn19fV0sWzgsMTEsIiIsMCx7ImxldmVsIjoyLCJzdHlsZSI6eyJib2R5Ijp7Im5hbWUiOiJzcXVpZ2dseSJ9LCJoZWFkIjp7Im5hbWUiOiJub25lIn19fV1d
\[\begin{tikzcd}
	1 & {\Gamma(J)} & \Gamma & {\Gamma/\Gamma(J)} & 1 \\
	1 & {\Lambda(J)} & \Lambda & {\Lambda/\Lambda(J)} & 1 \\
	& {\Gamma(J)\cap \Lambda} && {\Lambda\Gamma(J)/\Gamma(J)}
	\arrow[from=1-1, to=1-2]
	\arrow[from=1-2, to=1-3]
	\arrow[from=1-3, to=1-4]
	\arrow[from=1-4, to=1-5]
	\arrow[from=2-1, to=2-2]
	\arrow[from=2-2, to=2-3]
	\arrow[from=2-3, to=2-4]
	\arrow[from=2-4, to=2-5]
	\arrow[hook, from=2-2, to=1-2]
	\arrow[hook, from=2-3, to=1-3]
	\arrow["{e_\Lambda(J)}"', hook, from=2-4, to=1-4]
	\arrow[Rightarrow, no head, from=2-2, to=3-2]
	\arrow[Rightarrow, squiggly, no head, from=2-4, to=3-4]
\end{tikzcd}\]










\begin{comment}
% \part{temp}

% \section{probably useful references}
% \begin{itemize}
% 	\item \cite{chavelEigenvaluesRiemannianGeometry1984} for background on Riemannian geometry.
% 	\item \cite{patodiCollectedPapersPatodi1996} for reflection of curvature variation in spectrum on forms.
% \end{itemize}
\end{comment}
\part{dumpster}
\paragraph{Unramified local computations} In this paragraph, $k$ is a nonarchimedian local field of characteristic $0$, maximal compact subring $R$, with maximal ideal $\mfrak$ with residue field $\ffrak = R / \mfrak$ having characteristic $p$ and cardinality $q$. We fix a uniformizer $\pi \in \mfrak$.

Recall that:
\begin{itemize}
	\item A quaternion algebra $\Acal$ over $k$ is a $4$-dimensional central simple $k$-algebra. Any such algebra is equipped with a canonical pair of homomorphisms $\nrd: \Acal^\times \to k^\times$ and $\trd:\Acal \to k$ called the \textbf{reduced norm} and \textbf{reduced trace}, respectively. We extend $\nrd$ to $\Acal$ by setting $\nrd(x)=0$ if $x\in \Acal \setminus \Acal^\times$. Furthermore, $\Acal$ is equipped with an anti-involution $\sigma$, which is related to $\trd$ and $\nrd$ by the formulas:
	      \begin{align}
		      \trd(x) = x+x^\sigma, \quad \nrd(x)=xx^\sigma.
	      \end{align}
	\item An $R$-order $\Rcal$ in a quaternion algebra $\Acal$ over $k$ is a unital subring which contains $R$ and is such that $k\Rcal = \Acal$, where $k\Rcal$ is the $k$-subalgebra of $\Acal$ generated by $\Rcal$. We say $\Rcal$ is a \textbf{maximal order} if it is not properly contained in any other order.
	\item An order in a quaternion algebra is an \textbf{Eichler Order} if it is the intersection of two (not necessarily distinct) maximal orders. If $\Rcal=\Rcal_1 \cap \Rcal_2$ is an Eichler order, then $\Rcal$ is a maximal subring of both $\Rcal_1$ and $\Rcal_2$.
\end{itemize}
We define two algebras which will serve as representatives of the two isomorphism classes of quaternion algebras over $k$.

\paragraph{Split case} Let $V$ be a $2$-dimensional vectorspace over $k$. Then $\Acal=\End_k(V)$ is \textbf{the split quaternion algebra} over $k$. %If we pick a basis $e_1,e_2$ for $V$, and dual basis $e_1^*,e_2^*$ of $V^*$ and define coordinate functions $\Acal \to k$ by $a(x) = e_1^*(x e^1)$, $b(x) = e_2^*(x e^1)$, $c(x) = e_1^* (x e^2)$, $d(x) = e_2^*(x e_2)$. Relative to these coordinates,  one may identify $\Acal$ with the algebra $M(2,k)$ of $2\times 2$ matrices via the assignment $x \mapsto \tbt{a(x)}{b(x)}{c(x)}{d(x)}$.


We say a finitely generated $R$-submodule $L$ of $V$ is a(n R-)lattice (in $V$) if $kL=V$. Then $L$ is a lattice if and only if it is a compact open subgroup of $V$. For a lattice $L$, we set
\begin{align}
	\End(L)=\{x \in \End_k(V): xL\subset L\}
\end{align}
and for any other lattice $M$, we set
\begin{align}
	\Hom(L,M)=\{x \in \End_k(V): xL\subset M\}
\end{align}
and remark that these are understood \emph{as subrings of} $\End_k(V)$, not merely as abstract rings as the notation might suggest.


%\begin{def/prop}\label{def/prop:MaximalOrdersAndHomothetyClasses}
For any lattice $L$ in $V$, the subring $\End(L)$ is a maximal order in $\End_k(V)$. Coversely, any maximal order takes the form $\End(L)$ for some lattice $L$ in $V$. If $\End(L) = \End(M)$ for two lattices $L,M$, then there exists an $x\in k^\times$ such that $M=xL$. In this case, we say that $M$ and $L$ are \textbf{homothetic}. For a lattice $L$, we write $[L]= \{xL : x\in k^\times\}$ for its homothety class.

If we write $\Bcal(V) \subset V \times V$ for the set of ordered bases of $V$,  $\Lcal(V)$ for the set of lattices in $V$, and $\Hcal\Lcal(V)$ for the set of homothety classes of lattices in $V$, then there are projections:
\begin{itemize}
	\item $\Span_R:\Bcal(V) \to \Lcal(V)$, sending an ordered basis $(u,v)$ to the lattice $Ru+Rv$, and
	\item $[\cdot]:\Lcal(V) \to \HLcal(V)$, sending a lattice to its homothety class.
\end{itemize}

The group of units $\End_k(V)^\times$ in the quaternion algebra $\End_k(V)$ acts simply and transitively on $\Bcal(V)$. As the maps $\Bcal(V)\to \Lcal(V)$ and $\Lcal(V) \to \HLcal(V)$ are both $\GL(V)$ equivariant surjections, we find that $\Aut_k(V)$ acts transtively on these latter two.

For a lattice $L$, its stabilizer in $\Aut_k(V)$ is the group of units $\End(L)^\times$ in the maximal order $\End(L)$. Thus, we may identify $\Lcal(V)$ with \[\End_k(V)^\times / \End(M)^\times\].
In this model, the projection $M\mapsto [M]$ of a lattice to its homothety class amounts to the projection
\[\End_k(V)^\times / \End(M)^\times \to k^\times \lmod \End_k(V)^\times / \End(M)^\times\]
with $k^\times$ acting by scalars. As this action is central in $\End_k(V)$, we may write this latter quotient as  $\End_k(V)^\times / k^\times \End(M)^\times$.

	% For a lattice $L \in \Lcal(V)$, the group $\GL_R(M) = \End(M)^\times$ acts simply and transitively on the fiber 
	% \begin{align}
	% 	\span_R^\inv(M)=\{(u,v) \in \Bcal(V): Ru+Rv=M\}.
	% \end{align}

The group $k^\times$ acts transitively on the fibers of the projection $\Lcal(V) \to \HLcal(V)$ by definition, however, it does not do so freely. Rather, the subgroup $R^\times$ of $k^\times$ acts 			
\begin{rem}
	As $\Bcal(V)$ is a (Zariski open) subset of $V\times V \setminus {(0,0)}$ which is $k^\times$-stable, there is also a map $\Bcal(V) \to \Pbb(V\times V) \approx \Pbb^3(k)$....
\end{rem} 

Now, Write $\Mcal\Ocal(V)$ for the set of maximal orders in $\End_k(V)$.  The group $\End_k(V)^\times$ acts on $\mathcal{MO}(V)$ by conjugation $\Rcal \mapsto \Rcal^g=g\Rcal g^\inv$.
\begin{prop}
	The map $L \to \End(L)$ induces an $\End_k(V)^\times$ equivariant bijection between $\HLcal(V)$ to $\MOcal(V)$.
\end{prop}
As $\End_k(V)^\times$ acts transitively on the former, it also does so on the latter $\MOcal(V)$. In other words, we have:
\begin{prop}
	Any two maximal orders in $\End_k(V)$ are conjugate.
\end{prop}

To make all of this explicit, pick basis vectors $e_1,e_2$ and identify $V$ with $k^2$. Then $\End_k(V)$ is identified with the algebra $M(2,k)$ of $2\times 2$ matrices over $k$, the lattice $L_o=Re_1+Re_2$ corresponds to $R^2 \subset k^2$, and the maximal order $\End(L_o)$ corresponds to the subring $M(2,R)$ of $M(2,k)$ with entries in $R$. We may then identify $\MOcal(V)$ with the quotient  $\GL(2,k) / k^\times \GL(2,R)$.



%\item All maximal orders are conjugate by an element of $\Aut_k(V)$. 
%is an order, and is equal to $\End(L) \cap \End(M)$. In particular, it is a suborder of $\End(L)$ and $\End(M)$ and is the maximal suborder with this property. 
%It is contained in both $\End(L)$ and $\End(M)$ and is the maximal suborder of $\End(L)$ and $\End(M)$ with this property. In other words, $\Hom(L,M) = \End(L)\cap \End(M)$.

% \end{def/prop}



%Let $V$ be a $2$-dimensional vector space over $k$.  and fix basis vectors $e_1$, $e_2$ for $V$.

%We define an isomorphism $\cdot^t:V \to V^*$ extending the assignment of the basis $\{e_1,e_2\}$ to its dual basis $\{e_1^t,e_2^*\}$. 


\bibliography{references}{}
\bibliographystyle{plain}

\end{document}
