\documentclass{article}
\usepackage{amsthm,amssymb,amsmath,amsfonts,braket}
\usepackage{mycros}
\usepackage{graphicx} % Required for inserting images
\usepackage{url}
\usepackage{verbatim}
\usepackage{todonotes}
\usepackage{mathtools,thmtools}
\usepackage{tikz-cd}
\usepackage{quiver}
\usepackage{comment}

\newcommand{\LaSpec}{\operatorname{Spec}_\Delta}
\newcommand{\Vol}{\operatorname{Vol}}
\newcommand{\Comm}{\operatorname{Comm}}
\newcommand{\Clo}{\operatorname{Clo}}
\newcommand{\Autbf}{\operatorname{\mathbf{Aut}}}
\newcommand{\MOcal}{\mathcal{MO}}
\newcommand{\HLcal}{\mathcal{HL}}
\newcommand{\Span}{\operatorname{Span}}
\newcommand{\CDef}{\operatorname{Cdef}}
\newcommand{\Int}{\operatorname{Int}}
\newcommand{\disc}{\operatorname{disc}}
\newcommand{\hilbert}[3]{\left(\frac{#1, #2}{#3}\right)}
\newcommand{\sang}[1]{\langle\langle #1\rangle\rangle}
\newcommand{\sdang}[1]{\langle\langle\langle\langle #1 \rangle\rangle\rangle\rangle}
\newcommand{\Ram}{\operatorname{Ram}}
\newcommand{\Sing}{\operatorname{Sing}}
%\newcommand{\nrd}{\operatorname{nrd}}
%\newcommand{\trd}{\operatorname{trd}}
\title{SL2}
\author{justin katz}
\date{March 2023}

\setlength{\parindent}{0 em}
\setlength{\parskip}{6 pt}
\begin{document}
Notation as in \cite{sallyFourierTransformOrbital1983}:
\begin{itemize}
    \item $k$ p-adic fifeld, $R$ ring of integers, $P$ prime ideal, $q$ cardinality of residue field, $U = R^\times$ and $U_n = 1 + P^n$, $\dop x$ an additive haar measure with $\int_R \dop x =1$, define $|\cdot|$ on $k$ by: if $a\in k^\times$ then $\dop(ax) = |x| \dop x$ and $|0|=0$. 
    \item $\tau$ is a prime elemenmt, and $\eps$ is a fixed primitive $q-1$st root of $1$. 
    \item $G = \SL(2,k)$, and $G'$ the set of regular elements in $G$. 
\end{itemize}

For $\gamma \in G'$ or unipotent, an \textbf{orbital integral} is
\begin{align*}
    I_f(\gamma) = \int_{G/G_\gamma}f(x \gamma x^\inv)\dop \dot{x} 
\end{align*}
for $f$ locally constant compactly supported on $G$, and $\dop \dot{x}$ is left $G$ invariant measure on $G/G_\gamma$. When $\gamma \in G'$, then $G_\gamma =T$ is a maximal torus in $G$, and the \textbf{invariant integral} of $f$ relative $T$ is 
\begin{align*}
    F_f^T(t) = |D(t)|^{1/2} \int_{G/T} f(xtx^\inv)\dop \dot{x}
\end{align*}
for $t\in T' = T\cap G'$, where $D(x) = (\lambda-\lambda^\inv)^2$, if $\lambda^{\pm 1}$ are eigenvalues of $x$, is the \textbf{weyl discriminant}. 

For fixed $\gamma$, the map $f\mapsto I_f(\gamma)$ is a $G$-conjugation invariant distribution. 

Define subgroups of $G=\SL(2,k)$:

\begin{align*}
    A &= \set{\Tbt{x}{}{}{x^\inv}: x \in k^\times},\quad& \cdots   \\
    T_\tau &= \set{\Tbt{x}{y}{\tau y}{x}: x,y \in k^\times},    &T_\tau^\# = \set{\Tbt{x}{\eps y}{\tau\eps^\inv y}{x}: x,y \in k^\times} \\
    T_\tau &= \set{\Tbt{x}{y}{\tau y}{x}: x,y \in k^\times}, &T_\tau^\# = \set{\Tbt{x}{\eps y}{\tau\eps^\inv y}{x}: x,y \in k^\times} 
\end{align*}


\bibliographystyle{plain}
\bibliography{references}
\end{document}
