% !TeX root = main.tex
\section{Geometric Preliminaries}
\paragraph{Laplace-spectrum}
Let $(M,g)$ be a compact Riemannian manmifold. The eigenvalues of the Laplace operator of the Laplace operator $\Delta_{M,g}$ acting on $L^2(M,\dop \Vol_g)$ form a discrete set $\spec_{M,g}$ of nonnegative real numbers, tending to $\infty$. For each $\lambda \in \spec_{M,g}$, the dimension $m_{\Delta_{M,g}}(\lambda)$ of the $\lambda$-eigenspace $E_{M,g}(\lambda)=\ker (\Delta_{M,g}- \lambda \id)$ in $L^2(M, \dop \Vol_g)$ is finite. We encode the this data in the \textbf{Laplace-spectral counting function}, defined for $x\in \Rbb_{\geq 0}$ by
\begin{align}
    \pi_{\Delta_{M,g}}(x) = \sum_{\lambda \in \spec(\Delta_{M,g}) \cap [0,x]} \dim \ker (\Delta_{M,g} - \lambda \id )
\end{align}


% We say that two Riemannian manifolds $(M,g)$ and $(M',g')$ are Laplace-\textbf{isospectral} if $\spec_{M,g} = \spec_{M,g'}$. We say that $(M,g)$ is \textbf{Laplace-spectrally rigid} in the absolute sense, if any Riemannian manifold which isospectral to $(M,g)$ is in fact isometric to $(M,g)$.

\paragraph{Length spectrum}
Let $(M,g)$ be a closed Riemannian manifold.  Within each free homotopy class $\gamma$ of closed curves on $M$ there is a geodesic representative of minimal length; we write $\ell_g(\gamma)$ for that length. For a positive real number $l$, we write $\Lspec_{M,g}(\ell)$ for the number of free homotopy classes $\gamma$ of closed curves in $M$ for which $\ell_g(\gamma) = l$.


Two compact Riemannian manifolds $(M,g)$, $(M',g')$  are said to be \textbf{Laplace-isospectral} (resp. \textbf{length-isospectral})  if $\spec_{M,g} = \spec_{M',g'}$ (resp.  if $\Lspec_{M,g}=\Lspec_{M',g'}$). We say that $(M,g)$ is Laplace- (resp. length-) spectrally rigid if any $(M',g')$ to which it is Laplace- (resp. length-)isospectral, is in fact isometric to $(M,g)$.



\begin{remark}\label{remark:metricInvariants}
    Both the Laplace spectrum $\spec_{M,g}$ and the length spectrum $\Lspec_{M,g}$ are invariants of the Riemannian structure on $M$.
\end{remark}

\paragraph{abstract trace-formulae}
Let $D$ be an unbounded, self-adjoint operator acting on a hilbert space $\Hcal$ and suppose that the spectrum of $D$ is a discrete set of nonnegative real numbers, each occuring with finite multiplicity. We write $\spec D$ for the sequence $\lambda_0 \leq \lambda_1 \leq \dots$ of eigenvalues, repeated according to their multiplicity. Then there exists an orthonormal basis $\phi_k$ of eigenvectors 

Let $(M,g)$ be a compact Riemannian manifold, $\tilde{M}$ its universal cover, and $\Gamma =\pi_1(M)$ its fundamental group. There is a unique metric $g^*$ such that the covering map $(\Mti, \pubo{g}) \to (M,g)$ is a local isometry.  The fundamental group $\Gamma$ acts on $(\Mti,\pubo{g})$ by isometries and induces an isometric identification of $\Gamma lmod \Mti$ with $M$.

We identify functions on $M$ with $\Gamma$ invariant functions on $\Mti$.  For a smooth compactly supported function $f \in C_c^\infty(\Mti)$, the function $P_\Gamma(f) : x \mapsto \sum_{\gamma \in \Gamma}f(\gamma x)$ descends to a smooth function on the quotient $\Gamma \lmod \Mti \approx M$. The operator $P_\Gamma:C_c^\infty(\Mti)\to C^\infty(M)$ is surjective, and satisfies the relation
\begin{align}\label{eq:average}
    \int_{\Mti} f \dop \Vol_{\pubo{g}}  = \int_{\Gamma \lmod \Mti} P_\Gamma (f) \dop \Vol_{g}.
\end{align}
























\paragraph{Hyperbolic $2$- and $3$-manifolds}

\subparagraph{Surfaces}
We take the upper half plane
\begin{align*}
    \Hfrak = \{ x+iy \in \Cbb : y >0\}
\end{align*}
equipped with the Riemannian metric
\begin{align*}
    g_\hyp = \frac{\dop x^2 +\dop y^2}{y^2}
\end{align*}
is a model for the hyperbolic plane. The invariant volume element for this metric is given by
\begin{align*}
    \dop \vol_{g_\hyp} = \frac{\dop x \dop y}{y^2},
\end{align*}
and the Laplace operator is
\begin{align*}
    \Delta_{g_\hyp} = \frac{1}{y^2}\left( \frac{\partial^2 }{\partial x^2}+ \frac{\partial^2 }{\partial y^2} \right)
\end{align*}

The group
\begin{align*}
    \SL(2,\Rbb) = \left\{ \Tbt{a}{b}{c}{d}: ad-bc =1 \right\}
\end{align*}
acts isometrically and transitively on $(\Hfrak,\dop s^2)$ by linear fractional transformations
\begin{align*}
    \Tbt{a}{b}{c}{d} z = \frac{az+b}{cz+d} %= \frac{ac |z|^2 +(ad+bc)\re z  +bd}{|cz+d|^2} + i \frac{\im z}{|cz+d|^2}
\end{align*}
where $\tbt{a}{b}{c}{d} \in \SL(2,\Rbb)$ and $|\cdot|$ is the usual absolute value on $\Cbb$. This action factors through $\PSL(2,\Rbb)$ which can be identified with the full group of orientation preserving isometries of $(\Hfrak,\dop s^2)$. The stabilizer of $i$ in $\SL(2,\Rbb)$ is
\begin{align*}
    \SO(2) = \{\Tbt{c}{s}{-s}{c}: c^2 + s^2 =1\}
\end{align*}
so that the map $g \mapsto g i$ establishes an identification $\SL(2,|Rbb) / \SO(2) \approx \Hfrak$.
% Under this identification,the tangent space $T_i \Hfrak$ is identified with the quotient $\sl(2,\Rbb) / \so(2)$ where
% \begin{align*}
%     \sl(2,\Rbb) = \{ X \in M(2,\Rbb) : \tr X =0 \}
% \end{align*}
% and
% \begin{align*}
%     \so(2) = \{ X \in \sl(2,\Rbb) : X + ^t X = 0\}
% \end{align*}

% [OVEREXPOSITING?]

If $(M,g)$ is a compact Riemannian manifold let
\begin{align}\label{eq:universalCover}
    \tilde{M} \to M
\end{align}
be its universal cover, and let $g^*$ be the metric on $\Mti$ pulled back along \ref{eq:universalCover}. Then $\pi_1(M)$ acts freely and properly discontinuously on $(\Mti,g^*)$ by isometries, transitively on the fibers of \ref{eq:universalCover}. Consequently, one has an isometric identification
\begin{align}
    \pi_1(M) \lmod \tilde{M} \approx M.
\end{align}

If $M$ is a surface, and $g$ has constant scalar curvature $-1$, then there exists an isometry
\begin{align}\label{eq:uniformization}
    (\tilde{M},g^*) \to (\Hfrak , g_\hyp)
\end{align}
Pushing forward the action of $\pi_1(M)$ on $\Mti$ through the isometry \ref{eq:uniformization}, one obtains a homomorphism $\rho: \pi_1(M) \to \PSL(2,\Rbb) =\Isom(\Hfrak,g_\hyp)$ which induces an isometry of $(M,g)$ with $(\rho(\pi_1(M))\lmod \Hfrak,g_\hyp)$. The Riemannian structure  $(M,g)$ is uniquely determined by the $\PSL(2,\Rbb)$-conjugacy class of the homomorphism $\rho$. We refer to $\rho$ as a \textbf{Fuchsian}-representation of $\pi_1(M)$, and the map $\Hfrak \to \rho(\pi_1(M)) \lmod \Hfrak$ as a \textbf{uniformization} of $(M,g)$.




\paragraph{3-manifolds}
[TODO]
% Note that at each place $\nu \in V_k \setminus \Ram(A)$ one has isomorphisms
% \begin{align*}
%     \tilde{\Gbf}(k_\nu)\approx \GL(2,k_\nu),\quad \Gbf(k_\nu) \approx \SL(2,k_\nu), \bar{\Gbf}(k_\nu) \approx \PGL(2,k_\nu).
% \end{align*}
\section{Length spectrum, trace formulae}

Let $(M,g)$ be a compact Riemannian manifold with negative scalar curvature. The set of lengths of closed geodesics on $M$ is a discrete subset of $\Rbb_{>0}$ and for each length $\ell$, the number $\Lspec_{M,g}(\ell)$ of geodesics on $M$ with length $\ell$ is finite. We refer to the function $\Lspec_{M,g}$ on $\Rbb_{>0}$ as the \textbf{length spectrum} of $(M,g)$.  We say that two negatively curved closed Riemannian manifolds $(M,g)$ and $(M',g')$  are \textbf{length-isospectral} provided $\Lspec_{M,g} = \Lspec_{M',g'}$.


Let $\Kbb = \Rbb$ or $\Cbb$, and let $G=\SL(2,\Kbb)$, $K$ be a maximal compact subgroup of $G$, and write $X = G/K$ for the associated symmetric space, equipped with its metric of constant curvature $-1$.

$A$ be the copy of $\Kbb^\times$ embeded in $G$ via the $a(\lambda) = \tbt{\lambda}{0}{0}{\lambda^\inv}$.





\newpage
Let $(M,g)$ be a $d$-dimensional closed Riemannian manifold without boundary.  Letting $\dop \Vol_g$ denote the measure on $M$ associated to the volume form of $(M,g)$, we equip the space $C^\infty(M)$ of smooth complex valued functions on $M$ with the inner product
\begin{align}\label{eq:ip}
    \ip{f}{g} = \int_M f(x) \overline{g}(x) \dop \Vol_g(x)
\end{align}
and let $L^2(M,g)$ denote the completion of $C^\infty(M)$ with respect to the resulting norm.

The \textbf{Laplace-Beltrami operator} $\Delta$ acts on $C^\infty(M)$, in local coordinates $(x_1,\dots,x_d)$, by the formula
\begin{align}
    \Delta(f) = -\frac{1}{\sqrt{g}} \sum_{i,j=1}^d \frac{\partial \sqrt{g} g^{ij} (\partial f / \partial x_i)}{\partial x^j}
\end{align}
where $g = \det(g_{ij})$ and $(g^{ij})$ denotes the inverse of the matrix $(g_{ij})$. With respect to the inner product in \ref{eq:ip}, $\Delta$ is symmetric:
\begin{align*}
    \ip{\Delta f}{g} =\ip{ f}{\Delta g}, \quad \text{ for all $f,g \in C^\infty(M)$,}
\end{align*}
negative:
\begin{align*}
    \ip{\Delta f }{f} \leq 0, \quad \text{ for all $f \in C^\infty(M)$.}
\end{align*}
and unbounded. As $M$ is compact and without boundary, $\Delta$ admits a unique maximal extension to a negative, self-adjoint, unbounded operator  $\tilde{\Delta}$ on $L^2(M,g)$ with compact resovent.

By the spectral theorem for the latter class of operators, it follows that the resolvent set
\begin{align*}
    \sigma(\tilde{\Delta}) = \{ \lambda \in \Cbb : \tilde{\Delta} - \lambda \id \text{ is not invertible on $L^2(M,g)$}\}
\end{align*}
is a discrete set of nonnegative real numbers, accumulating at $-\infty$, and consists of eigenvalues of $\tilde{\Delta}$ with finite multiplicites. By elliptic regularity, any eigenfunction of $\tilde{\Delta}$ is smooth, and thus lies in the original domain $C^\infty(M)$ of $\Delta$. Consequently, $\sigma(\tilde{\Delta})$ coincides with the set
\begin{align*}
    \sigma(\Delta) = \{\lambda \in \Cbb : \Delta f  = \lambda f \text{ admits a nonzero  solution  $f \in C^\infty(M)$}.
\end{align*}
Set $E(\lambda) = \ker \Delta - \lambda \id$ and set $m(\lambda) = \dim E(\lambda)$. [TODO:finish rambling]

\paragraph{Heat kernel}
Let $\phi_0=\Vol(M)^{-1},\phi_1,\phi_2,\cdots$ be an orthonormal sequence of eigenfunctions for $\Delta$ corresponding to eigenvalues $\lambda_0=0 <\lambda_1 \leq \lambda_2 \leq \dots$. For $t >0$, the \textbf{heat kernel}
\begin{align}\label{eq:heatdef}
    K_t(x,y):= \sum_{j\geq 0} e^{-t\lambda_j} \phi_j(x)\phi_j(y)
\end{align}
is the unique function $k:\Rbb_{\geq 0} \times M \times M \to \Rbb$ satisfying the properties
\begin{enumerate}
    \item For each fixed $y\in M$, one has $\left(\frac{\partial}{\partial t} + \Delta \right)k_t(x,y) = 0$
    \item For each $u\in C^\infty(M)$, one has $ \int_M k_t(x,y) u(y)  \dop \Vol_g (x)(y) \to u(x)$  as $t \to 0^+$, uniformly in $x$
    \item $k_t(\cdot,\cdot)$ is $C^2$ in $(x,y)$ for each $t$
    \item $k_\cdot( x,y)$ is $C^1$  in $t$ for each $(x,y)$.
\end{enumerate}
In particular, the heat kernel $K_t(x,y)$ is independent of the choice of orthonormal basis of eigenfunctions in the defining equation \ref{eq:heatdef}.

\begin{proposition}\label{prop:asymptoticExpansion}
    Let $(M,g)$ be a compact $d$-dimensional Riemanniannian manifold without boundary, and let $K_t(x,y)$ be its heat kernel.  There exist a sequence of smooth functions $\abf_0,\abf_1, \abf_2,\cdots$ on $M$ such that
    \begin{align}\label{eq:asymptoticExpansion}
        K(t,x,x) \sim t^{-d/2}\left(\abf_0(x)+ \abf_1(x) t + \abf_2(x) t^2 +\cdots  \right)
    \end{align}
    as $t \to 0^+$. For each $n$, the function $\abf_n$ is a homogeneous polynonmial of degree $2n$ in the derivatives of the metric $g$.
    In particular,
    \begin{align}
        \abf_0 & = 1,                                                       \\
        \abf_1 & =\frac{1}{6} \tau,                                         \\
        \abf_2 & =\frac{1}{360}\left( 5\tau^2 - 2 |\rho|^2 +2 |R|^2 \right)
    \end{align}
    where $\tau \in C^\infty(M)$ is the scalar curvature,j
\end{proposition}













\newpage








