\begin{itemize}
  \item Consider a section describing a qualitative classification of instances of Q/Fp/Zp equivalence. I think guralnick has some good papers on this
\end{itemize}

\section{on cohomology}
First, on singular homology: the $q$-simplex is the topological space $\Delta^q = \{(t_0,\dots,t_q) | t_i\geq 0, \sum t_i =1 \}$.

There are \(q+1\) simplicial embeddings of \( \Delta^{q-1}$ into $\Delta^q \), given by

\[d_i : \Delta^{q-1} \to \Delta^{q},\quad d_i(t_0,\dots,t_{q-1}) =\left(t_0, \ldots, \underset{i}{0}, \ldots, t_{q-1}\right). \]

We refer to the map $d_i$ as the $i$-face of $\Delta^{q-1}$.

A $q$-simplex in a topological space $X$ is just a continuous map $\sigma: \Delta^q \to X$, and we write
\footnote{For any fixed topological space $\Delta$ the assignment $\Delta(\cdot):X \mapsto C(\Delta , X)$ is a covariant functor from the category of topological spaces to sets. Indeed, if $f:X \to Y$ is a continuous map, then the map $\Delta(f): \Delta(X) \to \Delta(Y)$ is given by post-composing a map $\sigma :\Delta \to X$ with $f:X\to Y$.

  Similarly if a target space $X$ is fixed and we vary among spaces $\Delta$ mapping into $X$ we obtain a contravariant functor, given on maps by \emph{pre-composition}.

  Slogan: maps \emph{from} a fixed space is covariant, maps \emph{to} a fixed space is contravariant.}
$\Delta^q X:=C(\Delta^q,X)$ for the set of $q$-simplices in $X$, and $C_q X :=\Zbb \Delta^q X$ for $q$-chains\footnote{The map sending a set $X$ to the free abelian group $\Zbb X$ on $X$ is covariant, so altogether, the map $X \mapsto C_q X$ sending a topological space to its (free abelian) group of $q$-cochains is covariant}



For each $i=0,..q$, the face maps $d_i$ induce natural transformations $\partial_i : \Delta^{q}(X)\to \Delta^{q-1}(X)$. We incorporate all of these maps together via the \textbf{singular boundary map} on $q$-chains, via the formula
\[\partial=\sum^q(-1)^i \partial_i: C_q X \rightarrow C_{q-1} X.\]

A key fact is that the composite maps $\partial \circ \partial C^{q}X\to C^{q-2}X $ are identically zero for each $q$. Symbolically: $B_qX \leq Z_q X \leq C_q$. In words: the $q$ boundary of any $q+1$ chain is a $q$-cycle; slightly less precisely: the boundary of a boundary is empty.

Chains in  $Z_qX=\operatorname{ker} \partial: C_q X \rightarrow C_{q-1} X$ are called $q$-cycles in $X$, and those in $B_q X=\operatorname{img} \partial: C_{q+1} X \rightarrow C_q X^{\bullet}$ are $q$-boundaries. The quotient $H_q X = Z_q X/B_q X$ is the $q$-th homology group of $X$.

\begin{remark}
  A $0$-chain in $X$ is just a formal $\Zbb$-linear combination of points. Every $0$-chain is a $0$-cycle, as $\partial:C_0(X) \to C_{-1}X=0$ is the zero map.
\end{remark}

Associated to a connected space $X$ is the \textbf{augmentation map}, as follows. Recalling that $C_0X = \Zbb X$ is simply the free abelian group generated by $X$, there is an adjunction: $\hom_{gp}(\Zbb X,A) = \hom_{set}(X,A)$ for any abelian group $A$. Taking $A = \Zbb$, we define a map $\eps_X : C_0 X \to \Zbb$ as the unique $\Zbb$-linear extension of the map $X\to \Zbb$ sending every element to $1\in \Zbb$.

\begin{lemma}
  The augmentation map induces an identification of $H_0X$ with the free abelian group $\Zbb \pi_0(X)$ on the path components of $X$.
\end{lemma}

\subsection*{$\pi_1$}

Recall that each point $v$ in the $q$-dimensional simplex $\Delta^q$ admits a unique expression $v = \sum_{i=0}^q t_i e_i$ subject to the condition $\sum t_i = 1$.


\begin{itemize}
  \item $\Delta^0 = \{e_0\} \subset \Rbb^1$ is a point, a $0$-simplex $\sigma: \Delta^0 \to X$ amounts to a choice of image $x=x_\sigma=\sigma(e_0) \in X$.
  \item $\Delta^1 = \{ t_0 e_0+t_1 e_1 \in \Rbb^2: t_0 + t_1 = 1\}$ is a line segment. The two face maps $\Delta^0 \to \Delta^1$ are the \emph{target} map \[t=d_0: e_0 \mapsto 0e_0+1e_1\] and the \emph{source} map \[s=d_1: e_0 \mapsto 1 e_0 +0 e_1.\] The boundary of the defining inclusion $\iota:\Delta^1 \to \Rbb^2$ is the formal difference (target) minus (source) $\partial \iota =t-s $.
  \item The three face maps $\Delta^1 \to \Delta^2$ are:
        \begin{align*}
          d^0 & : \tau e_0 + (1-\tau) e_1 \to 0 e_0 + (1-\tau) e_1 + \tau e_2 \\
          d^1 & : \tau e_0 + (1-\tau) e_1 \to \tau e_0 + 0 e_1 + (1-\tau)e_2  \\
          d^2 & : \tau e_0 + (1-\tau) e_1 \to \tau e_0 + (1-\tau) e_1 + 0e_2
        \end{align*}
        and the boundary of the inclusion $\Delta^2 \to \Rbb^3$ is the formal difference $d^0-d^1+d^2$.
  \item A path in $X$ is a continuous map of the interval $[0,1]$ to $X$. Two paths $\gamma,\gamma'$ are composable if $\gamma(1)=\gamma'(0)$ in which case we write $\gamma*\gamma'$ for the path
        \begin{align*}
          t \mapsto
          \begin{cases}
            \gamma(2t)    & \text{if }t\leq 1/2  \\ %if $t\leq 1/2$ \\
            \gamma'(2t-1) & \text{if }t \geq 1/2 %if $t\geq 1/2$
          \end{cases}
        \end{align*}
        and for any path $\gamma$ write $\gamma^\inv$ for the path $t \mapsto \gamma(1-t)$.
  \item Suppose we are given a pointed space $(X,x_o)$. Given a homotopy class $\alpha \in \pi_1(X,x_o)$, pick a pointed loop $\gamma : S^1 \to X$  representing $\alpha$ and such that $\gamma(1)=x_0$.
  \item Write $e: \Delta^1 \to \Delta^1/\partial \Delta^1$ for the quotient map, and identify the latter with $S^1$. Then we can pullback $\gamma$ along $e$ to obtain a $1$-chain $\gamma \circ e : \Delta^1 \to X$. As $\gamma$ is a closed loop, the $1$-chain $\gamma \circ e$ is a 1-cycle.
  \item Suppose $\gamma'$ is another representative pointed loop for $\alpha$. I claim that $\gamma'\circ e$ and $\gamma \circ e$ are then homologous cycles.
  \item Indeed, let $H: \Delta^1 \times \Delta^1 \to X$ be a pointed homotopy between $\gamma$ and $\gamma'$, so that $H(0,\cdot)=\gamma \circ e$, $H(1,\cdot )= \gamma'\circ e$ and $H(t,0)=H(t,1) = x_o$ for all $t\in \Delta^1$ and set $w(t)=H(t,t)$.
\end{itemize}
\subsection*{rescores}
Let $N$ be a normal subgroup of a group $G$, and let $A$ be a $G$-module. Then $G$ preserves the submodule $A^N$ of $N$-fixed points, and the action of $G$ on $A^N$ factors through $G/N$.

The inflation-restriction exact sequence is:
\begin{align*}
  H^1\left(G / N, A^N\right) \xhookrightarrow{{\rm infl}} H^1(G, A)\xrightarrow{{\rm res}} H^1(N, A)^{G / N} \xrightarrow{{\rm tran}} H^2\left(G / N, A^N\right) \xrightarrow{{\rm infl}} H^2(G, A)
\end{align*}

More generally, if $H$ is a subgroup of $G$, then there is a restriction map $C^q(G,A)\to C^q(H,A)$ at the level of chains.


\section{cohomology}
For any space $X$, there is a canonical diagonal map $\Delta: X\hookrightarrow X\times X$. This induces a map $H_*(X;R) \to H_*(X\times X;R)$ for any abelian group $R$. If moreover $R$ is a ring, then there is a map (cross product)
\begin{align*}
  \times : H_*(X;R)\otimes_R H_*(X;R) \to H_*(X\times X;R).
\end{align*}

Somehow $R$-being a PID+Kunneth's formula tells us that this map is a monomorphism, and ``if the remaining term of the kunneth sequence is zero'' then this map is an isomorphism.

When $R$ is a field, this map is universally defined and natural in $X$.

\subsection{}
Let $A$ be an abelian group. A singular $q$-cochain with values in $N$ is a(ny) function $\Sing_q X\to N$.

If $N$ is an $R$-module, a singular $q$-cochain with values in $N$ admits a unique extension $S_q(X;R)\to N$ which is $R$-linear.

In this case, we set
\begin{align*}
  S^n(X ; N)=\operatorname{Map}\left(\operatorname{Sin}_n(X), N\right)=\operatorname{Hom}_R\left(S_n(X ; R), N\right)
\end{align*}
which is naturally an $R$-module.

The boundary map
\begin{align*}
  d: S_{n+1}(X ; R) \rightarrow S_n(X ; R)
\end{align*}
induces the \emph{coboundary} map
\begin{align*}
  d: S^n(X ; N) \rightarrow S^{n+1}(X ; N)
\end{align*}
via the rule
\begin{align*}
  (d f)(\sigma)=(-1)^{n+1} f(d \sigma)
\end{align*}
and we define the $n$th singular cohomology groups with coefficients in $N$ as
\begin{align*}
  H^n(X ; N)=\frac{\operatorname{ker}\left(S^n(X ; N) \rightarrow S^{n+1}(X ; N)\right)}{\operatorname{im}\left(S^{n-1}(X ; N) \rightarrow S^n(X ; N)\right)}.
\end{align*}
If $N$ is an $R$-module then so too is $H^n(X;N)$ .
\section{$H^0$}
A $0$-cochain is a(ny) function $\Sing_0 X \to N$. Identifying $\Sing_0 X \to N$, see that a $0$-cochain is just a function $f:X\to N$.

If $\sigma: \Delta`^1 \to X$ is a $1$ simplex, then
\begin{align*}
  \left(d f\right)(\sigma) & = -f \left(\sigma(e_0) - \sigma(e_1)\right) \\
                           & = f(\sigma(e_1)) - f(\sigma(e_0))
\end{align*}


\section{}






\section{reminders}
Free-constructions are left-adjoints to forgetful functors:.







