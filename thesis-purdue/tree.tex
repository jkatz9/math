<<<<<<< HEAD
<<<<<<< HEAD
<<<<<<< HEAD
\documentclass[draft]{amsart}
\usepackage{amsthm,amssymb,amsmath,amsfonts,braket}

\usepackage[inline]{enumitem}
=======
\documentclass{article}
\usepackage{amsthm,amssymb,amsmath,amsfonts,braket}
>>>>>>> 7fd919c ( Changes to be committed:)
=======
\documentclass[draft]{article}
=======
\documentclass[draft]{amsart}
>>>>>>> 8cbdf03 ( On branch master)
\usepackage{amsthm,amssymb,amsmath,amsfonts,braket}

\usepackage[inline]{enumitem}
>>>>>>> 510c44e ( Changes to be committed:)
\usepackage{mycros}
\usepackage{graphicx} % Required for inserting images
\usepackage{url}
\usepackage{verbatim}
\usepackage{todonotes}
\usepackage{mathtools,thmtools}
\usepackage{tikz-cd}
\usepackage{quiver}
\usepackage{comment}
<<<<<<< HEAD
<<<<<<< HEAD
\input{macros.tex}

\usepackage[backend=biber]{biblatex}
\addbibresource{/Users/jukatz/texmf/bibtex/bib/bigBib.bib}



=======

\input{macros.tex}
>>>>>>> 7fd919c ( Changes to be committed:)
=======
\input{macros.tex}

\usepackage[backend=biber]{biblatex}
\addbibresource{/Users/jukatz/texmf/bibtex/bib/bigBib.bib}



>>>>>>> 8cbdf03 ( On branch master)
\title{tree}
\author{justin katz}
\date{March 2023}

\setlength{\parindent}{0 em}
\setlength{\parskip}{6 pt}
\begin{document}

<<<<<<< HEAD
<<<<<<< HEAD

Let $A$ be a quaternion algebra over a number field $k$, and let $\Ocal$ be a maximal order in $A$. For each finite place $\pfrak$ of $k$ over which $A$ is split, pick an isomorphism of $A_\pfrak =A\otimes_k k_\pfrak$ with $M(2,k_\pfrak)$ such that $\Ocal_\pfrak = \Ocal \otimes_R R_\pfrak$ maps to $M(2,R_\pfrak)$ and write $\rho_\pfrak:A^\times \to \GL(2,k_\pfrak)$ for the composition of the natural embedding $A^\times \to (A\otimes_k k_\pfrak)^\times$ followed by our chosen isomorphism $A_\pfrak^\times \approx \GL(2,k_\pfrak)$.
Let $A$ be a quaternion algebra over a number field $k$, and let $\Ocal$ be a maximal order in $A$. For each finite place $\pfrak$ of $k$ over which $A$ is split, pick an isomorphism of $A_\pfrak =A\otimes_k k_\pfrak$ with $M(2,k_\pfrak)$ such that $\Ocal_\pfrak = \Ocal \otimes_R R_\pfrak$ maps to $M(2,R_\pfrak)$ and write $\rho_\pfrak:A^\times \to \GL(2,k_\pfrak)$ for the composition of the natural embedding $A^\times \to (A\otimes_k k_\pfrak)^\times$ followed by our chosen isomorphism $A_\pfrak^\times \approx \GL(2,k_\pfrak)$.

\paragraph*{p-adic computation}
\begin{definition}[Bruhat-Tits tree]
    The Bruhat-Tits tree $\Tcal$ for $\GL(2,k)$ is the following graph: the vertices of $\Tcal$ are homothety classes of $R$-lattices in $k^2$.
    If $L$ is a lattice in $k^2$, we write $[L]$ for its homothety class.
    The Bruhat-Tits tree $\Tcal$ for $\GL(2,k)$ is the following graph: the vertices of $\Tcal$ are homothety classes of $R$-lattices in $k^2$.
    If $L$ is a lattice in $k^2$, we write $[L]$ for its homothety class.

    Two vertices $\Lambda_1,\Lambda_2 $ of $\Tcal$ are adjacent if there are representative lattices $L_i \in \Lambda_i$ such that $\pi L_1 < L_2 < L_1$ with each inclusion proper.
    Two vertices $\Lambda_1,\Lambda_2 $ of $\Tcal$ are adjacent if there are representative lattices $L_i \in \Lambda_i$ such that $\pi L_1 < L_2 < L_1$ with each inclusion proper.

    We write $V(\Tcal)$ for the vertices of $\Tcal$, and $E(\Tcal)$ for the edges of $\Tcal$ (which consist of ordered pairs of adjacent vertices.)
    We write $V(\Tcal)$ for the vertices of $\Tcal$, and $E(\Tcal)$ for the edges of $\Tcal$ (which consist of ordered pairs of adjacent vertices.)
\end{definition}

We regard $k^2$ as a space of column vectors, with distinguished basis vectors $e_1,e_2$. We write $L_0 = Re_1+Re_2$ and refer to this as the \textbf{distinguished lattice}, and refer to its homothety class $\Lambda_0 = [L_0]$ as the \textbf{distingushed vertex} of $\Tcal$. We let $L_1 = Re_1 + \pfrak e_2$, and set $\Lambda_1 = [L_1]$ and refer to $C=(\Lambda_0,\Lambda_1)$ as the \textbf{distinguished edge} of $\Tcal$.

$G=\GL(2,k)$ acts on the set of lattices in $k^2$: for $g\in \GL(2,k)$ and a lattice $L$, the set $gL = \{ g v : v \in L\}$ is also a lattice in $k^2$. This action is transitive: given a lattice $L$, pick an $R$ basis $(v,w)$ for $L$ and define $g_L$ by the assignment $ge_1 = v$ and $ge_2 = w$. Then $g_L L_0 =L$, where $L_0$ is our distinguished lattice $R^2$ in $k^2$ as above. Under this action, the stabilizer of $L_0$ is $K:=\GL(2,R)$. \footnote{If $L= gL_0$ is any other lattice, then its stabilizer in $\GL(2,k)$ is $\Aut_R(L)=g \GL(2,R)g^\inv$}. Thus, the orbit map $g \mapsto gL_0$ induces a bijection $G/K$ with the set of lattices in $k^2$.
We regard $k^2$ as a space of column vectors, with distinguished basis vectors $e_1,e_2$. We write $L_0 = Re_1+Re_2$ and refer to this as the \textbf{distinguished lattice}, and refer to its homothety class $\Lambda_0 = [L_0]$ as the \textbf{distingushed vertex} of $\Tcal$. We let $L_1 = Re_1 + \pfrak e_2$, and set $\Lambda_1 = [L_1]$ and refer to $C=(\Lambda_0,\Lambda_1)$ as the \textbf{distinguished edge} of $\Tcal$.

$G=\GL(2,k)$ acts on the set of lattices in $k^2$: for $g\in \GL(2,k)$ and a lattice $L$, the set $gL = \{ g v : v \in L\}$ is also a lattice in $k^2$. This action is transitive: given a lattice $L$, pick an $R$ basis $(v,w)$ for $L$ and define $g_L$ by the assignment $ge_1 = v$ and $ge_2 = w$. Then $g_L L_0 =L$, where $L_0$ is our distinguished lattice $R^2$ in $k^2$ as above. Under this action, the stabilizer of $L_0$ is $K:=\GL(2,R)$. \footnote{If $L= gL_0$ is any other lattice, then its stabilizer in $\GL(2,k)$ is $\Aut_R(L)=g \GL(2,R)g^\inv$}. Thus, the orbit map $g \mapsto gL_0$ induces a bijection $G/K$ with the set of lattices in $k^2$.

Under this bijection, the map $L\mapsto [L]$ sending a lattice to its homothety class, corresponds to the quotient $G / K \to k^\times \lmod G \rmod K$ with $k^\times$ acting by scaling. Thus, we have an identification of
Under this bijection, the map $L\mapsto [L]$ sending a lattice to its homothety class, corresponds to the quotient $G / K \to k^\times \lmod G \rmod K$ with $k^\times$ acting by scaling. Thus, we have an identification of
\begin{align*}
    V(\Tcal) \approx k^\times \lmod G/K.
\end{align*}
V(\Tcal) \approx k^\times \lmod G/K.
\end{align*}
As this action of $k^\times$ on $\GL(2,k)$ is central, this quotient can be identified with $G/k^\times K$, or with $\PGL(2,k)/\PGL(2,R)$ \footnote{We remark that the action of $\GL(2,k)$ on $\Tcal$ factors through an effective action by $\PGL(2,k)$ which is transitive both on vertices and ordered edges.}

The stabilizer in $\GL(2,k)$ of the distinguished vertex $\Lambda_0$ is $k^\times \cdot K$, and the stabilizer of the distinguished (oriented) edge $C = (\Lambda_0,\Lambda_1)$ is $k^\times K_0(\pfrak)$ where $K_0(\pfrak)=K\cap \tbt{1}{0}{0}{\pi}K\tbt{1}{0}{0}{\pi}^\inv$ consists of those $\tbt{a}{b}{c}{d} \in \GL(2,R)$  with $c \in \pfrak$.
The stabilizer in $\GL(2,k)$ of the distinguished vertex $\Lambda_0$ is $k^\times \cdot K$, and the stabilizer of the distinguished (oriented) edge $C = (\Lambda_0,\Lambda_1)$ is $k^\times K_0(\pfrak)$ where $K_0(\pfrak)=K\cap \tbt{1}{0}{0}{\pi}K\tbt{1}{0}{0}{\pi}^\inv$ consists of those $\tbt{a}{b}{c}{d} \in \GL(2,R)$  with $c \in \pfrak$.

The element $\tbt{0}{\pi}{-1}{0}$ of $\GL(2,k)$ induces an involution on $\Tcal$ which fixes no vertex, but interchanges the vertices $\Lambda_0$ and $\Lambda_1$ of the distinguished edge $C$. In particular,
The element $\tbt{0}{\pi}{-1}{0}$ of $\GL(2,k)$ induces an involution on $\Tcal$ which fixes no vertex, but interchanges the vertices $\Lambda_0$ and $\Lambda_1$ of the distinguished edge $C$. In particular,


\paragraph*{BT-Tree}
Setup from \cite{chinburgEmbeddingTheoremQuaternion1999}. If $D$ is a maximal order in $M(2,F)$, then there exists a $g\in \PGL(2,F)$ such that $g D g^\inv = M(2,R)$. If $D'$ is any other maximal order, we may pick among such $g$ one for which $gD'g^\inv = \Tbt{R}{\pfrak^n}{\pfrak^{-n}}{R}$ for a nonnegative integer $n:=d(D,D')$. Alternatively, $d(D,D')$ may be characterized as the the positive integer $n$ such that $D/(D\cap D') \approx R/\pfrak^n$ as $R$ modules. We write $\Tcal$ for the tree whose vertices are maximal orders in $M(2,F)$, and two orders $D,D'$ are adjacent, written $D \sim D'$ if $d(D,D')=1$.
Setup from \cite{chinburgEmbeddingTheoremQuaternion1999}. If $D$ is a maximal order in $M(2,F)$, then there exists a $g\in \PGL(2,F)$ such that $g D g^\inv = M(2,R)$. If $D'$ is any other maximal order, we may pick among such $g$ one for which $gD'g^\inv = \Tbt{R}{\pfrak^n}{\pfrak^{-n}}{R}$ for a nonnegative integer $n:=d(D,D')$. Alternatively, $d(D,D')$ may be characterized as the the positive integer $n$ such that $D/(D\cap D') \approx R/\pfrak^n$ as $R$ modules. We write $\Tcal$ for the tree whose vertices are maximal orders in $M(2,F)$, and two orders $D,D'$ are adjacent, written $D \sim D'$ if $d(D,D')=1$.

The function $g \mapsto \ord(\det g) \mod 2 $ is a homomorphism $\GL(2,F) \to \Zbb/2\Zbb$ which factors through $\PGL(2,F)$. We say $g\in \GL(2,F)$ or its image in $\PGL(2,F)$ is even or odd accordingly.

\begin{proposition}
    \begin{enumerate}
        \item $\PGL(2,F)$ acts on $\Tcal$ (by conjugation $D\mapsto gDg^\inv$) transitively on both vertices and edges. The restriction of this action to $\PSL(2,F)$ has two orbits of vertices: two orders are conjugate by an element of $\PSL(2,F)$ if and only if they are an even distance apart. An element $\overline{g} \in \PGL(2,F)$ interchanges these two orbits if and only if any lift $g \in \GL(2,F)$ has $\ord(\det(g))$ odd.
        \item Any maximal compact subgroup of $\PGL(2,F)$ is the stabilizer of an edge or a vertex.
        \item Any maximal compact subgroup of $\PGL(2,F)$ is the stabilizer of an edge or a vertex.
        \item For $g \in \GL(2,F)$ and $D\in \Tcal$, one has $d(D,g D g^\inv) \equiv \ord(\det (g))$ mod $2$.
        \item An odd element fixes no vertex.
        \item An even element of $\PGL(2,F)$ which fixes an edge setwise actually fixes that edge pointwise.
        \item An odd element fixes no vertex.
        \item An even element of $\PGL(2,F)$ which fixes an edge setwise actually fixes that edge pointwise.
        \item If an element of $\PGL(2,F)$ fixes a pair of vertices, then it fixes every vertex on the unique path between them.
    \end{enumerate}
\end{proposition}

The discriminant of the characteristic polynomial of an element $g\in \GL(2,F)$ is $\disc(g) =(\tr g)^2 - 4 \det g$. The function $g \mapsto \disc(g)/\det(g)$ is invariant under the center, and thus descends to a function on the quotient $\PGL(2,F)$, which we will call by the same name.
The discriminant of the characteristic polynomial of an element $g\in \GL(2,F)$ is $\disc(g) =(\tr g)^2 - 4 \det g$. The function $g \mapsto \disc(g)/\det(g)$ is invariant under the center, and thus descends to a function on the quotient $\PGL(2,F)$, which we will call by the same name.

\begin{proposition}
    For an element $g\in \GL(2,F)$, with image $\overline{g}$ in $\PGL(2,F)$, the following are equivalent:
    For an element $g\in \GL(2,F)$, with image $\overline{g}$ in $\PGL(2,F)$, the following are equivalent:
    \begin{enumerate}
        \item $\overline{g}$ fixes an edge or a vertex of $\Tcal$.
        \item $\overline{g}$ fixes an edge or a vertex of $\Tcal$.
        \item $\disc(g)/\det(g) \in R$
        \item $\overline{g}$ lies in a compact subgroup of $\PGL(2,F)$.
        \item $\overline{g}$ lies in a compact subgroup of $\PGL(2,F)$.
    \end{enumerate}
    In this case, $\overline{g}$ fixes an edge in $\Tcal$ if and only if at least one of the following conditions is true:
    In this case, $\overline{g}$ fixes an edge in $\Tcal$ if and only if at least one of the following conditions is true:
    \begin{enumerate}
        \item $g \in F^\times$
        \item $\ord \det (g)$ is odd
        \item the subalgebra $F[g]$ of $M(2,F)$ is not a field
        \item $\disc(g)/\det(g) \in \pfrak$.
        \item $\disc(g)/\det(g) \in \pfrak$.
    \end{enumerate}
    If all of these conditions fail, then $g$ fixes exactly one vertex and no edges.
    If all of these conditions fail, then $g$ fixes exactly one vertex and no edges.
\end{proposition}


\paragraph*{Subsets of $\Tcal$.}




\paragraph*{Invariant subtrees}
Let $H$ be a group, and $\rho: H \to \GL(2,k)$ be a representation. Then $H$ acts (through $\rho$) on $\Tcal$, and the set of fixed points $\Tcal(\rho)$ is a (possibly empty) connected subtree of $\Tcal$. First, observe the following:
\begin{proposition}
    The following are equivalent:
    \begin{enumerate}
        \item $\Tcal(\rho)$ is nonempty.
        \item the image of $\rho(H)$ in $\PGL(2,k)$ is relatively compact
        \item the image of $\rho(H)$ in $\PGL(2,k)$ is bounded
        \item for all $g\in H$,  $\frac{(\tr\rho(g))^2}{\det \rho (g)}-4 \in R$
              \ite m there is an equivalent representation $\rho': H \to \GL(2,k)$ such that $\rho'(H)$ is contained in $k^\times\cdot \GL(2,R)$ or the normalizer of $K_0(\pfrak)$.
    \end{enumerate}
    The following are equivalent:
    \begin{enumerate}
        \item $\Tcal(\rho)$ is nonempty.
        \item the image of $\rho(H)$ in $\PGL(2,k)$ is relatively compact
        \item the image of $\rho(H)$ in $\PGL(2,k)$ is bounded
        \item for all $g\in H$,  $\frac{(\tr\rho(g))^2}{\det \rho (g)}-4 \in R$
              \ite m there is an equivalent representation $\rho': H \to \GL(2,k)$ such that $\rho'(H)$ is contained in $k^\times\cdot \GL(2,R)$ or the normalizer of $K_0(\pfrak)$.
    \end{enumerate}
\end{proposition}



A theorem of \cite{bellaicheSousgroupesGLArbres2014} demonstrates that $\Tcal(\rho)$ must take a highly restricted form.
\begin{proposition}


\end{proposition}

% \begin{remark}
%     \begin{enumerate*}
%         \item The action of the center $k^\times$ of $\GL(2,k)$ on lattices $L$ factors through $R^\times \lmod k^\times $, as $xL = L$  if (and only if) $x \in R^\times$. 
%     \end{enumerate*}
% \end{remark}




\paragraph*{Local Computations}
For now, let $k$ nonarchimedian local field of characteristic $0$, with ring of integers $R$ and maximal ideal $\pfrak$, with uniformizer $\pi$. Let $V$ be a two dimensional $k$-vectorspace, and $A = \End_k(V)$. Then every maximal order in $A$ takes the form
\paragraph*{Local Computations}
For now, let $k$ nonarchimedian local field of characteristic $0$, with ring of integers $R$ and maximal ideal $\pfrak$, with uniformizer $\pi$. Let $V$ be a two dimensional $k$-vectorspace, and $A = \End_k(V)$. Then every maximal order in $A$ takes the form
\begin{align*}
    \End_R(L) = \{ x \in A : xL\leq L\}
\end{align*}
where $L$ is a free $R$-module of rank $2$ in $V$. If $M = xL$ for some $x\in A^\times = \Aut_k(V)$, then $\End_R(M) = x\End_R(L)x^\inv$. As $A^\times$ acts transitively on the set of lattices in $V$ (by left-translation), it acts transitively on the set of maximal orders in $A$ (by conjugation). Further, if $\End_R(L)=\End_R(M)$ for two lattices $L,M$ in $V$, then there exists an $x \in k^\times$ such that $L = x M$. In this case, we say $L$ and $M$ are homothetic, and conclude that there is an $k^\times \lmod A^\times = \PGL(V)$ equivariant bijection between homothety classes of lattices in $V$ and maximal orders in $A$.
where $L$ is a free $R$-module of rank $2$ in $V$. If $M = xL$ for some $x\in A^\times = \Aut_k(V)$, then $\End_R(M) = x\End_R(L)x^\inv$. As $A^\times$ acts transitively on the set of lattices in $V$ (by left-translation), it acts transitively on the set of maximal orders in $A$ (by conjugation). Further, if $\End_R(L)=\End_R(M)$ for two lattices $L,M$ in $V$, then there exists an $x \in k^\times$ such that $L = x M$. In this case, we say $L$ and $M$ are homothetic, and conclude that there is an $k^\times \lmod A^\times = \PGL(V)$ equivariant bijection between homothety classes of lattices in $V$ and maximal orders in $A$.

Given two lattices $L$ and $M$ in $V$, there exists a basis $u,v$ of $L$ such that $\pi^a u , \pi^b v$ is a basis for $M$ (elementary divisor theorem) and we set $d(L,M) = |a-b|$. If $x,y\in k^\times$ then $d(xL,yM)=d(L,M)$ so that $d$ descends to a function on homothety classes, or equivalently a function on maximal orders.
Given two lattices $L$ and $M$ in $V$, there exists a basis $u,v$ of $L$ such that $\pi^a u , \pi^b v$ is a basis for $M$ (elementary divisor theorem) and we set $d(L,M) = |a-b|$. If $x,y\in k^\times$ then $d(xL,yM)=d(L,M)$ so that $d$ descends to a function on homothety classes, or equivalently a function on maximal orders.



% \printbibliography
=======
=======

>>>>>>> 8cbdf03 ( On branch master)
Let $A$ be a quaternion algebra over a number field $k$, and let $\Ocal$ be a maximal order in $A$. For each finite place $\pfrak$ of $k$ over which $A$ is split, pick an isomorphism of $A_\pfrak =A\otimes_k k_\pfrak$ with $M(2,k_\pfrak)$ such that $\Ocal_\pfrak = \Ocal \otimes_R R_\pfrak$ maps to $M(2,R_\pfrak)$ and write $\rho_\pfrak:A^\times \to \GL(2,k_\pfrak)$ for the composition of the natural embedding $A^\times \to (A\otimes_k k_\pfrak)^\times$ followed by our chosen isomorphism $A_\pfrak^\times \approx \GL(2,k_\pfrak)$.

\paragraph*{p-adic computation}
\begin{definition}[Bruhat-Tits tree]
    The Bruhat-Tits tree $\Tcal$ for $\GL(2,k)$ is the following graph: the vertices of $\Tcal$ are homothety classes of $R$-lattices in $k^2$.
    If $L$ is a lattice in $k^2$, we write $[L]$ for its homothety class.

    Two vertices $\Lambda_1,\Lambda_2 $ of $\Tcal$ are adjacent if there are representative lattices $L_i \in \Lambda_i$ such that $\pi L_1 < L_2 < L_1$ with each inclusion proper.

    We write $V(\Tcal)$ for the vertices of $\Tcal$, and $E(\Tcal)$ for the edges of $\Tcal$ (which consist of ordered pairs of adjacent vertices.)
\end{definition}

We regard $k^2$ as a space of column vectors, with distinguished basis vectors $e_1,e_2$. We write $L_0 = Re_1+Re_2$ and refer to this as the \textbf{distinguished lattice}, and refer to its homothety class $\Lambda_0 = [L_0]$ as the \textbf{distingushed vertex} of $\Tcal$. We let $L_1 = Re_1 + \pfrak e_2$, and set $\Lambda_1 = [L_1]$ and refer to $C=(\Lambda_0,\Lambda_1)$ as the \textbf{distinguished edge} of $\Tcal$.

$G=\GL(2,k)$ acts on the set of lattices in $k^2$: for $g\in \GL(2,k)$ and a lattice $L$, the set $gL = \{ g v : v \in L\}$ is also a lattice in $k^2$. This action is transitive: given a lattice $L$, pick an $R$ basis $(v,w)$ for $L$ and define $g_L$ by the assignment $ge_1 = v$ and $ge_2 = w$. Then $g_L L_0 =L$, where $L_0$ is our distinguished lattice $R^2$ in $k^2$ as above. Under this action, the stabilizer of $L_0$ is $K:=\GL(2,R)$. \footnote{If $L= gL_0$ is any other lattice, then its stabilizer in $\GL(2,k)$ is $\Aut_R(L)=g \GL(2,R)g^\inv$}. Thus, the orbit map $g \mapsto gL_0$ induces a bijection $G/K$ with the set of lattices in $k^2$.

Under this bijection, the map $L\mapsto [L]$ sending a lattice to its homothety class, corresponds to the quotient $G / K \to k^\times \lmod G \rmod K$ with $k^\times$ acting by scaling. Thus, we have an identification of
\begin{align*}
    V(\Tcal) \approx k^\times \lmod G/K.
\end{align*}
As this action of $k^\times$ on $\GL(2,k)$ is central, this quotient can be identified with $G/k^\times K$, or with $\PGL(2,k)/\PGL(2,R)$ \footnote{We remark that the action of $\GL(2,k)$ on $\Tcal$ factors through an effective action by $\PGL(2,k)$ which is transitive both on vertices and ordered edges.}

The stabilizer in $\GL(2,k)$ of the distinguished vertex $\Lambda_0$ is $k^\times \cdot K$, and the stabilizer of the distinguished (oriented) edge $C = (\Lambda_0,\Lambda_1)$ is $k^\times K_0(\pfrak)$ where $K_0(\pfrak)=K\cap \tbt{1}{0}{0}{\pi}K\tbt{1}{0}{0}{\pi}^\inv$ consists of those $\tbt{a}{b}{c}{d} \in \GL(2,R)$  with $c \in \pfrak$.

The element $\tbt{0}{\pi}{-1}{0}$ of $\GL(2,k)$ induces an involution on $\Tcal$ which fixes no vertex, but interchanges the vertices $\Lambda_0$ and $\Lambda_1$ of the distinguished edge $C$. In particular,


\paragraph*{BT-Tree}
Setup from \cite{chinburgEmbeddingTheoremQuaternion1999}. If $D$ is a maximal order in $M(2,F)$, then there exists a $g\in \PGL(2,F)$ such that $g D g^\inv = M(2,R)$. If $D'$ is any other maximal order, we may pick among such $g$ one for which $gD'g^\inv = \Tbt{R}{\pfrak^n}{\pfrak^{-n}}{R}$ for a nonnegative integer $n:=d(D,D')$. Alternatively, $d(D,D')$ may be characterized as the the positive integer $n$ such that $D/(D\cap D') \approx R/\pfrak^n$ as $R$ modules. We write $\Tcal$ for the tree whose vertices are maximal orders in $M(2,F)$, and two orders $D,D'$ are adjacent, written $D \sim D'$ if $d(D,D')=1$.

The function $g \mapsto \ord(\det g) \mod 2 $ is a homomorphism $\GL(2,F) \to \Zbb/2\Zbb$ which factors through $\PGL(2,F)$. We say $g\in \GL(2,F)$ or its image in $\PGL(2,F)$ is even or odd accordingly.

\begin{proposition}
    \begin{enumerate}
        \item $\PGL(2,F)$ acts on $\Tcal$ (by conjugation $D\mapsto gDg^\inv$) transitively on both vertices and edges. The restriction of this action to $\PSL(2,F)$ has two orbits of vertices: two orders are conjugate by an element of $\PSL(2,F)$ if and only if they are an even distance apart. An element $\overline{g} \in \PGL(2,F)$ interchanges these two orbits if and only if any lift $g \in \GL(2,F)$ has $\ord(\det(g))$ odd.
        \item Any maximal compact subgroup of $\PGL(2,F)$ is the stabilizer of an edge or a vertex.
        \item For $g \in \GL(2,F)$ and $D\in \Tcal$, one has $d(D,g D g^\inv) \equiv \ord(\det (g))$ mod $2$.
        \item An odd element fixes no vertex.
        \item An even element of $\PGL(2,F)$ which fixes an edge setwise actually fixes that edge pointwise.
        \item If an element of $\PGL(2,F)$ fixes a pair of vertices, then it fixes every vertex on the unique path between them.
    \end{enumerate}
\end{proposition}

The discriminant of the characteristic polynomial of an element $g\in \GL(2,F)$ is $\disc(g) =(\tr g)^2 - 4 \det g$. The function $g \mapsto \disc(g)/\det(g)$ is invariant under the center, and thus descends to a function on the quotient $\PGL(2,F)$, which we will call by the same name.

\begin{proposition}
    For an element $g\in \GL(2,F)$, with image $\overline{g}$ in $\PGL(2,F)$, the following are equivalent:
    \begin{enumerate}
        \item $\overline{g}$ fixes an edge or a vertex of $\Tcal$.
        \item $\disc(g)/\det(g) \in R$
        \item $\overline{g}$ lies in a compact subgroup of $\PGL(2,F)$.
    \end{enumerate}
    In this case, $\overline{g}$ fixes an edge in $\Tcal$ if and only if at least one of the following conditions is true:
    \begin{enumerate}
        \item $g \in F^\times$
        \item $\ord \det (g)$ is odd
        \item the subalgebra $F[g]$ of $M(2,F)$ is not a field
        \item $\disc(g)/\det(g) \in \pfrak$.
    \end{enumerate}
    If all of these conditions fail, then $g$ fixes exactly one vertex and no edges.
\end{proposition}


\paragraph*{Subsets of $\Tcal$.}




\paragraph*{Invariant subtrees}
Let $H$ be a group, and $\rho: H \to \GL(2,k)$ be a representation. Then $H$ acts (through $\rho$) on $\Tcal$, and the set of fixed points $\Tcal(\rho)$ is a (possibly empty) connected subtree of $\Tcal$. First, observe the following:
\begin{proposition}
    The following are equivalent:
    \begin{enumerate}
        \item $\Tcal(\rho)$ is nonempty.
        \item the image of $\rho(H)$ in $\PGL(2,k)$ is relatively compact
        \item the image of $\rho(H)$ in $\PGL(2,k)$ is bounded
        \item for all $g\in H$,  $\frac{(\tr\rho(g))^2}{\det \rho (g)}-4 \in R$
              \ite m there is an equivalent representation $\rho': H \to \GL(2,k)$ such that $\rho'(H)$ is contained in $k^\times\cdot \GL(2,R)$ or the normalizer of $K_0(\pfrak)$.
    \end{enumerate}
\end{proposition}



A theorem of \cite{bellaicheSousgroupesGLArbres2014} demonstrates that $\Tcal(\rho)$ must take a highly restricted form.
\begin{proposition}

\end{proposition}

% \begin{remark}
%     \begin{enumerate*}
%         \item The action of the center $k^\times$ of $\GL(2,k)$ on lattices $L$ factors through $R^\times \lmod k^\times $, as $xL = L$  if (and only if) $x \in R^\times$. 
%     \end{enumerate*}
% \end{remark}



\paragraph*{Local Computations}
For now, let $k$ nonarchimedian local field of characteristic $0$, with ring of integers $R$ and maximal ideal $\pfrak$, with uniformizer $\pi$. Let $V$ be a two dimensional $k$-vectorspace, and $A = \End_k(V)$. Then every maximal order in $A$ takes the form
\begin{align*}
    \End_R(L) = \{ x \in A : xL\leq L\}
\end{align*}
where $L$ is a free $R$-module of rank $2$ in $V$. If $M = xL$ for some $x\in A^\times = \Aut_k(V)$, then $\End_R(M) = x\End_R(L)x^\inv$. As $A^\times$ acts transitively on the set of lattices in $V$ (by left-translation), it acts transitively on the set of maximal orders in $A$ (by conjugation). Further, if $\End_R(L)=\End_R(M)$ for two lattices $L,M$ in $V$, then there exists an $x \in k^\times$ such that $L = x M$. In this case, we say $L$ and $M$ are homothetic, and conclude that there is an $k^\times \lmod A^\times = \PGL(V)$ equivariant bijection between homothety classes of lattices in $V$ and maximal orders in $A$.

Given two lattices $L$ and $M$ in $V$, there exists a basis $u,v$ of $L$ such that $\pi^a u , \pi^b v$ is a basis for $M$ (elementary divisor theorem) and we set $d(L,M) = |a-b|$. If $x,y\in k^\times$ then $d(xL,yM)=d(L,M)$ so that $d$ descends to a function on homothety classes, or equivalently a function on maximal orders.


<<<<<<< HEAD
\cite{sallyFourierTransformOrbital1983}
\bibliographystyle{plain}
\bibliography{references}
>>>>>>> 7fd919c ( Changes to be committed:)
=======

% \printbibliography
>>>>>>> 8cbdf03 ( On branch master)
\end{document}
