\documentclass{amsart}
\usepackage{cleveref}
\usepackage{mathtools}
\usepackage{geometry}
\usepackage{amsmath}
\usepackage{amsthm}
\usepackage{amssymb}
\usepackage{quiver}
\usepackage{verbatim}
\usepackage{tikz}
\input{theorems}
\setlength{\parindent}{0em} \setlength{\parskip}{1em}

\usepackage{mycros}


\begin{document}
	%%%%%%%%%%%%%%%%%%%%%%%%%%%%%%%%%%%%%%%%%%%%%%%%%%%%%%%%%%%%%%%%%%%%%%%%%%%%%%%%
	
	\section{pseudo-introduction}
	%%%%%%%%%%%%%%%%%%%%%%%%%%%%%%%%%%%%%%%%%%%%%%%%%%%%%%%%%%%%%%%%%%%%%%%%%%%%%%%%
	
	Let $M$ be a compact Riemannian manifold. The eigenvalues of the Laplace operator $\Delta$ acting on $L^{2}(M)$ form a discrete sequence $0 =\lambda_{0}\leq \lambda_{1 }\leq \ldots $. The multiset of eigenvalues $\{\lambda_{i}\}$ is the \textbf{Laplace spectrum} of $M$. We say that two manifolds are \textbf{isospectral} if their Laplace spectra coincide. A manifold is \textbf{absolutely spectrally rigid} if the only manifolds to which it is isospectral are in fact isometric.
	
	In this article, we prove the following
	\begin{thm}
		Let $k$ be a totally real number field with ring of integers $R$, and $A$ a quaternion algebra over $k$ with type number $1$. Further, suppose that there is a unique real place of $k$ at which $A$ is unramified. Let $\Ocal$ be a maximal order in $A$, $\Ocal^{1}$ the multiplicative group of norm $1$ units. For an ideal $I$ in $R$, let $\Ocal^{1}(I) \leq \Ocal^{1}$ be the kernel of the reduction map. Denote by $\Gamma$ (resp. $\Gamma(I)$) the image of $\Ocal^{1}$ (resp. $\Ocal^{1}(I)$) in $\PSL(2,\Rbb)  = \Isom^{+}( \Hbb)$. Suppose that $I$ is not divisible by any prime in $R$ at which $A$ is ramified. Then the Riemann surface $\Gamma(I) \lmod \Hbb$ is absolutely spectrally rigid.
	\end{thm}
	
	\begin{comment}
		[TODO] identify the length spectrum for arithmetic hyperbolic surfaces coming
		from maximal orders in quaternion algebras
	\end{comment}
	
	\begin{comment}
		\section{background spectral theory}
		\subsection{The spectral theorem}% [TODO] cite LA TAKHTAJAN etudes
		Let $H$ be a Hilbert space, and $A$ a closed, self-adjoint linear operator on $H$. The values
		$\lambda \in \Cbb$ such for which the
		\begin{equation}
			\label{eq:resolvent}
			R_\lambda (A) = (A - \lambda \id )^\inv
		\end{equation}
		exists (as a bounded operator defined on all of $H$) is called the
		\textbf{resolvent set} $\rho(A)$ of $A$. The resolvent set is open in $\Cbb$;
		its complement $\sigma(A)$ is the \textbf{spectrum set} of $A$. If $A$ is a
		self-adjoint operator, then $\sigma(A) \subset \Rbb$.
		
		The spectral theorem of von Neumann associates to the self adjoint operator $A$
		a projection-valued countably additive measure $\mathbf{E}$ on the
		$\sigma$-algebra of borel subsets of $\Rbb$ such that $\Eloc(\emptyset) = 0 $
		and $\Eloc(\Rbb) = \id$, that
		\begin{equation}
			\label{eq:specthm}
			D(A) = \{ f \in H : \int_{\Rbb} \lambda^2 \dop \ip{\Eloc_\lambda f }{f} \},
		\end{equation}
		and for $f \in D(A)$,
		\begin{equation}
			\label{eq:specthm2}
			A f = \int_{\Rbb} \lambda \dop \Eloc_\lambda f,
		\end{equation}
		where $E_\lambda = \Eloc((-\infty, \lambda))$.
		\subsection{Determinants}
		In this section $C$ is a compact self-adjoint operator of trace class. Its
		\textbf{characteristic determinant} is the entire function
		\begin{equation}
			\label{eq:fredet}
			p_C (\lambda) : \lambda \mapsto \det ( \id - \lambda C ) : = \prod_i (1 - \lambda_i \lambda)
		\end{equation}
		where the eigenvalues $\lambda_i$ of $C$ are repeated according to their
		multiplicity. Much like the characteristic polynomial of an operator on a finite
		dimensional vectorspace, the characteristic determinant $p_C(\lambda)$ has
		zeroes at $\lambda in \sigma(C)$ with order $\dim \ker (C - \lambda \id))$.
		
		Now suppose that $C$ is invertible (so that, in particular, $p_\lambda(0) \neq
		0$) and set $A = C^\inv$. Then the formula
		\begin{equation}
			\label{eq:dlog}
			\frac{d}{d\lambda} \log \det (\id - \lambda C) = -\Tr R_\lambda (A)
		\end{equation}
		holds. When $R_\lambda(A)$ is an integral operator acting on some space $H =
		L^2(X)$ via an integral kernel $R_\lambda (x,y)$ on $X\times X$, the trace may be
		computed via
		\begin{equation}
			\label{eq:trace}
			\Tr R_\lambda (A) = \int_X R_\lambda (x,x) \dop x.
		\end{equation}
		\subsection{Storing spectral data}
		Now let $M$ be a compact connected, Riemannian manifold without boundary. Let
		$\Delta$ be the laplace-beltrami operator on $M$. Its resolvent
		$R_\lambda(\Delta) = (\Delta - \lambda \id)^\inv$ is compact and self-adjoint. We
		consider its characteristic determinant: [TODO bloviate]
		
		We associate to each $M$ some generating functions. For $t>0$, the \textbf{heat
			trace} of $M$ is
		\begin{equation}
			\label{eq:theta}
			\Theta_M(t) = \Tr (\exp(-t\Delta_M ) - \id ) = \sum_{\lambda \in \sigma(\Delta_M)} m_\lambda \exp (-t \lambda).
		\end{equation}
		For $s \in \Cbb$ the \textbf{resolvent determinant}
		\begin{equation}
			\label{eq:zeta}
			Z_M(\lambda) = \det (1 - \lambda R_\Delta(-1))
		\end{equation}
		
		
		\begin{prop}
			Let $M$ and $M'$ be compact connected Riemannian manifolds without boundary.
			The following properties are equivalent
			\begin{enumerate}
				\item $M$ and $M'$ are isospectral
				\item $\Theta_M = \Theta_{M'}$ (as functions on $\Rbb_{>0}$)
				\item $Z_M = Z_{M'}$ (as functions on $\Cbb$).
			\end{enumerate}
		\end{prop}
		[TODO bloviate]
	\end{comment}
	
	
	\section{Audible properties}
	\subsection{Heat invariants}
Let $M$ be a compact Riemannian manifold. Enumerate the spectrum of its Laplace-Beltrami operator as $0 = \lambda_0 < \lambda_1 \leq \lambda_2 \leq \cdots $, where each eigenvalue is repeated according to its multiplicity. Pick an $L^2(M)$-orthonormal basis $\{ \phi_j  \}$ of corresponding eigenfunctions.  The \textbf{heat kernel} of $M$ is the function $K : \R_{>0} \times M \times M$ defined by 
\begin{equation}
	K(t,x,y) = \sum_{j} e^{-t \lambda_j} \phi_j(x)\phi_j(y). 
\end{equation}	



	\subsection{Scope: Riemannian manifolds}
	\begin{thm}
		Volume is an audible property.
	\end{thm}
	\begin{proof}
		[TODO: use asymptotics of $\Theta_M$]
	\end{proof}
	\begin{thm}
		Dimension is an audible property.
	\end{thm}
	\begin{proof}
		[TODO: use asymptotics of $\Theta_M$]
	\end{proof}
	\subsection{Scope: smooth, compact Riemannian surfaces}
	\begin{thm}
		Suppose $\dim M = 2$ and that $M$ is smooth. The homeomorphism class of
		$M$ is audible.
	\end{thm}
	\begin{proof}
		[TODO: gauss-bonnet]
	\end{proof}
	\begin{thm}
		Suppose $\dim M = 2$, and that $M$ is smooth. The property of having constant
		curvature is audible.
	\end{thm}
	\begin{proof}
		[TODO: GCB the metric of constant curvature minimizes topological
		entropy, which is a spectral invariant.]
	\end{proof}
	\subsection{Scope: Riemann surfaces}
	\begin{thm}
		Suppose $M$ is a compact Riemann surface. Then arithmeticity is audible.
	\end{thm}
	\begin{proof}
		[TODO: apply Takeuichi]
	\end{proof}
	\subsection{Scope: Arithmetic Riemann surfaces}
	\begin{thm}
		Suppose $M$ is an arithmetic Riemann surface. The commensurability class
		of $M$ is audible. \end{thm}
	\begin{proof}
		[TODO: apply Reid]
	\end{proof}
	% Let $(M,g)$ be a compact, connected Riemannian manifold without boundary. For a function $f\in
	% C^2(M)$ define the \textbf{laplace-beltrami operator}
	% \begin{equation}
		%   \label{eq:laplacian}
		%   \Delta f = \delta d f,
		% \end{equation}
	% where $d : C^1(M) \to C^1 (T^* M)$ is the exterior derivative, and $\delta : C^1
	% (T^* M) \to C^1(M)$ is its formal adjoint, defined relative to the Riemannian
	% metric $g$ on $M$. The set of $\lambda \in \cbb$ such that the equation
	% \begin{equation}
		%   \label{eq:spec}
		%   \Delta f  = \lamba f  \quad \text{ for some } f\in L^2(M)
		% \end{equation}
	% has a solution is called the \textbf{eigenvalue set} of $M$. The
	% \textbf{eigenvalue spectrum} of $M$ is the multiset of such $\lambda$, repeated
	% according to their multiplicty $m_\lambda = \dim \ker (\Delta - \lambda \id)$.
	% \subsection{Storing spectral data: traces and determinants}
	\section{The proof}
	Now let us fix all of data $(k, A , \Ocal, I)$ as in the main theorem. Sequentially applying the
	theorems of the preceeding section, we may assume that a manifold $M$
	isospectral to $\Gamma(I) \lmod \Hbb$ is of the form $M = \Lambda \lmod \Hbb$
	for some subgroup $\Lambda$ of $\Gamma$ with the same index as  $\Gamma(I)$.
	
	To prove the theorem, it suffices to show that $\Lambda$ is in fact conjugate
	(in $A^\times$) to $\Gamma(I)$. To this end, we convert the problem into a local
	one.
	
	[TODO: get to the prime power setting].
	
	
	
	\section{Arithmetic Subgroups of algebraic groups}
	A matrix group $G \leq \GL(n,\Cbb)$ is said to be \textbf{algebraic} if it consists of all invertible matrices whose coefficients annihilate some set of polynomials on $M(n,\Cbb)$. If this set of polynomials can be taken with coefficients in some subring $R \leq \Cbb$, then this group is said to be \textbf{defined over} R.  
	
	
	
	
	%%%%%%%%%%%%%%%%%%%%%%%%%%%%%%%%%%%%%%%%%%%%%%%%%%%%%%%%%%%%%%%%%%%%%%%%%%%%%%%%
	\subsection{Base case}
	%%%%%%%%%%%%%%%%%%%%%%%%%%%%%%%%%%%%%%%%%%%%%%%%%%%%%%%%%%%%%%%%%%%%%%%%%%%%%%%%
	Let $X$ denote the \textbf{Bruhat-Tits tree} for $\SL_2(k)$. The vertices of $X$ are homothety classes of lattices in $k^2$, and two vertices $x,y \in X$ are adjacent if there exist lattices $L_x \in x$ and $L_y \in y$ so that $L_y \leq L_x$ and $L_y/ \pi L_x$ is a $\kfrak$-line in $L_x / \pi L_x \approx \kfrak^2$. We write $d: X \times X \to \Zbb_{\geq 0}$ for the graph-theoretic distance function on $X$.
	
	For a lattice $L \leq k^2$, we write $[L]$ for the vertex in $X$ corresponding to $L$. For a subset $A \subset X$ of vertices, write
	\begin{align*}
		B(A,r) & = \{ y \in X: d(x,y)\leq  r,\, \forall x \in A\} \\
		S(A,r) & = \{ y \in X: d(x,y) = r \, \forall x \in A\}    
	\end{align*}
	for the ball and sphere about $A$ of radius $r$, respectively.
	
	A \textbf{geodesic} is a non-backtracking path in $X$.  The length of a geodesic $c$ is $\sup_{x,y \in c} d(x,y) \in \Zbb_{\geq 0} \cup \{\infty\}$. For vertices $x,y
	\in X$ write $[x,y]$ for the unique geodesic starting at $x$ and finishing at $y$.
	
	For a subset $A \subset X$ we say that a vertex $x\in A$ is an {\bf interior point} of $A$ and write $x\in A^\inte$ if $B(x,1) \subset A$. Otherwise we say $x$ is a {\bf boundary point} of $A$ and write $x\in \partial A$. We say $x$ is an \textbf{extremal point} in $A$ if either: $A=\{x\}$ or $\card(S(x,1) \cap A) = 1$.
	
	Let $\mathcal{G}= \SL_{2}$, and write  $G=\mathcal{G}(k)$  and $K= \mathcal{G}(R)$. $G$ acts on the set of lattices in $k^2$, and this action passes to one on $X$.  For a subgroup $H \leq G$, we write $X^H$ for its fixed point set on $X$. For a subset $A \subset X$,  and a subgroup $H\leq G$, we write $H_A$ for its (pointwise) stabilizer in $H$.
	
	% \paragraph{A homothety section} Fix once and for all a root vertex $x_o \in X$, and a representative lattice $L_o \in x_o$. With this choice, we may associate to each vertex $x\in X$ a unique lattice $L_x \leq L_o$ satisfying the property:  $\pi^{d(x,x_o)} L_o \leq L_x \leq L_o$.
	
	For each vertex $x$ and integer $n\geq 1$, we have a $R/\Pcal^{n}$-representation $\rho_{x} : G_{x} \to \SL(L_x / \pi^{n} L_{x}) \approx (R / \Pcal^{n})$
	
	Upon picking a basis $e_{1},e_{2}$ for $L_{o}$ we may identify $G_{x_{o}}$ with $\SL_{2}(R)$. The action of $\SL_{2}(k)$ on $X$ has exactly two orbits. Letting $\alpha = \tbt{1}{0}{0}{p}$, the set $\{x_{o}, \alpha x_{o}\}$ constitutes a fundamental domain for this action. The respective stabilizers $G_{x_{o}}$ and $G_{\alpha x_{o}} = \alpha G_{x_{o}} \alpha^{\inv }$ are representatives of the two conjugacy classes of maximal compact subgroups.
	The following lemma is standard:
	\begin{lemma}\label{lem:nball}
		For a vertex $x \in X$, there is a $G_{x}$ equivariant bijection between $n$-sphere $S(x,n)$ and cyclic $R/\Pcal^{n}$-submodules of $L_{x} / \pi^{n} L_{x}$. Using a basis for $L_{x}$, the latter set is naturally identified with $\Pbb^{1}(R/\Pcal^{n})$
	\end{lemma}
	
	\begin{lemma}\label{lem:XH}
		Let $H \leq G$ be a subgroup such that $X^{H}$ is nonempty.
		\begin{enumerate}
			\item\label{item:XH1} For a subgroup $H\leq G$ and a vertex $x\in X^{H}$, the bijection in \cref{lem:nball} restricts to one between $H$-fixed vertices $X^{H}\cap S(x,n)$, and  $H$-stable $R/\Pcal^{n}$-submodules of $L_{x}/\pi^{n}L_{x}$.
			\item\label{item:XH2} For $x\in X^{H}$, one has $B(x,n) \leq X^{H}$ if and only if $H$ acts by scalars under $\rho_{x,n}$.
			\item\label{item:XH3} A vertex $x$ is a non-extremal boundary point of $X^{H}$ if and only if $\rho_{x,1}$ decomposes as a direct sum of (distinct) nontrivial characters.
			\item\label{item:XH4} A vertex $x$ is extremal in $X^{H}$ if and only if the $\mathfrak{k}$-representation $\rho_{x,1}$ is reducible but indecomposable.
		\end{enumerate}
	\end{lemma}
	\begin{proof}
		\cref{item:XH1} and \cref{item:XH2} follow immediately from the definitions and the equivariance in \cref{lem:nball}.
		
		\cref{item:XH3}: Suppose $x$ is a nonextremal boundary point of $X^{H}$. There are then exactly two vertices $y,z \in X^{H}$ adjacent to $x$. Under the identification of $S(x,1)$ with $\Pbb^{1}(\kfrak)$, we see that $H$ must act as a group of hyperbolic transformations with common fixed points $y,z$.
		
		\cref{item:XH4} Suppose $x$ is extremal in $X^{H}$. Then $x$ has exactly one neighbor in $X^{H}$. Under the identification of $S(x,1)$ with $\Pbb^{1}(\kfrak)$, we find that $H$ must act as a group of parabolic transformations.
	\end{proof}
	
	
	\begin{prop}\label{prop:ribetgl2}[ref ribetGL2]
		Let $g\in \SL_{2}(R)$ and $x \in \partial X^{\langle g \rangle }$. For an integer $n>0$, the following are equivalent.
		\begin{enumerate}
			\item\label{item:ribet1} The characteristic polynomial $p_{g}(T)$ is reducible mod $\Pcal^{n}$.
			%\item $\tr(g)^{2}-4$ is a square modulo $\Pcal^{n}$.
			\item\label{item:ribet2} There exists a geodesic of length $n$ based at $x$ and contained in $X^{\langle g \rangle}$.
		\end{enumerate}
	\end{prop}
	\begin{proof}
		Assuming \cref{item:ribet1}, pick $\alpha,\beta \in R$ such that $p_{g}(T) \equiv (T-\alpha)(T-\beta)$ mod $\Pcal^{n}$. Then since $x$ is a boundary point, $g$ does not act as a scalar on $L_{x}/\pi L_{x}$. Consequently, $L_{x}/\pi L_{x}$ is a cyclic $R[g]$ module. Pick a vector $v \in L_{x}$ such that $\{v, gv\}$ projects to a basis of $L_{x}/\pi L_{x}$. By Nakayama's lemma, $\{v,gv\}$ projects to a basis of $L_{x}/\pi^{n}L_{x}$.
		
		Let $w = (g-\beta)v$. Then, compute modulo $\Pcal^{n}$:
		\begin{align*}
			g w & \equiv (g^{2} -g \beta ) v                            \\
			& \equiv ((\alpha +\beta )g -\alpha\beta g - g\beta  )v \\
			& \equiv \alpha(g-\beta )v                              \\
			& \equiv \alpha w                                       
		\end{align*}
		Thus, the cyclic $R/\Pcal^{n}$-submodule of $L_{x}/\pi^{n}L_{x}$ spanned by $w$ is $g$ stable. The corresponding vertex lies in $X^{\langle g \rangle}$ and is at distance $n$ from $x$ as desired.
		
		Now assume that $y \in X^{\langle g \rangle }$ has $d(x,y)=n$. Pick a lattice
		$L_y$ representing $y$ with $\pi^n L_x < L_y <L_x$. Then $L_y$ projects to a
		$g$-stable free rank 1 $R/\Pcal^n$ submodule of $L_x / \pi^n L_y$ on which $g$
		must act by an element $\alpha \in (R/\Pcal^n)^\times$. It follows that
		$(t-\alpha)$ divides $p_g(t)$ modulo $\Pcal^n$, proving reducibility.
	\end{proof}
	
	
	\begin{prop}
		Let $g\in \SL_{2}(R)$. For an integer $n >0$, the following are equivalent:
		\begin{enumerate}
			\item $\tr(g) = \pm 2 \mod \Pcal^{2n}$
			\item There is a point $x \in X^{g}$ such that $B(x,n)\subset X^{g}$.
		\end{enumerate}
	\end{prop}
	\begin{proof}
		The argument is by induction on $n$.
		
		Base case: suppose $\tr(g)=2 \pm 2 \mod \Pcal^{2}$. By \cref{prop:ribetgl2} there is a geodesic $\gamma$ of length $2$ contained in $X^{g}$.  Conjugating by an element of $\GL_{2}(k)$ if needed, we may assume that $\gamma$ takes the form $(y, x_{o},z)$. Since $g$ has two fixed points in the $1$ neighborhood of $x_{o}$, it acts semisimply on $L_{x_o}/\pi L_{x_{o}}$ say with eigenvalues $\alpha,\beta \in R/\Pcal$. These eigenvalues must satisfy $\alpha + \beta =\pm 2 \mod \Pcal$ and $\alpha \beta = 1  \mod \Pcal$, but then $g$ acts as $\pm \id$ and the claim follows.
		
		Suppose now that for all $k<n$ that $\tr(g) = \pm 2 \mod \Pcal^{2k}$ iff and only if $g$ fixes some $k$ ball in $X$. Let $\tr(g) = \pm 2 \mod \Pcal^{2n}$. Applying \cref{prop:ribetgl2}  as before, there is some geodesic $\gamma$ passing through $x_{o}$ of length $2n$ which is fixed pointwise by $g$.
		
		
	\end{proof}
	
	
	%%%%%%%%%%%%%%%%%%%%%%%%%%%%%%%%%%%%%%%%%%%%%%%%%%%%%%%%%%%%%%%%%%%%%%%%%%%%%%%%
	\subsection{Some projective geometry}
	%%%%%%%%%%%%%%%%%%%%%%%%%%%%%%%%%%%%%%%%%%%%%%%%%%%%%%%%%%%%%%%%%%%%%%%%%%%%%%%%
	Let $F$ be a field, let $V$ be a two dimensional vectorspace over $F$. The \textbf{projective line} over $F$ is the set of $F$-lines in $V$. There are several useful coordinate systems that one can put on $\Pbb^{1}(F)$.
	
	The first system requires no choices: the association of a nonzero vector $v$ with the $F$-line that it spans yeilds a surjection $V\setminus \{0\} \to \Pbb^{1}(F)$. The multiplicative group $F^{\times}$ acts transitively on the fibers, yeilding our first identification: $\Pbb^{1}(F) = F^{\times}\rmod (V - \{0\})$
	
	A bijection $\Pbb^{1}(F) \to \Pbb^{1}(F)$ is a \textbf{projective transformation} if it lifts to a linear automorphism of $V$. Write $G'$ for the group of projective transformations.
	
	
	\begin{lemma}
		A nonidentity projective transformation can have at most $2$ fixed points on $\Pbb^{1}(F)$
	\end{lemma}
	\begin{proof}
		The points in $\Pbb^{1}(F)$ fixed by $g$ correspond to eigenlines of its lifts to $\GL(V)$. Since $V$ is two dimensional, it supports at most two linearly independent lines.
	\end{proof}
	We classify nonidentity transformations accordingly:
	\begin{itemize}
		\item say $g$ is \textbf{hyperbolic} if it fixes two points on $\Pbb^{1}(F)$
		\item say $g$ is \textbf{parabolic} if it fixes exactly one point on $\Pbb^{1}(F)$
		\item say $g$ is \textbf{elliptic} if it fixes no points in $\Pbb^{1}(F)$.
	\end{itemize}
	The following lemma demonstrates how to determine the category of a projective transformation in terms of its lifts. For a linear transformation $\tilde{g} \in \GL(V)$ write $p_{\tilde{g}}(T) = \det(\tilde{g}-T \id) \in F[T]$ for its characteristic polynomial and $\delta(\tilde{g}) = \tr(g)^{2} -4 \det(g)$ for its discriminant.
	\begin{lemma}
		Let $g$ be a nonidentity projective transformation.
		\begin{itemize}
			\item $g$ is hyperbolic if and only if any lift $\tilde{g}$ is semisimple and diagonalizable over $F$, for any lift $\tilde{g}$. This is so  if and only if $\delta(\tilde{g})  \in (F^{\times})^{2}$.
			\item $g$ is parabolic if and only if $\tilde{g}$ is not semisimple for any lift $\tilde{g}$. This is so if and only if $\delta(\tilde{g})= 0$.
			\item $g$ is parabolic if and only if $\tilde{g}$ is not semisimple for any lift $\tilde{g}$. This is so if and only if $\delta(\tilde{g}) \in F^\times \setminus (F^{\times})^{2}$.
		\end{itemize}
	\end{lemma}
	
	
	%%%%%%%%%%%%%%%%%%%%%%%%%%%%%%%%%%%%%%%%%%%%%%%%%%%%%%%%%%%%%%%%%%%%%%%%%%%%%%%%
	% \section{another attempt}
	%%%%%%%%%%%%%%%%%%%%%%%%%%%%%%%%%%%%%%%%%%%%%%%%%%%%%%%%%%%%%%%%%%%%%%%%%%%%%%%%
	% The main theorem of this section is
	% \begin{thm}
		% 	Let $G \leq	\SL_2(R)$ be a subgroup satisfying $\tr(g)^2 - 4 \in
		% 	\Pcal^{2n}$ for all $g\in G$. Then there exists a $g\in \GL_2(k)$ such that
		% 	$gGg^\inv \leq K(n) = \{g \in \SL_2(R) : g - \id \in \Pcal^n
		% 	M_2(R)\}$.
		% \end{thm}
	
	
	
	% \begin{prop}
		% 	For a subgroup $H \subseteq \GL_2(k)$, fix a vertex $x\in X^G$ and let $L_x \in
		% 	x$ be a representative lattice. There is a bijection between
		% 	\begin{enumerate}
			% 		\item Vertices $x\in X^G$ with $d(x,y)=n$, and
			% 		\item $G$-stable, rank-1, free $R/\Pcal^n$ submodules of $L_x / \pi^n
			% 			L_x$.
			% 	\end{enumerate}
		% 	Given a point $y\in X^G$ with $d(x,y)=n$, the corresponding submodule of $L_x
		% 	/ \pi^n L_x$ is given by $L_y / \pi^n L_x$ where $L_y$ is chosen so that
		% 	$\pi^n L_x < L_y < L_x$.
		% \end{prop}
	
	% \begin{prop}
		% 	For a subgroup $G \subseteq \GL_2(K)$, fix a vertex $x\in X^G$ and a representative $G$-stable lattice $L_x \in x$. Then $G$ acts on $L_x /\pi^r L_x$ by scalars if and only if $B(x,r) \subseteq X^G$.
		% \end{prop}
	
	
	% For an element $g\in M_2(k)$, we write $p_g(t)= \det(g -t \id) \in k[t]$ for its characteristic polynomial.
	
	% \subsection{Elementwise results}
	% 	Our first lemma characterizes those elements of $\GL_2(k)$ which fix some vertex. \begin{lemma}
		% 	Let $g\in \GL_2(k)$. Then $X^g$ is nonempty if and only $p_g(t) \in R[t]$.
		% \end{lemma}
	% 	\begin{proof}
		% 			The $R$ module $R[g]$ is finitely generated if and only if $p_g(t) \in R[t]$. In this case, $R[g]$ is a a $g$ stable lattice in $k[g] \approx k^2$.
		% 	\end{proof}
	
	
	% \begin{cor}
		%   Suppose $g,h \in \GL_2(k)$ satisfy $X^g, X^h \neq \varnothing $. Then $X^{g} \cap X^{h} = X^{\langle g,h\rangle} \neq \varnothing $ if and only if $p_{gh}(t) \in R[t]$.
		% \end{cor}
	% \begin{cor}
		%   For a subgroup $G\leq \GL_2(k)$, $X^G\neq \varnothing $ if and only if $p_g(t) \in R[t]$ for all $g\in G$.
		% \end{cor}
	
	
	
	% The next lemma characterizes elements $g\in \GL_2(k)$ for which fix a geodesic of length $n$.
	% \begin{lemma}
		%    Let $g\in \GL_2(k)$ and $x \in \partial X^g$ be an boundary point. Then there exists a vertex $y \in X^g$ with $d(x,y) = n $ if and only if $p_g(t)$ is reducible mod $\Pcal^n$.
		% \end{lemma}
	% 	\begin{proof}
		% 		Let $\alpha,\beta \in R$ satisfying $p_g(t) \equiv (t - \alpha)(t-\beta) \mod \Pcal^n$. Since $x$ is a boundary point of $X^g$, the action of $g$ is nonscalar on $L_x/\pi L_x$. Thus, we may pick a vector $u \in L_x$ such that $\{u,gu\}$ projects to a basis of $L_x/\pi L_x$. By Nakayama's lemma, we may assume that $\{u,gu\}$ projects to an $R/\Pcal^n$-basis for $L_x / \pi^n L_x$.
		
		% 		By the cayley-hamilton theorem, the vector $v = (g-\beta ) u$ satisfies $g v \equiv \alpha v \mod \Pcal^n$. Let $L = R v + \pi^n L_x$. Then the vertex $y=[L]$ lies in $X^g$ and satisfies $d(x,y)=n$ as desired.
		% 	\end{proof}

\begin{lemma}
	For a subgroup $H \leq \SL(2,k)$, suppose $X(H)$ contains a geodesic segment $[y,z]$ of length two. Denote its midpoint $x$. Then there exists a character $\chi : H \to \mathfrak{f}^{\times}$ such that $\rho_{x} \approx \chi \oplus \chi^{\inv}$. 
\end{lemma}	
	
\end{document}