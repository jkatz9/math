\documentclass{amsart}
\usepackage{geometry}
\usepackage{amsmath}
\usepackage{amssymb}
\usepackage{quiver}
\usepackage{mycros}
\input{theorems}
\setlength{\parindent}{0em}
\setlength{\parskip}{1em}

\begin{document}
\section{Reductive Groups}
  Let $G$ be a connected algebraic group over an algebraically closed field $k$.  Say that $G$ is semisimple if the only smooth connected solvable normal subgroup of $G$ is trivial, and reductive if the only smooth connected unipotent normal subgroup of $G$ is trivial. Any unipotent group over an algebraically closed field has a composition series in which each quotient is isomorphic to $\Gbb_a$. For reductive $G$, the inner action of $G$ on itself induces a homomorphism of $k$-group functors $G \to \Aut(G)$, and automorphisms of $G$ can be differentiated to elements of $\Aut(\gfrak)$: this is the adjoint action of $G$ on $\gfrak$.

  A representation of a torus $T$ on a vectorspace $V$ is tantamount to a grading of $V$ by $X(T) = \Hom(T,\Gbb_m)$. When $T$ is a (maximal) torus  in reductive $G$ and $V = \gfrak$, the decomposition is
        \begin{align*}
          \gfrak = \tfrak \oplus \bigoplus_{\alpha \in R(T,G)} \gfrak_\alpha
        \end{align*}
        where $R(G,T) \leq X(T)$ are the relative to $T$, and $\gfrak_\alpha$ is the subspace on which $T$ acts by $\alpha$. Each $\gfrak_\alpha$ (since $k$ is algebraically closed) is one dimensional: hence may be identified with $\Gbb_a$. Pulling back the natural action of $\Gbb_m$ on $\Gbb_a$ by scaling through $\alpha$, we obtain an action of $T$ on $\Gbb_a$. Up to scalar, there is a unique \emph{root homomorphism} $x_\alpha : \Gbb_a \to \gfrak$ intertwining the actions of $T$ on $\Gbb_a$ and on $\gfrak$, inducing an isomorphism $\dop x_\alpha : \Lie(\Gbb_a) \approx \gfrak_\alpha$. Let $U_\alpha$ denote the corresponding subgroup of $G$.

        After normalizing $x_\alpha$ and $x_{-\alpha}$ suitably, there is a unique homomorphism $\phi_\alpha:\SL_2 \to g$ such that $\phi_\alpha(\tbt{1}{a}{0}{1}) = x_\alpha(a)$ and $\phi_{\alpha} (\tbt{1}{0}{a}{1}) = x_{-\alpha}(a)$

    The dual coroots $\alpha^v \in \hom(\Gbb_m,T)$ are defined by the relation $\alpha^v(\lambda) = \phi_\alpha(\tbt{\lambda}{0}{0}{\lambda^\inv})$

    For each $\alpha \in R$, there is an involution $s_\alpha : X(T) \to X(T)$ defined by $s_\alpha (x) = x - \ip{x, \alpha^v}\alpha$, which restricts to a permutation on $R$.

    The \emph{finite weyl group} associated to the root datum $(R,X,R^v,X^v)$ is the group generated by the $s_\alpha$ for $\alpha \in R$.

    The weyl group acts transitively on the choices of simple roots $\sigma \subset R$, and subordinate to any such choice on defines the \emph{positive roots}$R_+=\{\alpha \in R : \alpha \in \sum_{\sigma \in \Sigma} \Zbb_{\geq 0} \sigma \}$, \emph{simple reflections} $S_f = \{s_\alpha : \alpha \in \Sigma\}$, and the \emph{dominant weights} $X_+ = \{ \lambda \in X : \ip{\lambda}{\alpha^v}\geq 0, \, \alpha \in\Sigma \}$.
    (easymotion-prefix)ll
    A choice of $R_+$ yeilds a \emph{Borel subgroup} $B^+$ containing $T$ such that $B^+=TU^+$ where $U^+$ is the subgroup generated by the $U_\alpha$ for $\alpha \in R$

    \subsection{Parabolic subgroups: tautological representations from flag variety quotients}zo
       At the level of algebraic groups (and algebraic representations,) every rep of $G$ embeds in
       some number of copies of $k[G]$. As an affine coordinate ring, $k[G]$ is in many regards too large to deal with on its own. \emph{Parabolic subgroups} $P$ of $G$ are those for which the quotient variety $G/P$ is as small (in the algebro-geometric context) as possible.

       When $G=SL_2$, the quotient $G/B^+$ identfies with $\Pbb^1$ viz. the set of lines in $k^2$: indeed the action of $G$ on such lines is transitive, and $B^+$ is the stabilizer of the line spanned by $e_1 = (1,0)$. More generally, when $G=GL_n$, the quotient $G/B^+$ identifies with the variety $\Fcal$ of full flags $0 \leq V_1 \leq \ldots \leq V_n = k^n$ where each $V_i$ is $i$-dimensional.
\begin{mydef}
  Suppose $G$ acts on a $k$-scheme $X$ through $\sigma G \times X \to X$. A $G$-equivariant sheaf $\Fcal$ on $X$ is a sheaf of $\Ocal_X$ modules together with an isomorphism $\phi: \sigma^* \Fcal \to p_2^* \Fcal$ of $\Ocal_{G\times X}$ modules, which satisfies the cocycle condition $p^*_{23}\phi \circ (1_G \times \sigma)^*\phi = (m \times 1_X )^*\phi$.  The isomorphism $\phi$ yeilds a $G$-equivariant identification of stalks: $\Fcal_{gx} \approx \Fcal_x$ and the cocycle condition ensures that the identifications are compatible: $\Fcal_{ghx} \approx \Fcal_{hx} \approx \Fcal_x$.
\end{mydef}

For any such sheaf, the $k$-vectorspace of global sections $\Gamma(X,\Fcal)$ admits a natural representation of $G$. Conversely, for any $G$ module $V$, $G$ acts on $\Pbb(V^*)$, and the tautological bundle $\Ocal(1)$ is an equivariant line bundle for this action. One recovers the action of $G$ on $V$ from the action on global sections: $\Gamma(\Pbb(V))$

\begin{thm}[Borel fixed point theorem]
  Let $H$ be a connected solvable algebraic group acting through regular functions on a nonempty complete variety $W$ over an algebraically closed field. Then there exists a point of $W$ fixed by $H$.
\end{thm}

\begin{mydef}\label{def:associated-sheaf}
  Let $G$ be a $k$-group scheme acting freely on a $k$-scheme $X$ in such a way that $X/H$ is a scheme; let $\pi: X \to X/H$ be the projection map. The \textbf{associated sheaf functor} is
  \[ \Lcal=\Lcal_{X,H} : \{H -\text{modules}\} \to \{\text{vector bundles on} \, X/H\} \]
  defined on objects as follows: if $U \subset X/H$ is open, then
  \[\Lcal(M)(U) = \{ f \in \Hom_{\text{scheme}}(\pi^\inv (U), M_a) : f(xh) = h^\inv f(x)\}. \]
   Note: if $\pi^\inv(U)$ is affine, these sections coincide with $(M \otimes k[\pi^\inv U])^H$.
\end{mydef}


For any $\lambda \in X(T)=\Hom(X,\Gbb_m)$, let $k_\lambda$ be the representation of $B$ pulled back through the projection $B \to B/[B,B] \approx T$, and define the sheaf $\Ocal(\lambda) = \Lcal_{G,B}(k_{-\lambda})$ on $G/B$.

Given a choice of positive roots $R_+$ and corresponding Borel $B$, let $\bar{B}$ be the opposite Borel (corresponding to the choice of $-R_+$ as positive rooots) and $\bar{U}$ its unipotent radical. A consequence of the Bruhat decomposition of $G$ is that the map $\bar{U} \to G / \bar{B}$ sending $u$ to $u\bar{B}/\bar{B}$ is an open inclusion. Furthermore, the (cartesian) product map $(x_\alpha)_{\alpha \in R^+}$ yeilds parametrization of $\bar{U}$ (identiftying the latter with $\Abb^{|R_+|}$.


\section{Witt Vectors}
\begin{thm}
Let $K$ be a perfect ring of characteristic $p$.
\begin{enumerate}
        \item There is a strict $p$-ring $R$ with residue ring $K$, unique up to canonical isomorphism.
        \item There is a unique system of representatives $\tau : K \to R$ (teichmulller representatives) such that $\tau(xy)=\tau(x)\tau(y)$ for $x,y\in K$.
        \item Every element $x \in  R$ can be written uniquely in the form $x = \tau(x_n) p^n$ for $x_n \in K$.
        \item Formation of $R$ and $\tau$ is functorial in $K$.
\end{enumerate}
\end{thm}

The simplest example: take $R= \Zbb_p$ and $K = \Fbb_p$, then by Hensel's lemma, each nonzero $x \in \Fbb_p$ has a unique lift $\tau(x)$ to $\Zbb_p$, and extending $\tau$ by $0$ to $\Fbb_p$ completes the definition.

A central question: given $x = \sum \tau(x_n)p^n$ and $y = \sum \tau (y_n) p^n $ write $xy = \sum \tau(m_n) p^n$ and $x+y = \sum \tau(s_n) p^n$. How can we determine $\tau(s_n)$ and $\tau(m_n)$ in terms of $x$ and $y$?

An important
\begin{lemma}
Let $A$ be a ring, and $x,y\in A$ such that $x=y \mod pA$. Then for all $i\geq 0$ we have $x^{p^i} = y^{p^i} \mod p^{i+1}A$.
\end{lemma}

Note the two maps in play: there is the teichmuller lift $\tau : K \to R$, and an infinite sequence of maps $\pi_n=(\cdot)_n : R \to K$ such that the mapping $\cdot \mapsto \sum \tau( (\cdot)_n ) p^n$  is the identity on $R$. A preliminary goal is to understand the compositions $(x,y) \mapsto \pi_n(x+y)$ and $(x,y) \mapsto \pi_n(xy)$.

The answer is as follows:
\begin{equation*}
        s_1(x,y) = x_1 + y_1 - \sum_{n=1}^{p-1} (p/n){p \choose n} x_0^{n/p}y_0^{(p-n)/p}
\end{equation*}

\begin{definition}
        A set $P$ of natural numbers is \bf{divisor-stable} if it is nonempty and for all $n \in P$, all divisors of $n$ are also in $P$. For a divisor stable set $P$ let $\mathcal{p}{P}$ be the set of prime numbers in $P$. Let $P_p = \{p^n: n\geq 0\}$ and $P_{p(n)} = \{ p^j: 0 \leq j \leq n \}$ (these are both divisor stable).
\end{definition}


\begin{definition}
        Let $n\in \Nbb$, define the \textbf{$n$-th witt polynomial} as
        \begin{equation*}
                w_n = \sum_{d \vert n} d x_d^{n/d} \in \Zbb[ \{ X_d : d \vert n \} ].
        \end{equation*}
       For any divisor stable $P$ and any ring $A$, define
       \begin{equation*}
               W_P(A) = \prod_{n \in P} A.
       \end{equation*}
       And for $x\in W_P(A)$ write $\pi_n(x) = x_n \in A$ for the projection to the $n$-th factor. For $P = \Nbb$ write $W(A)$ for $W_P(A)$ and if $P=P_p$m, write $W_p(A)$ for $W_P(A)$.
\end{definition}

 The witt polynomials $w_n$ are then (set theoretic) maps $w_n : W_P(A) \to A$. Write $w_*$ for the cartesian product of these maps.  For $x\in W_P(A)$, the values $w_n(x)$ are called the \bf{ghost components of $x$}.

\begin{thm}
  Let $P$ be a divisor stable set. There is a unique covariant functor $W_P : \rm{Alg}_\Zbb \to \rm{Alg}_\Zbb$, such
  that for any ring $A$,
  \begin{enumerate}
    \item $W_P(A) = \prod_{n \in P} A = A^P$ as sets, and for a ring hom $f : A \to B$, one has
      \begin{equation*}
        W_P(f) ((a_n)_{n\in P})  = (f(a_n))_{n \in P}.
      \end{equation*}
    \item The maps $W_P(A) \to A$ are ring homomorphisms for all $n \in P$.
  \end{enumerate}
  Furthermore the zero element is  $(0,0,\ldots)$ and the unit element is $(1,0,\ldots)$.
\end{thm}

A remark: If $A$ is a $K$ algebra, then $W_P(A)$ need not be a $K$ algebra. For example, when $A = \Fbb_p$ and $P=\{
    p^\Nbb\}$, then $W_P(\Fbb_p)= \Zbb_p$ but the latter is not an algebra over $\Fbb_p$. Nonetheless, $W_P$ sends
    $K$-algebras to $\Zbb$-algebras.

    For a ring $A$, let $\Lambda (A) = 1 + t A[[t]]$ (a multiplicative abelian group). Then for any element $f= 1 +
    \sum_{n=1}^\infty x_n t^n  \in \Lambda(A)$, there is a unique expression $f = \prod (1 - y_n t^n)$ for $y_n \in A$.
    Furthermore, there exist polynomials $Y_n \in \Zbb[X_1,...,X_n]$ and $X'_n \in \Zbb[Y'_n,...,Y'_n]$ independent of
    $A$ such that $y_n = Y_n ( x_1 , ..., x_n )$ and $x_n = X'_n(y_1, ... , y_n)$.

    Consequently: for any ring $A$ the map $x \mapsto f_x : W(A) \to \Lambda(A)$ defined by
\begin{equation}
    f_x (t) = \prod (1-x_n t^n),
\end{equation}
where $x = (x_1, ...)$ is a bijection.


For any $\Qbb$-algebra $A$, the mercator series defines a bijection $\log : \Lambda(A) \to t A[[t]]$ with inverse given
by the exponential series $\exp : t A [[t]] \to \Lambda(A)$. In fact, $\log$ is a homomorphism of abelian groups (the
former being multiplicative and the latter additive). The map $f \mapsto -t \dop f / \dopt t $ is an automorphism of
$tA[[t]]$ (additive), and its inverse is $\int - t^\inv (\cdot ) \dop t$.  Let $D = -t \frac{\dop }{\dop t} \log
(\cdot): \Lambda(A) \to t A[[t]]$.


%\bibliographystyle{plain}
%\bibliography{bigbib}
\end{document}
