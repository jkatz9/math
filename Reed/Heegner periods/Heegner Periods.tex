\documentclass[11pt]{amsart}
\usepackage{geometry}                % See geometry.pdf to learn the layout options. There are lots.
\geometry{letterpaper}                   % ... or a4paper or a5paper or ... 
%\geometry{landscape}                % Activate for for rotated page geometry
%\usepackage[parfill]{parskip}    % Activate to begin paragraphs with an empty line rather than an indent
\usepackage{graphicx}
\usepackage{amssymb}
\usepackage{epstopdf}
\DeclareGraphicsRule{.tif}{png}{.png}{`convert #1 `dirname #1`/`basename #1 .tif`.png}
\newcommand{\half}{\mathfrak{H}}
\newcommand{\Q}{\mathbb{Q}}
\newcommand{\R}{\mathbb{R}}
\newcommand{\C}{\mathbb{C}}
\newcommand{\Z}{\mathbb{Z}}
\newcommand{\GL}{\text{GL}}
\newcommand{\SL}{\text{SL}}
\newcommand{\intring}{\mathcal{O}}
\newcommand{\gothic}[1]{\mathfrak{#1}}
\newcommand{\im}{\text{Im}}
\newcommand{\re}{\text{Re}}

\title{Zeta functions of real quadratic fields as periods of Eisenstein series}
\author{Justin Katz}
%\date{}                                           % Activate to display a given date or no date

\begin{document}
\maketitle

\section{Introduction}

Set $k=\Q(\sqrt{D})$ and $\intring_k$ its the ring of integers. The zeta function attached to $k$ is
\begin{equation*}
\zeta_k(s)=\sum_{\gothic{a}\subset \intring_k} \frac{1}{|N(\gothic{a})|^{-s}}
\end{equation*}
where the sum is over nonzero integral ideals in $\intring_k$ and $\re(s)>1$. A suitable modification of Riemann's argument for the continuation of $\zeta=\zeta_\Q$ shows that $\zeta_k$ has meromorphic continuation to the entire $s$-plane.

Let $G=\SL_2(\R)$, $\Gamma=\SL_2(\Z)$, and $P$ be the parabolic of upper triangular elements of $\SL_2(\R)$. As usual, $G$ acts on the upper half plane $\half$ by fractional linear transformations. The for complex $s$ with $\re(s)>1$, the $s^{th}$ Eisenstein series on the upper half plane is
\begin{equation*}
E_s(z)=\sum_{\gamma \in (P\cap \Gamma)\backslash \Gamma} (\im(\gamma z))^s,
\end{equation*}
which is $\Gamma$-invariant by design. For fixed $z$, the map $s\mapsto E_s(z)$ has meromorphic continuation to the entire $s$-plane.

This writeup shows that $\zeta_k$ are integrals of Eisenstein series over closed geodesics.

\section{Backstory}
In setting the stage for real quadratic fields, let's quickly rehearse an argument on imaginary quadratic fields.  In this section $k=\Q(\sqrt{D})$ with $D<0$. The sum defining $\zeta_k$ decomposes into a double sum, first over classes of ideals, then over representatives within each class,
\begin{equation*}
\zeta_k(s)=\sum_{\gothic{a}\subset \intring_k} \frac{1}{|N(\gothic{a})|^{-s}} = \sum_{\text{Classes } \gothic{a}} \hspace{10pt}  \sum_{\gothic{b}\approx \gothic{a}} \frac{1}{|N(\gothic{b})|^{-s}}.
\end{equation*}
Quadratic fields have finitely many distinct classes of ideals, so the outer sum is finite. The ideal norm is multiplicative....

\section{Real quadratic}
In this section $k=\Q(\sqrt{D})$ with $D<0$. As a $\Q$ module, $k$ is $\Q^2$. The multiplicative subgroup $k^\times$ acts transitively on $\Q^2$. Choose the basis $\{\sqrt{D},1\}$ and compute for $a+b\sqrt{D}\in k^\times$,
\begin{align*}
(a+b\sqrt{D}) \times \sqrt{D}&= a\cdot \sqrt{D} + bD \cdot 1  \\
(a+b \sqrt{D}) \times 1        &=  b \cdot \sqrt{D} + a \cdot 1 .
\end{align*}
In coordinates, we have an embedding
\begin{align*}
k^\times &\to \GL_2(\Q) \\ 
a+b\sqrt{D} &\mapsto   \begin{pmatrix}
    a & b D \\ 
    b & a     \end{pmatrix}
\end{align*}
Let $G'$ be the image of $k^\times$ in $\GL_2(\Q)$. Note that the determinant of the image of $a+b \sqrt{D}$ is $a^2-b^2 D=N(a+b\sqrt{D})$. The existence of nontrivial units in $\intring_k$ implies that the subgroup $H'=G'\cap G$ is nontrivial. As a subgroup of $G=\SL_2(\R)$, the group $H'$ sensibly acts on the upper half-plane. Taking the trace of a generic matrix  $ Tr \begin{pmatrix}
    a & b D \\ 
    b & a     \end{pmatrix}=2a$ and recalling that the units of $\intring_k$ are integral shows that all nonidentity elements of $H_1$ are hyperbolic. As such, any nonidentity element of $H'$ fixes two distinct points on $\R \cup \{ \infty \}$.

Although $H'$ is discrete in $G$, it lies in a one parameter subgroup $H$ of $G'$ defined by parameterizing the solutions of $a^2-b^2 D=1$ viz
\begin{equation*}
H'\subset H=\{ \begin{pmatrix}
    \cosh(t) & \sinh(t)\sqrt{D} \\ 
    \sinh(t)/\sqrt{D} & \cosh(t)     \end{pmatrix} : \hspace{10pt} t\in \R \}
\end{equation*}
In fact, $H'=H\cap \Gamma$. Denote an element of $H$ by $h_t$. One can compute that each $h_t$ fixes $-\sqrt{D}$ and $\sqrt{D}$. Consequently, $H$ fixes the geodesic $mathcal{C}_{\sqrt{D}}$ running from $-\sqrt{D}$ to $\sqrt{D}$, set-wise. In particular, the radius of the semicircle defining $\mathcal{C}_{\sqrt{D}}$ is $\sqrt{D}$, so the point $i\sqrt{D}\in\mathcal{C}_{\sqrt{D}}$.  Pointwise, each $h_t$ translates a point rightward along $\mathcal{C}_{\sqrt{D}}$. The orbit of $i \sqrt{D}$ under the nontrivial discrete subgroup $H'=H\cap \Gamma$ partitions the orbit $H \cdot i\sqrt{D}$ into congruent intervals. The resulting quotient $H'\backslash H \cdot i\sqrt{D}$ is compact.
\end{document}  