
\documentclass[11pt]{article} % use larger type; default would be 10pt

\usepackage[utf8]{inputenc} % set input encoding (not needed with XeLaTeX)
\usepackage{geometry} % to change the page dimensions
\geometry{a4paper} % or letterpaper (US) or a5paper or....

\usepackage{amsmath}
\usepackage{amsthm}
\usepackage{amsfonts}

\title{Personal Statement}
\author{Justin Katz}
\theoremstyle{definition}
\newtheorem*{fact}{Fact}
\begin{document}
\maketitle
Let $X$ be either a locally compact Hausdorff space or a smooth manifold and let $G$ be either a topological or Lie group that acts on $X$ transitively, compatibly in the relevent category (e.g. continuously, smoothly, respectively). Fix some point $x\in X$ and let $G_x$ be the \emph{isotropy} subgroup of $x$ in $G$, those $G$ elments that fix $x$.

\begin{fact}
The $G$-space $X$  is $G$-isomorphic (in the relevent category) to the \emph{quotient space} $G/G_x$ under the assignment
	\begin{equation*}
		gG_x \longmapsto g \cdot x
	\end{equation*}
\end{fact}

To see the power of the above fact, consider the following sketch of the existence of Fourier series of square integrable functions on the circle $S^1=\left \{e^{2\pi i \theta}:\theta \in \mathbb{R} \right \}$. Ignore for the moment that $S^1$ caries a natrual group structure, instead just viewing it as a smooth manifold acted on transitively by the Lie group $\mathbb{R}$ via rotation, $r \times e^{2\pi i\theta}\mapsto e^{2\pi i (\theta+r)}$. The complex exponential is $2\pi i$ periodic, so the the isotropy subgroup of any point is $\mathbb{Z}$. Thus, by the above fact we have a model of $S^1$ as a quotient space $\mathbb{R}/\mathbb{Z}$. 

As a locally compact group, $\mathbb{R}$ has a unique translation invariant measure (this is the Lebesgue measure), and that measure descends to a unique translation invariant measure the quotient $\mathbb{R}/\mathbb{Z}$ so it is sensible to talk about the Hilbert space $L^2(\mathbb{R}/\mathbb{Z})$. Similarly, the Lie group $\mathbb{R}$ admits a canonical translation invariant differential operator (the Laplacian) $-d^2/dx^2$ and similarly that operator descends to the quotient so we can talk about the smooth structure of the quotient $C^\infty(\mathbb{R}/\mathbb{Z})$.  Furthermore, the smooth and Hilbert structures on the quotient are perfectly compatible in the sense that the differential operator is symmetric and positive with respect to the inner product. 

Solving the differential equation on the circle $-d^2/dx^2 f =\lambda f$ shows that the only eigenfunctions for the Laplacian in $C^\infty(\mathbb{R}/\mathbb{Z})$ are the integer frequency complex exponentials, and the eigenvalues are integer multiples of $2\pi$. By a density argument, these are \emph{all} of the eigenfunctions for the Laplacian in $L^2(\mathbb{R}/\mathbb{Z})$. Despite the Laplacian initially only being defined on a dense subspace of $L^2(\mathbb{R}/\mathbb{Z})$, and being screamingly discountinuous (it is not even continuous on \emph{polynomials}!), it provably has a unique self adjoint extension with positive self adjoint \emph{compact} resolvent. Further, the compact resolvent of the extension of the Laplacian provably has the same eigenfunctions as the Laplacian, so by the \emph{spectral theorem} for compact self adjoint operators, $L^2(\mathbb{R}/\mathbb{Z})$ has an orthonormal basis of eigenfunctions for the Laplacian. That is, the integer frequency oscillations are an orthonormal basis for $L^2(\mathbb{R}/\mathbb{Z})$, which is to say that square integrable functions on the circle admit Fourier series. 

In the above we did not use the fact that $\mathbb{R}/\mathbb{Z}$ is a group, we did not use approximate identities, not a single epsilon was brought to bear on the situation. All that we used was the fact that we can view a \emph{space} as a quotient of a \emph{group}, groups admit natural \emph{structure}, and that structure descends to the \emph{quotient}. The rest of the proof was fairly technical but uncomplicated functional analysis that applies to analogous situations in overwhelming generality. Thinking intrinsically \emph{clarified} the causual mechanism driving a well known phenomenon, rather than obscuring the problem in coincidental specifics. Indeed, as I demonstrate in my thesis (following Professor Paul Garrett at UMinn) analagous techniques allow one to prove an orthogonal decomposition of suitible restrictions of level one \emph{cuspforms} on the quotient of the upper halfplane $L^2(\Gamma\backslash \mathfrak{H})$, an important result in the theory of \emph{autmorphic forms}.

The paradigm of doing mathematics intrinsically, instilled in me by my mentor Professor Jerry Shurman, has radically influenced the way I see and in turn \emph{do} mathematics.  As a concrete example of this influence, consider my Summer 2014 project (funded by a RCSCF grant) on computer vision. Computer vision is a \emph{crusingly} applied science, and my task was to use the abstract techniques of algebraic geometry and commutative aglebra to solve problems that arise in multi-camera scenarios. A single camera is modeled by a point (the focal point), a projective plane (the film) in an ambient  projective $3$-space (the world). A picture is taken of a point by intersecting the unique line between the point in space and the focal point with the image plane to obtain a unique image point. A multi-camera scenario is then modeled by a collection of distinct focal points, distinct film planes, all in a single ambient space. A multi-photo is taken by doing the above process with each camera to obtain a unique point on each film plane. Following Professors Anders Heyden and Kalle Astrom from ULund, we unembedded the scenario by viewing each camera as an abstract projective linear mapping projective $3$-space onto projective $2$-space. We then modeled the whole scenario as a conglomerate map from $3$-space into a product of $2$-spaces, where each component map corresponds to a single camera. The image of the conglomerate map carries all concievable information about point correspondences between film planes. An abstract product of projective spaces is ammenable to the techniques of algebraic geometry, so one might wonder if the image under the conglomerate map is a variety. Heyden and Astrom proved that this is never so in 1996, the image is never a variety, but its Zariski closure is the variety associated to a collection of polynomials that arise naturally from geometric considerations:  the so called bilinear and trilinear constraints. That said, generating \emph{all} of these constraints is prohibitively computationally taxing, so one would hope that the same variety could be generated by an adroitly chosen subset of the constraints with cardinality the codimension of the variety (because that is the lower bound for a set of generators). Heyden and Astrom proved that this also cannot happen, but conjectured that a particular frugal choice of generators will always generate a variety that has the variety of interest as an irreducible subvariety. Professor Irena Swanson and I were able to prove this conjecture on the ideal side by demonstrating a primary decomposition of the relevent ideal. Fascinatingly, the primary decomposition is independent of the camera parameters (i.e. the coefficients in the matrices that respresent the camera maps) so is a truly intrinsic fact about the structure of multiview scenarios. 
\end{document}