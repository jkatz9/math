\documentclass[11pt]{amsart}
\usepackage{geometry}                % See geometry.pdf to learn the layout options. There are lots.
\geometry{letterpaper}                   % ... or a4paper or a5paper or ... 
%\geometry{landscape}                % Activate for for rotated page geometry
%\usepackage[parfill]{parskip}    % Activate to begin paragraphs with an empty line rather than an indent
\usepackage{graphicx}
\usepackage{amssymb}
\usepackage{epstopdf}
\usepackage{amsthm}
\usepackage{hyperref}
\renewcommand{\tilde}{\widetilde}
\newcommand{\hmod}{H^{1}(\modcurve)}
\newcommand{\cmod}{C^{\infty}_{b}(\modcurve)}
\newcommand{\tr}{\Tr}
\newcommand{\Ch}{\operatorname{Ch}}
\newcommand{\ind}{\operatorname{ind}}
\newcommand{\ip}[2]{\langle #1, #2 \rangle}
\newcommand{\C}{\mathbb{C}}
\newcommand{\Z}{\mathbb{Z}}
\newcommand{\R}{\mathbb{R}}
\newcommand{\Q}{\mathbb{Q}}
\newcommand{\til}[1]{\tilde{#1}}
\newcommand{\PSL}{\operatorname{PSL}}
\newcommand{\diag}{\text{diag}}
\newcommand{\half}{\mathfrak{H}}
\newcommand{\intring}{\mathcal{O}}
\newcommand{\sch}{\mathscr{S}}
\newcommand{\tor}{\mathbb{T}}
\newcommand{\bep}{\mathfrak{B}}
\newcommand{\ipd}{\ip{\cdot}{\cdot}}
\newcommand{\fredth}{\tilde{\theta}}
\newcommand{\resth}{\tilde{\theta}^{-1}}
\renewcommand{\sl}{\mathfrak{sl}}
\renewcommand{\phi}{\varphi}
\newcommand{\ltmod}{L^{2}(\modcurve)}
\renewcommand{\k}{\mathfrak{k}}
\newcommand{\halfred}{\tilde{\laphalf}}
\newcommand{\halfres}{(\lambda-\halfred)^{-1}}
\newcommand{\zpz}{\Z/p\Z}
\newcommand{\fredlapsn}{\tilde{\Delta}^{\nsphere}}
\newcommand{\ressn}{(1-\fredlapsn)^{-1}}
\newcommand{\vol}{\operatorname{vol}}
\newcommand{\ijsum}{\sum_{i < j}}
\newcommand{\poly}{\C[x_{1},\ldots,x_{n}]}
\newcommand{\rest}{\big|}
\newcommand{\dtx}{\frac{\partial^{2}}{\partial x^{2}}}
\newcommand{\dty}{\frac{\partial^{2}}{\partial y^{2}}}
\newcommand{\g}{\mathfrak{g}}
\newcommand{\dophalf}{\frac{\dop x \dop y}{y^{2}}}
\newcommand{\tbt}[4]{\left[ \begin{smallmatrix}
		#1 & #2 \\
		#3 & #4 
	\end{smallmatrix} \right] }
\newcommand{\Tbt}[4]{\left[ \begin{matrix}
		#1 & #2 \\
		#3 & #4 
	\end{matrix} \right] }
\newcommand{\Tr}{\operatorname{tr}}
\renewcommand{\r}{\mathfrak{r}}
\newcommand{\ciamod}{C^{\infty}_{a}(\modcurve)}
\newcommand{\cicmod}{C^{\infty}_{c}(\modcurve)}
\newcommand{\hamod}{H^{1}_{a}(\modcurve)}
\newcommand{\ltmoda}{L^{2}_{a}(\modcurve)}
\newcommand{\ltmodcts}{L^{2}_{\text{cts}}(\modcurve)}
\newcommand{\mel}{\mathcal{M}}
\newcommand{\ltmodcfm}{L^{2}_{\text{cfm}}(\modcurve)}
\newcommand{\res}{\operatorname{res}}
\newcommand{\re}{\operatorname{Re}}
\newcommand{\im}{\operatorname{Im}}
\newcommand{\Ad}{\operatorname{Ad}}
\newcommand{\Aut}{\operatorname{Aut}}
\renewcommand{\O}{\operatorname{O}}
\newcommand{\To}{\longrightarrow}
\newcommand{\Mapsto}{\longmapsto}
\newcommand{\inc}{\operatorname{inc}}
\newcommand{\gothic}[1]{\mathfrak{#1}}
\newcommand{\so}{\mathfrak{so}}
\newcommand{\Fund}{\mathcal{F}}
\newcommand{\partone}[1]{\frac{\partial}{\partial x_{#1}}}
\newcommand{\parttwo}[1]{\frac{\partial^{2}}{\partial x_{#1}^{2}}}
\newcommand{\PGL}{\operatorname{PGL}}
\newcommand{\F}{\mathbb{F}}
\newcommand{\ol}{\overline}
\newcommand{\inj}{\hookrightarrow}
\newcommand{\surj}{\twoheadrightarrow}
\newcommand{\trace}{\operatorname{Tr}}
\newcommand{\proj}{\operatorname{proj}}
\newcommand{\der}{\frac{d^2}{dx^2}}
\newcommand{\four}{\mathcal{F}}
\newcommand{\laphalf}{\Delta^{\half}}
\newcommand{\eps}{\varepsilon}
\newcommand{\dom}{\operatorname{dom}}
\newcommand{\id}{\operatorname{id}}
\newcommand{\Ind}{\operatorname{Ind}}
\newcommand{\Res}{\operatorname{Res}}
\newcommand{\End}{\operatorname{End}}
\newcommand{\SL}{\operatorname{SL}}
\newcommand{\GL}{\operatorname{GL}}
\newcommand{\SO}{\operatorname{SO}}
\newcommand{\Orth}{\operatorname{O}}
\newcommand{\dop}{\,{\rm d}}
\newcommand{\nsphere}{S}
\newcommand{\ltnsphere}{L^{2}(\nsphere)}
\newcommand{\honsphere}{H^{1}(\nsphere)}
\newcommand{\lthalf}{L^{2}(\half)}
\newcommand{\ltg}{L^{2}(G)}   
\newcommand{\modcurve}{\Gamma \backslash \half}
\newcommand{\Gal}{\operatorname{Gal}}
\newcommand{\ipn}[2]{\langle #1, #2 \rangle_1}
\newcommand{\Graph}{\operatorname{graph}}
\newcommand{\mhaar}[1]{\frac{\operatorname{d}#1}{#1}}
\newcommand{\rn}{\R^{n}}
\newcommand{\laprn}{\Delta^{\rn}}
\newcommand{\lapsn}{\Delta^{\nsphere}}
\newcommand{\ltrn}{L^{2}(\rn)}
\newcommand{\cirn}{C^{\infty}(\rn)}
\newcommand{\horn}{H^{1}(\rn)}
\newcommand{\inv}{{-1}}
\newcommand{\p}{\mathfrak{p}}
\renewcommand{\P}{\mathfrak{P}}
\newcommand{\frob}[1]{\operatorname{frob}(#1)}
\newcommand{\Ell}{\mathcal{L}}
\newcommand{\arccosh}{\operatorname{arccosh}}
\newcommand{\I}{\mathbb{I}}
\newcommand{\A}{\mathbb{A}}
\newcommand{\Of}{\mathcal{O}}
\newcommand{\Isom}{\operatorname{Isom}}
\newcommand{\lmod}{\backslash}
\newcommand{\rmod}{/}
\newcommand{\Com}{\operatorname{Com}}
\newcommand{\hecke}{\mathcal{H}}
\newcommand{\ord}{\operatorname{ord}}
\newcommand{\mf}{\mathfrak}
\newcommand{\q}{\textbf{q}}
\newcommand{\normset}{\mathcal{N}}
\renewcommand{\Ell}{\mathcal{L}}
\newcommand{\infl}{\operatorname{Infl}}
\newcommand{\vchar}{\operatorname{Vchar}}
\newcommand{\nspec}{\mathcal{N}}
\newcommand{\prim}{\operatorname{prim}}

\theoremstyle{definition}
\newtheorem{claim}{Claim}
\newtheorem*{question*}{Question}
\newtheorem{thm}{Theorem}
\newtheorem{prop}{Proposition}
\newtheorem{remark}{Remark}
\newtheorem*{remark*}{Remark}
\newtheorem{mydef}{Definition}
\newtheorem{fact}{Fact}
\newtheorem{lemma}{Lemma}
\newtheorem{cor}{Corollary}


\title{Homogeneous Spaces as Quotients of Groups}
\author{Justin Katz}
%\date{}                                           % Activate to display a given date or no date

\begin{document}
\maketitle
This statement and proof draws heavily on the exposition of Paul Garrett, in the appendix of \url{http://www.math.umn.edu/~garrett/m/mfms/notes/02_solenoids.pdf}.

Let $X$ be a locally compact Hausdorff space and $G$ be a topological group acting continuously, transitively on $X$. Fix a point $x\in X$ and let $G_x$ be the isotropy subgroup of $G$ at $x$. 
\begin{claim}
	The $G$-\emph{space} $X$ is homeomorphic to the \emph{quotient space} $G/G_x$ under the assignment
	\begin{equation*}
		gG_x \mapsto gx
	\end{equation*}
\end{claim}
\begin{proof}
	By the transitivity of the $G$-action on $X$, the map $gG_x\mapsto gx$ surjects. Because $G_x$ fixes $x$, the map injects. To prove the claim, it suffices to show that the map is continuous and open.

The topology on the quotient with projection $\pi:G\to G/G_x$ is uniquely characterized by the condition that any continuous map out of $G$ that is constant on $G_x$ factors uniquely through $\pi$ to a continuous map out of $G/G_x$. The map $g \mapsto gx$ is continuous as a restriction of the action, and is constant on $G_x$ by definition of isotropy. Thus $g \mapsto gx$ factors uniquely through $\pi$ to a continuous map out of $G/G_x$. The map $gG_x \mapsto gx$ fits the bill, so is continuous.

To prove that $gG_x\mapsto gx$ is open, let $U$ be a neighborhood of $g \in G$. For reasons that will become apparent later, we want a compact neighborhood $V$ of $1$ so that $gV^2=\{gvh:v,h\in V\}\subset U$. To show such a compact set $V$ exists first show the result at $g=1$.The inverse image of the open $U$ under the (continuous) product map $h\times k\mapsto hk$ is again open. Open sets in the product topology are generated by products of opens in the producands, so the inverse image of $U$ under multiplication contains a product of opens $W_1\times W_2$ each containing $1$. Let $W=W_1\cap W_2$ so that $W^2\subset W_1 \cdot W_2 \subset U$ where the last containment comes from the definition of $W_1\times W_2$ as a subset of the inverse image of $U$ under multiplication. Furthermore $G$ is Hausdorff, so there is some neighborhood $W'$ of $1$ contained in $W$ such that $\overline{W'}$ is compact and sits inside $W$. Let $V=\overline{W'}$ so that $V^2\subset W^2\subset U$. For generic $g$ with neighborhood $U$, the open $g^{-1}U$ is a neighborhood of $1$, and the above discussion gives the result. We can \emph{balance} $V$ about $1$ (i.e. make it such that $V=V^{-1}$) by setting $V\mapsto V \cap V^{-1}$.

Next, we show that the whole group $G$ can be covered by countably many translates of the compact $V$. First, we show the result for some open $W$ in $V$. Let $\{U_1,U_2,...\}$ be a (countable) basis for $G$. For each $g\in G$, by the definition of basis, the open $gW$ is the union of those $U_i\subset gW$. As such, for each $g\in G$ there is a smallest index index $j(g)$ such that $g \in U_{j(g)} \subset gW$. For each index $i$ pick some $g_i$ in $j^{-1}(i)$ so that $g_i \in U_i \subset g_i W$. By definition of the map $g\mapsto j(g)$, we have $j^{-1}(i)\subset U_i \subset g_i W$. Taking the union over all (countably many) indices $i$, $\cup j^{-1}(i)=G \subset \cup g_i W$ as desired. We can certainly replace $W$ by its compact superset $V$ so that $G=\cup g_i V$ as claimed.

We are now ready to prove that the map $gG_x\mapsto gx$ is open. Recall that $U$ is a neighborhood of some point $g\in G$, $V$ is a balanced compact in $U$ such that $V^2\subset U$. We want to show that $Ux$ is open.  Recall the version of the \emph{Baire category theorem}:
\begin{quote}
	\emph{A locally compact Hausdorff space is not a countable union of nowhere dense sets}
\end{quote}
In particular, by transitivity of the group action we can cover the space $X$ by countably many $Vx$ translates $X=\cup g_i V x$. Note that each translate $g_i V x$ is closed, being the continuous image of a compact $g_i V$ in a Hausdorff space.  By Baire, some $g_m V x$ contains a nonempty open $S$. Let $h$ be such that $g_m h x \in S$ and write 
	\begin{equation*}
		gx=g(g_m h)^{-1}(g_m h)x\in g h^{-1}g_{m}^{-1} S
	\end{equation*} 
The rightmost set in the above display is again open in $X$ because translation in $X$ by a fixed element of $G$ is a homeomorphism. Compute
	\begin{align*}
		gx \in g h^{-1}g_{m}^{-1} S  & \subset gh^{-1}g_{m}^{-1} g_m V x  &\qquad \left(\text{By definition of S} \right) \\
						& \subset g h^{-1} V x  \\
						& \subset g V^{-1} V x \\
						& = g V^2 x \qquad &\left( V \text{ is balanced about } 1 \right) \\
						& \subset Ux  \qquad &\left( \text{By definition of } V \right)	,			
	\end{align*}
meaning $gx$ is an interior point of $Ux$. The group element $g \in U$ was arbitrary so $Ux$ is open, proving the claim.
\end{proof}
\begin{remark}
If\footnote{I essentially follow Warner in his text \emph{Foundations of Differntiable Manifolds and Lie Groups}, roughly page 120} $X$ is a smooth manifold and $G$ is a Lie group acting on $X$ smoothly, then the homeomorphism in the conclusion of the above claim is actually a diffeomorphism. Indeed as the isotropy subgroup $G_x$ is closed, the quotient $G/G_x$ has a unique smooth structure so that any smooth map out of $G$ constant on $G_x$ factors uniquely through the projection $\pi$ to a smooth map out of $G/G_x$. Because $G/G_x$ is already homeomorphic to $G/G_x$ and (by the mapping property of quotients) the map $f:gG_x\mapsto gx$ is smooth, (by the inverse function theorem) it suffices to show that the differential  $df_{1G_x}:T(G/G_x)_{1G_x}\to TX_x$ is nonsingular. Note that the map $h:G\to X$ defined by $g\mapsto gx$ is the composition $f\circ \pi$. Thus, to show that $df_{1G_x}$ is nonsingular, it suffices to show that the kernel of $dh_1$ is exactly the kernel of $d\pi_1$, i.e. the tangent space $T(G_x)_{1}$. One direction is easy: $\ker dh_1$ certainly contains $T(G_x)_1$, because $h$ is constant on $G_x$. To prove the other direction, let $z \in \ker dh_1$ and let $Z$ be the corresponding left invariant vector field on $G$. Recall that the left invariance of $Z$ is the equality $d(L_\gamma)_{Z(\cdot)}= Z\circ L_\gamma(\cdot)$ where $L_\gamma :g\mapsto \gamma g$ is the (smooth) left action of $G$ on itself. That $Z$ corresponds to $z$ means that $Z$ is the unique vector field such that $\frac{d}{dr}\exp(rZ)|_0=z$. To show $z\in T(G_x)_1$ it suffices to show that $\exp(tZ) \in G_x$ for all $t\in \R$, meaning $\exp(tZ)$ fixes $x$ for all $t$. Consider the curve $\alpha: t\mapsto h(\exp(tZ))$ in $M$. If the tangent vector to $\alpha$ is zero at every $t$ then $\alpha$ is constant. In particular, $\alpha(0)=h(1)=x$ so if $\alpha$ is constant then $\exp(tZ)$ fixes $x$ for all $t$ and is thus in $G_x$. To prove that the tangent vector to $\alpha$ is zero, first compute for $t=0$
	\begin{align*}
		d(\alpha)_0&=d(h)_{1}\circ \frac{d}{dr}\exp(r Z)|_0\\
				 &=dh_1(z) &\qquad \left(Z \text{ corresponds to } z\right)\\
				 &=0		    &\qquad \left(z\in \ker{d(h)_1} \right).
	\end{align*}
To prove that $\frac{d}{dr}\alpha(r)|_t=0$ for all $t$ notice that the map $h$ is invariant under conjugation by a group element $\gamma$ i.e. $\gamma \cdot h \circ L_\gamma^{-1} (g)=\gamma \cdot \gamma^{-1} \cdot g x=gx=h(g)$. In particular, for $\gamma=\exp(tZ)$ compute
	\begin{align*}
		\frac{d}{dr}\alpha(r)|_t &= d(h)_{\exp(tZ)} \circ \frac{d}{dr} \exp(r Z)|_t &\left( \text{Chain rule} \right) \\
				  &= d(\exp(tZ)\cdot h \circ L_{\exp(-tZ)})_{\exp(tZ)}\circ \frac{d}{dr}e^{r Z})|_t  &\left( \text{Invariance of $h$ under conjugation} \right)\\
				  &= d(\exp(tZ) \cdot h )_{L_{\exp(-tZ)}(\exp(tZ))}\circ \frac{d}{dr}L_{\exp(-tZ)} \exp(r Z)|_t &\left( \text{Chain rule} \right) \\
				  &=d(\exp(tZ) \cdot h )_1 \circ \frac{d}{dr} \exp(r Z)|_0 &\left( \text{Definition of left action, changing variables $t\mapsto r-t$} \right) \\
				  &=0 &\left(\frac{d}{dr} \exp(r Z)|_0=z\in \ker dh_1\right).
	\end{align*}
	Thus the curve $\alpha$ is constant, so $\exp(tZ)$ fixes $x$ for all $t$, meaning the tangent vector $z$ corresponding to $Z$ is in $T_1 G_x$. Therefore $\ker dh_1 = T_1 G_x$, so that smooth bijection $f:gG_x \mapsto gx$ has nonsingular derivative, and is thus a diffeomorphism as desired.
\end{remark}
\end{document}  