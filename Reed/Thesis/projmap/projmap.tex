
\footline={}
\rightline{4 June 2014}
\bigskip

\font\tenbbold=msbm10
\def\bbP{{\hbox{\tenbbold P}}}
\def\midline#1{\raise 0.20em\hbox to #1{\vrule depth0pt height 0.405pt width #1}}
\def\dashrightarrow{\midline{0.4em} \mkern 3.6mu\midline{0.4em} \mkern -0.6mu \mathrel{\rightarrow}}
\def\qedbox{\hbox{\vbox{\hrule\hbox{\vrule\kern3pt\vbox{\kern6pt}\kern3pt\vrule}\hrule}}}% in eqalignno, use: "& \qedbox \cr"
\def\qed{\unskip\hfill\qedbox}

\noindent
{\bf Theorem:}
Let $k$ be a field,
$X_1, \ldots, X_c$ variables over $k$,
and for $j = 1, \ldots, n$ and $i = 0, \ldots, r_i$
let $f_{ij} \in k[X_1, \ldots, X_c]$.
Let
$$
F : k^c \dashrightarrow \bbP^{r_1} \times \cdots \times \bbP^{r_n}
$$
be given by the projection of $F(\underline x)$ onto $\bbP^{r_j}$
as $(f_{0j}(\underline x), \ldots, f_{r_jj}(\underline x))$.
In the polynomial ring $R = k[X_1, \ldots, X_c, s_1, \ldots, s_n, Y_{ij}: j =
1, \ldots, n, i = 0, \ldots, r_j]$
define the ideal
$$
I =
\sum_{j=1}^n (Y_{0j} - s_j f_{0j}(\underline X),
\ldots,
Y_{r_jj} - s_j f_{r_jj}(\underline X)).
$$
Then the defining ideal in $k[Y_{ij}: i, j]$
of the closure of the image of $F$ is
$$
J = \left( I : \left( \prod_{j=1}^n s_j (Y_{0j}, \ldots, Y_{r_jj})
\right)^\infty
\right)
\cap k[Y_{ij}: i, j].
$$

\noindent
{\it Proof:}
Let $g \in J$, and $b$ in the image of $F$.
Then there exists $p \in k^c$ such that $b = F(p)$.
Also,
there exists a large integer $e$ such that
$\left( \prod_{j=1}^n s_j (Y_{0j}, \ldots, Y_{r_jj})
\right)^e g \subseteq I$.
Hence
$\left( \prod_{j=1}^n s_j (f_{0j}(p), \ldots, f_{r_jj}(p)) \right)^e g(F(p))
= 0$.
Since $b = F(p)$,
for each $j = 1, \ldots, n$,
one of $f_{ij}(p)$ is non-zero,
and since $s_1 \cdots s_n \not = 0$,
we have that $g(b) = g(F(p)) = 0$.
Thus $g$ vanishes on the image of $F$,
and so on the closure of the image of $F$.
It follows that the closure of the image of $F$ is in $Z(J)$.

We call an element $g$ of $k[Y_{ij}: i, j]$ multihomogeneous
if for each $j = 1, \ldots, n$,
the total $(Y_{0j}, \ldots, Y_{r_jj})$-degree of all the terms appearing
in $g$ is the same.
Any $g$ can be written as the sum of multihomogeneous polynomials.
Now let $g \in k[Y_{ij}: i, j]$ vanish on the image of $F$.
Since the codomain is a product of projective spaces,
each multihomogeneous component of $g$ also vanishes on the image of $F$.
If we can prove that each
multihomogeneous component of $g$ is in $J$,
then $g$ is in $J$.
So we may assume that $g$ is multihomogeneous and vanishes on the image
of $F$.
So for any $\underline x \in k^c$,
$g(\underline f_{ij} (\underline x)) = 0$
(since $g$ is multihomogeneous,
this is true whether $\underline x$ is in the domain of $F$ or not).
In short,
$g(\underline f_{ij} (\underline X)) = 0$ (variables $X$).
Again by multihomogeneity,
$g(\underline {s_j f_{ij}} (\underline X))
= s_1^{e_1} \cdots s_n^{e_n} g(\underline f_{ij} (\underline X)) = 0$
for some non-negative integers $e_1, \ldots, e_n$.
But then
$s_1^{e_1} \cdots s_n^{e_n} g
\in (Y_{ij} - s_j f_{ij}(\underline X))$,
so that $g \in J$.
It follows that the polynomials that vanish on the image of $F$ are in
$J$,
and so that the zeros of $J$ are contained
in the zeros of the polynomials that vanish on the image of $F$,
i.e.,
that the zeros of $J$ are contained in the closure of the image of $F$.
\qed

